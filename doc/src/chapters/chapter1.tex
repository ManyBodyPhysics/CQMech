
\chapter{Introduction}

\section{Choice of programming language}

As programming language we have ended up with preferring 
C++, but all examples discussed in the text have their 
corresponding Fortran and Python programs on the webpage of this text.
 
Fortran (FORmula TRANslation) was introduced in 1957 and remains in many 
scientific computing environments the language of choice.
The latest standard, see Refs.~\cite{f95ref,metcalf1996,marshall1995,f2003}, 
includes extensions that are
familiar to users of C++. 
Some of the most important features of Fortran  include recursive
subroutines, dynamic storage allocation and pointers, 
user defined data structures, modules,
and the ability to manipulate entire arrays. 
However, there are several good reasons for 
choosing C++ as programming language for scientific and engineering
problems. Here are some:
\begin{itemize}
\item C++ is now the dominating language in Unix and Windows environments. It is widely available and is
the language of choice for system programmers.  It is very widespread for developments of non-numerical  software 
\item The C++ syntax has inspired lots of popular languages, such as Perl, Python and Java.
\item It is an extremely portable language, all Linux and Unix operated machines have a 
C++ compiler.
\item In the last years there has been an enormous effort towards developing numerical libraries
for C++. Numerous tools (numerical libraries such as MPI\cite{gropp1999,mpiref,cmpi}) are written in C++
and interfacing them requires knowledge of C++. 
Most C++ and Fortran compilers compare fairly well when it comes to speed and
numerical efficiency. Although Fortran 77 and C are regarded as slightly faster than C++ or Fortran,
compiler improvements during the last few years have diminshed such differences. The Java numerics project
has lost some of its steam recently, and Java is therefore normally slower than C++ or Fortran.
\item Complex variables, one of Fortran's strongholds, can also be defined in the new 
ANSI C++ standard. 
\item C++ is a language which catches most of the errors as early as possible, typically at compilation
time. Fortran has some of these features if one omits implicit variable declarations.
\item C++ is also an object-oriented language, to be contrasted with C and Fortran.
This means that it supports three fundamental ideas, namely objects, class hierarchies and polymorphism.
Fortran has, through the \verb? MODULE?  declaration the capability of defining classes, but lacks 
inheritance, although polymorphism is possible. Fortran is then considered as an object-based
programming language, to be contrasted with C++ which has the capability of relating classes
to each other in a hierarchical way.
\end{itemize}

An important aspect of C++ is its richness with more than 60 keywords allowing for a good balance between object orientation
and numerical efficiency. Furthermore, careful programming can results in an efficiency close to
Fortran 77.  The language is well-suited for large projects and has presently good standard libraries suitable
for computational science projects, although many of these still lag behind the large body of libraries for numerics
available to Fortran programmers. However, it is not difficult to interface libraries written in Fortran with C++
codes, if care is exercised.
Other weak sides are the fact that it can be easy to write inefficient code  and that there are many ways of writing the
same things, adding to the confusion for beginners  and professionals as well.  The language is also under continuous
development, which often causes portability problems.

C++ is also a difficult language to learn. Grasping the basics is rather straightforward, but takes time
to master. A specific problem which often causes 
unwanted or odd errors is dynamic memory management.

The efficiency of C++ codes are close to those provided by Fortran. This means often that a code
written in Fortran 77 can be faster, however  for large numerical projects C++ and Fortran 
are to be preferred. If speed is an issue, one could port critical parts of the code to Fortran 77.


