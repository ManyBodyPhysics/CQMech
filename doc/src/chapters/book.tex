\documentclass[graybox,envcountchap,sectrefs]{svmono}

\usepackage{mathptmx}
\usepackage{helvet}
\usepackage{courier}

\usepackage{type1cm}         
\usepackage{exercise}
\usepackage{makeidx}         % allows index generation
\usepackage{graphicx}        % standard LaTeX graphics tool
                             % when including figure files
\usepackage{multicol}        % used for the two-column index
\usepackage[bottom]{footmisc}% places footnotes at page bottom
%\usepackage{graybox}
\usepackage[usenames,dvipsnames,x11names]{xcolor}
\usepackage{tikz}
\usetikzlibrary{arrows,snakes,shapes}

 \usepackage{listings}
 \usepackage{graphicx}
 \usepackage{epic}
 \usepackage{eepic}
 \usepackage{a4wide}
 \usepackage{color}
 \usepackage{amsmath}
 \usepackage{amssymb}
% \usepackage[dvips]{epsfig}
% \usepackage{psfig}
 \usepackage[T1]{fontenc}
 \usepackage{cite} % [2,3,4] --> [2--4]
 \usepackage{shadow}
 \usepackage{hyperref}
 \usepackage{bezier}
 \usepackage{pstricks}
% \usepackage{refcheck}
\setcounter{tocdepth}{2}
%\usepackage{gnuplot-lua-tikz}


\usepackage{textcomp,type1ec,pdfpages}
\usepackage{bera}

\definecolor{dkgreen}{rgb}{0,0.6,0}
\definecolor{gray}{rgb}{0.5,0.5,0.5}
\definecolor{mauve}{rgb}{0.58,0,0.82}

 \lstset{language=c++}
 \lstset{alsolanguage=[90]Fortran}
 \lstset{alsolanguage=python}
% \lstset{basicstyle=\small}
 \lstset{backgroundcolor=\color{white}}
 \lstset{frame=single}
 \lstset{stringstyle=\ttfamily}
 \lstset{keywordstyle=\color{red}\bfseries}
 \lstset{commentstyle=\itshape\color{blue}}
 \lstset{showspaces=false}
 \lstset{showstringspaces=false}
 \lstset{showtabs=false}
 \lstset{breaklines}
 

% Default settings for code listings
% \lstnewenvironment{Python}[1]{
\lstset{%frame=tb,
  language=c++,
  alsolanguage=python,
  %aboveskip=3mm,
 % belowskip=3mm,
  showstringspaces=false,
  columns=flexible,
  basicstyle={\footnotesize\ttfamily},
  numbers=none,
  numberstyle=\tiny\color{gray},
  commentstyle=\color{dkgreen},
  stringstyle=\color{mauve},
  frame=single,  
  breaklines=true,
  %%%% FOR PYTHON 
  otherkeywords={\ , \}, \{},
  keywordstyle=\color{blue},
  emph={void, ||, &&, break, class,continue, delete, else,
  for, if, include, return,try,while},
  emphstyle=\color{black}\bfseries,
  emph={[2]True, False, None, self},
  emphstyle=[2]\color{dkgreen},
  emphstyle=[2]\color{red},
  emph={[3]from, import, as},
  emphstyle=[3]\color{blue},
  upquote=true,
  morecomment=[s]{"""}{"""},
  commentstyle=\color{green}\slshape, %%% cambie gray por green
  emph={[4]1, 2, 3, 4, 5, 6, 7, 8, 9, 0},
  emphstyle=[4]\color{blue},
  breakatwhitespace=true,
  tabsize=2
}

\renewcommand{\lstlistlistingname}{Code Listings}
\renewcommand{\lstlistingname}{Code Listing}
\definecolor{gray}{gray}{0.5}
\definecolor{green}{rgb}{0,0.5,0}

\lstnewenvironment{Python}[1]{
\lstset{
language=python,
basicstyle=\footnotesize\setstretch{1},
stringstyle=\color{red},
showstringspaces=false,
alsoletter={1234567890},
otherkeywords={\ , \}, \{},
keywordstyle=\color{blue},
emph={access,and,break,class,continue,def,del,elif ,else,%
except,exec,finally,for,from,global,if,import,in,is,%
lambda,not,or,pass,print,raise,return,try,while},
emphstyle=\color{black}\bfseries,
emph={[2]True, False, None, self},
emphstyle=[2]\color{red},
emph={[3]from, import, as},
emphstyle=[3]\color{blue},
upquote=true,
morecomment=[s]{"""}{"""},
commentstyle=\color{dkgreen}\slshape, % el color era gray pero lo cambie a verde
emph={[4]1, 2, 3, 4, 5, 6, 7, 8, 9, 0},
emphstyle=[4]\color{blue},
framexleftmargin=1mm, framextopmargin=1mm, rulesepcolor=\color{blue},
breakatwhitespace=true,
tabsize=2
}}{}


\lstnewenvironment{C++}[1]{
\lstset{
language=c++,
% basicstyle=\ttfamily\small\setstretch{1},
basicstyle=\footnotesize\setstretch{1},
stringstyle=\color{red},
showstringspaces=false,
alsoletter={1234567890},
otherkeywords={\ , \}, \{},
keywordstyle=\color{blue},
emph={access,and,break,class,continue,def,del,elif ,else,%
except,exec,finally,for,from,global,if,import,in,is,%
lambda,not,or,pass,print,raise,return,try,while},
emphstyle=\color{black}\bfseries,
emph={[2]True, False, None, self},
emphstyle=[2]\color{red},
emph={[3]from, import, as},
emphstyle=[3]\color{blue},
upquote=true,
morecomment=[s]{"""}{"""},
commentstyle=\color{dkgreen}\slshape, % el color era gray pero lo cambie a verde
emph={[4]1, 2, 3, 4, 5, 6, 7, 8, 9, 0},
emphstyle=[4]\color{blue},
% literate=*{:}{{\textcolor{blue}:}}{1}%
% {=}{{\textcolor{blue}=}}{1}%
% {-}{{\textcolor{blue}-}}{1}%
% {+}{{\textcolor{blue}+}}{1}%
% {*}{{\textcolor{blue}*}}{1}%
% {!}{{\textcolor{blue}!}}{1}%
% {(}{{\textcolor{blue}(}}{1}%
% {)}{{\textcolor{blue})}}{1}%
% {[}{{\textcolor{blue}[}}{1}%
% {]}{{\textcolor{blue}]}}{1}%
% {<}{{\textcolor{blue}<}}{1}%
% {>}{{\textcolor{blue}>}}{1},%
framexleftmargin=1mm, framextopmargin=1mm, rulesepcolor=\color{blue},
breakatwhitespace=true,
tabsize=2
}}{}




\usepackage{tikz}
\usetikzlibrary{shapes,arrows}

% Define block styles
\tikzstyle{decision} = [diamond, draw, fill=blue!20,
    text width=3.5em, text badly centered, node distance=2.5cm, inner sep=0pt]
\tikzstyle{block} = [rectangle, draw, fill=blue!20,
    text width=8em, text centered, rounded corners, minimum height=4em]
\tikzstyle{line} = [draw, very thick, color=black!50, -latex']
\tikzstyle{cloud} = [draw, ellipse,fill=red!20, node distance=2.5cm,
    minimum height=2em]

\def\radius{.7mm} 
\tikzstyle{branch}=[fill,shape=circle,minimum size=3pt,inner sep=0pt]


\newcommand{\bfv}[1]{\boldsymbol{#1}} 
\newcommand{\Div}[1]{\nabla \bullet \vbf{#1}}           % define divergence
\newcommand{\Grad}[1]{\boldsymbol{\nabla}{#1}}
 \newcommand{\OP}[1]{{\bf\widehat{#1}}}
 \newcommand{\be}{\begin{equation}}
 \newcommand{\ee}{\end{equation}}
\newcommand{\beN}{\begin{equation*}}
\newcommand{\bea}{\begin{eqnarray}}
\newcommand{\beaN}{\begin{eqnarray*}}
\newcommand{\eeN}{\end{equation*}}
\newcommand{\eea}{\end{eqnarray}}
\newcommand{\eeaN}{\end{eqnarray*}}
\newcommand{\bdm}{\begin{displaymath}}
\newcommand{\edm}{\end{displaymath}}
\newcommand{\bsubeqs}{\begin{subequations}}
\newcommand{\esubeqs}{\end{subequations}}
\newcommand{\Obs}[1]{\langle{\Op{#1}\rangle}}             % define observable
\newcommand{\PsiT}{\bfv{\Psi_T}(\bfv{R})}                       % symbol for trial wave function
%\newcommand{\braket}[2]{\langle{#1}|\Op{#2}|{#1}\rangle}
\newcommand{\Det}[1]{{|\bfv{#1}|}}
\newcommand{\uvec}[1]{\mbox{\boldmath$\hat{#1}$\unboldmath}}
\newcommand{\Op}[1]{{\bf\widehat{#1}}}    
\newcommand{\eqbrace}[4]{\left\{
\begin{array}{ll}
#1 & #2 \\[0.5cm]
#3 & #4
\end{array}\right.}
\newcommand{\eqbraced}[4]{\left\{
\begin{array}{ll}
#1 & #2 \\[0.5cm]
#3 & #4
\end{array}\right\}}
\newcommand{\eqbracetriple}[6]{\left\{
\begin{array}{ll}
#1 & #2 \\
#3 & #4 \\
#5 & #6
\end{array}\right.}
\newcommand{\eqbracedtriple}[6]{\left\{
\begin{array}{ll}
#1 & #2 \\
#3 & #4 \\
#5 & #6
\end{array}\right\}}

\newcommand{\mybox}[3]{\mbox{\makebox[#1][#2]{$#3$}}}
\newcommand{\myframedbox}[3]{\mbox{\framebox[#1][#2]{$#3$}}}

%% Infinitesimal (and double infinitesimal), useful at end of integrals
%\newcommand{\ud}[1]{\mathrm d#1}
\newcommand{\ud}[1]{d#1}
\newcommand{\udd}[1]{d^2\!#1}

%% Operators, algebraic matrices, algebraic vectors

%% Operator (hat, bold or bold symbol, whichever you like best):
\newcommand{\op}[1]{\widehat{#1}}
%\newcommand{\op}[1]{\mathbf{#1}}
%\newcommand{\op}[1]{\boldsymbol{#1}}

%% Vector:
\renewcommand{\vec}[1]{\boldsymbol{#1}}

%% Matrix symbol:
\newcommand{\matr}[1]{\boldsymbol{#1}}
%\newcommand{\bb}[1]{\mathbb{#1}}

%% Determinant symbol:
\renewcommand{\det}[1]{|#1|}

%% Means (expectation values) of varius sizes
\newcommand{\mean}[1]{\langle #1 \rangle}
\newcommand{\meanb}[1]{\big\langle #1 \big\rangle}
\newcommand{\meanbb}[1]{\Big\langle #1 \Big\rangle}
\newcommand{\meanbbb}[1]{\bigg\langle #1 \bigg\rangle}
\newcommand{\meanbbbb}[1]{\Bigg\langle #1 \Bigg\rangle}


%% Big-O (typically for specifying the speed scaling of an algorithm)
\newcommand{\bigO}{\mathcal{O}}

%% Real value of a complex number
%\newcommand{\real}[1]{\mathrm{Re}\!\left\{#1\right\}}

%% Quantum mechanical state vectors and matrix elements (of different sizes)
\newcommand{\brab}[1]{\big\langle #1 \big|}
\newcommand{\brabb}[1]{\Big\langle #1 \Big|}
\newcommand{\brabbb}[1]{\bigg\langle #1 \bigg|}
\newcommand{\brabbbb}[1]{\Bigg\langle #1 \Bigg|}
\newcommand{\ketb}[1]{\big| #1 \big\rangle}
\newcommand{\ketbb}[1]{\Big| #1 \Big\rangle}
\newcommand{\ketbbb}[1]{\bigg| #1 \bigg\rangle}
\newcommand{\ketbbbb}[1]{\Bigg| #1 \Bigg\rangle}
\newcommand{\overlap}[2]{\langle #1 | #2 \rangle}
\newcommand{\overlapb}[2]{\big\langle #1 \big| #2 \big\rangle}
\newcommand{\overlapbb}[2]{\Big\langle #1 \Big| #2 \Big\rangle}
\newcommand{\overlapbbb}[2]{\bigg\langle #1 \bigg| #2 \bigg\rangle}
\newcommand{\overlapbbbb}[2]{\Bigg\langle #1 \Bigg| #2 \Bigg\rangle}
\newcommand{\bracket}[3]{\langle #1 | #2 | #3 \rangle}
\newcommand{\bracketb}[3]{\big\langle #1 \big| #2 \big| #3 \big\rangle}
\newcommand{\bracketbb}[3]{\Big\langle #1 \Big| #2 \Big| #3 \Big\rangle}
\newcommand{\bracketbbb}[3]{\bigg\langle #1 \bigg| #2 \bigg| #3 \bigg\rangle}
\newcommand{\bracketbbbb}[3]{\Bigg\langle #1 \Bigg| #2 \Bigg| #3 \Bigg\rangle}
\newcommand{\projection}[2]
{| #1 \rangle \langle  #2 |}
\newcommand{\projectionb}[2]
{\big| #1 \big\rangle \big\langle #2 \big|}
\newcommand{\projectionbb}[2]
{ \Big| #1 \Big\rangle \Big\langle #2 \Big|}
\newcommand{\projectionbbb}[2]
{ \bigg| #1 \bigg\rangle \bigg\langle #2 \bigg|}
\newcommand{\projectionbbbb}[2]
{ \Bigg| #1 \Bigg\rangle \Bigg\langle #2 \Bigg|}





\makeindex             % used for the subject index
                       % please use the style svind.ist with
                       % your makeindex program

%%%%%%%%%%%%%%%%%%%%%%%%%%%%%%%%%%%%%%%%%%%%%%%%%%%%%%%%%%%%%%%%%%%%%

\begin{document}

\author{Svenn-Arne Dragly, Henrik Eiding, Morten Hjorth-Jensen, Milad Hobbi Moborhan, and Olte Tobias Norli}
\title{Tools for Computational Quantum Mechanics}
 
\subtitle{University of Oslo,  2014}
\date{August 2014}
\maketitle

\frontmatter%%%%%%%%%%%%%%%%%%%%%%%%%%%%%%%%%%%%%%%%%%%%%%%%%%%%%%

%
%%%%%%%%%%%%%%%%%%%%%%% dedic.tex %%%%%%%%%%%%%%%%%%%%%%%%%%%%%%%%%
%
% sample dedication
%
% Use this file as a template for your own input.
%
%%%%%%%%%%%%%%%%%%%%%%%% Springer %%%%%%%%%%%%%%%%%%%%%%%%%%

\begin{dedication}
Use the template \emph{dedic.tex} together with the Springer document class SVMono for monograph-type books or SVMult for contributed volumes to style a quotation or a dedication\index{dedication} at the very beginning of your book in the Springer layout
\end{dedication}





%%%%%%%%%%%%%%%%%%%%%%%foreword.tex%%%%%%%%%%%%%%%%%%%%%%%%%%%%%%%%%
% sample foreword
%
% Use this file as a template for your own input.
%
%%%%%%%%%%%%%%%%%%%%%%%% Springer %%%%%%%%%%%%%%%%%%%%%%%%%%

\foreword

%% Please have the foreword written here
Use the template \textit{foreword.tex} together with the Springer document class SVMono (monograph-type books) or SVMult (edited books) to style your foreword\index{foreword} in the Springer layout. 

The foreword covers introductory remarks preceding the text of a book that are written by a \textit{person other than the author or editor} of the book. If applicable, the foreword precedes the preface which is written by the author or editor of the book.


\vspace{\baselineskip}
\begin{flushright}\noindent
Place, month year\hfill {\it Firstname  Surname}\\
\end{flushright}



\preface
%  last update : 24/8/2013  mhj

\begin{quotation}
So, ultimately, in order to understand nature it may be necessary to
have a deeper understanding of mathematical relationships. But the
real reason is that the subject is enjoyable, and although we humans
cut nature up in different ways, and we have different courses in
different departments, such compartmentalization is really artificial,
and we should take our intellectual pleasures where we find them. 
{\em Richard Feynman, The Laws of Thermodynamics.}
\end{quotation}

Why a preface you may ask? Isn't that just a mere exposition of a
raison d'$\mathrm{\hat{e}}$tre of an author's choice of material,
preferences, biases, teaching philosophy etc.?  To a large extent I
can answer in the affirmative to that. A preface ought to be personal.
Indeed, what you will see in the various chapters of these notes
represents how I perceive computational physics should be taught.

 This set of lecture notes serves the scope of presenting to you and
train you in an algorithmic approach to problems in the sciences,
represented here by the unity of three disciplines, physics,
mathematics and informatics. This trinity outlines the emerging field
of computational physics.

Our insight in a physical system, combined with numerical mathematics
gives us the rules for setting up an algorithm, viz.~a set of rules
for solving a particular problem.  Our understanding of the physical
system under study is obviously gauged by the natural laws at play,
the initial conditions, boundary conditions and other external
constraints which influence the given system. Having spelled out the
physics, for example in the form of a set of coupled partial
differential equations, we need efficient numerical methods in order
to set up the final algorithm.  This algorithm is in turn coded into a
computer program and executed on available computing facilities.  To
develop such an algorithmic approach, you will be exposed to several
physics cases, spanning from the classical pendulum to quantum
mechanical systems. We will also present some of the most popular
algorithms from numerical mathematics used to solve a plethora of
problems in the sciences.  Finally we will codify these algorithms
using some of the most widely used programming languages, presently C,
C++ and Fortran and its most recent standard Fortran
2008\footnote{Throughout this text we refer to Fortran 2008 as
Fortran, implying the latest standard.}. However, a high-level and fully
object-oriented language like Python is now emerging as a good
alternative although C++ and Fortran still outperform Python when it
comes to computational speed.  In this text we offer an approach where
one can write all programs in C/C++ or Fortran.  We will also show you
how to develop large programs in Python interfacing C++ and/or Fortran
functions for those parts of the program which are CPU intensive.
Such an approach allows you to structure the flow of data in a
high-level language like Python while tasks of a mere repetitive and
CPU intensive nature are left to low-level languages like C++ or
Fortran. Python allows you also to smoothly interface your program
with other software, such as plotting programs or operating system
instructions. A typical Python program you may end up writing contains
everything from compiling and running your codes to preparing the body
of a file for writing up your report.



Computer simulations are nowadays an integral part of contemporary
basic and applied research in the sciences.  Computation is becoming
as important as theory and experiment. In physics, computational
physics, theoretical physics and experimental physics are all equally
important in our daily research and studies of physical
systems. Physics is the unity of theory, experiment and
computation\footnote{We mentioned previously the trinity of physics,
mathematics and informatics. Viewing physics as the trinity of theory,
experiment and simulations is yet another example. It is obviously
tempting to go beyond the sciences. History shows that triunes,
trinities and for example triple deities permeate the Indo-European
cultures (and probably all human cultures), from the ancient Celts and
Hindus to modern days.  The ancient Celts revered many such trinues,
their world was divided into earth, sea and air, nature was divided in
animal, vegetable and mineral and the cardinal colours were red,
yellow and blue, just to mention a few.  As a curious digression, it
was a Gaulish Celt, Hilary, philosopher and bishop of Poitiers (AD
315-367) in his work {\em De Trinitate} who formulated the Holy
Trinity concept of Christianity, perhaps in order to accomodate
millenia of human divination practice.}.  Moreover, the ability "to
compute" forms part of the essential repertoire of research
scientists. Several new fields within computational science have
emerged and strengthened their positions in the last years, such as
computational materials science, bioinformatics, computational
mathematics and mechanics, computational chemistry and physics and so
forth, just to mention a few.  These fields underscore the importance
of simulations as a means to gain novel insights into physical
systems, especially for those cases where no analytical solutions can
be found or an experiment is too complicated or expensive to carry
out.  To be able to simulate large quantal systems with many degrees
of freedom such as strongly interacting electrons in a quantum dot
will be of great importance for future directions in novel fields like
nano-techonology.  This ability often combines knowledge from many
different subjects, in our case essentially from the physical
sciences, numerical mathematics, computing languages, topics from
high-performace computing and some knowledge of computers.


In 1999, when I started this course at the department of physics in
Oslo, computational physics and computational science in general were
still perceived by the majority of physicists and scientists as topics
dealing with just mere tools and number crunching, and not as subjects
of their own.  The computational background of most students enlisting
for the course on computational physics could span from dedicated
hackers and computer freaks to people who basically had never used a
PC. The majority of undergraduate and graduate students had a very
rudimentary knowledge of computational techniques and methods.
Questions like 'do you know of better methods for numerical
integration than the trapezoidal rule' were not uncommon. I do happen
to know of colleagues who applied for time at a supercomputing centre
because they needed to invert matrices of the size of $10^4\times
10^4$ since they were using the trapezoidal rule to compute
integrals. With Gaussian quadrature this dimensionality was easily
reduced to matrix problems of the size of $10^2\times 10^2$, with much
better precision.

More than a decade later most students have now been exposed to a
fairly uniform introduction to computers, basic programming skills and
use of numerical exercises.  Practically every undergraduate student
in physics has now made a Matlab or Maple simulation of for example
the pendulum, with or without chaotic motion.  Nowadays most of you
are familiar, through various undergraduate courses in physics and
mathematics, with interpreted languages such as Maple, Matlab and/or
Mathematica. In addition, the interest in scripting languages such as
Python or Perl has increased considerably in recent years.  The modern
programmer would typically combine several tools, computing
environments and programming languages. A typical example is the
following. Suppose you are working on a project which demands
extensive visualizations of the results. To obtain these results, that
is to solve a physics problems like obtaining the density profile of a
Bose-Einstein condensate, you need however a program which is fairly
fast when computational speed matters.  In this case you would most
likely write a high-performance computing program using Monte Carlo
methods in languages which are tailored for that. These are
represented by programming languages like Fortran and C++.  However,
to visualize the results you would find interpreted languages like
Matlab or scripting languages like Python extremely suitable for your
tasks.  You will therefore end up writing for example a script in
Matlab which calls a Fortran or C++ program where the number crunching
is done and then visualize the results of say a wave equation solver
via Matlab's large library of visualization tools. Alternatively, you
could organize everything into a Python or Perl script which does
everything for you, calls the Fortran and/or C++ programs and performs
the visualization in Matlab or Python. Used correctly, these tools,
spanning from scripting languages to high-performance computing
languages and vizualization programs, speed up your capability to
solve complicated problems.  Being multilingual is thus an advantage
which not only applies to our globalized modern society but to
computing environments as well.  This text shows you how to use C++
and Fortran as programming languages.

There is however more to the picture than meets the eye.  Although
interpreted languages like Matlab, Mathematica and Maple allow you
nowadays to solve very complicated problems, and high-level languages
like Python can be used to solve computational problems, computational
speed and the capability to write an efficient code are topics which
still do matter. To this end, the majority of scientists still use
languages like C++ and Fortran to solve scientific problems.  When you
embark on a master or PhD thesis, you will most likely meet these
high-performance computing languages.  This course emphasizes thus the
use of programming languages like Fortran, Python and C++ instead of
interpreted ones like Matlab or Maple. You should however note that
there are still large differences in computer time between for example
numerical Python and a corresponding C++ program for many numerical
applications in the physical sciences, with a code in C++ or Fortran
being the fastest.

Computational speed is not the only reason for this choice of
programming languages. Another important reason is that we feel that
at a certain stage one needs to have some insights into the algorithm
used, its stability conditions, possible pitfalls like loss of
precision, ranges of applicability, the possibility to improve the
algorithm and taylor it to special purposes etc etc.  One of our major
aims here is to present to you what we would dub 'the algorithmic
approach', a set of rules for doing mathematics or a precise
description of how to solve a problem. To device an algorithm and
thereafter write a code for solving physics problems is a marvelous
way of gaining insight into complicated physical systems. The
algorithm you end up writing reflects in essentially all cases your
own understanding of the physics and the mathematics (the way you
express yourself) of the problem.  We do therefore devote quite some
space to the algorithms behind various functions presented in the
text. Especially, insight into how errors propagate and how to avoid
them is a topic we would like you to pay special attention to. Only
then can you avoid problems like underflow, overflow and loss of
precision. Such a control is not always achievable with interpreted
languages and canned functions where the underlying algorithm and/or
code is not easily accesible.  Although we will at various stages
recommend the use of library routines for say linear
algebra\footnote{Such library functions are often taylored to a given
machine's architecture and should accordingly run faster than user
provided ones.}, our belief is that one should understand what the
given function does, at least to have a mere idea.  With such a
starting point, we strongly believe that it can be easier to develope
more complicated programs on your own using Fortran, C++ or Python.

We have several other aims as well, namely:
\begin{itemize}
\item We would like to give you  an opportunity to gain a 
      deeper understanding of the physics you have learned in other
      courses. In most courses one is normally confronted with simple
      systems which provide exact solutions and mimic to a certain
      extent the realistic cases. Many are however the comments like
      'why can't we do something else than the particle in a box
      potential?'.  In several of the projects we hope to present some
      more 'realistic' cases to solve by various numerical
      methods. This also means that we wish to give examples of how
      physics can be applied in a much broader context than it is
      discussed in the traditional physics undergraduate curriculum.
\item To encourage you to "discover" physics in a way similar to how 
researchers learn in the context of research.
\item Hopefully also to introduce numerical methods and new areas of physics that 
      can be studied with the methods discussed.
\item To teach   structured programming in the context of doing science. 
\item The projects we propose are meant to mimic to a certain extent 
      the situation encountered during a thesis or project work. You
      will tipically have at your disposal 2-3 weeks to solve
      numerically a given project. In so doing you may need to do a
      literature study as well. Finally, we would like you to write a
      report for every project.
\end{itemize}
Our overall goal is to encourage you to learn about science through
experience and by asking questions. Our objective is always
understanding and the purpose of computing is further insight, not
mere numbers!  Simulations can often be considered as
experiments. Rerunning a simulation need not be as costly as rerunning
an experiment.


 
Needless to say, these lecture notes are upgraded continuously, from
typos to new input.  And we do always benefit from your comments,
suggestions and ideas for making these notes better.  It's through the
scientific discourse and critics we advance.  Moreover, I have
benefitted immensely from many discussions with fellow colleagues and
students. In particular I must mention Hans Petter Langtangen, Anders
Malthe-S\o renssen, Knut M\o rken and \O yvind Ryan, whose input
during the last fifteen years has considerably improved these lecture
notes.  Furthermore, the time we have spent and keep spending together
on the Computing in Science Education project at the University, is
just marvelous. Thanks so much. Concerning the Computing in Science
Education initiative, you can read more
at \url{http://www.mn.uio.no/english/about/collaboration/cse/}.


Finally, I would like to add a petit note on referencing. These notes
have evolved over many years and the idea is that they should end up
in the format of a web-based learning environment for doing
computational science. It will be fully free and hopefully represent a
much more efficient way of conveying teaching material than
traditional textbooks.  I have not yet settled on a specific format,
so any input is welcome. At present however, it is very easy for me to
upgrade and improve the material on say a yearly basis, from simple
typos to adding new material.  When accessing the web page of the
course, you will have noticed that you can obtain all source files for
the programs discussed in the text.  Many people have thus written to
me about how they should properly reference this material and whether
they can freely use it. My answer is rather simple.  You are
encouraged to use these codes, modify them, include them in
publications, thesis work, your lectures etc.  As long as your use is
part of the dialectics of science you can use this material freely.
However, since many weekends have elapsed in writing several of these
programs, testing them, sweating over bugs, swearing in front of a
f*@?\%g code which didn't compile properly ten minutes before monday
morning's eight o'clock lecture etc etc, I would dearly appreciate in
case you find these codes of any use, to reference them properly. That
can be done in a simple way, refer to M.~Hjorth-Jensen, {\em
Computational Physics}, University of Oslo (2013). The weblink to the
course should also be included. Hope it is not too much to ask
for. Enjoy!

%%%%%%%%%%%%%%%%%%%%%%%acknow.tex%%%%%%%%%%%%%%%%%%%%%%%%%%%%%%%%%%%%%%%%%
% sample acknowledgement chapter
%
% Use this file as a template for your own input.
%
%%%%%%%%%%%%%%%%%%%%%%%% Springer %%%%%%%%%%%%%%%%%%%%%%%%%%

\extrachap{Acknowledgements}

Use the template \emph{acknow.tex} together with the Springer document class SVMono (monograph-type books) or SVMult (edited books) if you prefer to set your acknowledgement section as a separate chapter instead of including it as last part of your preface.



\tableofcontents

%%%%%%%%%%%%%%%%%%%%%%%acronym.tex%%%%%%%%%%%%%%%%%%%%%%%%%%%%%%%%%%%%%%%%%
% sample list of acronyms
%
% Use this file as a template for your own input.
%
%%%%%%%%%%%%%%%%%%%%%%%% Springer %%%%%%%%%%%%%%%%%%%%%%%%%%

\extrachap{Acronyms}

Use the template \emph{acronym.tex} together with the Springer document class SVMono (monograph-type books) or SVMult (edited books) to style your list(s) of abbreviations or symbols in the Springer layout.

Lists of abbreviations\index{acronyms, list of}, symbols\index{symbols, list of} and the like are easily formatted with the help of the Springer-enhanced \verb|description| environment.

\begin{description}[CABR]
\item[ABC]{Spelled-out abbreviation and definition}
\item[BABI]{Spelled-out abbreviation and definition}
\item[CABR]{Spelled-out abbreviation and definition}
\end{description}


\mainmatter%%%%%%%%%%%%%%%%%%%%%%%%%%%%%%%%%%%%%%%%%%%%%%%%%%%%%%%

\chapter{Introduction}

\section{Choice of programming language}

As programming language we have ended up with preferring 
C++, but all examples discussed in the text have their 
corresponding Fortran and Python programs on the webpage of this text.
 
Fortran (FORmula TRANslation) was introduced in 1957 and remains in many 
scientific computing environments the language of choice.
The latest standard, see Refs.~\cite{f95ref,metcalf1996,marshall1995,f2003}, 
includes extensions that are
familiar to users of C++. 
Some of the most important features of Fortran  include recursive
subroutines, dynamic storage allocation and pointers, 
user defined data structures, modules,
and the ability to manipulate entire arrays. 
However, there are several good reasons for 
choosing C++ as programming language for scientific and engineering
problems. Here are some:
\begin{itemize}
\item C++ is now the dominating language in Unix and Windows environments. It is widely available and is
the language of choice for system programmers.  It is very widespread for developments of non-numerical  software 
\item The C++ syntax has inspired lots of popular languages, such as Perl, Python and Java.
\item It is an extremely portable language, all Linux and Unix operated machines have a 
C++ compiler.
\item In the last years there has been an enormous effort towards developing numerical libraries
for C++. Numerous tools (numerical libraries such as MPI\cite{gropp1999,mpiref,cmpi}) are written in C++
and interfacing them requires knowledge of C++. 
Most C++ and Fortran compilers compare fairly well when it comes to speed and
numerical efficiency. Although Fortran 77 and C are regarded as slightly faster than C++ or Fortran,
compiler improvements during the last few years have diminshed such differences. The Java numerics project
has lost some of its steam recently, and Java is therefore normally slower than C++ or Fortran.
\item Complex variables, one of Fortran's strongholds, can also be defined in the new 
ANSI C++ standard. 
\item C++ is a language which catches most of the errors as early as possible, typically at compilation
time. Fortran has some of these features if one omits implicit variable declarations.
\item C++ is also an object-oriented language, to be contrasted with C and Fortran.
This means that it supports three fundamental ideas, namely objects, class hierarchies and polymorphism.
Fortran has, through the \verb? MODULE?  declaration the capability of defining classes, but lacks 
inheritance, although polymorphism is possible. Fortran is then considered as an object-based
programming language, to be contrasted with C++ which has the capability of relating classes
to each other in a hierarchical way.
\end{itemize}

An important aspect of C++ is its richness with more than 60 keywords allowing for a good balance between object orientation
and numerical efficiency. Furthermore, careful programming can results in an efficiency close to
Fortran 77.  The language is well-suited for large projects and has presently good standard libraries suitable
for computational science projects, although many of these still lag behind the large body of libraries for numerics
available to Fortran programmers. However, it is not difficult to interface libraries written in Fortran with C++
codes, if care is exercised.
Other weak sides are the fact that it can be easy to write inefficient code  and that there are many ways of writing the
same things, adding to the confusion for beginners  and professionals as well.  The language is also under continuous
development, which often causes portability problems.

C++ is also a difficult language to learn. Grasping the basics is rather straightforward, but takes time
to master. A specific problem which often causes 
unwanted or odd errors is dynamic memory management.

The efficiency of C++ codes are close to those provided by Fortran. This means often that a code
written in Fortran 77 can be faster, however  for large numerical projects C++ and Fortran 
are to be preferred. If speed is an issue, one could port critical parts of the code to Fortran 77.



 \clearemptydoublepage
\chapter{Practical Hartree-Fock approaches}\label{chap:hartreefock}

\abstract{This chapters aims at catching two birds with a stone;  to introduce to you essential features of the programming languages
C++ and Fortran with a brief reminder on Python specific topics, and to stress problems like
overflow, underflow, round off errors and eventually loss of precision due to the finite amount 
of numbers a computer can represent.  
The programs we discuss are tailored to these aims.}

\section{Getting Started}

The Hartree-Fock method, initially developed by Hartree \cite{hartree} and improved by Fock \cite{Fock},
is probably the most popular \emph{ab initio} method of quantum chemistry. There are mainly two reasons for this.
Firstly, it  provides an excellent first approximation to the wave function and energy of the system, often accounting
for about 90\%-99\% of the total energy. Secondly, in cases where an even higher degree of precision is needed, the result from a Hartree-Fock calculation
is a very good starting point for other so-called \emph{post-Hartree-Fock methods}. We will look into one such method, namely perturbation
theory, in chapter \ref{chapter:perturbation_theory}.

The general form of the equations is
\begin{equation}
 \mathcal{F}\psi_k = \varepsilon_k \psi_k,
\end{equation}
where $\mathcal{F}$ is the Fock operator, defined as
\begin{equation}
\begin{split}
 \mathcal{F}(\vec x)\psi_k(\vec x)  = & \Big[-\frac{1}{2}\nabla^2 - \sum_{n=1}^K\frac{Z_n}{|\vec R_n - \vec r|}\Big]\psi_k(\vec x) \\
                                      & +  \sum_{l=1}^N\int d\vec x'|\psi_l(\vec x')|^2\frac{1}{|\vec r - \vec r'|}\psi_k(\vec x) \\
                                      & - \sum_{l=1}^N \int d\vec x'\psi^*_l(\vec x')\frac{1}{|\vec r - \vec r'|}\psi_k(\vec x')\psi_l(\vec x).
\end{split}
\end{equation}
These are a set of coupled one-electron eigenvalue equations for the spin orbitals $\psi_k$.
The equations are non-linear because the orbitals we are seeking are
actually needed in order to obtain the operator $\mathcal{F}$ which determine them. They are therefore often referred to as self consistent field (SCF)
equations, and they must be solved iteratively.

The term in the square brackets is the one-body operator which we have called $h(\vec r)$ in equation (\ref{eq:H1}). The two extra sums are due to the interactions between
the electrons. The first of these represent the Coulomb potential from the mean field set up by the electrons of the system. The last is similar to the first
except that the indices of two orbitals have been switched. This is a direct consequence of the fact that in the derivation of the equations, the state
is assumed to be a Slater determinant. Note that due to the last sum, the Fock operator is non-local, that is to say, the value of $\mathcal{F}(\vec x)\psi_k(\vec x)$
depends on the value of $\psi_k(\vec x')$ at all coordinates $\vec x' \in \mathbb{R}^3 \oplus \{\uparrow, \downarrow\}$.

In the following section the general Hartree-Fock equations presented above are derived. Thereafter, we will see how to reformulate the equations to a more
implementation friendly form. We do this by removing the spin part so that the resulting equations are in terms of spatial orbitals only.
However, before doing this, it is necessary to decide how to relate the spin orbitals to the spatial orbitals. We discuss the two most common ways to
do this. These result in the so-called restricted and unrestricted Slater determinants, which are discussed in section \ref{sec:res_and_unres_dets}.
In section \ref{sec:RHF} we show how the restricted determinant leads to the restricted Hartree-Fock (RHF) equations, which is a set of integro-differential equations for
the spatial orbitals. In order to solve the equations, the spatial orbitals are expanded in a known basis, which leads to a set of self consistent algebraic equations
called the \emph{Roothaan equations}. Thereafter, in section \ref{sec:UHF}, we derive the unrestricted Hartree-Fock (UHF)
equations from the unrestricted determinant. These are also solved by introducing a set of known basis functions, leading to the so-called
\emph{Pople-Nesbet} equations, which is the unrestricted analogue of the Roothaan equations.

The theory of this chapter is covered by Szabo and Ostlund \cite{Szabo} and Thijssen \cite{Thijssen}.


% The Hartree-Fock method is a popular method for calculating the ground state and ground state energy of many-particle systems.
% It is a variational method which aims to find the minimum of the expectation value of the Hamiltonian. However, instead of minimizing in
% the space of all possible wave functions, the search is restricted to the set of all Slater determinants.
% 
% The Hartree-Fock equations are given by
% \begin{equation}
%  \mathcal{F}\psi_k = \varepsilon_k \psi_k,
% \end{equation}
% where $\mathcal{F}$ is the Fock operator, defined as
% \begin{equation}
% \begin{split}
%  \mathcal{F}(\vec x)\psi_k(\vec x)  = & \Big[-\frac{1}{2}\nabla^2 - \sum_{i=1}^N\frac{Z_n}{|\vec R_n - \vec r|}\Big]\psi_k(\vec x) \\
%                                       & +  \sum_{l=1}^N\int d\vec x'|\psi_l(\vec x')|^2\frac{1}{|\vec r - \vec r'|}\psi_k(\vec x) \\
%                                       & - \sum_{l=1}^N \int d\vec x'\psi^*_l(\vec x')\frac{1}{|\vec r - \vec r'|}\psi_k(\vec x')\psi_l(\vec x).
% \end{split}
% \end{equation}
% These are a set of coupled one-electron eigenvalue equations for the spin orbitals $\psi_k$.
% The equations are non-linear because the orbitals we are seeking are
% actually needed in order to obtain the operator $\mathcal{F}$ which determine them. They are therefore often referred to as self consistent field (SCF)
% equations and must be solved iteratively. The procedure goes along the following lines. First make an initial guess for the spin orbitals $\psi_k$ and calculate the Fock
% operator $\mathcal{F}$. Then, solve the Hartree-Fock equations to obtain a new set of spin orbitals and use these as input to calculate a new Fock operator.
% The process continues like this until a convergence criterium is reached.
% 
% The term in the square brackets is the one-body operator which we have called $h(\vec r)$ in equation (\ref{eq:H1}). The two extra sums are due to the interactions between
% the electrons. The first of these represent the Coulomb potential from the mean field set up by the electrons of the system. The last is similar to the first
% except that the indices of two orbitals have been switched. This is a direct consequence of the fact that in the derivation of the equations, the state
% is assumed to be a Slater determinant. Note that due to the last sum, the Fock operator is non-local, that is to say, the value of $\mathcal{F}(\vec x)\psi_k(\vec x)$
% depends on the value of $\psi_k(\vec x')$ at all coordinates $\vec x' \in \mathbb{R}^3 \oplus \{\uparrow, \downarrow\}$.
% 
% As discussed in section \ref{sec:fermion_basis}, any state $\ket{\Psi}$ can be written as a linear combination of Slater determinants
% \begin{equation}
%  \ket{\Psi} = C_0\ket{c} + \sum_{ia}C^a_i\ket{\Psi^a_i} + \sum_{ijab}C^{ab}_{ij}\ket{\Psi^{ab}_{ij}} + \dots
% \end{equation}
% The most crude approximation is to neglect all terms except the reference Slater $\ket{c}$. The Hartree-Fock method answers the following problem:
% Given the approximation $\ket{\Psi} \approx \ket{c}$, what is the optimal set of spin orbitals to choose as constituents of $\ket{c}$, and what is the resulting approximation
% for the energy, $E_{HF}$? The energy $E_{HF}$ is defined as the energy found from an exact solution of the Hartree-Fock equations (which presupposes a complete basis).
% 
% In most cases the Hartree-Fock method provides an excellent first approximation to the wave function and energy of the system, and it often accounts
% for about 99\% of the total energy. However, in quantum chemistry an even higher degree of precision is sometimes needed. In such cases, the solution of
% the Hartree-Fock equations is a very good starting point to use as input to other so-called \emph{post-Hartree-Fock methods}. The other
% methods discussed in this thesis can be considered to belong to this class of methods.
% 
% The difference between the exact energy $E$ and the Hartree-Fock energy $E_{HF}$ is referred to as the correlation energy $\Delta E_{corr}$:
% \begin{equation}
%  \Delta E_{corr} = E - E_{HF}.
% \end{equation}


\section{Derivation of the Hartree-Fock equations}
\label{sec:hartree_fock_derivation}
As discussed in section \ref{sec:fermion_basis}, the exact ground state can be written as a
linear combination of Slater determinants
\begin{equation}
 \ket{\Phi_0} = C_0\ket{\Psi_0} + \sum_{ia}C^a_i\ket{\Psi^a_i} + \sum_{i<j,a<b}C^{ab}_{ij}\ket{\Psi^{ab}_{ij}} + \dots
\end{equation}
where $\ket{\Psi_0}$ is some chosen reference determinant. In the Hartree-Fock method all determinants except $\ket{\Psi_0}$ are neglected, and the
spin orbitals from which $\ket{\Psi_0}$ is constructed are chosen in such a way that the expectation value $E_0 = \bra{\Psi_0}H\ket{\Psi_0}$ comes as
close to the excact energy $\mathscr E_0$ as possible. According to the variational principle, the expectation value $E_0$ is an upper bound
to the exact energy $\mathscr E_0$, which means that the optimal choice of spin orbitals are those which minimise $E_0$. When $E_0$ is at its minimum, any
infinitesimal variation of the spin orbitals will leave $E_0$ unchanged, which in mathematical terms means that
\begin{equation}
 \delta E_0 = \sum_{k=1}^N[\bra{\delta\psi_k}\mathcal F\ket{\psi_k} + \bra{\psi_k}\mathcal F\ket{\delta\psi_k}] = 0.
\end{equation}
If each variation could be chosen independently, this would immediately give us $N$ equations to be solved for the $N$ spin orbitals.
Unfortunately, the spin orbitals cannot be varied independently, but must remain orthonormal throughout the variation. The orthonormality condition reads
\begin{equation}
 \langle\psi_k|\psi_l\rangle - \delta_{kl} = 0.
\end{equation}
This type of \emph{constrained} optimisation problem can be solved elegantly by the method of Lagrangian multipliers. A good review of the
method is given in Boas \cite{boas2005mathematical}. The method simply says that if we construct the new functional
\begin{equation}
\label{eq:L_operator}
 \mathscr L = E_0 - \sum_{k,l=1}^N\Lambda_{lk}[\langle\psi_k|\psi_l\rangle - \delta_{kl}],
\end{equation}
we can find the stationary value of $E_0$ by solving the \emph{unconstrained} variational problem for $\mathscr L$. By unconstrained we mean that the variation
of each spin orbital can be chosen freely. The multipliers $\Lambda_{lk}$ are called Lagrange multipliers and will also be determined as part of the solution.

% In the Hartree-Fock method,
% all but the first term is neglected and the orbitals from which this determinant is
% constructed are chosen such that the reference energy $E_0 = \bra{\Psi_0}H\ket{\Psi_0}$ comes
% as close as possible to the exact energy. The variational principle says that the reference
% energy $E_0$ gives an upper bound to the exact energy. We are therefore seeking the spin orbitals
% which minimize $E_0$.

Recall from equation (\ref{eq:E_ref}) that the reference energy is given by
\begin{equation}
 E_0 = \bra{\Psi_0}H\ket{\Psi_0} = \sum_{k=1}^N\bra{\psi_k} h\ket{\psi_k} + \frac{1}{2}\sum_{k,l=1}^N[\bra{\psi_k\psi_l}g\ket{\psi_k\psi_l}
                                    - \bra{\psi_k\psi_l}g\ket{\psi_l\psi_k}].
\end{equation}
Taking the variation of this yields
\begin{equation}
\begin{split}
 \delta E_0  = & \sum_{k=1}^N\bra{\delta \psi_k}h\ket{\psi_k} \\
          &   + \frac{1}{2}\sum_{k,l = 1}^N[\bra{\delta \psi_k \psi_l}g\ket{\psi_k \psi_l} + \bra{\psi_k \delta \psi_l}g\ket{\psi_k \psi_l} \\
          &  \hspace{12mm} - \bra{\delta \psi_k \psi_l}g\ket{\psi_l \psi_k} - \bra{\psi_k \delta \psi_l}g\ket{\psi_l \psi_k}] + \text{c.c.}\\
        = & \sum_{k=1}^N\bra{\delta \psi_k}h\ket{\psi_k} \\
          & + \frac{1}{2}\sum_{k,l = 1}^N[\bra{\delta \psi_k \psi_l}g\ket{\psi_k \psi_l} + \bra{\delta\psi_l \psi_k}g\ket{\psi_l \psi_k} \\
          & \hspace{12mm} - \bra{\delta \psi_k \psi_l}g\ket{\psi_l \psi_k} - \bra{\delta\psi_l \psi_k}g\ket{\psi_k \psi_l}] + \text{c.c.}\\
        = & \sum_{k=1}^N\bra{\delta \psi_k}h\ket{\psi_k} + \sum_{k,l = 1}^N[\bra{\delta \psi_k \psi_l}g\ket{\psi_k\psi_l} - \bra{\delta \psi_k \psi_l}g\ket{\psi_l \psi_k}] + \text{c.c.}
\end{split}
\end{equation}
where c.c. represents complex conjugate terms.
In the expression after the second equality sign, we have used that $\bra{\psi_k\delta\psi_l}g\ket{\psi_k\psi_l} = \bra{\delta\psi_l\psi_k}g\ket{\psi_l\psi_k}$, and the last
line follows from the fact that the indices $k$ and $l$ can be switched since they are dummy indices.
We next define the two operators
\begin{eqnarray}
\label{eq:J_operator}
 \mathcal{J}(\vec x)\psi_k(\vec x) & = & \sum_{l=1}^N\Big[\int d\vec x' \psi^*_l(\vec x') g(\vec r, \vec r')\psi_l(\vec x')\Big]\psi_k(\vec x), \\ \label{eq:K_operator}
 \mathcal{K}(\vec x)\psi_k(\vec x) & = & \sum_{l=1}^N\Big[\int d\vec x' \psi^*_l(\vec x') g(\vec r, \vec r')\psi_k(\vec x')\Big]\psi_l(\vec x),
\end{eqnarray}
so that the variation of the energy can be written more compactly as
\begin{equation}
 \delta E_0 = \sum_{k=1}^N\bra{\delta \psi_k} h + \mathcal J - \mathcal K \ket{\psi_k} + \text{c.c.}
\end{equation}
Next we consider the variation of the constraint:
\begin{equation}
 \sum_{k,l=1}^N[\Lambda_{lk}\langle\delta\psi_k|\psi_l\rangle + \Lambda_{lk}\langle\psi_k|\delta\psi_l\rangle].
\end{equation}
We will show that the second term of this equation is in fact the complex conjugate of the first. To realise this, consider the functional $\mathscr L$
of equation (\ref{eq:L_operator}). First note that it must be real because $E_0$ is real and the added constraints are equal to zero.
Taking the complex conjugate of $\mathscr L$ therefore gives
\begin{equation}
\begin{split}
 \mathscr L = & E_0 - \sum_{k,l=1}^N\Lambda^*_{lk}[\langle\psi_k|\psi_l\rangle^* - \delta_{kl}] \\
            = & E_0 - \sum_{k,l=1}^N\Lambda^*_{lk}[\langle\psi_l|\psi_k\rangle - \delta_{lk}] \\
            = & E_0 - \sum_{k,l=1}^N\Lambda^*_{kl}[\langle\psi_k|\psi_l\rangle - \delta_{kl}].
\end{split}
\end{equation}
This form of $\mathscr L$ is identical to the original of (\ref{eq:L_operator}) except that $\Lambda_{lk}$ has been replaced by $\Lambda^*_{kl}$.
Since both will yield exactly the same Lagrange multipliers (assuming that the solution is unique), this means that
\begin{equation}
 \Lambda_{lk} = \Lambda^*_{kl},
\end{equation}
that is to say, $\Lambda_{lk}$ are the elements of a Hermitian matrix. Thus we can write the variation of the constraint as
\begin{equation}
\begin{split}
 \sum_{k,l=1}^N[\Lambda_{lk}\langle\delta\psi_k|\psi_l\rangle + \Lambda_{lk}\langle\psi_k|\delta\psi_l\rangle] = &
 \sum_{k,l=1}^N\Lambda_{lk}\langle\delta\psi_k|\psi_l\rangle + \sum_{k,l=1}^N\Lambda^*_{kl}\langle\delta\psi_l|\psi_k\rangle^* \\
= & \sum_{k,l=1}^N\Lambda_{lk}\langle\delta\psi_k|\psi_l\rangle + \sum_{k,l=1}^N\Lambda^*_{lk}\langle\delta\psi_k|\psi_l\rangle^* \\
= & \sum_{k,l=1}^N\Lambda_{lk}\langle\delta\psi_k|\psi_l\rangle + \text{c.c.}
\end{split}
\end{equation}
Putting it all together, the variation of $\mathscr L$ is now
\begin{equation}
\begin{split}
 \delta \mathscr L &  = \sum_{k=1}^N\bra{\delta \psi_k}\Big[ (h + \mathcal J - \mathcal K)\ket{\psi_k}
                     - \sum_{l=1}^N\Lambda_{lk}\ket{\psi_l}\Big] + \text{c.c.} \\
                   & = 0.
\end{split}
\end{equation}
By defining the Fock operator
\begin{equation}
\label{eq:Fock_operator}
\mathcal{F} = h + \mathcal J - \mathcal K
\end{equation}
the above equation can be written even more compactly as
\begin{equation}
\begin{split}
  \delta \mathscr L & =\sum_{k=1}^N\bra{\delta \psi_k}\Big[\mathcal F\ket{\psi_k}
                       - \sum_{l=1}^N\Lambda_{lk}\ket{\psi_l}\Big] + \text{c.c.} \\
                    & = 0.
\end{split}
\end{equation}
Since the variations of the spin orbitals can be chosen freely, each term
in the square brackets must be equal to zero, which implies that
\begin{equation}
\label{eq:non_canonical_hartree_fock}
 \mathcal F\psi_k = \sum_{l=1}^N\Lambda_{lk}\psi_l.
\end{equation}
This equation\footnote{We are actually talking about a set of $N$ equations for all the spin orbitals
$\{\psi_k\}_{k=1}^N$, but since we observe that they are all equal, we may refer to \emph{the equation}.}
is not on the same form as the one introduced at the beginning of the chapter. This reason is as follows.
Given a solution $\{\psi_k\}$ of the equation above, it is possible to obtain a new set of spin orbitals
$\{\psi'_k\}$ via a unitary transformation
\begin{equation}
 \psi'_k = \sum_l \psi_l\ U_{lk}
\end{equation}
which keeps the expectation value $E_0 = \bra{\Psi'_0}H\ket{\Psi'_0}$ as well as the form of the Fock operator
unchanged, see Szabo and Ostlund \cite{Szabo}. Thus there is some flexibility in the choice of spin orbitals.
One particular choice of spin orbitals are the eigenfunctions of the Fock operator
\begin{equation}
\label{eq:canonical_hartree_fock}
 \mathcal F\psi_k = \varepsilon_k\psi_k,
\end{equation}
which are guaranteed to exist since $\mathcal F$ is Hermitian.
These particular spin orbitals are solutions of equation (\ref{eq:non_canonical_hartree_fock}) for the specific case where
$\Lambda_{lk} = \varepsilon_k\delta_{lk}$. Equation (\ref{eq:canonical_hartree_fock}) is called the canonical Hartree-Fock
equation.

If we are studying a molecular system, that is, if the system has more than a single nucleus,
the eigenfunctions of the Hartree-Fock equations are called \emph{molecular orbitals} (MOs). The word
molecular is used to emphasise that the orbitals are characterisitic of the molecular system.
It is important to distinguish between these and the familiar atomic orbitals because they are usually
very different. This means that for molecular systems it is no longer helpful to think of the
electrons as occupying atomic orbitals; the atomic orbitals are solutions of the Hartree-Fock equations
for \emph{isolated atoms}, but the molecule is an entirely different system with often entirely 
different solutions.

When the Slater determinant $\ket{\Psi_0}$ is composed of the $N$ lowest
eigenfunctions of the Hartree-Fock equations we will call it the Hartree-Fock determinant,
and we will refer to $E_0$ as the Hartree-Fock energy. This will be the case for the
remainder of the thesis unless stated otherwise.

% Note that the Hartree-Fock equations are the same for all orbitals, which is reassuring. Recall that the Slater determinant was motivated from the fact that we wanted
% a many-particle wave function which did not distinguish between the particles. Had we gone through the exact same variational procedure as above, but instead started with the energy
% corresponding to the Hartree product, the equations would be different for different orbitals.


The Hartree-Fock energy can be written in terms of the operators $\mathcal J$ and $\mathcal K$ as
\begin{equation}
 E_0 = \sum_{k=1}^N\bra{\psi_k}h + \frac{1}{2}(\mathcal J - \mathcal K) \ket{\psi_k},
\end{equation}
which shows that the eigenvalues of the Hartree-Fock equations (\ref{eq:canonical_hartree_fock}) do not add up to the ground state energy; the term $(\mathcal J-\mathcal K)$ in the Fock operator
is a factor of two too large. However, the energy can be calculated via the eigenvalues in the following two equivalent ways:
\begin{equation}
\label{eq:E_ref2}
\begin{split}
 E_0 & = \frac{1}{2}\sum_{k=1}^N[\varepsilon_k + \bra{\psi_k}h\ket{\psi_k}] \\
         & = \sum_{k=1}^N[\varepsilon_k - \frac{1}{2}\bra{\psi_k}\mathcal J - \mathcal K\ket{\psi_k}].
 \end{split}
\end{equation}




\section{Restricted and unrestricted determinants}
\label{sec:res_and_unres_dets}
In (\ref{eq:canonical_hartree_fock}) the Hartree-Fock equations are written on their most general form. The unknowns are the eigenvalues $\varepsilon_k$ and the spin orbitals $\psi_k$.
However, before solving the equations, it is useful to rewrite them in terms of spatial orbitals $\phi_k$ instead of spin orbtitals $\psi_k$. This is done by
integrating out the spin part, as will be shown in the next section. But first we must decide how to construct the spin orbitals from the spatial orbitals. There are two
ways to do this: One can either form so-called restricted spin orbitals or unrestricted spin orbitals. The two approaches will lead to two different Hartree-Fock methods,
namely the restricted Hartree-Fock method (RHF) and the unrestricted Hartree-Fock method (UHF), respectively.

\subsection{Restricted determinants}
Recall from equation (\ref{eq:spin_orbital}) that a spin orbital $\psi_k$ is a spatial orbital $\phi_k$ multiplied with a spin function which is either spin up, $\alpha$,
or spin down, $\beta$. This means that we can create spin orbitals in the following way
\begin{equation}
\label{eq:restricted_spinorbitals}
\psi_k(\vec x) = \left\{\begin{array}{c} \phi_l(\vec r)\alpha(s) \\
                                          \text{or} \\
                                          \phi_l(\vec r)\beta(s). \\
                       \end{array}\right.
\end{equation} 
Spin orbitals on this form are called restricted spin orbitals, and the Slater determinants they form are called restricted determinants. In such determinants, a spatial orbital is either
occupied by a single electron or two electrons, see figure \ref{fig:restricted_determinant}. A determinant which has
every spatial orbital doubly occupied, is called a closed shell determinant (left figure), whereas a
determinant that has one or more partially filled spatial orbitals, is called an open shell determinant (right figure). 
If the system has an odd number of electrons, the determinant will always be open shell.
However, an even number of electrons does not imply that the determinant is closed shell; if degeneracies apart from that due to spin are present, it can still be open shell.
% Recall from equation (\ref{eq:spin_orbital}) that a spin orbital $\psi_k$ is a spatial orbital $\phi_k$ multiplied with a spin function which is either spin up, $\alpha$,
% or spin down, $\beta$. This means that a set of $N$ spin orbitals can be generated from a set of $\lceil N/2\rceil$ spatial orbitals in the following way:
% \begin{equation}
% \label{eq:restricted_spinorbitals}
%  \{\psi_{2k-1}(\vec x),\,\psi_{2k}(\vec x)\} = \{\phi_k(\vec r)\alpha(s),\,\phi_k(\vec r)\beta(s)\}, \qquad k = 1,\dots,\lceil N/2\rceil.
% \end{equation} 
% Spin orbitals on this form are called restricted spin orbitals, and the Slater determinants they form are called restricted determinants. In such determinants, a spatial orbital is either
% occupied by a single electron or two electrons, one with spin up and one with spin down, see figure \ref{fig:restricted_determinant}. A determinant which has
% every spatial orbital doubly occupied, is called a closed shell determinant (left figure), whereas a
% determinant that has one or more partially filled spatial orbitals, is called an open shell determinant (right figure). 
% If the system has an odd number of electrons, the determinant will always be open shell.
% However, an even number of electrons does not imply that the determinant is closed shell; if degeneracies apart from that due to spin are present, it can still be open shell.
\begin{figure}
 \begin{center}
  \includegraphics[scale=0.7]{hartree_fock/figures/restricted_determinant.pdf}
  \caption{Illustration of the restricted determinant comprised of spin orbitals on the form (\ref{eq:restricted_spinorbitals}).
  The left and right figures illustrate closed and open shell determinants, respectively.}
  \label{fig:restricted_determinant}
 \end{center}
\end{figure}

Throughout this thesis we will limit the use of restricted determinants (and restricted Hartree-Fock) to closed shell systems. This means that the
spin orbitals are given by
\begin{equation}
\label{eq:restricted_spinorbitals2}
 \{\psi_{2k-1}(\vec x),\,\psi_{2k}(\vec x)\} = \{\phi_k(\vec r)\alpha(s),\,\phi_k(\vec r)\beta(s)\}, \qquad k = 1,\dots,N/2,
\end{equation}
where $N$ is the number of electrons.


\subsection{Unrestricted determinants}
In equation (\ref{eq:restricted_spinorbitals}) the spin-up electrons are described by the same set of spatial orbitals as the spin-down electrons.
For closed shell systems this is often a good assumption.
However, consider the open shell determinant illustrated to the right of figure \ref{fig:restricted_determinant}. The electron occupying the spin orbital 
$\phi_3\alpha$ will have an exchange interaction with the other spin-up electrons, but not with the spin-down electrons.
Hence, it could be energetically favourable to let the spin-up levels shift with respect to the spin-down levels, as shown in figure \ref{fig:unrestricted_determinant}.
This can be accomplished by letting the spin-up and spin-down electrons be described by different sets of spatial orbitals.
Spin orbitals formed in this way, are called unrestricted spin orbitals, and the Slater determinants they form are called unrestricted determinants.
\begin{equation}
\label{eq:unrestricted_spinorbitals}
\psi_k(\vec x) = \left\{\begin{array}{c} \phi^\alpha_l(\vec r)\alpha(s) \\
                                          \text{or} \\
                                          \phi^\beta_l(\vec r)\beta(s). \\
                       \end{array}\right.
\end{equation}
% In addition to  the case of open shell systems, unrestricted determinants will give lower energies when studying the dissociation of molecules. To see why, consider the
% $H_2$-molecule as an example. The restricted orbitals will force both electrons to be located at either of the nuclei with equal probability, and although
% this is reasonable for small internuclear distances, as the distance increases, the electrons will in fact be located at differenct nuclei. This can only be
% realised by the unrestricted spin orbitals which allow the electrons to have different spatial orbitals.
\begin{figure}
 \begin{center}
  \includegraphics[scale=0.7]{hartree_fock/figures/unrestricted_determinant.pdf}
  \caption{Illustration of the unrestricted determinant comprised of spin orbitals on the form (\ref{eq:unrestricted_spinorbitals}).
  The left and right figures illustrate closed and open shell determinants, respectively.}
  \label{fig:unrestricted_determinant}
 \end{center}
\end{figure}


% \subsection{Determinants and spin operators}
% The Hamiltonians which are considered in this thesis have no spin dependence. This means that it commutes with the total spin operator $\mathcal S^2$ and the spin operator
% in the z-direction $\mathcal S_z$. If we assume that the ground state of the Hamiltonian $\ket{\Psi_0}$ is nondegenerate, this means that $\ket{\Psi_0}$ is
% an eigenstate of $\mathcal S^2$ and $\mathcal S_z$ as well. To see this consider the following:
% \begin{equation}
% \begin{split}
%  H(\mathcal S^2\ket{\Psi_0}) = & ([\mathcal S^2,H] + \mathcal S^2H)\ket{\Psi_0} \\
%                              = & (0 + \mathcal S^2H)\ket{\Psi_0} \\
%                              = & E_0 \mathcal S^2\ket{\Psi_0}.
% \end{split}
% \end{equation}
% This shows that $\mathcal S^2\ket{\Psi_0}$ is also an eigenstate of $H$ with eigenvalue $E_0$, and since the ground state of $H$ is nondegenerate, this means that
% \begin{equation}
%  \mathcal S^2\ket{\Psi_0} = A \ket{\Psi_0}
% \end{equation}
% for some constant $A$.

\section{Slater determinants and the spin operators}
\label{sec:determinants_and_spin}
In this section we define the total spin operator for a system of $N$ particles and discuss how it acts on the restricted
and unrestricted determinants.
The reader is referred to the texts by Griffiths \cite{griffiths} and Shankar \cite{shankar} for introductory treatments of spin.

\subsection{Single-particle spin operators}
The spin operator for a single particle is given by
\begin{equation}
 \vec s = s_x\vec i + s_y\vec j + s_z\vec k,
\end{equation}
where $s_x$, $s_y$ and $s_z$ are the operators for the spin components along the coordinate axes. The latter satisfy the commutation relations
\begin{equation}
\label{eq:spin_commutation}
 [s_x,\, s_y] = i s_z, \qquad [s_y,\, s_z] = i s_x, \qquad [s_z,\, s_x] = i s_y.
\end{equation}
The squared magnitude of the spin operator is a scalar operator
\begin{equation}
 \vec s^2 = s_x^2 + s_y^2 + s_z^2.
\end{equation}
It commutes with all of the component operators, and it is therefore possible to find a common set of eigenstates for
$\vec s^2$ and \emph{one} of the operators $s_x$, $s_y$ or $s_z$. The standard choice in the literature is $s_z$.
A particle with spin $s$ has eigenvalues $s(s+1)$ and $m_s$
\begin{equation}
\begin{split}
 s^2 \ket{s,m_s} & = s(s+1) \ket{s,m_s}, \\
 s_z \ket{s,m_s} & = m_s \ket{s,m_s},
\end{split}
\end{equation}
where $m_s$ can take the values $\{-s,\, -s+1,\dots, s-1,\,s\}$. Electrons have spin 1/2, and the Hilbert space describing the
spin of electrons is therefore spanned by the two states $\ket{1/2,1/2}$ and $\ket{1/2,-1/2}$, which we until now have simply called $\ket{\alpha}$
and $\ket{\beta}$ for convenience.



\subsection{Many-particle spin operators}
Analogously, the total spin operator for a system of $N$ particles is given by
\begin{equation}
 \vec S = \sum_{i=1}^N \vec s(i),
\end{equation}
where $\vec s(i)$ is the spin operator of particle $i$. Also, the squared magnitude of the total spin operator is given by
\begin{equation}
 \vec S^2 = S_x^2 + S_y^2 + S_z^2,
\end{equation}
where
\begin{equation}
 S_I = \sum_{i=1}^N s_I(i), \qquad I\in\{x,\,y,\,z\}.
\end{equation}
Since the Hamiltonian does not depend on any of the spin coordinates, it commutes with $\vec S^2$ as well as $S_z$
\begin{equation}
 [H,\,\vec S^2] = [H, S_z] = 0.
\end{equation}
This means that the exact eigenstates of the Hamiltonian $\ket{\Phi_i}$ are also eigenstates of $\vec S^2$ and $S_z$ \cite{Szabo}
\begin{align}
 \vec S^2\ket{\Phi_i} & = S(S+1)\ket{\Phi_i},\\
 S_z\ket{\Phi_i} & = M_S\ket{\Phi_i},
\end{align}
where $S$ and $M_S$ are the quantum numbers for the total spin and its projection along the $z$-axis, respectively. A natural question
to ask at this point is whether or not the restricted and unrestricted determinants are eigenstates of $\vec S^2$ and $S_z$. The answer to this question is as follows:
\begin{enumerate}
 \item Both restricted and unrestricted determinants are eigenstates of $S_z$.
 \item The restricted closed shell determinant is an eigenstate of $\vec S^2$ with eigenvalue 0, making it a pure singlet state.
 \item The unrestricted determinant is generally not an eigenstate of $\vec S^2$.
\end{enumerate}
A consequence of point three is that unrestricted determinants can often have a total spin which is larger than the exact value. This is often referred
to as spin contamination of the unrestricted determinant. The reader is referred to appendix \ref{chapter:appendix_spin} for a further discussion of this topic.




\section{Restricted Hartree-Fock (RHF)}
\label{sec:RHF}
The most general form of the Hartree-Fock equations is given in (\ref{eq:canonical_hartree_fock}), where the Fock operator $\mathcal{F}$ is given by (\ref{eq:Fock_operator}) and
(\ref{eq:J_operator}) - (\ref{eq:K_operator}). We will now derive the equations which result from the restricted assumption in (\ref{eq:restricted_spinorbitals}).
Without loss of generality, we may assume that the spin orbital on which the Fock operator acts is spin up:
\begin{equation}
\mathcal F(\vec x)\phi_k(\vec r) \alpha(s) = \varepsilon_k\phi_k(\vec r)\alpha(s).
\end{equation}
Writing out the Fock operator explicitly:
\begin{equation}
\begin{split}
 \mathcal{F}(\vec x)\phi_k(\vec r)\alpha(s)  = &\ h(\vec r)\,\phi_k(\vec r)\alpha(s) \\
                      &  + \Big[\sum_{l=1}^N\int d\vec x'\psi^*_l(\vec x')g(\vec r, \vec r')\psi_l(\vec x')\Big]\phi_k(\vec r)\alpha(s) \\
                       &    - \Big[\sum_{l=1}^N\int d\vec x'\psi^*_l(\vec x')g(\vec r, \vec r')\phi_k(\vec r')\alpha(s)\Big]\psi_l(\vec x) \\
                       = &\  \varepsilon_k\phi_k(\vec r)\alpha(s).
%                    & = & h(\vec r)\,\phi_k(\vec r)\xi_k(s) + 2\Big[\sum_{l=1}^{N/2}\int d\vec r'\phi^*_l(\vec r')g(\vec r, \vec r')\phi_l(\vec r')\Big]\phi_k(\vec r)\xi_k(s) \\
%                    &   & - \Big[\sum_{l=1}^{N/2}\int d\vec r'\phi^*_l(\vec r')g(\vec r, \vec r')\phi_k(\vec r')\Big]\phi_l(\vec r)\xi_k(s).
\end{split}
\end{equation}
Our goal is to integrate out the spin of this equation. To do this, we must also express the spin orbitals in the sums in terms of spatial orbitals. We will assume that
$N$ is an even number of electrons and that we are dealing with a closed shell determinant. This means that both sums, which run from 0 to $N$, can be split
into two sums which run from 0 to $N/2$:
\begin{equation}
\begin{split}
 \mathcal{F}(\vec x)\phi_k(\vec r)\alpha(s)  = &\ h(\vec r)\,\phi_k(\vec r)\alpha(s) \\
                      &  + \Big[\sum_{l=1}^{N/2}\int d\vec x'\phi^*_l(\vec r')\alpha^*(s')g(\vec r, \vec r')\phi_l(\vec r')\alpha(s')\Big]\phi_k(\vec r)\alpha(s) \\
                      &  + \Big[\sum_{l=1}^{N/2}\int d\vec x'\phi^*_l(\vec r')\beta^*(s')g(\vec r, \vec r')\phi_l(\vec r')\beta(s')\Big]\phi_k(\vec r)\alpha(s) \\
                       & - \Big[\sum_{l=1}^{N/2}\int d\vec x'\phi^*_l(\vec r')\alpha^*(s')g(\vec r, \vec r')\phi_k(\vec r')\alpha(s')\Big]\phi_l(\vec r)\alpha(s) \\
                       & - \Big[\sum_{l=1}^{N/2}\int d\vec x'\phi^*_l(\vec r')\beta^*(s')g(\vec r, \vec r')\phi_k(\vec r')\alpha(s')\Big]\phi_l(\vec r)\beta(s) \\
                     = &\ \varepsilon_k\phi_k(\vec r)\alpha(s).
\end{split}
\end{equation}
We note that the first two sums are in fact equal when the spin coordinate $s'$ is integrated out. Furthermore, the last sum is equal to zero because the integration
is over unequal spins. Thus we are left with
\begin{equation}
\begin{split}
 \mathcal{F}(\vec x)\phi_k(\vec r)\alpha(s)  = &\ h(\vec r)\,\phi_k(\vec r)\alpha(s) \\
                      &  + 2\Big[\sum_{l=1}^{N/2}\int d\vec r'\phi^*_l(\vec r')g(\vec r, \vec r')\phi_l(\vec r')\Big]\phi_k(\vec r)\alpha(s) \\
                       & - \Big[\sum_{l=1}^{N/2}\int d\vec r'\phi^*_l(\vec r')g(\vec r, \vec r')\phi_k(\vec r')\Big]\phi_l(\vec r)\alpha(s) \\
                       = &\ \varepsilon_k\phi_k(\vec r)\alpha(s).
\end{split}
\end{equation}
If we now multiply both sides of the equation by $\alpha^*(s)$ and integrate over the spin coordinate $s$ we finally arrive at
\begin{equation}
\begin{split}
 \big[\sum_{s=\uparrow\downarrow}\alpha^*(s)\mathcal F(\vec x)\alpha(s)\Big]\phi_k(\vec r) = &\ h(\vec r)\,\phi_k(\vec r) \\
                       & + 2\Big[\sum_{l=1}^{N/2}\int d\vec r'\phi^*_l(\vec r')g(\vec r, \vec r')\phi_l(\vec r')\Big]\phi_k(\vec r) \\
                       & - \Big[\sum_{l=1}^{N/2}\int d\vec r'\phi^*_l(\vec r')g(\vec r, \vec r')\phi_k(\vec r')\Big]\phi_l(\vec r) \\
                     = &\ \varepsilon_k\phi_k(\vec r).
\end{split}
\end{equation}
Defining the restricted spatial Fock operator as
\begin{equation}
 F(\vec r) = \sum_{s=\uparrow\downarrow}\alpha^*(s)\mathcal F(\vec x)\alpha(s),
\end{equation}
the equation can be written as
\begin{equation}
\label{eq:RHF}
 F(\vec r)\phi_k(\vec r) = \varepsilon_k\phi_k(\vec r),
\end{equation}
where
\begin{equation}
\begin{split}
 F(\vec r)\phi_k(\vec r) = & h(\vec r)\,\phi_k(\vec r)  + 2\Big[\sum_{l=1}^{N/2}\int d\vec r'\phi^*_l(\vec r')g(\vec r, \vec r')\phi_l(\vec r')\Big]\phi_k(\vec r) \\
                       & - \Big[\sum_{l=1}^{N/2}\int d\vec r'\phi^*_l(\vec r')g(\vec r, \vec r')\phi_k(\vec r')\Big]\phi_l(\vec r).
\end{split}
\end{equation}
The spatial Fock operator can be written more compactly as
\begin{equation}
\label{eq:restricted_Fock_operator}
 F(\vec r) = h(\vec r) + 2 J(\vec r) - K(\vec r),
\end{equation}
where
\begin{align}
 J(\vec r)\phi_k(\vec r) & = \sum_{l=1}^{N/2}\int d\vec r'\phi^*_l(\vec r')g(\vec r, \vec r')\phi_l(\vec r')\Big]\phi_k(\vec r), \\
 K(\vec r)\phi_k(\vec r) & = \sum_{l=1}^{N/2}\int d\vec r'\phi^*_l(\vec r')g(\vec r, \vec r')\phi_k(\vec r')\Big]\phi_l(\vec r).
\end{align}
Note the factor 2 in front of the direct term, or more to the point, the absence of the factor 2 in front of the exchange term. This is due to the fact that
the exchange interaction is only present between electrons of equal spins.

To find the energy, consider the first line of equation (\ref{eq:E_ref2}):
\begin{equation}
\begin{split}
 E_0 = & \frac{1}{2}\sum_{k=1}^N[\varepsilon_k + \bra{\psi_k}h\ket{\psi_k}] \\
         = & \frac{1}{2}\sum_{k=1}^{N}\bra{\psi_k}(\mathcal{F} + h)\ket{\psi_k}.
\end{split}
\end{equation}
If, as we did above, insert the assumption (\ref{eq:restricted_spinorbitals}) and split the sum in two sums, one with spin up and one with spin down, we get
\begin{equation}
\label{eq:E_ref3}
 E_0 = \sum_{k=1}^{N/2}\bra{\phi_k}(F + h)\ket{\phi_k}.
\end{equation}


\subsection{Introducing a basis}
We have now eliminated the spin from the general Hartree-Fock equations (\ref{eq:canonical_hartree_fock}) and arrived at equation (\ref{eq:RHF}), which represents
a set of integro-differential equations for the spatial orbitals. There is presently no feasible way to solve these as they stand.
However, by expressing the orbitals in terms of some known basis
\begin{equation}
\label{eq:linear_expansion}
 \phi_k(\vec r) = \sum_{\mu=1}^M C_{\mu k}\chi_\mu(\vec r),
\end{equation}
where $M$ is the number of basis functions, the equations can be converted to a set of algebraic equations. Inserting this expansion into equation (\ref{eq:RHF}) gives
\begin{equation}
 \sum_{\nu=1}^M F(\vec r)\chi_\nu(\vec r) C_{\nu k} = \varepsilon_k \sum_{\nu=1}^M C_{\nu k} \chi_{\nu}(\vec r).
\end{equation}
If we now multiply by $\chi^*_\mu(\vec r)$ and integrate with respect to $\vec r$, we get the so-called \emph{Roothaan equations} \cite{roothan}
\begin{equation}
 \sum_{\nu=1}^M F_{\mu\nu}C_{\nu k} = \varepsilon_k\sum_{\nu=1}^M S_{\mu\nu}C_{\nu k},
\end{equation}
or in matrix notation
\begin{equation}
\label{eq:Roothaan}
 \mathbf{FC}_k = \varepsilon_k\mathbf{SC}_k,
\end{equation}
where
\begin{equation}
 F_{\mu\nu} = \int d\vec r \chi^*_\mu(\vec r)F(\vec r)\chi_\nu(\vec r)
\end{equation}
is the Fock matrix and
\begin{equation}
 S_{\mu\nu} = \int d\vec r \chi^*_\mu(\vec r)\chi_\nu(\vec r)
\end{equation}
is the overlap matrix. If the basis is orthonormal, the overlap matrix is the identity matrix. However, in molecular calculations Gaussian functions are
most often used, and these are not orthogonal. In section \ref{sec:gen_eig} we show how equation (\ref{eq:Roothaan}) can be transformed to a regular
eigenvalue problem
\begin{equation}
 \mathbf{F'C'_k} = \varepsilon_k\mathbf C'_k.
\end{equation}


Let us take a closer look at the Fock matrix:
\begin{equation}
 \begin{split}
  F_{\mu\nu} = & \int d\vec r \chi^*_\mu(\vec r) F(\vec r)\chi_\nu(\vec r)\\
                  = & \int d\vec r \chi^*_\mu(\vec r)[h(\vec r) + 2J(\vec r) - K(\vec r)]\chi_\nu(\vec r) \\
                  = & \bra\mu h\ket\nu + 2 \bra\mu J\ket\nu - \bra\mu K\ket\nu.
 \end{split}
\end{equation}
To get the final expression for the matrix, we insert the expansion (\ref{eq:linear_expansion}) into the operators $J(\vec r)$ and $K(\vec r)$.
For $J(\vec r)$ this gives
\begin{equation}
\begin{split}
 J(\vec r)\chi_\nu(\vec r) & = \Big[\sum_{k=1}^{N/2}\int d\vec r'\phi^*_k(\vec r')g(\vec r, \vec r')\phi_k(\vec r')\Big]\chi_\nu(\vec r) \\
                             & = \Big[\sum_{k=1}^{N/2}\sum_{\sigma,\lambda=1}^M\int d\vec r'\chi^*_\sigma(\vec r')g(\vec r, \vec r')\chi_\lambda(\vec r')\Big]C^*_{\sigma k}C_{\lambda k}\chi_\nu(\vec r),
\end{split}
\end{equation}
so that
\begin{equation}
 \bra\mu J\ket\nu = \sum_{k=1}^{N/2}\sum_{\sigma,\lambda=1}^M\bra{\mu\sigma}g\ket{\nu\lambda}C^*_{\sigma k}C_{\lambda k},
\end{equation}
where the matrix element $\bra{\mu\sigma}g\ket{\nu\lambda}$ is defined in equation (\ref{eq:pqgrs_def}).
The expression for $\bra\mu K\ket\nu$ is similar, except that the indices $\nu$ and $\lambda$ switch places in the integral:
\begin{equation}
 \bra\mu K\ket\nu = \sum_{k=1}^{N/2}\sum_{\sigma,\lambda=1}^M\bra{\mu\sigma}g\ket{\lambda\nu}C^*_{\sigma k}C_{\lambda k}.
\end{equation}
Hence, the Fock matrix is given explicitly in terms of the basis functions by
\begin{equation}
 F_{\mu\nu} = \bra\mu h\ket\nu + \sum_{k=1}^{N/2}\sum_{\sigma,\lambda=1}^M[2\bra{\mu\sigma}g\ket{\nu\lambda} - \bra{\mu\sigma}g\ket{\lambda\nu}]C^*_{\sigma k}C_{\lambda k}.
\end{equation}
It is useful to define the so-called density matrix
\begin{equation}
\label{eq:density_matrix}
 P_{\lambda\sigma} = 2\sum_{k=1}^{N/2}C_{\lambda k}C^*_{\sigma k},
\end{equation}
which allows us to write the Fock matrix more compactly as
\begin{equation}
\label{eq:Fock_matrix}
 F_{\mu\nu} = \bra\mu h\ket\nu + \frac{1}{2}\sum_{\sigma,\lambda=1}^M P_{\lambda\sigma}[2\bra{\mu\sigma}g\ket{\nu\lambda} - \bra{\mu\sigma}g\ket{\lambda\nu}].
\end{equation}

The energy is found by expanding equation (\ref{eq:E_ref3}) in the known basis:
\begin{equation}
 \label{eq:E_ref4}
 E_0 = \frac{1}{2}\sum_{\mu,\nu=1}^M P_{\nu\mu}[\bra\mu h \ket\nu + F_{\mu\nu}].
\end{equation}






% Inserting this into equation (\ref{eq:RHF}), multiplying by $\chi^*_p(\vec r)$ and integrating with respect to $\vec r$ leads to the following set of eigenvalue equations
% \begin{equation}
%  \mathbf{FC_k} = \varepsilon_k\mathbf{SC_k}
% \end{equation}
% where the elements of the Fock matrix $\mathbf{F}$ are
% \begin{equation}
%  F_{pq} = \bra{p}h\ket{q} + \sum_l\sum_{rs}(2\bra{pr}g\ket{qs} - \bra{pr}g\ket{sq})C^*_{rl}C_{sl}
% \end{equation}
% and the elements of the overlap matrix $\mathbf{S}$ are
% \begin{equation}
%  S_{pq} = \langle p|q\rangle.
% \end{equation}
% It is convenient to introduce the density matrix $P_{rs} = 2\sum_lC^*_{rl}C_{sl}$. The Fock matrix is then written more compactly as
% \begin{equation}
%  F_{pq} = \bra{p}h\ket{q} + \frac{1}{2}\sum_{rs}P_{rs}(2\bra{pr}g\ket{qs} - \bra{pr}g\ket{sq}).
% \end{equation}
% The energy is given by
% \begin{equation}
%  E_{ref} = \sum_{pq}P_{pq}\bra{p}h\ket{q} + \frac{1}{2}\sum_{pqrs}P_{pq}P_{rs}(\bra{pr}g\ket{qs} - \frac{1}{2}\bra{pr}g\ket{sq})
% \end{equation}


\section{Unrestricted Hartree-Fock (UHF)}
\label{sec:UHF}
We now derive the unrestricted Hartree-Fock equations which result from the assumption (\ref{eq:unrestricted_spinorbitals}). The Hartree-Fock equations (\ref{eq:canonical_hartree_fock})
are now split into two sets of equations
\begin{align}
\mathcal F^\alpha(\vec x) \phi^\alpha_k(\vec r)\alpha(s) & = \varepsilon^\alpha_k\phi^\alpha_k(\vec r)\alpha(s), \qquad k=1,2,\dots, N_\alpha, \\
\mathcal F^\beta(\vec x) \phi^\beta_k(\vec r)\beta(s) & = \varepsilon^\beta_k\phi^\beta_k(\vec r)\beta(s), \qquad k=1,2,\dots, N_\beta,
\end{align}
where $N^\alpha$ and $N^\beta$ are the number of electrons with spin up and spin down, respectively.
Again, we want to take out the spin part of the equations. We can insert the unrestricted spin orbitals (\ref{eq:unrestricted_spinorbitals}) and do the calculations
explicitly in the same way as we did for the restricted case. However, using the insight gained during our previous calculation, we can come up with the answer directly. Consider
for example the operator $\mathcal F^\alpha(\vec x)$. It contains a kinetic term and the potential due to the atomic nuclei. Furthermore, it contains
a direct interaction term due to the mean field set up by all electrons, both spin up and spin down. Finally it contains an exchange term due to the mean field
set up by the electrons \emph{with spin up only}. Thus we can conclude that the equations for the spatial orbitals $\phi^\alpha_k$ are given by
\begin{equation}
\label{eq:HF^alpha}
 F^\alpha(\vec r)\phi^\alpha_k(\vec r) = \varepsilon^\alpha_k\phi^\alpha_k(\vec r),
\end{equation}
where the unrestricted spin up Fock operator is defined as
\begin{equation}
\label{eq:Foperator^alpha}
 F^\alpha(\vec r) = h(\vec r) + [J^\alpha(\vec r) - K^\alpha(\vec r)] + J^\beta(\vec r),
\end{equation}
with
\begin{align}
 J^\alpha(\vec r)\phi^\alpha_k(\vec r) = & \sum_{l=1}^{N^\alpha}\Big[\int d\vec r' \phi^\alpha_l(\vec r')g(\vec r, \vec r')\phi^\alpha_l(\vec r')\Big]\phi^\alpha_k(\vec r), \\
 J^\beta(\vec r)\phi^\alpha_k(\vec r) = & \sum_{l=1}^{N^\beta}\Big[\int d\vec r' \phi^\beta_l(\vec r')g(\vec r, \vec r')\phi^\beta_l(\vec r')\Big]\phi^\alpha_k(\vec r), \\
 K^\alpha(\vec r)\phi^\alpha_k(\vec r) = & \sum_{l=1}^{N^\alpha}\Big[\int d\vec r' \phi^\alpha_l(\vec r')g(\vec r, \vec r')\phi^\alpha_k(\vec r')\Big]\phi^\alpha_l(\vec r).
\end{align}
The equations for the spatial orbitals $\phi^\beta_k$ are the same, except that the indices $\alpha$ and $\beta$ switch places.


\subsection{Introducing a basis}
To solve the unrestricted Hartree-Fock equations we expand the spatial orbitals in terms of a known basis
\begin{align}
 \phi^\alpha_k(\vec r) & = \sum_{\mu=1}^M C^\alpha_{\mu k} \chi_\mu(\vec r), \label{eq:phi^alpha_expansion}\\
 \phi^\beta_k(\vec r) & = \sum_{\mu=1}^M C^\beta_{\mu k} \chi_\mu(\vec r), \label{eq:phi^beta_expansion}
\end{align}
just as we did in the restricted case. Inserting the expansion (\ref{eq:phi^alpha_expansion}) into equation (\ref{eq:HF^alpha}) gives
\begin{equation}
 \sum_{\nu=1}^M F^\alpha(\vec r)\chi_\nu(\vec r)C^\alpha_{\nu k} = \varepsilon^\alpha_k\sum_{\nu=1}^M C^\alpha_{\nu k}\chi_\nu(\vec r).
\end{equation}
If we multiply this equation by $\chi^*_\mu(\vec r)$ and integrate over $\vec r$, we get
\begin{equation}
 \sum_{\nu=1}^M F^\alpha_{\mu\nu}C^\alpha_{\nu k} = \varepsilon^\alpha_k\sum_{\nu=1}^M S_{\mu\nu}C^\alpha_{\nu k},
\end{equation}
where $S_{\mu\nu}$ is the overlap matrix and $F^\alpha_{\mu\nu}$ is the matrix representation of the unrestricted spatial Fock operator
\begin{equation}
\label{eq:Fmatrix^alpha}
 F^\alpha_{\mu\nu} = \int d\vec r \chi^*_\mu(\vec r) F^\alpha(\vec r)\chi_\nu(\vec r).
\end{equation}
The corresponding equations for the spin down particles are derived in exactly the same way, of course. In total we have the two sets of equations
\begin{align}
 \mathbf F^\alpha \mathbf C^\alpha_k = & \varepsilon^\alpha_k\mathbf{SC}^\alpha_k, \label{eq:pople-nesbet_up}\\
 \mathbf F^\beta \mathbf C^\beta_k = & \varepsilon^\beta_k\mathbf{SC}^\beta_k, \label{eq:pople-nesbet_down}
\end{align}
called the \emph{Pople-Nesbet equations} \cite{pople_nesbet}, which are the unrestricted generalisation of the Roothaan equations derived in the previous section. They are nonlinear
and coupled since the matrices $\mathbf F^\alpha$ and $\mathbf F^\beta$ are functions of both $\{\mathbf C^\alpha_k\}$ and $\{\mathbf C^\beta_k\}$.

The final expression for the Fock matrix $\mathbf F^\alpha$ is obtained by inserting the expansion (\ref{eq:phi^alpha_expansion}) into equation (\ref{eq:Fmatrix^alpha}).
Recalling that the Fock operator $F^\alpha(\vec r)$ is given by (\ref{eq:Foperator^alpha}) this leads to
\begin{equation}
 F^\alpha_{\mu\nu} = \bra{\mu}h\ket{\nu} + \sum_{k=1}^{N^\alpha}\sum_{\sigma,\lambda=1}^M\langle\mu\sigma||\nu\lambda\rangle(C^\alpha_{\sigma k})^* C^\alpha_{\lambda k}
                                         + \sum_{k=1}^{N^\beta}\sum_{\sigma,\lambda=1}^M\bra{\mu\sigma}g\ket{\nu\lambda}(C^\beta_{\sigma k})^* C^\beta_{\lambda k},
\end{equation}
and similarly we find
\begin{equation}
 F^\beta_{\mu\nu} = \bra{\mu}h\ket{\nu} + \sum_{k=1}^{N^\beta}\sum_{\sigma,\lambda=1}^M\langle\mu\sigma||\nu\lambda\rangle(C^\beta_{\sigma k})^* C^\beta_{\lambda k}
                                         + \sum_{k=1}^{N^\alpha}\sum_{\sigma,\lambda=1}^M\bra{\mu\sigma}g\ket{\nu\lambda}(C^\alpha_{\sigma k})^* C^\alpha_{\lambda k},
\end{equation}
where $\bra{\mu\sigma}g\ket{\nu\lambda}$ and $\langle\mu\sigma||\nu\lambda\rangle$ are defined in equations (\ref{eq:pqgrs_def}) and (\ref{eq:pqgrs_as_def}),
respectively. If we introduce the density matrices
\begin{align}
  P^\alpha_{\lambda\sigma} = & \sum_{k=1}^{N^\alpha}C^\alpha_{\lambda k} (C^\alpha_{\sigma k})^*, \label{eq:density_matrix_up}\\
  P^\beta_{\lambda\sigma} = & \sum_{k=1}^{N^\beta}C^\beta_{\lambda k} (C^\beta_{\sigma k})^*, \label{eq:density_matrix_down}\\
  P^T_{\lambda\sigma} = & P^\alpha_{\lambda\sigma} + P^\beta_{\lambda\sigma},
\end{align}
the Fock matrices can be written more compactly as
\begin{equation}
\label{eq:Fock_matrix_up}
 F^\alpha_{\mu\nu} = \bra{\mu}h\ket{\nu} + \sum_{\sigma,\lambda=1}^M\left[\langle\mu\sigma||\nu\lambda\rangle P^\alpha_{\lambda\sigma}
                                                                      + \bra{\mu\sigma}g\ket{\nu\lambda}P^\beta_{\lambda\sigma}\right],
\end{equation}
and
\begin{equation}
\label{eq:Fock_matrix_down}
 F^\beta_{\mu\nu} = \bra{\mu}h\ket{\nu} + \sum_{\sigma,\lambda=1}^M\left[\langle\mu\sigma||\nu\lambda\rangle P^\beta_{\lambda\sigma}
                                                                  + \bra{\mu\sigma}g\ket{\nu\lambda}P^\alpha_{\lambda\sigma}\right].
\end{equation}


The energy can be found from the expectation value of these matrices, keeping in mind that the double counting of the interaction terms must be taken into account.
The expression is
\begin{equation}
 E_0 = \frac{1}{2}\sum_{k=1}^{N^\alpha}\sum_{\mu,\nu=1}^M \big[\bra\mu h\ket\nu + F^\alpha_{\mu\nu} \big](C^\alpha_{\mu k})^* C^\alpha_{\nu k}
          +\frac{1}{2}\sum_{k=1}^{N^\beta}\sum_{\mu,\nu=1}^M\left[\bra\mu h\ket\nu + F^\beta_{\mu\nu}\right](C^\beta_{\mu k})^* C^\beta_{\nu k},
\end{equation}
or in terms of the density matrices
\begin{equation}
 E_0 = \frac{1}{2}\sum_{\mu,\nu=1}^M\left[P^T_{\nu\mu}\bra\mu h\ket\nu + P^\alpha_{\nu\mu}F^\alpha_{\mu\nu} + P^\beta_{\nu\mu}F^\beta_{\mu\nu}\right].
\end{equation}




\section{Solving the generalised eigenvalue problem}
\label{sec:gen_eig}

In this section we show how the generalised eigenvalue problem
\begin{equation}
 \mathbf{FC}_k = \varepsilon_k\mathbf{SC}_k,
\end{equation}
can be transformed to the regular eigenvalue problem
\begin{equation}
 \mathbf F'\mathbf C'_k = \varepsilon_k\mathbf C'_k.
\end{equation}
We can achieve this if there exists a matrix $\mathbf X$ such that
\begin{equation}
 \mathbf{X^\dagger S X = I},
\end{equation}
because, if we then let $\mathbf{C_k = X C'_k}$, we get
\begin{equation}
\begin{split}
 \mathbf{FC}_k & = \varepsilon_k\mathbf{SC}_k \\
 \mathbf{FXC}'_k & = \varepsilon_k\mathbf{SXC}'_k \\
 \mathbf{X^\dagger FXC}'_k & = \varepsilon_k\mathbf{X^\dagger SXC}'_k \\
 \mathbf F' \mathbf C'_k & = \varepsilon_k\mathbf C'_k,
\end{split}
\end{equation}
where
\begin{equation}
 \mathbf{F' = X^\dagger FX}.
\end{equation}
It only remains to show that the matrix $\mathbf X$ does indeed exist and how to construct it. First note that the overlap matrix $\mathbf S$ is Hermitian:
\begin{equation}
\begin{split}
 S_{\mu\nu} = & \int d\vec r \chi^*_\mu(\vec r) \chi_\nu(\vec r) \\
            = & \int d\vec r \chi_\nu(\vec r) \chi^*_\mu(\vec r) \\
            = & S^*_{\nu\mu}.
\end{split}
\end{equation}
This means that there exists a unitary matrix $\mathbf U$ such that
\begin{equation}
\label{eq:similarity_transf}
 \mathbf{U^\dagger S U = s},
\end{equation}
where $\mathbf s =$ diag$(s_1, s_2, \dots, s_M)$ is a diagonal matrix containing the eigenvalues of $\mathbf S$, which are all real, and the columns of $\mathbf U$ are the eigenvectors of $\mathbf S$.
Furthermore, the eigenvalues $\{s_i\}$ are positive. To see this, consider the expansion of some function $f(\vec r)$, not identically equal to zero, in terms of the basis functions $\chi_\mu(\vec r)$:
\begin{equation}
 f(\vec r) = \sum_\mu A_\mu\chi_\mu(\vec r).
\end{equation}
No matter how the coefficients are chosen, the norm of $f(\vec r)$ will be positive. In particular, if we choose $\mathbf{A}=[A_\mu]$ to be equal to the $i$'th eigenvector of $\mathbf S$, we get
\begin{equation}
\begin{split}
 0 < \langle f|f\rangle = & \sum_{\mu\nu}A^*_\mu S_{\mu\nu} A_\nu = \mathbf{A^\dagger S A} \\
                    = & s_i\mathbf{A^\dagger A} =  s_i ||\mathbf A||^2.
\end{split}                  
\end{equation}
Now, since all eigenvalues are positive, one can define the matrix
\begin{equation}
 \mathbf s^{-1/2} = \left[\begin{array}{c c c c}
                          s_1^{-1/2} &&& \\
                          & s_2^{-1/2} && \\
                          && \ddots & \\
                          &&& s_M^{-1/2} 
                         \end{array}\right].
\end{equation}
Multiplying equation (\ref{eq:similarity_transf}) from the left and right by $\mathbf s^{-1/2}$ yields
\begin{equation}
\begin{split}
 \mathbf s^{-1/2} \mathbf U^\dagger \mathbf{S U s}^{-1/2} =  \mathbf I \\
 (\mathbf{Us}^{-1/2})^\dagger \mathbf S (\mathbf{Us}^{-1/2}) =  \mathbf I,
\end{split}
\end{equation}
which means that
\begin{equation}
 \mathbf X = \mathbf{Us}^{-1/2}.
\end{equation}
 \clearemptydoublepage

\chapter{Basis sets}\label{chap:basisets}

%\section{Introduction}
\abstract{Numerical integration and differentiation
are some of the most frequently needed methods in computational
physics. Quite often we are confronted with the need of evaluating
either the derivative $f'$ or an integral  $\int f(x)dx$.  
The aim of this chapter is to introduce some of these methods
with a critical eye on numerical accuracy, following the discussion
in the previous chapter. 
 The next section deals essentially with topics from numerical differentiation.
There we present also the most commonly used formulae for computing
first and second derivatives, formulae which in turn find their most important
applications in the numerical solution of ordinary and partial 
differential equations. We discuss also selected methods for numerical 
interpolation. 
This  chapter serves also the scope of introducing
some more advanced C++ programming concepts, such as call
by reference and value, reading and writing to a file and the use
of dynamic memory allocation.  We will also discuss several object-oriented features of C++,
ending the chapter with an analogous discussion of Fortran features.}


In the previous chapter the general Hartree-Fock equations, which is a set of
integro-differential equations, were converted to a set of algebraic equations (the Roothaan
equations in the restricted case and the Pople-Nesbet equations in the unrestricted case) by
expanding the unknown orbitals in a known set of basis functions. The Fock operator was then
reduced to a matrix, the elements of which are integrals involving the chosen basis functions.

We start this chapter by discussing two popular types of basis functions, namely the 
Slater-type orbtials (STOs) and the Gaussian-type orbitals (GTOs). The latter is more suited for
molecular calculations and is the one we will use in this thesis. Thereafter we derive the
integration scheme for the Gaussian type orbitals.



\section{Basis functions}
For a molecular system, the eigenfunctions of the Hartree-Fock equations are called
\emph{molecular orbitals} (MOs). As discussed earlier, it is important to distinguish these
from the perhaps more familiar \emph{atomic orbitals}, and it is erroneous to
think that the electrons of molecular systems are occupying atomic orbitals.
Consider for example the $H_2$-molecule.
In the ground state the electrons are not occupying the 1$s$-orbitals of atomic hydrogen. The
molecular system is entirely different from the atomic one, with an entirely different Hamiltonian,
and the eigenstates of the Hartree-Fock equations will therefore also be different.

In order to solve the Hartree-Fock equations, we need to expand the molecular orbitals in a known
set of basis functions
\begin{equation}
 \phi_k(\vec r) = \sum_{\mu=1}^M\chi_{\mu k}(\vec r).
\end{equation}
The importance of choosing suitable basis functions can hardly be overemphasised; it completely determines
the accuracy of the results as well as the computational cost of the calculations. In choosing basis functions,
the following criteria should be met:
\begin{enumerate}
 \item The functions must be physically reasonable, i.e., they should have large probability where
      the electrons are likely to be and small probability elsewhere.
 \item It should be possible to integrate the functions efficiently.
 \item The solution of the Hartree-Fock equations must converge towards the Hartree-Fock limit (see chapter \ref{chapter:electron_correlation})
       as the number of basis functions increases.
\end{enumerate}
The first point suggests that we choose atomic orbitals as basis functions,  which is often
referred to as ``linear combination of atomic orbitals'' (LCAO). In this thesis we will let the
atomic orbitals be centered at the nuclei. However, this is not strictly required since the
atomic orbitals are merely being used as basis functions, and they are not to be thought of
as orbitals occupied by electrons. In the following two subsections, we
discuss two common types of atomic orbitals, namely the Slater type orbitals (STOs) and
Gaussian type orbitals (GTOs), respectively. Only the GTOs will be applied to the 
calculations in this thesis.


\subsection{Slater-type orbitals (STOs)}
The Slater type orbitals are defined as \cite{Cramer}
\begin{equation}
 \chi^{STO}(r,\theta,\phi,n,l,m) = \frac{(2a)^{n+1/2}}{[(2n)!]^{1/2}}r^{n-1}\exp(-a r) Y^m_l(\theta,\phi),
\end{equation}
where $n$ is the principal quantum number, $l$ and $m$ are the angular momentum quantum numbers,
$Y^m_l(\theta,\phi)$ are the spherical harmonics familiar from the solution of the Schrödinger
equation for the hydrogen atom, and $a$ is an exponent which determines the radial decay of the
function. The main attractive features of the STOs are that they have the correct exponential decay with
increasing $r$ and that the angular components are hydrogenic. For this reason, they are often
used in atomic Hartree-Fock calculations. When doing molecular calculations, however, they have the
disadvantage that the two-particle integrals $\bra{\mu\sigma}g\ket{\nu\lambda}$ occuring in the
Fock matrix $F_{\mu\nu}$ have no known analytical expression. This is because integrals of products
of exponentials centered on different nuclei are difficult to handle. They can of course be calculated
numerically, but for large molecules this is very time consuming.

\subsection{Gaussian-type orbitals (GTOs)}
A clever trick which makes multiple center integrals easier to handle is to replace the exponential
term $\exp(-a r)$ with $\exp(-a r^2)$, i.e., to use Gaussian functions. This greatly simplifies the
integrals because the product of two Gaussians centered on nuclei with positions
$\vec A$ and $\vec B$ is equal to \emph{one} Gaussian centered on some point $\vec P$ on
the line between them:
\begin{equation}
\label{eq:gaussian_product}
 \exp(-a|\vec r - \vec A|^2)\cdot \exp(-b|\vec r - \vec B|^2) = K_{AB}\,\exp(-p|\vec r - \vec P|^2),
\end{equation}
where
\begin{eqnarray}
 K_{AB} & = & \exp\Big(-\frac{ab}{a + b}|\vec A - \vec B|^2\Big), \\
 \vec P & = & \frac{a \vec A + b \vec B}{a + b}, \\
 p & = & a + b.
\end{eqnarray}
This is the so-called \emph{Gaussian product theorem}. It is illustrated in the one-dimensional case in figure \ref{fig:Gaussians}.
\begin{figure}
 \begin{center}
  \includegraphics[scale=0.5]{basis_functions_and_integral_evaluation/figures/Gaussians.pdf}
  \caption{Illustration of the Gaussian product theorem which says that the product of two Gaussians with centers at points A and B is another Gaussian
           with center somewhere between A and B.}
  \label{fig:Gaussians}
 \end{center}
\end{figure}


The general functional form of a normalised Gaussian type orbital 
centered at $\vec A$ is given by \cite{Cramer}
\begin{equation}
\label{eq:gaussian_primitive}
 G_{ijk}(a, \vec r_A) = \Big(\frac{2a}{\pi}\Big)^{3/4}\Big[\frac{(8a)^{i+j+k}\,i!\,j!\,k!}{(2i)!\,(2j)!\,(2k)!}\Big]x_A^i\,y_A^j\,z_A^k\,\exp(-a r_A^2),
\end{equation}
where $\vec r_A = \vec r - \vec A$ and the integers $i$, $j$, $k$ determine the angular momentum quantum number $l=i+j+k$. 

\subsection{Contracted GTOs}
\label{subsec:contracted_gtos}
The greatest drawback with Gaussians is that they do not have the proper exponential radial decay. This can be remedied by forming linear combinations of GTOs
to resemble the STOs
\begin{equation}
 \chi^{CGTO}(\vec r_A,i,j,k) = \sum_{p=1}^L d_p G_{ijk}(a_p, \vec r_A).
\end{equation}
These are called STO-LG basis functions, where L refers to the number of Gaussians used in the linear combination. Hehre, Stewart and Pople \cite{pople}
were the first to systematically calculate optimal coefficients $d_p$ and exponentials $a_p$, and today the STO-LG basis sets are available
for most atoms.
The individual Gaussians are called \emph{primitive} basis functions and the linear combinations are called \emph{contracted} basis functions, hence the label
CGTO (Contracted Gaussian Type Orbital). In this thesis we will only consider CGTOs, and whenever the symbol $\chi$ appears without any label,
we will always mean CGTO. 

A very common choice for the STO-LG basis sets is $L=3$. Figure \ref{fig:STOvsSTO3G} shows how the 1$s$ STO for hydrogen is approximated by 3 GTOs.
The exponents $a_p$ and coefficients $d_p$ have been set so that the contracted basis function lies as close to the STO as possible, see
table \ref{tab:STO3G}. It is important to note that the parameters $(a_p,d_p)$ are static and that the linear combination of Gaussians constitute \emph{one single}
basis function. Throughout this text the phrase ``basis function'' will always refer to a contracted basis function.



\begin{figure}
 \begin{center}
  \includegraphics[scale=0.7]{basis_functions_and_integral_evaluation/figures/STO3G.pdf}
  \caption{The whole line shows the 1s STO basis function, while the broken line shows a linear combination of three Gaussians.}
  \label{fig:STOvsSTO3G}
 \end{center}
\end{figure}

% Note that the GTOs and STOs do not have any radial nodes as the hydrogenic functions do. This means that no single GTO or STO can mimic the 2$s$ hydrogenic orbital for example. With contractions, however,
% this problem is eliminated; nodes can be introduced by using a combination of positive and negative coefficients $d_p$.

The STO-LG basis sets belong to the family of \emph{minimal basis sets}.
It means that there is one and only one basis function per atomic orbital.
The STO-LG basis sets for the hydrogen and helium atoms, for example, contain only one basis
function for the 1$s$ atomic orbital. This basis function is, as explained above, composed of
a linear combination of L primitives. For the atoms lithium through neon the STO-LG basis
sets contain 5 basis functions; one for each of the atomic orbitals
$1s$, $2s$, $2p_x$, $2p_y$ and $2p_z$.

\begin{table}
 \begin{center}
 \caption{Coefficients and exponents used in the STO-3G basis shown in figure \ref{fig:STOvsSTO3G}.}
 \label{tab:STO3G}
  \begin{tabular}{|c|c|c|c|}\hline
   $p$        &  1  &  2  &   3 \\ \hline
   $d_p$      &  0.1543   & 0.5353    & 0.4446  \\ \hline
   $a_p$ &  3.4252   & 0.6239    & 0.1688  \\ \hline
  \end{tabular}
 \end{center}
\end{table}

The reader might be asking herself why the coefficients $d_p$ in the linear combination of the STO-LGs are static. Shouldn't the accuracy of our results
actually improve if we let the coefficients vary? The answer to this is yes. However, the linear system to be solved (the Hartree-Fock equations) will then
be larger. Thus there is a trade off between accuracy and computational efficiency which must be considered.

However, basis sets where the STO-LG sets have been ``decontracted'' as described above have actually been used. They belong to the family of $\zeta$-basis sets.
The double-$\zeta$ and triple-$\zeta$ basis sets have two and three basis functions, respectively, for each atomic orbital. As an example, we could
create a double-$\zeta$ basis set from the STO-3G basis set by contracting the two first primitives and leave the third as a normalised primitive. Similarly, we could
counstruct a triple-$\zeta$ basis set by treating each primitive as a basis function.

Let us for a moment assume that we use a triple-$\zeta$ basis set constructed from the STO-3G to do Hartree-Fock calculations on atomic oxygen.
What will the resulting orbitals look like? The lowest orbital will be very close to the definition of the 1$s$ orbital of the STO-3G set.
This is to be expected since the STO-3G basis functions are constructed to resemble the STO atomic orbitals.

What if we now apply the same triple-$\zeta$ basis to calculations on the CO molecule, say? In this case we would probably also find an orbital which resembles the 1$s$
orbital of the STO-3G basis for oxygen. This is because it is mostly the valence electrons which contribute in the bonding between atoms, and the core electrons are more or less unaffected.
Thus decontracting basis functions corresponding to the core atomic orbitals will generally not pay off, but will merely increase the computational load. Therefore, basis
sets have been constructed where only the functions corresponding to the valence atomic orbitals are decontracted. These are the so-called \emph{split-valence} basis sets. 
An example of a split-valence basis set is the 3-21G. The number before the hyphen (in this case 3) is the number of primitives per core atomic orbital. The fact that there are
two numbers after the hyphen signifies that there are two basis functions for each valence atomic orbital. The numbers themselves (in this case 2 and 1) indicate how many primitives
the first and second of these are composed of. As an example, the 3-21G basis set for the oxygen atom has one single
basis function for the 1$s$ orbital (since this is the core orbital), and this basis function consists of three primitives. Furthermore, it has two basis functions for the 2$s$,
2$p_x$, 2$p_y$ and 2$p_z$ atomic orbitals (since these are the valence orbitals). The first consists of two primitives, and the second consists of only one
primitive. In sum the oxygen atom thus has 9 basis functions which are built up from a total of 15 primitives.
Other examples of split-valence basis sets are 4-31G, 6-31G and 6-311G \cite{Cramer}.

In many molecular calculations, the split-valence basis sets mentioned thus far do not provide enough flexibility to describe the chemistry appropriately.
This is often fixed by adding functions
corresponding to atomic orbitals with angular momentum $l_{max}+1$, where $l_{max}$ is the highest angular momentum of the atom. 
Such functions are called \emph{polarisation functions}. For example, the polarisation functions for the Oxygen atom are the $d$-functions. Asterisks (*) are added to the name of the basis set to indicate that polarisation
functions are included. One asterisk (as in 6-31G*) indicates that $d$-functions are added to polarise the $p$-functions of first row atoms (Li-Ne). Two asterisks (as in 6-31G**)
mean that $p$-functions are added to polarise the $s$-functions of hydrogen and helium as well.

% Note that the 3-21G basis set for Oxygen contains basis functions for each atomic orbital up to the $p$-orbital. Similarly, all the split-valence basis sets mentioned above
% contain basis functions for each atomic orbital up to the the highest present in the atom which is considered. However, in many molecular calculations this does not provide enough flexibility to
% describe the chemistry correctly. One way to increase the flexibility is to include basis functions for the orbital \emph{above} the highest present in the atom. These functions
% are called \emph{polarization functions}. The polarization functions for the Oxygen atom are the $d$-orbitals. An asterix (*) is added to the name of the basis set to indicate that polarization
% functions are included. One asterisk (as in 3-21G*, for example) indicate that $d$-functions are added to polarize the $p$-functions. Two asterisks (as in 3-21G**, for example)
% are added to show that also the $s$-functions of Hydrogen and helium are polarized by $p$-functions.

It should be noted that the list of basis sets mentioned here is in no way exhaustive. There is a flora of basis sets out there, see for example
Cramer \cite{Cramer} or Helgaker \emph{et al} \cite{Helgaker}.




\section{Integral evaluation}
As discussed in the previous section, using Gaussian basis functions significantly improves the speed of the integrations which must be done when setting up the Fock matrix.
This section discusses the details of how the integration is performed.

We start by summarising the most important properties of the Cartesian Gaussians. Thereafter, we
change basis to the so-called Hermite Gaussians, as proposed by \v{Z}ivkovi\'{c} and Maksi\'{c} \cite{zivkovic_maksic}.
Then, following the work of McMurchie and Davidson \cite{mcmurchie_davidson}, we show how the one- and two-particle integrals can be expressed compactly in
terms of some auxiliary functions. The auxiliary functions are computed via a set of recurrence relations.

A thorough review of the techniques presented can be found in Helgaker \emph{et al} \cite{Helgaker}.

\subsection{Cartesian Gaussians}
The Cartesian Gaussian functions centered at $\vec A$ are given by
\begin{equation}
 G_{ijk}(a, \vec r_A) = x^i_A\,y^j_A\,z^k_A\,\exp(-a r^2_A),
\end{equation}
where $\vec r_A = \vec r - \vec A$. These will be our primitive basis functions. They factorise in the Cartesian components
\begin{equation}
 G_{ijk}(a, \vec r_A) = G_i(a, x_A)\,G_j(a, y_A)\,G_k(a, z_A),
\end{equation}
where
\begin{equation}
 G_i(a, x_A) = x^i_A\,\exp(-a x^2_A),
\end{equation}
and the other factors are defined similarly. Each of the components obey the simple recurrence relation
\begin{equation}
 x_A\,G_i = G_{i+1}.
\end{equation}


\subsection{Gaussian overlap distribution}
We introduce the following shorthand notation
\begin{align}
  G_a(\vec r) & = G_{ikm}(a, \vec r_A), \label{eq:Ga}\\
  G_b(\vec r) & = G_{jln}(b, \vec r_B), \label{eq:Gb}
\end{align}
and define the overlap distribution
\begin{equation}
 \Omega_{ab}(\vec r) = G_a(\vec r)\,G_b(\vec r).
\end{equation}
Using the Gaussian product theorem (\ref{eq:gaussian_product}) this can be written as
\begin{equation}
 \Omega_{ab}(\vec r) = K_{AB}\,x^i_A\,x^j_B\,y^k_A\,y^l_B\,z^m_A\,z^n_B\,\exp(-p\,r^2_P),
\end{equation}
where
\begin{equation}
\label{eq:gaussian_product_defs}
 \begin{split}
  K_{AB} & = \exp\Big(-\frac{ab}{a + b}R^2_{AB}\Big) \\
  \vec R_{AB} & =  \vec A - \vec B \\
  p & = a + b\\
  \vec r_P & = \vec r - \vec P \\
  \vec P & = \frac{a\vec A + b\vec B}{a + b}.
 \end{split}
\end{equation}
Because the Gaussians $G_a$ and $G_b$ factorise in their Cartesian components, so does the overlap distribution
\begin{equation}
 \Omega_{ab}(\vec r) = \Omega_{ij}(x)\,\Omega_{kl}(y)\,\Omega_{mn}(z),
\end{equation}
where
\begin{equation}
 \Omega_{ij} = K^x_{AB}\,x^i_A\,x^j_B\,\exp(-px^2_P)
\end{equation}
and
\begin{equation}
\begin{split}
 K^x_{AB} = & \exp\Big(-\frac{ab}{a + b}X^2_{AB}\Big) \\
   X_{AB} = & A_x - B_x.
\end{split}
\end{equation}
The distributions $\Omega_{kl}(y)$ and $\Omega_{mn}(z)$ are defined similarly.




\subsection{Hermite Gaussians}
Later we will expand the Cartesian Gaussians in terms of the so-called Hermite Gaussians.
This will simplify the integrations significantly. The Hermite Gaussians centered at $\vec P$ are defined by
\begin{equation}
 \Lambda_{tuv}(p, \vec r_p) = \Big(\frac{\partial}{\partial P_x}\Big)^t \Big(\frac{\partial}{\partial P_y}\Big)^u \Big(\frac{\partial}{\partial P_z}\Big)^v \exp(-p\, r^2_P),
\end{equation}
where $\vec r_p = \vec r - \vec P$. They factorise in the same way as the Cartesian Gaussians do:
\begin{equation}
 \Lambda_{tuv}(p, \vec r_P) = \Lambda_t(p,x_P)\,\Lambda_u(p,y_P)\,\Lambda_v(p,z_P),
\end{equation}
where
\begin{equation}
\label{eq:HermiteGaussian_x}
 \Lambda_t(p,x_P) = \Big(\frac{\partial}{\partial P_x}\Big)^t \exp(-p\,x^2_P),
\end{equation}
and the other factors are defined similarly. However, their recurrence relation is quite different from that of the Cartesian Gaussians:
\begin{equation}
\begin{split}
 \Lambda_{t+1}(p,x_P) & = \Big(\frac{\partial}{\partial P_x}\Big)^t \frac{\partial}{\partial P_x}\exp(-px^2_P) \\
                      & = \Big(\frac{\partial}{\partial P_x}\Big)^t 2px_P \exp(-px^2_P)  \\
                      & = 2p[-t\Big(\frac{\partial}{\partial P_x}\Big)^{t-1} + x_P \Big(\frac{\partial}{\partial P_x}\Big)^t] \exp(-px^2_P) \\
                      & = 2p[-t\Lambda_{t-1} + x_P \Lambda_t],
\end{split}
\end{equation}
where we have used that
\begin{equation}
\label{eq:derivation_rule}
 \Big(\frac{\partial}{\partial x}\Big)^t x f(x) = t\Big(\frac{\partial}{\partial x}\Big)^{t-1}f(x) + x\Big(\frac{\partial}{\partial x}\Big)^t f(x).
\end{equation}
Thus the recurrence relation reads
\begin{equation}
\label{eq:hermite_gaussian_recurrence}
 x_P \Lambda_t = \frac{1}{2p}\Lambda_{t+1} + t\Lambda_{t-1}.
\end{equation}




\subsection{Overlap integral $S_{ab}$}
Our goal is to compute the overlap integral
\begin{equation}
S_{ab}  = \langle G_a|G_b\rangle = \int d\vec r \,\Omega_{ab}(\vec r)
\end{equation}
between two Gaussians centered at the points $\vec A$ and $\vec B$.
Note that since the overlap distribution $\Omega_{ab}$ factorise in the Cartesian components, the integrals over $x$, $y$ and $z$ can be calculated independently of each other:
\begin{equation}
\begin{split}
 S_{ab} = & \langle G_i|G_j\rangle \langle G_k|G_l\rangle \langle G_m|G_n\rangle \\
        = & S_{ij}\,S_{kl}\,S_{mn}.
\end{split}
\end{equation}
The $x$ component of the overlap integral, for example, is given by
\begin{equation}
\label{eq:intG_ij}
\begin{split}
 S_{ij} = & \int dx \,\Omega_{ij}(x) \\
        = & K_{AB}^x\int dx \,x_A^ix_B^j\exp(-px_P^2).
\end{split}
\end{equation}
In equation (\ref{eq:intG_ij}) the two-center Gaussians have been reduced to a one-center Gaussian.
However, the integral is still not straightforward to calculate because of the powers $x_A^i$ and $x_B^j$. A smart way to deal with this is to express the Cartesian Gaussian
in terms of the Hermite Gaussians. Note that (\ref{eq:HermiteGaussian_x}) is a polynomial of order $t$ in $x$ multiplied by the exponential function. In equation (\ref{eq:intG_ij}) the polynomial
is of order $i+j$. This means that we can express the overlap distribution $\Omega_{ij}(x)$ in equation (\ref{eq:intG_ij}) in terms of the Hermite Gaussians in (\ref{eq:HermiteGaussian_x}) in the following way:
\begin{equation}
\label{eq:LinCombOfHermGauss}
 \Omega_{ij}(x) = \sum_{t=0}^{i+j} E^{ij}_t \Lambda_t(p, x_P),
\end{equation}
where $E^{ij}_t$ are constants.
Note that the sum is over $t$ only. The indices $i$ and $j$ are static and are determined from the powers of $x$ in $G_i$ and $G_j$.
We use them as labels on the coefficients $E^{ij}_t$ because different sets of indices will lead to different sets of coefficients.

To get the overlap integral in the $x$-direction we integrate (\ref{eq:LinCombOfHermGauss}) over $\mathbb{R}$, which now turns out to be extremely easy;
the only term that survives the integration is the term for $t=0$:
\begin{eqnarray}
 \int dx\,\Lambda_t(p,x_P) & = & \int dx\,\Big(\frac{\partial}{\partial P_x}\Big)^t\exp(-p\,x^2_P), \\
                           & = & \Big(\frac{\partial}{\partial P_x}\Big)^t \int dx\,\exp(-p\,x^2_P), \\
                           & = & \sqrt{\frac{\pi}{p}}\,\delta_{t0}.
\end{eqnarray}
We have used Leibniz' rule, which says that the differentiation of an integrand with respect to a variable which is not an integration variable can
be moved outside the integral. Thus the integral in (\ref{eq:intG_ij}) is simply
\begin{equation}
 S_{ij} = E^{ij}_0\,\sqrt{\frac{\pi}{p}}.
\end{equation}
The exact same procedure can be used for the integrals with respect to $y$ and $z$, which means that the total overlap integral is
\begin{equation}
\label{eq:S_ab}
 S_{ab} = E^{ij}_0\,E^{kl}_0\,E^{mn}_0\,\Big(\frac{\pi}{p}\Big)^{3/2}.
\end{equation}
So far nothing has been said about how we actually determine the coefficients $E^{ij}_t$. First observe that when $i=j=0$ in equation (\ref{eq:LinCombOfHermGauss})
we obtain
\begin{equation}
 E^{0,0}_0 = K_{AB}^x.
\end{equation}
The other coefficients are found via the following recurrence relations
\begin{equation}
\label{eq:E_recurrence}
\begin{split}
 E^{i+1,j}_t & = \frac{1}{2p}E^{ij}_{t-1} + X_{PA}E^{ij}_t + (t+1)E^{ij}_{t+1} \\
 E^{i,j+1}_t & = \frac{1}{2p}E^{ij}_{t-1} + X_{PB}E^{ij}_t + (t+1)E^{ij}_{t+1}.
\end{split}
\end{equation}
Analogous expressions hold for the coefficients $E^{kl}_u$ and $E^{mn}_v$. The first equation in (\ref{eq:E_recurrence}) can be derived by comparing two equivalent ways of expanding the product $G_{i+1}G_j$
in Hermite Gaussians. The first way is
\begin{equation}
 G_{i+1}\,G_j= \sum_{t=0}^{i+j+1}E^{i+1,j}_t \Lambda_t,
\end{equation}
and the second way is
\begin{equation}
\label{eq:derivation_E_coeffs}
\begin{split}
 G_{i+1}\,G_j & = x_A G_i\,G_j \\
              & = [(x - P_x) + (P_x - A_x)]\sum_{t=0}^{i+j} E^{ij}_t \Lambda_t\\
              & = \sum_{t=0}^{i+j}[x_P + X_{PA}] E^{ij}_t \Lambda_t \\
              & = \sum_{t=0}^{i+j}[\frac{1}{2p}\Lambda_{t+1} + t\Lambda_{t-1} + X_{PA}\Lambda_t]E^{ij}_t \\
              & = \sum_{t=0}^{i+j+1}[\frac{1}{2p}E^{ij}_{t-1} + X_{PA}E^{ij}_t + (t+1)E^{ij}_{t+1}] \Lambda_t,
\end{split}
\end{equation}
where we have used the recurrence relation (\ref{eq:hermite_gaussian_recurrence}) on the fourth line and changed the summation indices on the fifth line. Comparing the two expressions gives the
desired result.

Note that the change in summation indices in equation (\ref{eq:derivation_E_coeffs}) implies that we must define
\begin{equation}
 E^{ij}_t = 0, \qquad \text{if }t<0\text{ or }t > i + j.
\end{equation}



\subsection{Kinetic integral $T_{ij}$}
Next we turn to the evaluation of the kinetic integral:
\begin{equation}
\begin{split}
T_{ab} & = -\frac{1}{2}\bra{G_a}\nabla^2\ket{G_b} \\
       & = -\frac{1}{2}\bra{G_{ikm}(a, \vec r_A)}\nabla^2\ket{G_{jln}(b, \vec r_B)} \\
       & = -\frac{1}{2}(T_{ij}\,S_{kl}\,S_{mn} + S_{ij}\,T_{kl}\,S_{mn} + S_{ij}\,S_{kl}\,T_{mn}),
\end{split}
\end{equation}
where
\begin{equation}
 T_{ij} = \int dx \,G_i(a,x_A)\frac{\partial^2}{\partial x^2}G_j(b,x_B),
\end{equation}
and the other factors are defined in the same way. Performing the differentiation yields
\begin{equation}
 T_{ij} = 4b^2\,S_{i,j+2} - 2b(2j + 1)S_{i,j} + j(j-1)S_{i,j-2}.
\end{equation}
Thus we see that the kinetic integrals are calculated easily as products of the overlap integrals.


\subsection{Coulomb integral $V_{ab}$}
\label{sec:V_ab}
We now turn to the Coulomb integral due to the interaction between the electrons and the nuclei
\begin{equation}
 V_{ab} = \bra{G_a}\frac{1}{r_C}\ket{G_b},
\end{equation}
where $r_C = |\vec r - \vec C|$. As before, the overlap distribution is expanded in the Hermite Gaussians:
\begin{equation}
\begin{split}
 V_{ab} & = \int d\vec r \,\frac{\Omega_{ab}(\vec r)}{r_C} \\
        & = \sum_{tuv}E^{ij}_t E^{kl}_u E^{mn}_v\int d\vec r \, \frac{\Lambda_{tuv}(p,\vec r_P)}{r_C} \\
        & = \sum_{tuv}E^{ab}_{tuv}\int d\vec r \, \frac{\Lambda_{tuv}(p,\vec r_P)}{r_C}. \\
\end{split}
\end{equation}
Here we have used the shorthand notation
\begin{equation}
 E^{ab}_{tuv} = E^{ij}_t E^{kl}_u E^{mn}_v.
\end{equation}
In this integral other terms besides $\Lambda_{000}$ will survive due to the factor $1/r_C$. Let us nonetheless start by evaluating this term
\begin{equation}
 V_p = \int d\vec r \, \frac{\Lambda_{000}(p,\vec r_P)}{r_C} = \int d\vec r\, \frac{\exp(-p\,r_P^2)}{r_C}.
\end{equation}
We will show that this three-dimensional integral can actually be converted to a one-dimensional one. The trick is to observe that the factor $1/r_C$ can be replaced by the integral
\begin{equation}
 \frac{1}{r_C} = \frac{1}{\sqrt{\pi}}\int_{-\infty}^\infty dt\,\exp(-r^2_C\,t^2).
\end{equation}
Inserting this into $V_p$ and using the Gaussian product theorem gives
\begin{eqnarray}
 V_p & = & \int \exp(-p\,r_P^2)\Big(\frac{1}{\sqrt{\pi}}\int_{-\infty}^\infty\exp(-r^2_C\,t^2)\,dt\Big)\,d\vec r \\
     & = & \frac{1}{\sqrt{\pi}}\int_{-\infty}^\infty\int\exp\Big(-\frac{pt^2}{p + t^2}R^2_{PC}\Big)\,\exp[-(p + t^2)r^2_S] d\vec r\, dt,
\end{eqnarray}
where $\vec R_{PC} = \vec P - \vec C$ and $\vec r_S = \vec r - \vec S$ for some point $\vec S$. Doing the integral over the spatial coordinates reveals that the specific value of $\vec S$ is immaterial:
\begin{eqnarray}
 V_p & = & \frac{1}{\sqrt{\pi}}\int_{-\infty}^\infty\exp\Big(-\frac{pt^2}{p + t^2}R^2_{PC}\Big)\Big(\frac{\pi}{p + t^2}\Big)^{3/2}\,dt \\
     & = & 2\pi\int_0^\infty\exp\Big(-\frac{pt^2}{p + t^2}R^2_{PC}\Big)\frac{dt}{(p + t^2)^{3/2}}.
\end{eqnarray}
Next we change integration variable from $t$ to $u$ by defining
\begin{equation}
 u^2 = \frac{t^2}{p + t^2}.
\end{equation}
This will change the range of integration from $[0,\infty\rangle$ to $[0,1]$. This is beneficial because the final integral at which we arrive will be calculated numerically.
The change of variables leads to
\begin{align}
\label{eq:V_p}
 V_p & = \frac{2\pi}{p}\int_0^1\exp(-p\,R^2_{PC}\,u^2)\,du \\
     & = \frac{2\pi}{p}F_0(p\,R^2_{PC}),
\end{align}
where $F_0(x)$ is a special instance of the Boys function $F_n(x)$ which is defined as
\begin{equation}
 F_n(x) = \int_0^1\exp(-xt^2)\,t^{2n}\,dt.
\end{equation}
How to actually evaluate the Boys function will be discussed in section \ref{section:Boys}.

We have now a tremendously simplified way of calculating the integral of $\Lambda_{000}/r_C$. However, we need to integrate $\Lambda_{tuv}/r_C$ for general values of $t$, $u$ and $v$.
These integrals are actually not that hard to do once the Boys function is calculated:
\begin{align}
 V_{ab} & = \sum_{tuv}E^{ab}_{tuv}\int d\vec r \, \frac{\Lambda_{tuv}(p,\vec r_p)}{r_C} \\
        & = \frac{2\pi}{p}\sum_{tuv}E^{ab}_{tuv} \frac{\partial^{t+u+v} F_0(p R^2_{PC})}{\partial P_x^t \partial P_y^u \partial P_z^v} \\
        & = \frac{2\pi}{p}\sum_{tuv}E^{ab}_{tuv} R_{tuv}(p,\vec R_{PC}), \label{eq:V_ab}
\end{align}
where we have defined
\begin{equation}
 R_{tuv}(a,\vec A) = \frac{\partial^{t+u+v} F_0(a A^2)}{\partial A_x^t \partial A_y^u \partial A_z^v}.
\end{equation}
So we need to know how to calculate derivatives of the function $F_0$. Note first that
\begin{equation}
 \frac{d}{dx}F_n(x) = -F_{n+1}(x).
\end{equation}
This means that it is possible to derive analytical expressions for the Coulomb term $V_{ab}$. However, in practice they are calculated recursively in a manner similar to the way we calculate
the coefficients $E^{ij}_t$. Before presenting the recursion relations, we introduce the so-called auxiliary Hermite integrals
\begin{equation}
 R^n_{tuv}(a,\vec A) = (-2a)^n\,\frac{\partial^{t+u+v} F_n(a A^2)}{\partial A_x^t \partial A_y^u \partial A_z^v}.
\end{equation}
By starting with the source terms $R^n_{000}(a,\vec A) = (-2a)^n\,F_n(a A^2)$ we can reach the targets $R^0_{tuv}(a,\vec A) = R_{tuv}(a,\vec A)$ through the following
recurrence relations
\begin{equation}
\label{eq:R_recurrence}
 \begin{split}
  R^n_{t+1,u,v} & = tR^{n+1}_{t-1,u,v} + A_x R^{n+1}_{tuv} \\
  R^n_{t,u+1,v} & = uR^{n+1}_{t,u-1,v} + A_y R^{n+1}_{tuv} \\
  R^n_{t,u,v+1} & = vR^{n+1}_{t,u,v-1} + A_z R^{n+1}_{tuv}.
 \end{split}
\end{equation}
The first of these are derived as follows
\begin{align}
 R^{n}_{t+1,u,v} & = (-2a)^n\frac{\partial^{t+u+v}}{\partial A_x^t \partial A_y^u \partial A_z^v} [2aA_xF'_n(aA^2)] \\
                 & = (-2a)^{n+1}\frac{\partial^{t+u+v}}{\partial A_x^t\partial A_y^u \partial A_z^v}\Big[A_xF_{n+1}(aA^2)\Big] \\
                 & = (-2a)^{n+1}\frac{\partial^{u+v}}{\partial A_y^u \partial A_z^v}\Big[t\frac{\partial^{t-1}}{\partial A_x^{t-1}} + A_x\frac{\partial^t}{\partial A_x^t}\Big]F_{n+1}(aA^2) \\
                 & = tR^{n+1}_{t-1,u,v} + A_xR^{n+1}_{tuv},
\end{align}
where we have used equation (\ref{eq:derivation_rule}) and the fact that $F'_n(x) = -F_{n+1}(x)$.
% From these relations it is clear that in order to calculate the values of $R_{tuv}(p,\vec R_{PC})$ needed in (\ref{eq:V_ab}), the Boys function $F_n(p, R^2_{PC})$ has to be
% evaluated for $n\in\{0,\dots,\mathrm{max}(t_{max}, u_{max}, v_{max})\}$.


\subsection{Coulomb integral $g_{acbd}$}
\label{sec:g_abcd}
Finally we show how to calculate the Coulomb integral due to the interaction between the electrons. It is given by\footnote{Here $G_a$ is combined with $G_b$ and $G_c$ combined with
$G_d$ using the Gaussian product rule. In many books on quantum chemistry this is written as \newline
$\left(G_a(1) G_b(1)|r^{-1}_{12}|G_c(2) G_d(2)\right) =\int\int d\vec r_1 d\vec r_2 G_a(\vec r_1) G_b(\vec r_1) r^{-1}_{12} G_c(\vec r_2) G_d(\vec r_2)$. However, since this
departs from the usual notation of quantum physics, it will not be used in this thesis.}
\begin{equation}
\begin{split}
  g_{acbd} & = \bra{G_a G_c}\frac{1}{r_{12}}\ket{G_b G_d} \\
           & = \int\int \frac{\Omega_{ab}(\vec r_1)\Omega_{cd}(\vec r_2)}{r_{12}} d\vec r_1 d\vec r_2 \\
           & = \sum_{tuv}\sum_{\tau\nu\phi}E^{ab}_{tuv}E^{cd}_{\tau\nu\phi}\int\int\frac{\Lambda_{tuv}(p,\vec r_{1P})\Lambda_{\tau\nu\phi}(q,\vec r_{2Q})}{r_{12}}d \vec r_1 d\vec r_2 \\
           & = \sum_{tuv}\sum_{\tau\nu\phi}E^{ab}_{tuv}E^{cd}_{\tau\nu\phi}\frac{\partial^{t+u+v}}{\partial P_x^t \partial P_y^u \partial P_z^v}
                \frac{\partial^{\tau+\nu+\phi}}{\partial Q_x^\tau \partial Q_y^\nu \partial Q_z^\phi} \\
           &    \hspace{30mm} \int\int\frac{\exp(-pr^2_{1P})\exp(-qr^2_{2Q})}{r_{12}}d \vec r_1 d\vec r_2,
\end{split}
\end{equation}
where, analogous to $p$ and $\vec r_{1P}$, we have defined 
\begin{equation}
 \begin{split}
    q = & c + d\\
  \vec r_{2Q} = & \vec r_2 - \vec Q \\
  \vec Q = & \frac{c\,\vec C + d\,\vec D}{c + d}.
 \end{split}
\end{equation}
Thus we need to evaluate the integral
\begin{equation}
 V_{pq} = \int\int\frac{\exp(-pr^2_{1P})\exp(-qr^2_{2Q})}{r_{12}}d \vec r_1 d\vec r_2.
\end{equation}
By first integrating over $\vec r_1$ and using equation (\ref{eq:V_p}) this can be written as
\begin{equation}
 V_{pq} = \int\Big(\frac{2\pi}{p}\int_0^1\exp(-p\,r^2_{2P}\,u^2)\,du\Big)\exp(-qr^2_{2Q}) d \vec r_2.
\end{equation}
Next we change the order of integration and use the Gaussian product theorem to get
\begin{equation}
\begin{split}
 V_{pq} & = \frac{2\pi}{p}\int_0^1\int\exp(-\frac{pqu^2}{pu^2+q}R^2_{PQ})\exp[-(pu^2+q)r_{2S}^2] d\vec r_2 du \\
        & = \frac{2\pi}{p}\int_0^1\exp(-\frac{pqu^2}{pu^2+q}R^2_{PQ})\Big(\frac{\pi}{pu^2+q}\Big)^{3/2} du,
\end{split}
\end{equation}
where $\vec R_{PQ} = \vec P - \vec Q$ and $\vec r_{2S} = \vec r_2 - \vec S$ for some point $\vec S$. Again, the actual coordinates of 
$\vec S$ are immaterial. If we now make the change of variable
\begin{equation}
 \frac{v^2}{p+q} = \frac{u^2}{pu^2+q},
\end{equation}
we get the result
\begin{equation}
 V_{pq} = \frac{2\pi^{5/2}}{pq\sqrt{p+q}}F_0\Big(\frac{pq}{p+q}R^2_{PQ}\Big).
\end{equation}
From this we get the final answer
\begin{equation}
\begin{split}
 g_{acbd} & = \frac{2\pi^{5/2}}{pq\sqrt{p+q}}\sum_{tuv}\sum_{\tau\nu\phi}(-1)^{\tau+\nu+\phi}E^{ab}_{tuv}E^{cd}_{\tau\nu\phi} \\
          & \hspace{40mm} \frac{\partial^{t+u+v+\tau+\nu+\phi}}{\partial P_x^{t+\tau} \partial P_y^{u+\nu} \partial P_z^{v+\phi}}F_0\Big(\frac{pq}{p+q}R^2_{PQ}\Big) \\
          & = \frac{2\pi^{5/2}}{pq\sqrt{p+q}}\sum_{tuv}\sum_{\tau\nu\phi}(-1)^{\tau+\nu+\phi}E^{ab}_{tuv}E^{cd}_{\tau\nu\phi}R_{t+\tau,u+\nu,v+\phi}(\alpha,\vec R_{PQ}),
\end{split}
\end{equation}
where $\alpha = pq/(p+q)$. The term $(-)^{\tau+\nu+\phi}$ arises due to the fact that
\begin{equation}
\frac{\partial}{\partial Q_x} F_0\Big(\frac{pq}{p+q}R^2_{PQ}\Big) = - \frac{\partial}{\partial P_x} F_0\Big(\frac{pq}{p+q}R^2_{PQ}\Big).
\end{equation}




\section{The Boys function}
\label{section:Boys}

As shown in the previous section, calculating the Coulomb integrals boils down to evaluating the Boys function
\begin{equation}
\label{eq:boys}
 F_n(x) = \int_0^1\exp(-xt^2)\,t^{2n}\,dt.
\end{equation}
Doing this by standard numerical procedures is compuationally
expensive and should therefore be avoided. This section describes one possible way to calculate the Boys function efficiently.

First note that if $x$ is very large, the function value will hardly be affected by changing the upper limit of the integral from $1$ to $\infty$. Doing this is beneficial
because then the integral can be calculated exactly. Thus, we have the following approximation for the Boys function for large $x$:
\begin{equation}
 F_n(x) \approx \frac{(2n-1)!!}{2^{n+1}}\sqrt{\frac{\pi}{x^{2n+1}}}. \hspace{15mm} (x\hspace{2mm}\mathrm{large})
\end{equation}
For small values of $x$ there seems to be no escape from numerical calculation. However, instead of doing the integral at the time of computation, 
it can be tabulated once and for all at regular values of $x$. For values between the tabulated ones, the function can be calculated by a Taylor expansion centered at the nearest tabulated point $x_t$:
\begin{equation}
 F_n(x_t+\Delta x) = \sum_{k=0}^\infty\frac{F_{n+k}(x_t) (-\Delta x)^k}{k!}. \hspace{15mm} (x\hspace{2mm}\mathrm{small})
\end{equation}
Computational cost can be reduced even further by calculating the Boys function according to the description above only for the highest values of $n$ needed; for lower values of $n$ the function
can be found via the recursion relation
\begin{equation}
 F_n(x) = \frac{2xF_{n+1}(x)+e^{-x}}{2n+1},
\end{equation}
which can be shown by integrating the function by parts.


\section{Summary of the integration scheme}
In the previous sections the integration scheme for GTOs has been derived. We summarise the results in this section. Some of the results are only elaborated fully for the $x$-component as
the others components are defined similarly.

\subsection*{Gaussian functions}
The Gaussian functions are given by
\begin{equation}
 \begin{split}
  G_a(\vec r) & = G_{ikm}(a, \vec r_A) = x^i_A\,y^k_A\,z^m_A\exp(-a r^2_A), \\
  G_b(\vec r) & = G_{jln}(b, \vec r_B) = x^j_B\,y^l_B\,z^n_B\exp(-b r^2_B),
 \end{split}
\end{equation}
where $\vec r_A = \vec r - \vec A$ and $\vec r_B = \vec r - \vec B$. We further define
\begin{equation}
\begin{split}
  p & = a + b, \\
 \vec P & = \frac{a\vec A + b\vec B}{a + b}.
\end{split}
\end{equation}


\subsection*{Overlap integral $S_{ab}$}
The overlap integral
\begin{equation}
 S_{ab} = \langle G_a|G_b\rangle
\end{equation}
is calculated as
\begin{equation}
\label{eq:Sab_compute}
 S_{ab} = E^{ij}_0\,E^{kl}_0\,E^{mn}_0\,\Big(\frac{\pi}{p}\Big)^{3/2},
\end{equation}
where
\begin{equation}
 E^{i=0,j=0}_0 = \exp(-\frac{ab}{a+b}X_{AB}^2),
\end{equation}
and the desired coefficients are found via
\begin{equation}
\begin{split}
 E^{i+1,j}_t & = \frac{1}{2p}E^{ij}_{t-1} + X_{PA}E^{ij}_t + (t+1)E^{ij}_{t+1}, \\
 E^{i,j+1}_t & = \frac{1}{2p}E^{ij}_{t-1} + X_{PB}E^{ij}_t + (t+1)E^{ij}_{t+1}.
\end{split}
\end{equation}



\subsection*{Kinetic integral $T_{ab}$}
The kinetic integral is calculated as
\begin{equation}
\label{eq:Tab_compute}
 T_{ab} = -\frac{1}{2}(T_{ij}\,S_{kl}\,S_{mn} + S_{ij}\,T_{kl}\,S_{mn} + S_{ij}\,S_{kl}\,T_{mn}),
\end{equation}
where
\begin{equation}
 T_{ij} = 4b^2\,S_{i,j+2} - 2b(2j + 1)S_{i,j} + j(j-1)S_{i,j-2}.
\end{equation}

\subsection*{Coulomb integral $V_{ab}$}
The Coulomb integral
\begin{equation}
 V_{ab} = \bra{G_a}\frac{1}{r_C}\ket{G_b}
\end{equation}
is calculated as
\begin{equation}
\label{eq:Vab_compute}
 V_{ab} = \frac{2\pi}{p}\sum_{tuv}E^{ab}_{tuv} R_{tuv}(p,\vec R_{PC}),
\end{equation}
where
\begin{equation}
 E^{ab}_{tuv} = E^{ij}_t\,E^{kl}_u\,E^{mn}_v,
\end{equation}
and $R_{tuv}(a,\vec A)$ is found by first calculating the source term
\begin{equation}
 R^n_{000}(a,\vec A) = (-2a)^n\,F_n(a A^2)
\end{equation}
and then iterating towards the target $R^0_{tuv}(a,\vec A) = R_{tuv}(a,\vec A)$ via the recurrence relations
\begin{equation}
 \begin{split}
  R^n_{t+1,u,v} & = tR^{n+1}_{t-1,u,v} + A_x R^{n+1}_{tuv}, \\
  R^n_{t,u+1,v} & = uR^{n+1}_{t,u-1,v} + A_y R^{n+1}_{tuv}, \\
  R^n_{t,u,v+1} & = vR^{n+1}_{t,u,v-1} + A_z R^{n+1}_{tuv}.
 \end{split}
\end{equation}


\subsection*{Coulomb integral $g_{acbd}$}
The Coulomb integral
\begin{equation}
 g_{acbd} = \bra{G_a G_c}\frac{1}{r_{12}}\ket{G_b G_d}
\end{equation}
is calculated as
\begin{equation}
\label{eq:gabcd_compute}
 g_{acbd} = \frac{2\pi^{5/2}}{pq\sqrt{p+q}}\sum_{tuv}\sum_{\tau\nu\phi}(-1)^{\tau+\nu+\phi}E^{ab}_{tuv}E^{cd}_{\tau\nu\phi}R_{t+\tau,u+\nu,v+\phi}(\alpha,\vec R_{PQ}),
\end{equation}
where
\begin{equation}
  \alpha = \frac{pq}{p + q}.
\end{equation}



 \clearemptydoublepage

\chapter{Practical Many-body Perturbation Theory} \label{chap:mbpt}


\abstract{In physics we often encounter the problem of determining the root of a function $f(x)$. 
Especially, we may need to solve non-linear equations of one variable. 
Such equations are usually divided into two classes, algebraic equations involving
roots of polynomials and transcendental equations.
When there is only one independent variable,
the problem is one-dimensional, namely to find the root or roots of a function.
Except in linear problems, root finding invariably proceeds by iteration, and
this is equally true in one or in many dimensions. 
This means that we cannot solve exactly the equations at hand. Rather, we 
start with some approximate
trial solution. The chosen algorithm will in turn improve the solution until some predetermined
convergence criterion is satisfied. The algoritms we discuss below attempt to implement
this strategy. We will deal mainly with 
one-dimensional problems.}

\section{Formal perturbation theory}
The basic starting point of perturbation theory is to divide the total Hamiltonian $H$ into two parts: One part, $H_0$, of which we are able to find the eigenstates and eigenvalues and
the remaining part, $V$, which is called the perturbation:
\begin{equation}
 H = H_0 + V.
\end{equation}
%The Hamiltonian can be split in many different ways, each of which will lead to different perturbation schemes.
%In this section we will keep the discussion general and not specify $H_0$. Later we will set $H_0$ equal to the Fock operator $\mathcal{F}$.
When $V=0$, the known solutions to the Schrödinger equation are given by
\begin{equation}
 H_0\ket{\Psi^{(0)}_i} = E^{(0)}_i\ket{\Psi^{(0)}_i},
\end{equation}
where the superindices indicate that the solutions are of zeroth order, that is, with complete disregard of $V$. Most often we will be interested in the ground state. Of course, $\ket{\Psi^{(0)}_0}$ is not the actual
ground state, but merely an approximation. To obtain the exact ground state $\ket{\Phi_0}$, an (unknown) correction term $\ket{\gamma}$ must be added:
\begin{equation}
 \ket{\Phi_0} = \ket{\Psi^{(0)}_0} + \ket{\gamma}.
\end{equation}
The same is true for the energy:
\begin{equation}
\label{eq:def_deltaE}
 \mathscr E_0 = E^{(0)}_0 + \Delta E.
\end{equation}
In perturbation theory, the goal is to estimate the corrections $\ket{\gamma}$ and $\Delta E$ order by order in terms of the perturbation $V$:
\begin{align}
 \ket{\gamma} & = \ket{\Psi^{(1)}_0} + \ket{\Psi^{(2)}_0} + \dots \\
 \Delta E  & = E^{(1)}_0 + E^{(2)}_0 + \dots
\end{align}
where, as before, the superindices indicate the order of the perturbation $V$.
The hope is that most of the physics is captured by the zeroth order Hamiltonian $H_0$ so that the two series above converge as fast as possible.

It will be assumed that $\langle\Psi^{(0)}_0|\gamma\rangle = 0$, i.e., that there is no overlap between the unperturbed solution and the correction. Furthermore, we will let the unperturbed
solution be normalised, which means that the overlap between the unperturbed and exact solution is equal to unity:
\begin{equation}
\label{eq:intermediate_normalisation}
 \langle\Psi^{(0)}_0|\Phi_0\rangle = \bra{\Psi^{(0)}_0}\big(\ket{\Psi^{(0)}_0} + \ket{\gamma}\big) = 1 + 0 = 1.
\end{equation}
This is often referred to as intermediate normalisation.

The general expression for the energy correction $\Delta E$ can be derived from the Schrödinger equation as follows:
\begin{equation}
\begin{split}
 (H_0 + V)\ket{\Phi_0} & = \mathscr E_0\ket{\Phi_0}, \\
 \bra{\Psi^{(0)}_0}(H_0 + V)\ket{\Phi_0} & = \mathscr E_0\langle\Psi^{(0)}_0|\Phi_0\rangle, \\
 \bra{\Psi^{(0)}_0}H_0\ket{\Phi_0} + \bra{\Psi^{(0)}_0}V\ket{\Phi_0} & = \mathscr E_0, \\
 \langle H_0\Psi^{(0)}_0|\Phi_0\rangle + \bra{\Psi^{(0)}_0}V\ket{\Phi_0} & = \mathscr E_0, \\
 E^{(0)}_0 + \bra{\Psi^{(0)}_0}V\ket{\Phi_0} & = \mathscr E_0,
\end{split}
\end{equation}
so that
\begin{equation}
\label{eq:PT_DeltaE}
 \Delta E = \bra{\Psi^{(0)}_0}V\ket{\Phi_0}.
\end{equation}
Here we have used equations (\ref{eq:def_deltaE}) and (\ref{eq:intermediate_normalisation}) and the fact that $H_0$ is Hermitian.
Once we have an order by order expansion of $\ket{\Phi_0}$, this will give us an order by order expansion also of $\Delta E$.

To make further progress, we define the projection operators $P$ and $Q$:
\begin{align}
 P & = \ket{\Psi^{(0)}_0}\bra{\Psi^{(0)}_0}, \\
 Q & = \sum_{i=1}^\infty \ket{\Psi^{(0)}_i}\bra{\Psi^{(0)}_i}.
\end{align}
When acting on the state $\ket{\Phi_0}$, $P$ picks out the part which is parallel with $\ket{\Psi^{(0)}_0}$.
This is easily shown by expressing $\ket{\Phi_0}$ in the basis $\left\{\ket{\Psi^{(0)}_i}\right\}_{i=0}^\infty$:
\begin{equation}
 P\ket{\Phi_0} = \ket{\Psi^{(0)}_0}\bra{\Psi^{(0)}_0} \sum_{j=0}^\infty C_j\ket{\Psi^{(0)}_j} = C_0\ket{\Psi^{(0)}_0}.
\end{equation}
Similarly, $Q$ picks out the part which is orthogonal to $\ket{\Psi^{(0)}_0}$ since $PQ = QP = 0$. Note also that $P^2 = P$, $Q^2 = Q$ and $P+Q = I$.
Furthermore, $P$ commutes with $H_0$:
\begin{equation}
\begin{split}
 H_0 P \ket{\Phi_0} & = H_0 P \sum_{i=0}^\infty C_i \ket{\Psi^{(0)}_i} = H_0 C_0 \ket{\Psi^{(0)}_0} =  C_0 E^{(0)}_0 \ket{\Psi^{(0)}_0}, \\
 P H_0 \ket{\Phi_0} & = P H_0 \sum_{i=0}^\infty C_i \ket{\Psi^{(0)}_i} = P \sum_{i=0}^\infty C_i E^{(0)}_i \ket{\Psi^{(0)}_i} = C_0 E^{(0)}_0 \ket{\Psi^{(0)}_0}.
\end{split}
\end{equation}
Also, since $Q = I - P$, $Q$ commutes with $H_0$ as well.

We now have the ingredients necessary to find a perturbative expansion of $\ket{\Phi_0}$. The starting point is a slight rewrite of the Schrödinger equation:
\begin{equation}
 (\zeta - H_0) \ket{\Phi_0} = (V - \mathscr E_0 + \zeta)\ket{\Phi_0},
\end{equation}
where $\zeta$ is a hitherto unspecified parameter. Different choices of $\zeta$ will lead to different perturbation schemes. Two very common choices are
$\zeta = \mathscr E_0$ and $\zeta = E^{(0)}_0$ which lead to Brillouin-Wigner and Rayleigh-Schrödinger perturbation theory, respectively \cite{Bartlett}.
Acting from the left with the operator $Q$ on both sides, and using the fact that $Q^2 = Q$ and $[Q,H_0]=0$ leads to
\begin{equation}
\label{eq:Q(z-H0)QPhi}
 Q(\zeta - H_0)Q \ket{\Phi_0} = Q(V - \mathscr E_0 + \zeta)\ket{\Phi_0}.
\end{equation}
We now need an expression for the inverse of $Q(\zeta - H_0)Q$. As long as $\zeta$ is not equal to any of the eigenvalues $\{E^{(0)}_i\}_{i=1}^\infty$, this
exists and is equal to
\begin{equation}
 R_0(\zeta) = \frac{Q}{\zeta - H_0} = \sum_{i=1}^\infty\sum_{j=1}^\infty\ket{\Psi^{(0)}_i}\bra{\Psi^{(0)}_i}(\zeta - H_0)^{-1}\ket{\Psi^{(0)}_j}\bra{\Psi^{(0)}_j},
\end{equation}
which is called the resolvent of $H_0$. Writing $R_0(\zeta)$ as the fraction above is justified by the fact that $Q$ commutes with $H_0$.

Acting with $R_0(\zeta)$ from the left on both sides of equation (\ref{eq:Q(z-H0)QPhi}) gives
\begin{equation}
 Q\ket{\Psi_0} = R_0(\zeta)(V - \mathscr E_0 + \zeta)\ket{\Phi_0},
\end{equation}
and using the fact that $(P+Q)\ket{\Phi_0} = \ket{\Phi_0}$ leads to
\begin{equation}
 \ket{\Phi_0} = \ket{\Psi^{(0)}_0} + R_0(\zeta)(V - \mathscr E_0 + \zeta)\ket{\Phi_0}.
\end{equation}
Substituting the expression into itself gives
\begin{equation}
\begin{split}
 \ket{\Phi_0} = &             \ket{\Psi^{(0)}_0} + R_0(\zeta)(V - \mathscr E_0 + \zeta)\ket{\Psi^{(0)}_0} \\
	      & +  [R_0(\zeta)(V - \mathscr E_0 + \zeta)]^2\ket{\Phi_0},
 \end{split}
\end{equation}
and repeating this process yields the expression we are seeking:
\begin{equation}
 \ket{\Phi_0} = \sum_{n=0}^\infty [R_0(\zeta)(V - \mathscr E_0 + \zeta)]^n \ket{\Psi^{(0)}_0}.
\end{equation}
The corresponding expansion for the energy is found by inserting this into equation (\ref{eq:PT_DeltaE}):
\begin{equation}
 \Delta E = \sum_{n=0}^\infty \bra{\Psi^{(0)}_0}V[R_0(\zeta)(V - \mathscr E_0 + \zeta)]^n \ket{\Psi^{(0)}_0}.
\end{equation}


\section{Rayleigh-Schrödinger perturbation theory}
Rayleigh-Schrödinger perturbation theory, named after Lord Rayleigh \cite{rayleigh} and Erwin Scrhrödinger \cite{schrodinger},
is obtained by setting $\zeta = E^{(0)}_0$, which gives
\begin{equation}
 \Delta E = \sum_{n=0}^\infty \bra{\Psi^{(0)}_0}V[R_0(E^{(0)}_0)(V - \Delta E)]^n \ket{\Psi^{(0)}_0}.
\end{equation}
Writing out the first three terms explicitly:
\begin{equation}
 \begin{split}
  \Delta E = & \bra{\Psi^{(0)}_0}V\ket{\Psi^{(0)}_0} \\
             & + \bra{\Psi^{(0)}_0}V R_0(V - \Delta E) \ket{\Psi^{(0)}_0} \\
             & + \bra{\Psi^{(0)}_0}V R_0(V - \Delta E)R_0(V - \Delta E) \ket{\Psi^{(0)}_0} + \dots
 \end{split}
\end{equation}
where the dependence of $R_0$ on $E^{(0)}_0$ has been supressed.
This is still not a proper perturbative expansion since $\Delta E$ is present also on the right hand side of the equation. We get the correct expansion
by inserting the expression itself into every occurence of $\Delta E$ on the right hand side. Doing this, and noting that $R_0\ket{\Psi^{(0)}_0} = 0$, gives
\begin{equation}
 \begin{split}
  \Delta E = & \bra{\Psi^{(0)}_0}V\ket{\Psi^{(0)}_0} \\
             & + \bra{\Psi^{(0)}_0}V R_0 V\ket{\Psi^{(0)}_0} \\
             & + \bra{\Psi^{(0)}_0}V R_0(V - \bra{\Psi^{(0)}_0}V\ket{\Psi^{(0)}_0})R_0 V\ket{\Psi^{(0)}_0} + \cdots
 \end{split}
\end{equation}
The three lines are the first, second and third order corrections, respectively, to the zero order energy.



% These expressions are still quite general, and in order to obtain programmable expressions, the parameter $\zeta$ and the zero order Hamiltonian $H_0$ needs to be
% specified. In this thesis we will be using the Rayleigh-Schrödinger perturbation theory which is obtained by setting $\zeta = E^{(0)}_0$. Furthermore we will
% set $H_0$ equal to the Fock operator $\mathcal{F}$. The resulting perturbation scheme is often referred to as Møller-Plesset perturbation theory.


\section{Møller-Plesset perturbation theory}
To make further progress, it is necessary to specify the zero order Hamiltonian $H_0$. Setting it equal to the Hartree-Fock Hamiltonian
\begin{equation}
 H_0 = \sum_{pq}\bra{p}\mathcal{F}\ket{q}a^\dagger_p a_q,
\end{equation}
yields the so-called Møller-Plesset perturbation theory, after Møller and Plesset \cite{MP}.
If we choose the eigenfunctions of the Fock operator as our single-particle basis,
then $H_0$ simplifies to
\begin{equation}
 H_0 = \sum_{pq}\bra{p}\varepsilon_q\ket{q}a^\dagger_p a_q = \sum_{pq}\varepsilon_q\delta_{pq} a^\dagger_p a_q = \sum_p \varepsilon_p a^\dagger_p a_p.
\end{equation}
Furthermore, the unperturbed state $\ket{\Psi_0^{(0)}}$ is then equal to the Hartree-Fock
determinant $\ket{\Psi_0}$. It is per definition an eigenstate of the
unperturbed Hamiltonian
\begin{equation}
 H_0 \ket{\Psi_0} = \sum_p \varepsilon_p a^\dagger_p a_p \ket{123\dots N} = \sum_{i=1}^N \varepsilon_i \ket{\Psi_0},
\end{equation}
Thus the zero order energy is the sum of the Hartree-Fock orbital energies
\begin{equation}
\label{eq:pert_zero_order}
 E^{(0)}_0 = \sum_{i=1}^N \varepsilon_i.
\end{equation}

Next we show how this choice of $H_0$ determines the perturbation $V$. The Hamiltonian can be written as
\begin{equation}
 H = H_0 + (H_1 - H_0 + H_2),
\end{equation}
where $H_1$ and $H_2$ are the one- and two-body terms defined in equations (\ref{eq:H1_2Q}) and (\ref{eq:H2_2Q_AS}), respectively.
This means that 
\begin{equation}
 \begin{split}
  V & = H_1 - H_0 + H_2 \\
    & = \sum_{pq}[\bra{p}h\ket{q} - \bra{p}\mathcal{F}\ket{q}]a^\dagger_p a_q + \frac{1}{4}\sum_{pqrs}\langle pq||rs\rangle a^\dagger_p a^\dagger_q a_s a_r \\
    & = -\sum_{pq}\bra{p}(\mathcal J - \mathcal K)\ket{q}a^\dagger_p a_q + \frac{1}{4}\sum_{pqrs}\langle pq||rs\rangle a^\dagger_p a^\dagger_q a_s a_r,
 \end{split}
\end{equation}
where $\mathcal J$ and $\mathcal K$ are defined in equations (\ref{eq:J_operator}) and (\ref{eq:K_operator}), respectively.
This form of $V$ seems to suggest that it has a one-body as well as a two-body part. However, normal ordering the operator with respect to $\ket{\Psi^{(0)}_0}$ as the Fermi vacuum
reveals that the one-body part of $V$ cancels out. Doing exactly the same calculations as in section \ref{sec:Normal_ordered_operators} leads to
\begin{equation}
 \begin{split}
  V  = & -\sum_{pq}\bra{p}(\mathcal J - \mathcal K)\ket{q}\{a^\dagger_p a_q\} - \sum_i\bra{i}(\mathcal J - \mathcal K)\ket{i} \\
       & + \frac{1}{4}\sum_{pqrs}\langle pq||rs\rangle \{a^\dagger_p a^\dagger_q a_s a_r\} + \sum_{pq}\sum_i \langle pi||qi\rangle \{a^\dagger_p a_q\} \\
       & + \frac{1}{2}\sum_{ij}\langle ij||ij\rangle
 \end{split}
\end{equation}
Noting that $\bra{p}(J - K)\ket{q} = \langle pi||qi\rangle$, the one-body parts now cancel, and we arrive at
\begin{equation}
\label{eq:V}
 V = \frac{1}{4}\sum_{pqrs}\langle pq||rs\rangle \{a^\dagger_p a^\dagger_q a_s a_r\} - \frac{1}{2}\sum_{ij}\langle ij||ij\rangle.
\end{equation}
This form of the perturbation makes it easier to evaluate the energy corrections.

The first order correction is:
\begin{equation}
\label{eq:pert_1st_order}
 E^{(1)}_0 = \bra{\Psi_0}V\ket{\Psi_0} = - \frac{1}{2}\sum_{ij}\langle ij||ij\rangle,
\end{equation}
since the expectation of the normal product is equal to zero. Comparing equations (\ref{eq:pert_zero_order}) and (\ref{eq:pert_1st_order}) with equation
(\ref{eq:E_ref2}) shows that the first order correction is included in the Hartree-Fock energy
\begin{equation}
 E_0 = E^{(0)}_0 + E^{(1)}_0.
\end{equation}
To clearify, $E_0$ is the Hartree-Fock energy (reference energy), and $E_0^{(i)}$ is the $i'$th order correction to the exact energy $\mathscr E_0$.

\section{Second order perturbation theory (MP2)}
The expression for the second order correction is somewhat more complicated. It is given by
\begin{equation}
 E^{(2)}_0 = \bra{\Psi^{(0)}_0}V R_0 V\ket{\Psi^{(0)}_0}.
\end{equation}
Because of the resolvent $R_0$, only the normal product, $W$, of equation (\ref{eq:V}) survives, so that we get
\begin{equation}
 E^{(2)}_0 = \bra{\Psi^{(0)}_0}W R_0 W\ket{\Psi^{(0)}_0}.
\end{equation}
We have already calculated an almost identical expression in section \ref{sec:Diagrammatic_notation}. The only difference is that the resolvent
\begin{equation}
 R_0 = \frac{Q}{E_0 - H_0} = \sum_{ia}^\infty \frac{\ket{\Psi^a_i}\bra{\Psi^a_i}}{\varepsilon_i - \varepsilon_a}
                             + \sum_{i<j,a<b}\frac{\ket{\Psi^{ab}_{ij}}\bra{\Psi^{ab}_{ij}}}{\varepsilon_i + \varepsilon_j - \varepsilon_a - \varepsilon_b} + \dots
\end{equation}
is squeezed between the two $W$ operators. This simply has the effect that each term in equation (\ref{eq:W2}) is divided by an energy denominator
\begin{equation}
 \frac{1}{\varepsilon_i + \varepsilon_j - \varepsilon_a - \varepsilon_b}
\end{equation}
which yields the result
\begin{equation}
\label{eq:MP2}
 E^{(2)}_0 = \frac{1}{4}\sum_{ijab}\frac{|\langle ij||ab\rangle|^2}{\varepsilon_i+\varepsilon_j-\varepsilon_a-\varepsilon_b}.
\end{equation}
This can be read off directly from the diagram of figure \ref{fig:MP2} according to the rules of section \ref{sec:Diagrammatic_notation} and
the following additional rule. Draw a horizontal line between each operator in the diagram.
Each such line, called resolvent line, will contribute with a factor in the energy denominator. The factor is equal to the sum of all hole lines passing through minus the sum
of all particle lines passing through.
\begin{figure}
 \begin{center}
  \includegraphics[scale=1.0]{perturbation_theory/figures/MP2.pdf}
  \caption{Diagrammatic representation of the second order correction to the energy of Møller-Plesset perturbation theory.}
  \label{fig:MP2}
 \end{center}
\end{figure}


% The resolvent $R_0$ is given by
% \begin{equation}
%  R_0 = \frac{Q}{E_0 - H_0} = \sum_{i=1}^\infty \frac{\ket{\Psi^{(0)}_i}\bra{\Psi^{(0)}_i}}{E_0 - E^{(0)}_i},
% \end{equation}
% where the states $\{\ket{\Psi^{(0)}_i}\}_{i=1}^\infty$ include all eigenstates of the zero order Hamiltonian $H_0$ except the ground state $\ket{\Psi^{(0)}_0}$.
% In terms of particle- and hole-states, the resolvent can be written as
% \begin{equation}
%  R_0 = \sum_{ia}\frac{\ket{\Psi^a_i}\bra{\Psi^a_i}}{\varepsilon_i - \varepsilon_a}
%           + \sum_{i<j, a<b}\frac{\ket{\Psi^{ab}_{ij}}\bra{\Psi^{ab}_{ij}}}{\varepsilon_i + \varepsilon_j - \varepsilon_a - \varepsilon_b} + \cdots
% \end{equation}
% We will use Wick's generalised theorem to calculate the second order correction. From the theorem it follows that only the
% two-particle-two-hole states will give non-zero terms. This means that
% \begin{equation}
%  \begin{split}
%   E^{(2)}_0 = & \frac{1}{16}\sum_{pqrs}\sum_{tuvw}\sum_{i<j,a<b}\frac{\langle pq||rs\rangle\langle tu||vw\rangle}{\varepsilon_i + \varepsilon_j - \varepsilon_a - \varepsilon_b} 
%                     \bra{\Psi^{(0)}_0}\{a^\dagger_p a^\dagger_q a_s a_r\} \{a^\dagger_a a^\dagger_b a_j a_i\} \ket{{\Psi^{(0)}_0}} \\
%                    & \bra{{\Psi^{(0)}_0}} \{a^\dagger_i a^\dagger_j a_b a_a\} \{a^\dagger_t a^\dagger_u a_w a_v\}\ket{{\Psi^{(0)}_0}}.
%  \end{split}
% \end{equation}
% The only non-zero contractions are those between creation and annihilation operators. Thus
% \newpage
% \begin{displaymath}
%  \contraction[4ex]{E^{(2)}_0 =  \frac{1}{16}\sum_{pqrs}\sum_{tuvw}\sum_{i<j,a<b}\frac{\langle pq||rs\rangle\langle tu||vw\rangle}{\varepsilon_i + \varepsilon_j - \varepsilon_a - \varepsilon_b}
%               \bra{\Psi^{(0)}_0}\{}{a}{{}^\dagger_p a^\dagger_q a_s a_r\} \{a^\dagger_a a^\dagger_b a_j}{a}
%  \contraction[3ex]{E^{(2)}_0 =  \frac{1}{16}\sum_{pqrs}\sum_{tuvw}\sum_{i<j,a<b}\frac{\langle pq||rs\rangle\langle tu||vw\rangle}{\varepsilon_i + \varepsilon_j - \varepsilon_a - \varepsilon_b}
%                    \bra{\Psi^{(0)}_0}\{a^\dagger_p}{a}{{}^\dagger_q a_s a_r\} \{a^\dagger_a a^\dagger_b}{a}
%  \contraction[2ex]{E^{(2)}_0 =  \frac{1}{16}\sum_{pqrs}\sum_{tuvw}\sum_{i<j,a<b}\frac{\langle pq||rs\rangle\langle tu||vw\rangle}{\varepsilon_i + \varepsilon_j - \varepsilon_a - \varepsilon_b}
%                    \bra{\Psi^{(0)}_0}\{a^\dagger_p a^\dagger_q}{a}{{}_s a_r\} \{a^\dagger_a}{a}
%  \contraction{E^{(2)}_0 =  \frac{1}{16}\sum_{pqrs}\sum_{tuvw}\sum_{i<j,a<b}\frac{\langle pq||rs\rangle\langle tu||vw\rangle}{\varepsilon_i + \varepsilon_j - \varepsilon_a - \varepsilon_b}
%                    \bra{\Psi^{(0)}_0}\{a^\dagger_p a^\dagger_q a_s}{a}{{}_r\} \{}{a}
%  E^{(2)}_0 =  \frac{1}{16}\sum_{pqrs}\sum_{tuvw}\sum_{i<j,a<b}\frac{\langle pq||rs\rangle\langle tu||vw\rangle}{\varepsilon_i + \varepsilon_j - \varepsilon_a - \varepsilon_b}
%                    \bra{\Psi^{(0)}_0}\{a^\dagger_p a^\dagger_q a_s a_r\} \{a^\dagger_a a^\dagger_b a_j a_i\} \ket{{\Psi^{(0)}_0}}
% \end{displaymath}
% \begin{equation}
%  \contraction[4ex]{\bra{{\Psi^{(0)}_0}} \{}{a}{{}^\dagger_i a^\dagger_j a_b a_a\} \{a^\dagger_t a^\dagger_u a_w}{a}
%  \contraction[3ex]{\bra{{\Psi^{(0)}_0}} \{a^\dagger_i}{a}{{}^\dagger_j a_b a_a\} \{a^\dagger_t a^\dagger_u}{a}
%  \contraction[2ex]{\bra{{\Psi^{(0)}_0}} \{a^\dagger_i a^\dagger_j}{a}{{}_b a_a\} \{a^\dagger_t}{a}
%  \contraction{\bra{{\Psi^{(0)}_0}} \{a^\dagger_i a^\dagger_j a_b}{a}{{}_a\} \{}{a}
%  \bra{{\Psi^{(0)}_0}} \{a^\dagger_i a^\dagger_j a_b a_a\} \{a^\dagger_t a^\dagger_u a_w a_v\}\ket{{\Psi^{(0)}_0}} + \text{15 more terms}
% \end{equation}
% The 15 other terms are due to the fact that there are four possible ways of contracting the first as well as the second vacuum expectation, resulting
% in a total of 16 terms. Each change of order in the contractions introduces a minus sign which in turn can be compensated for by using the antisymmetry of
% the matrix elements. This means that all the terms are actually equal, and we get
% \begin{equation}
% \label{eq:pert_2nd_order}
%  E^{(2)}_0 = \frac{1}{4}\sum_{i,j}\sum_{a,b}\frac{|\langle ij||ab\rangle|^2}{\varepsilon^{ab}_{ij}},
% \end{equation}
% where $\varepsilon^{ab}_{ij} = \varepsilon_i + \varepsilon_j - \varepsilon_a - \varepsilon_b$.
% Note that a factor of $1/4$ has been introduced by removing the restrictions $i<j$ and $a<b$. The sums over the hole states range from 0 to the number of particles $N$,
% and the sums over the particle states range from $N+1$ to the number of basis functions $M$.
% 
% Clearly, the calculations become more and more laborious as we increase the order of correction. In practice, diagrammatic notation is used to evaluate higher order
% terms. It will not be introduced here, and the interested reader is referred to Shavitt and Bartlett \cite{Bartlett}. In any case, only second order Møller-Plesset
% perturbation theory (MP2) will be considered in this thesis.


\subsection{MP2 in the RHF-case}
In the case of RHF, the spin orbitals are assumed to be on the from
\begin{equation}
 \{\psi_{2k}(\vec x),\, \psi_{2k+1}(\vec x)\} = \{\phi_k(\vec r)\alpha(s),\, \phi_k(\vec r)\beta(s)\}.
\end{equation}
This means that the sum over each spin orbital in equation (\ref{eq:MP2}) can be
replaced by two sums. The sum over $i$, for example, is
\begin{equation}
 \sum_{i=1}^N\psi_i = \sum_{i=1}^{N/2}\phi_i\alpha + \sum_{i=1}^{N/2}\phi_i\beta,
\end{equation}
and we get similar sums for the other indices. Inserting this yields
\begin{equation}
\begin{split}
 E^{(2)}_0 = & \frac{1}{4}\sum_{i,j=1}^{N/2}\sum_{a,b=N/2+1}^{M}\frac{1}{\varepsilon^{ab}_{ij}}\Big[|\langle (\phi_i\alpha) (\phi_j\alpha)||(\phi_a\alpha)(\phi_b\alpha)\rangle|^2 \\
                  & + |\langle (\phi_i\beta) (\phi_j\beta)||(\phi_a\beta)(\phi_b\beta)\rangle|^2 + |\langle (\phi_i\alpha) (\phi_j\beta)||(\phi_a\alpha)(\phi_b\beta)\rangle|^2 \\
                  & + |\langle (\phi_i\beta) (\phi_j\alpha)||(\phi_a\beta)(\phi_b\alpha)\rangle|^2 + |\langle (\phi_i\alpha) (\phi_j\beta)||(\phi_a\beta)(\phi_b\alpha)\rangle|^2 \\
                  & + |\langle (\phi_i\beta) (\phi_j\alpha)||(\phi_a\alpha)(\phi_b\beta)\rangle|^2\Big],
\end{split}
\end{equation}
where $M$ is the number of spatial orbitals (equal to the number of spatial basis functions) and
\begin{equation}
 \varepsilon^{ab}_{ij} = \varepsilon_i + \varepsilon_j - \varepsilon_a - \varepsilon_b.
\end{equation}
Terms which are automatically zero due to spin have not been included. Integrating out spin now yields
\begin{equation}
 \begin{split}
  E^{(2)}_0 = & \frac{1}{4}\sum_{i,j=1}^{N/2}\sum_{a,b=N/2+1}^{M}\frac{1}{\varepsilon^{ab}_{ij}}\Big[|\langle \phi_i\phi_j||\phi_a\phi_b\rangle|^2 \\
                   & +|\langle \phi_i\phi_j||\phi_a\phi_b\rangle|^2 + |\bra{\phi_i\phi_j}g\ket{\phi_a\phi_b}|^2 \\
                   & + |\bra{\phi_i\phi_j}g\ket{\phi_a\phi_b}|^2 + |\bra{\phi_i\phi_j}g\ket{\phi_b\phi_a}|^2 \\
                   & + |\bra{\phi_i\phi_j}g\ket{\phi_b\phi_a}|^2 \Big].
 \end{split}
\end{equation}
Note that the first two terms in the sum have both the direct term and the exchange term, whereas in the rest of the terms only one of these (direct or exchange) survives.
Collecting equal terms finally gives
\begin{equation}
\label{eq:RMP2}
 E^{(2)}_0 = \sum_{i,j=1}^{N/2}\sum_{a,b=N/2+1}^{M}\frac{\bra{ij}g\ket{ab}\big(2\bra{ab}g\ket{ij} - \bra{ab}g\ket{ji}\big)}{\varepsilon^{ab}_{ij}}.
\end{equation}
Note that in this expression the explicit appearance of $\phi$ has been supressed. To avoid
confusion as to whether the sum is over spin orbitals or spatial orbitals (in this case it
is the latter), we shall use the following convention. If the summation ranges are given
explicitly, the sum is over spatial orbitals. Otherwise, the sum is over spin orbitals.




\subsection{MP2 in the UHF-case}
In the case of UHF, the spinorbitals are assumed to be on the form
\begin{equation}
 \{\psi_{2k}(\vec x),\, \psi_{2k+1}(\vec x)\} = \{\phi^\alpha_{k}(\vec r)\alpha(s),\,\phi^\beta_k(\vec r)\beta(s)\}.
\end{equation}
Inserting this in equation (\ref{eq:MP2}) and integrating out spin in the same manner as above yields
\begin{equation}
 \begin{split}
  E^{(2)}_0 = & \frac{1}{4}\sum_{i=1}^{N^\alpha}\sum_{j=1}^{N^\alpha}\sum_{a=N^\alpha+1}^M\sum_{b=N^\alpha+1}^M\frac{|\langle\phi_i^\alpha \phi_j^\alpha||\phi_a^\alpha \phi_b^\alpha\rangle|^2}{\varepsilon^\alpha_i + \varepsilon^\alpha_j - \varepsilon^\alpha_a - \varepsilon^\alpha_b}  \\
              & +\frac{1}{4}\sum_{i=1}^{N^\beta}\sum_{j=1}^{N^\beta}\sum_{a=N^\beta+1}^M\sum_{b=N^\beta+1}^M\frac{|\langle\phi_i^\beta \phi_j^\beta||\phi_a^\beta \phi_b^\beta\rangle|^2}{\varepsilon^\beta_i + \varepsilon^\beta_j - \varepsilon^\beta_a - \varepsilon^\beta_b}  \\
              & +\frac{1}{4}\sum_{i=1}^{N^\alpha}\sum_{j=1}^{N^\beta}\sum_{a=N^\alpha+1}^M\sum_{b=N^\beta+1}^M\frac{|\langle\phi_i^\alpha \phi_j^\beta|g|\phi_a^\alpha \phi_b^\beta\rangle|^2}{\varepsilon^\alpha_i + \varepsilon^\beta_j - \varepsilon^\alpha_a - \varepsilon^\beta_b}  \\
              & +\frac{1}{4}\sum_{i=1}^{N^\beta}\sum_{j=1}^{N^\alpha}\sum_{a=N^\beta+1}^M\sum_{b=N^\alpha+1}^M\frac{|\langle\phi_i^\beta \phi_j^\alpha|g|\phi_a^\beta \phi_b^\alpha\rangle|^2}{\varepsilon^\beta_i + \varepsilon^\alpha_j - \varepsilon^\beta_a - \varepsilon^\alpha_b}  \\
              & +\frac{1}{4}\sum_{i=1}^{N^\alpha}\sum_{j=1}^{N^\beta}\sum_{a=N^\beta+1}^M\sum_{b=N^\alpha+1}^M\frac{|\langle\phi_i^\alpha \phi_j^\beta|g|\phi_b^\alpha \phi_a^\beta\rangle|^2}{\varepsilon^\alpha_i + \varepsilon^\beta_j - \varepsilon^\alpha_b - \varepsilon^\beta_a}  \\
              & +\frac{1}{4}\sum_{i=1}^{N^\beta}\sum_{j=1}^{N^\alpha}\sum_{a=N^\alpha+1}^M\sum_{b=N^\beta+1}^M\frac{|\langle\phi_i^\beta \phi_j^\alpha|g|\phi_b^\beta \phi_a^\alpha\rangle|^2}{\varepsilon^\beta_i + \varepsilon^\alpha_j - \varepsilon^\beta_b - \varepsilon^\alpha_a}.
 \end{split}
\end{equation}
Since the last four terms are equal, this is reduced to
\begin{equation}
\label{eq:UMP2}
 \begin{split}
  E^{(2)}_0 = & \frac{1}{4}\sum_{i=1}^{N^\alpha}\sum_{j=1}^{N^\alpha}\sum_{a=N^\alpha+1}^M\sum_{b=N^\alpha+1}^M\frac{|\langle\phi_i^\alpha \phi_j^\alpha||\phi_a^\alpha \phi_b^\alpha\rangle|^2}{\varepsilon^\alpha_i + \varepsilon^\alpha_j - \varepsilon^\alpha_a - \varepsilon^\alpha_b}  \\
              & +\frac{1}{4}\sum_{i=1}^{N^\beta}\sum_{j=1}^{N^\beta}\sum_{a=N^\beta+1}^M\sum_{b=N^\beta+1}^M\frac{|\langle\phi_i^\beta \phi_j^\beta||\phi_a^\beta \phi_b^\beta\rangle|^2}{\varepsilon^\beta_i + \varepsilon^\beta_j - \varepsilon^\beta_a - \varepsilon^\beta_b}  \\
              & +\sum_{i=1}^{N^\alpha}\sum_{j=1}^{N^\beta}\sum_{a=N^\alpha+1}^M\sum_{b=N^\beta+1}^M\frac{|\langle\phi_i^\alpha \phi_j^\beta|g|\phi_a^\alpha \phi_b^\beta\rangle|^2}{\varepsilon^\alpha_i + \varepsilon^\beta_j - \varepsilon^\alpha_a - \varepsilon^\beta_b}  \\
 \end{split}
\end{equation} 



% \begin{equation}
%  \begin{split}
%   \Delta E^{(2)} = & \frac{1}{4}\sum_{i,j=1}^{N/2}\sum_{a,b=N/2+1}^{M/2}\Big[\frac{|\langle i_+ j_+||a_+ b_+\rangle|^2}{\varepsilon^{a_+b_+}_{i_+j_+}}  \\
%                    & +\frac{|\langle i_- j_-||a_- b_-\rangle|^2}{\varepsilon^{a_-b_-}_{i_-j_-}} + \frac{|\bra{i_+ j_-}g\ket{a_+ b_-}|^2}{\varepsilon^{a_+b_-}_{i_+j_-}}\\
%                    & +\frac{|\bra{i_- j_+}g\ket{a_- b_+}|^2}{\varepsilon^{a_-b_+}_{i_-j_+}} + \frac{|\bra{i_+ j_-}g\ket{b_+ a_-}|^2}{\varepsilon^{a_-b_+}_{i_+j_-}}\\
%                    & +\frac{|\bra{i_- j_+}g\ket{b_-a_+}|^2}{\varepsilon^{a_+b_-}_{i_-j_+}}\Big].
%  \end{split}
% \end{equation}



\section{Third order perturbation theory (MP3)}
The third order correction is given by
\begin{equation}
 E^{(3)}_0 = \bra{\Psi^{(0)}_0}V R_0(V - E^{(1)}_0)R_0 V\ket{\Psi^{(0)}_0},
\end{equation}
 which, when using that $V = W + E^{(1)}$ and $R_0\ket{\Psi^{(0)}_0} = 0$, can be written as
\begin{equation}
 E^{(3)}_0 = \bra{\Psi^{(0)}_0}W R_0 W R_0 W\ket{\Psi^{(0)}_0}.
\end{equation}
We can evaluate this expression in the same manner as the second order correction, the only difference now being that there are two resolvent lines.
The diagrams for this term was set up and discussed in section \ref{sec:Diagrammatic_notation}.
The result is shown in figure \ref{fig:MP3}.

\begin{figure}
 \begin{center}
  \includegraphics[scale=1.0]{perturbation_theory/figures/MP3.pdf}
  \caption{Diagrammatic representation of the third order correction to the energy of Møller-Plesset perturbation theory.}
  \label{fig:MP3}
 \end{center}
\end{figure}

Integrating out the spin part of these equations are a little bit more involved and tedious than for the second order case, and the derivation and
results are therefore shown in appendix \ref{chapter:appendix_perturbation}.



 \clearemptydoublepage
\chapter{Computational aspects}

\abstract{In this chapter
we discuss some of the classical methods for integrating a function. The methods we discuss are  the trapezoidal, rectangular and Simpson's rule for equally spaced 
abscissas  and integration approaches  
based on Gaussian quadrature. The latter are more suitable
for the case where the abscissas are not equally spaced. 
The emphasis is on 
methods for evaluating few-dimensional (typically up to four dimensions) integrals. In
chapter \ref{chap:mcint} 
we show how Monte Carlo methods can be used to compute multi-dimensional
integrals.
We discuss also how to compute 
singular integrals.
We end this chapter with an extensive discussion on MPI and parallel computing.
The examples focus on parallelization of algorithms for computing integrals. }


 \clearemptydoublepage

\chapter{Coupled-Cluster theory at the level of singles and doubles}

\abstract{This chapter introduces several matrix related topics, from the solution of linear equations, computing determinants, conjugate-gradient methods, spline interpolation to efficient handling of matrices}


Coupled Cluster is an important ab initio technique in computational chemistry. It is considered the most reliable and also computational affordable method for solving the electronic Schr\"{o}dinger equation. It was introduced in Quantum Chemistry by Paldus and Cizek in the late 1960s. Then it was derived using Feynman-like diagrams, however about ten years later Hurley re-derived the equations in terms more familiar to most physicists. In this chapter we will look at the derivation of coupled cluster singles and doubles (CCSD), using the results from Hartree Fock. \\

This chapter is almost solely based on a book by T. Daniel Crawford and Henry S. Schaeffer III, Ref.\cite{ccsdbook11}. 

\section{Creation and Annihilation operators}
In Coupled Cluster, CC, we will solve the Schr\"{o}dinger equation.

\begin{equation}
\textbf{H} | \Psi \rangle = E | \Psi \rangle \label{SE} .
\end{equation}
From Hartree Fock calculations we have created a $|\Psi\rangle_{HF}$ which contains MOs in a Slater determinant. Dirac notation provides a simple representation of this. In Dirac notation only the diagonal terms in the slater determinant are listed. If $|\Psi_0 \rangle$ has four electrons Dirac notation would be

\begin{equation}
|\Psi_0 \rangle =  |\psi_i(r_1), \psi_j(r_2), \psi_k(r_3), \psi_l(r_4) \rangle . \label{dirac_not} 
\end{equation}
Eq. \eqref{dirac_not} will be used to introduce a few new operators needed. The creation operator $\textbf{a}^{\dag}_m$ creates a new electron in orbital m.

\begin{equation}
\textbf{a}^{\dag}_m |\psi_i(r_1), \psi_j(r_2), \psi_k(r_3), \psi_l(r_4) \rangle = |\psi_i(r_1), \psi_j(r_2), \psi_k(r_3), \psi_l(r_4), \psi_m(r_5) \rangle .
\end{equation}
The annihilation operator, $\textbf{a}_n$, destroys an electron in orbital n.

\begin{equation}
\textbf{a}_n |\psi_i(r_1), \psi_j(r_2), \psi_k(r_3), \psi_l(r_4), \psi_n(r_5) \rangle = |\psi_i(r_1), \psi_j(r_2), \psi_k(r_3), \psi_l(r_4) \rangle  .
\end{equation}
These two operators working together can destroy one electron in orbital n, and create another in orbital m. The result is that one electron now occupies a different orbital, as such

\begin{equation}
\textbf{a}^{\dag}_m \textbf{a}_n |\psi_i(r_1), \psi_j(r_2), \psi_k(r_3), \psi_n(r_4) \rangle = |\psi_i(r_1), \psi_j(r_2), \psi_k(r_3), \psi_m(r_4) \rangle .
\end{equation}
These operators have a few interesting features. Such as the annihilation operator acting on the vacuum state produces 0.

\begin{equation}
a_n | \rangle = 0  .
\end{equation}
Interchanging two rows in the Slater determinant introduce a change in the sign. Hence we have

\begin{equation}
\textbf{a}^{\dag}_m \textbf{a}^{\dag}_n | \rangle = |\psi_m, \psi_n \rangle = - |\psi_n, \psi_m \rangle = -\textbf{a}^{\dag}_n \textbf{a}^{\dag}_m | \rangle  .
\end{equation}

\begin{equation}
\Rightarrow \textbf{a}^{\dag}_m \textbf{a}^{\dag}_n + \textbf{a}^{\dag}_n \textbf{a}^{\dag}_m = 0 .
\end{equation}
The same applies to the annihilation operator.

\begin{equation}
\textbf{a}_m \textbf{a}_n + \textbf{a}_n \textbf{a}_m = 0 .
\end{equation}
These are known as anti commutation relations. It can be shown that the anti commutation relation when mixing $\textbf{a}$ and $\textbf{a}^{\dag}$ is

\begin{equation}
\textbf{a}^{\dag}_m \textbf{a}_n + \textbf{a}_n \textbf{a}^{\dag}_m = \delta_{mn} . \label{ccsd_anni_creato_operator_combo}
\end{equation}

\section{CCSD Wavefunction}
The first step in coupled cluster is rewriting the wavefunction,

\begin{equation}
|\Psi_{CC} \rangle \equiv e^{\textbf{T}} | \Psi_{HF} \rangle .
\end{equation} 
$\textbf{T}$ is known as the cluster operator. This includes all possible excitations. $|\Psi_{CC} \rangle$ is thus a linear combination of Slater determinants of all possible excitations and is an exact solution to Eq. \eqref{SE}. $\textbf{T}$ can be defined in terms of a one-orbital excitation operator, a two-orbital excitation operator and so on.

\begin{equation}
\textbf{T} \equiv \textbf{T}_1 + \textbf{T}_2 + \textbf{T}_3 + \textbf{T}_4 \dots .
\end{equation}
When doing CCSD only single excitations, $\textbf{T}_1$, and double excitations, $\textbf{T}_2$, are included. 

\begin{equation}
\textbf{T} = \textbf{T}_1 + \textbf{T}_2 .
\end{equation}
Other CC methods include more terms. If $\textbf{T}_3$ is included the method is called CCSDT. CCSDTQ includes also $\textbf{T}_4$. \\

$\textbf{T}_1$ is defined using one creation and one annihilation operator, because we will have one single electron excited. Also defining $\textbf{T}_1$ is an amplitude $t_i^a$ and a summation over all possible excitations.  

\begin{equation}
\textbf{T}_1 \equiv \sum_{a,i} t_i^a \textbf{a}^{\dag}_a \textbf{a}_i . \label{t1defi}
\end{equation}
$\textbf{T}_2$ is defined by two creation and two annihilation operators and an amplitude $t_{ij}^{ab}$.

\begin{equation}
\textbf{T}_2 \equiv \frac{1}{4} \sum_{a,b,i,j} t_{ij}^{ab} \textbf{a}^{\dag}_a \textbf{a}^{\dag}_b \textbf{a}_i \textbf{a}_j . \label{t2defi}
\end{equation}

\section{Derivation of Equations}
This section contains the formal derivation of coupled cluster theory, starting from Eq. \eqref{SE} and using the CCSD wavefunction. 

\begin{equation}
\textbf{H} e^{\textbf{T}} |\Psi \rangle_{HF} = E e^{\textbf{T}} |\Psi \rangle_{HF} .
\end{equation}
For this derivation $|\Psi \rangle_{HF}$ will be shortened to $|\Psi_0 \rangle$. 

\begin{equation}
E = \langle \Psi_0 |e^{-\textbf{T}} \textbf{H} e^{\textbf{T}} |\Psi_0 \rangle .
\end{equation}
We also assume an orthonormal basis, meaning

\begin{equation}
\langle \Psi_m |e^{-\textbf{T}} \textbf{H} e^{\textbf{T}} |\Psi_0 \rangle = 0 .
\end{equation}

\subsection{Baker-Campbell-Hausdorff formula}
The Baker-Campbell-Hausdorff formula is used to expand $e^{-\textbf{T}} \textbf{H} e^{\textbf{T}}$.

\begin{equation}
\begin{split}
e^{-\textbf{T}} \textbf{H} e^{\textbf{T}} = 
\textbf{H} 
+ \left[ \textbf{H}, \textbf{T} \right] 
+ \frac{1}{2} \left[ [\textbf{H}, \textbf{T}], \textbf{T} \right]  + \frac{1}{6} \left[ [ [\textbf{H}, \textbf{T}], \textbf{T}], \textbf{T} \right] \\
+ \frac{1}{24} \left[ [ [ [\textbf{H}, \textbf{T}], \textbf{T}],\textbf{T}], \textbf{T} \right] \dots .
\end{split} \label{baker}
\end{equation}
$\textbf{T}$ is expressed in terms of $\textbf{a}^{\dag}$ and $\textbf{a}$. $\textbf{H}$ contains a maximum of two orbital interactions. It can be showed that $\textbf{H}$ can also be expressed in terms of these operators. 

\begin{equation}
\textbf{H} = \sum_{a,i} h_{a,i}
\textbf{a}^{\dag}_a 
\textbf{a}_i 
+ \frac{1}{4} \sum_{a,b,i,j} \langle ab||ij \rangle
\textbf{a}^{\dag}_a  
\textbf{a}^{\dag}_b
\textbf{a}_i 
\textbf{a}_j . \label{annih}
\end{equation}
Where $h_{a,i} = \langle \psi_a | \textbf{h} | \psi_i \rangle$. With $\textbf{h}$ the one-particle part of $\textbf{H}$. This is the same Hamiltonian as before expressed slightly different and will be discussed further later on. Eq. \eqref{baker} can be simplified using commutators.

\begin{equation}
[\textbf{a}^{\dag}_a  \textbf{a}_i 
, \textbf{a}^{\dag}_b \textbf{a}_j] =
\textbf{a}^{\dag}_a  \textbf{a}_i \textbf{a}^{\dag}_b \textbf{a}_j
- \textbf{a}^{\dag}_b \textbf{a}_j \textbf{a}^{\dag}_a  \textbf{a}_i .
\end{equation} 
Using the anti commutator relations this commutator itself can be simplified.

\begin{equation}
[\textbf{a}^{\dag}_a  \textbf{a}_i 
, \textbf{a}^{\dag}_b \textbf{a}_j] =
\textbf{a}^{\dag}_a  \delta_{ib} \textbf{a}_j
- \textbf{a}^{\dag}_b \delta_{ja} \textbf{a}_i .
\end{equation}
This simplification reduces the number of indices from 4 to 3, replacing two operators with a Kronecker delta. Each nested commutator in Eq. \eqref{baker} will reduce the number of indexes by 1. The maximum number of creation/annihilation operators in $\textbf{H}$ was 4. This means Eq. \eqref{baker} will naturally truncate after exactly 4 terms, and we can remove the dots.

\begin{equation}
\begin{split}
e^{-\textbf{T}} \textbf{H} e^{\textbf{T}} = 
\textbf{H} 
+ \left[ \textbf{H}, \textbf{T} \right] 
+ \frac{1}{2} \left[ [\textbf{H}, \textbf{T}], \textbf{T} \right] 
+ \frac{1}{6} \left[ [ [ \textbf{H},[\textbf{T}], \textbf{T}], \textbf{T} \right] \\
+ \frac{1}{24} \left[ [ [ [\textbf{H}, \textbf{T}], \textbf{T}], \textbf{T}], \textbf{T}  \right] .
\end{split}  \label{variationalccsd}
\end{equation}

\subsection{Normal Order and Contractions}
When deriving CCSD equations it is common to introduce a concept called normal ordering of second quantized operators. This means that all creation operators are placed to the left and the annihilation operators to the right. The mathematics of swapping the order of creation and annihilation operators are well defined. To show this we define an example operator $\textbf{O}$ and use the anti commutator relations.

\begin{align}
\textbf{O} & = 
\textbf{a}_i 
\textbf{a}^{\dag}_a 
\textbf{a}_j 
\textbf{a}^{\dag}_b \label{normalorder} \\ &
= \delta_{ia} 
\textbf{a}_j 
\textbf{a}^{\dag}_b - 
\textbf{a}^{\dag}_a 
\textbf{a}_i  
\textbf{a}_j 
\textbf{a}^{\dag}_b \nonumber \\ &
= \delta_{ia} \delta_{jb} -
\delta_{ia} 
\textbf{a}^{\dag}_b
\textbf{a}_j -
\delta_{jb}
\textbf{a}^{\dag}_a
\textbf{a}_i +
\textbf{a}^{\dag}_a
\textbf{a}_i
\textbf{a}_j
\textbf{a}^{\dag}_b \nonumber \\ &
= \delta_{ia} \delta_{jb} - 
\delta_{ia} 
\textbf{a}^{\dag}_b
\textbf{a}_j +
\delta_{ib} 
\textbf{a}^{\dag}_a
\textbf{a}_j -
\delta_{jb} 
\textbf{a}^{\dag}_a
\textbf{a}_i -
\textbf{a}^{\dag}_a
\textbf{a}^{\dag}_b
\textbf{a}_i
\textbf{a}_j .
\end{align}
The final expression of $\textbf{O}$ is in normal order since all creation operators are to the left and all annihilation operators to the right. Notice that we now have five terms, four of which have a reduced number of operators compared to our first definition of $\textbf{O}$. Any combination of annihilation and creation operators can be expressed as a linear combination of normal ordered combinations of these operators. \\

The four terms with reduced number of operators arise from contractions between operators. A contraction between two operators $\textbf{A}$ and $\textbf{B}$, that each contain an arbitrary number of creation or/and annihilation operators, can be defined as such:

\begin{equation}
\contraction{}{\textbf{A}}{}{\textbf{B}}
\textbf{A}\textbf{B}
 \equiv \textbf{A}\textbf{B} - \left\{ \textbf{A}\textbf{B} \right\}_{\nu} .
\end{equation}
Here $\left\{ \textbf{A}\textbf{B} \right\}_{\nu}$ is the normal ordered form of $\textbf{A}\textbf{B}$. $\contraction{}{\textbf{A}}{}{\textbf{B}}
\textbf{A}\textbf{B}$ is called the contraction between $\textbf{A}$ and $\textbf{B}$. As an example $\textbf{A} = \textbf{a}_i$ and $\textbf{B} = \textbf{a}_j$ will give a contraction of

\begin{equation}
\contraction{}{\textbf{a}_i}{}{\textbf{a}_j}
\textbf{a}_i\textbf{a}_j
= \textbf{a}_i \textbf{a}_j - \left\{ \textbf{a}_i \textbf{a}_j \right\}_{\nu} = \textbf{a}_i \textbf{a}_j - \textbf{a}_i \textbf{a}_j = 0 . \label{wickexample1}
\end{equation}
From the example above we see the contraction of all annihilation operators will be zero since there will be no swapping of operators when creating the normal ordered form. The same will apply to contractions between operators formed from only creation operators. Also the contraction between already normal ordered operators will be zero. \\

However the contraction between different operators not in normal order will not be zero. The simplest example is one annihilation operator in front of one creation operator.

\begin{equation}
\contraction{}{\textbf{a}_i}{}{\textbf{a}^{\dag}_a}
\textbf{a}_i\textbf{a}^{\dag}_a
= \textbf{a}_i \textbf{a}^{\dag}_a - \left\{ \textbf{a}_i \textbf{a}^{\dag}_a \right\}_{\nu} = \textbf{a}_i \textbf{a}^{\dag}_a + \textbf{a}^{\dag}_a \textbf{a}_i = \delta_{ia} .
\end{equation}
Here we used Eq. \eqref{ccsd_anni_creato_operator_combo}.

\subsection{Wick's Theorem}
Wick's Theorem provides a schematic way of defining any string of annihilation and creation operators in terms of these contractions. A string of annihilation and creation operators can be defined as $ABC \dots XYZ$ where $A, B, C, X, Y, Z$ $\dots$ represent either a creation or an annihilation operator.  Wick's Theorem is defined as such

\begin{align}
\textbf{A} \textbf{B} \textbf{C} \dots \textbf{X} \textbf{Y} \textbf{Z} = & \left\{\textbf{A} \textbf{B} \textbf{C} \dots \textbf{X} \textbf{Y} \textbf{Z} \right\}_{\nu} \label{wicks} \\
& + \sum_{singles} \{
\contraction{}{\textbf{A}}{}{\textbf{B}}
\textbf{A}\textbf{B}
\textbf{C} \dots \textbf{X} \textbf{Y} \textbf{Z}
\}_{\nu} \nonumber \\
& + \sum_{doubles} \{
\contraction{}{\textbf{A}}{\textbf{B}}{\textbf{C}}
\contraction[2ex]{\textbf{A}}{\textbf{B}}{\textbf{C} \dots \textbf{X} \textbf{Y}}{\textbf{Z}}
\textbf{A} \textbf{B} \textbf{C} \dots \textbf{X} \textbf{Y} \textbf{Z} 
\}_{\nu} \nonumber \\
& \dots \nonumber
\end{align}
The right side of Eq. \eqref{wicks} should represent every possible contraction of $\textbf{A} \textbf{B} \textbf{C} \dots \textbf{X} \textbf{Y} \textbf{Z}$. To specify the notation we apply Wick's theorem as an example to the operator $\textbf{O}$ defined in Eq. \eqref{normalorder} and repeated here.

\begin{equation}
\textbf{O} = \textbf{a}_i \textbf{a}^{\dag}_a \textbf{a}_j \textbf{a}^{\dag}_b \nonumber
\end{equation}
Applying Wick's Theorem provides

\begin{align}
\textbf{O} = & 
\{
\textbf{a}_i \textbf{a}^{\dag}_a \textbf{a}_j \textbf{a}^{\dag}_b
\}_{\nu} \\ &
+ \{
\contraction{}{\textbf{a}_i}{}{\textbf{a}^{\dag}_a}
\textbf{a}_i \textbf{a}^{\dag}_a
\textbf{a}_j \textbf{a}^{\dag}_b
\}_{\nu} \\ &
+ \{
\textbf{a}_i
\contraction{}{\textbf{a}^{\dag}_a}{}{\textbf{a}_j}
\textbf{a}^{\dag}_a \textbf{a}_j
\textbf{a}^{\dag}_b
\}_{\nu} \label{wickex1} \\ &
+ \{
\textbf{a}_i \textbf{a}^{\dag}_a
\contraction{}{\textbf{a}^j}{}{\textbf{a}^{\dag}_b}
\textbf{a}_j \textbf{a}^{\dag}_b
\}_{\nu} \\ &
+ \{
\contraction{}{\textbf{a}_i}{\textbf{a}^{\dag}_a}{\textbf{a}_j}
\textbf{a}_i \textbf{a}^{\dag}_a \textbf{a}_j
\textbf{a}^{\dag}_b
\}_{\nu} \label{wickex2} \\ &
+ \{
\textbf{a}_i
\contraction{}{\textbf{a}^{\dag}_a}{\textbf{a}_j}{\textbf{a}^{\dag}_b}
\textbf{a}^{\dag}_a \textbf{a}_j \textbf{a}^{\dag}_b
\}_{\nu} \label{wickex3} \\ &
+ \{
\contraction{}{\textbf{a}_i}{\textbf{a}^{\dag}_a \textbf{a}_j}{\textbf{a}^{\dag}_b}
\textbf{a}_i \textbf{a}^{\dag}_a \textbf{a}_j \textbf{a}^{\dag}_b
\}_{\nu} \label{wickex4} \\ &
+ \{
\contraction{}{\textbf{a}_i}{}{\textbf{a}^{\dag}_a}
\textbf{a}_i \textbf{a}^{\dag}_a 
\contraction{}{\textbf{a}_j}{}{\textbf{a}^{\dag}_b}
\textbf{a}_j \textbf{a}^{\dag}_b
\}_{\nu} . \label{wicksdoubles}
\end{align}
Eq. \eqref{wicksdoubles} is from the doubles summation. The other terms are from the singles summation. Eqs. \eqref{wickex1}, \eqref{wickex2} and \eqref{wickex3} are zero when using the rules such as \eqref{wickexample1}. This leaves the terms

\begin{align}
\textbf{O} = & 
\{
\textbf{a}_i \textbf{a}^{\dag}_a \textbf{a}_j \textbf{a}^{\dag}_b
\}_{\nu}
+ \{
\contraction{}{\textbf{a}_i}{}{\textbf{a}^{\dag}_a}
\textbf{a}_i \textbf{a}^{\dag}_a
\textbf{a}_j \textbf{a}^{\dag}_b
\}_{\nu}
+ \{
\textbf{a}_i \textbf{a}^{\dag}_a
\contraction{}{\textbf{a}^j}{}{\textbf{a}^{\dag}_b}
\textbf{a}_j \textbf{a}^{\dag}_b
\}_{\nu} \nonumber \\ &
+ \{
\contraction{}{\textbf{a}_i}{\textbf{a}^{\dag}_a \textbf{a}_j}{\textbf{a}^{\dag}_b}
\textbf{a}_i \textbf{a}^{\dag}_a \textbf{a}_j \textbf{a}^{\dag}_b
\}_{\nu}
+ \{
\contraction{}{\textbf{a}_i}{}{\textbf{a}^{\dag}_a}
\textbf{a}_i \textbf{a}^{\dag}_a 
\contraction{}{\textbf{a}_j}{}{\textbf{a}^{\dag}_b}
\textbf{a}_j \textbf{a}^{\dag}_b
\}_{\nu} .
\end{align}
These must be evaluated. One trick needed is when swapping two operators inside a contraction the sign is changed. Eq. \eqref{wickex4} is used as an example.

\begin{equation}
\{
\contraction{}{\textbf{a}_i}{\textbf{a}^{\dag}_a \textbf{a}_j}{\textbf{a}^{\dag}_b}
\textbf{a}_i \textbf{a}^{\dag}_a \textbf{a}_j \textbf{a}^{\dag}_b
\}_{\nu}
= - \{
\contraction{}{\textbf{a}_i}{\textbf{a}^{\dag}_a}{\textbf{a}^{\dag}_b}
\textbf{a}_i \textbf{a}^{\dag}_a \textbf{a}^{\dag}_b 
\textbf{a}_j 
\}_{\nu}
= \{
\contraction{}{\textbf{a}_i}{}{\textbf{a}^{\dag}_b}
\textbf{a}_i  \textbf{a}^{\dag}_b 
\textbf{a}^{\dag}_a \textbf{a}_j 
\}_{\nu} . \label{wikssign}
\end{equation}
Remembering $\{\contraction{}{\textbf{a}_i}{}{\textbf{a}^{\dag}_b}
\textbf{a}_i  \textbf{a}^{\dag}_b \}_{\nu} = \delta_{ib}$ then the terms in $\textbf{O}$ reduces in order to

\begin{equation}
\textbf{O} =  \{ 
\textbf{a}_i \textbf{a}^{\dag}_a
\textbf{a}_j \textbf{a}^{\dag}_b
\} +
\delta_{ia} \{
\textbf{a}_j \textbf{a}^{\dag}_b
\} +
\delta_{jb} \{
\textbf{a}_i \textbf{a}^{\dag}_a
\} +
\delta_{ib} \{
\textbf{a}^{\dag}_a \textbf{a}_j
\} +
\delta_{ia} \delta_{ib} . 
\end{equation}

Using $\{
\textbf{a} \textbf{a}^{\dag}
\} = -\textbf{a}^{\dag} \textbf{a}$ and $\{
\textbf{a}^{\dag} \textbf{a} 
\} = \textbf{a}^{\dag} \textbf{a}$ we get

\begin{equation}
\textbf{O} = 
\textbf{a}^{\dag}_a \textbf{a}^{\dag}_b
\textbf{a}_i \textbf{a}_j  
- \delta_{ia} \textbf{a}^{\dag}_b \textbf{a}_j
- \delta_{jb} \textbf{a}^{\dag}_a \textbf{a}_i
+ \delta_{ib} \textbf{a}^{\dag}_a \textbf{a}_j 
+ \delta_{ia} \delta_{ib} ,
\end{equation}
which is identical to Eq. \eqref{normalorder}. The sign rules can sometimes be complicated, when there is more than one contraction present. Swapping two operators can fulfil the positioning for two contractions at once, as seen in example Eq. \eqref{examplebullshit}. This provides a minus sign which must not be neglected.

\begin{equation}
\{
\contraction{}{\textbf{a}_i}{\textbf{a}^{\dag}_a}{\textbf{a}_j}
\contraction[2ex]{\textbf{a}_i}{\textbf{a}^{\dag}_a}{\textbf{a}_j}{\textbf{a}^{\dag}_b}
\textbf{a}_i \textbf{a}^{\dag}_a \textbf{a}_j \textbf{a}^{\dag}_b
\}_{\nu}
= - \{
\contraction{}{\textbf{a}_i}{}{\textbf{a}^{\dag}_a}
\contraction{\textbf{a}_i \textbf{a}^{\dag}_a}{\textbf{a}_j}{}{\textbf{a}^{\dag}_b}
\textbf{a}_i \textbf{a}^{\dag}_a \textbf{a}_j \textbf{a}^{\dag}_b
\} . \label{examplebullshit}
\end{equation}

\subsection{Fermi Vacuum and Particle Holes}

When using creation and annihilation operators it is common they work on the vacuum state, $| \rangle$. Eq. \eqref{dirac_not} would commonly be represented as such

\begin{equation}
|\Psi_0 \rangle = \textbf{a}^{\dag}_i
\textbf{a}^{\dag}_j
\textbf{a}^{\dag}_k
\textbf{a}^{\dag}_l |\rangle .
\end{equation}
An excited state $|\Psi_m\rangle$ would for example be noted as the following

\begin{equation}
|\Psi_m \rangle = \textbf{a}^{\dag}_m
\textbf{a}^{\dag}_i
\textbf{a}^{\dag}_j
\textbf{a}^{\dag}_k |\rangle .
\end{equation}
Where the m orbital is occupied and the l orbital is not occupied. In our CCSD derivation we will not use this kind of notation. The Fermi Vacuum is introduced and is later defined as the Hartree Fock result. However continuing this example the Fermi Vacuum could be defined as $|\Psi_0\rangle$, and the excited state would be

\begin{equation}
|\Psi_m\rangle = \textbf{a}^{\dag}_m \textbf{a}_l |\Psi_0\rangle .
\end{equation}
This creates a "hole state" in orbital l, since an occupied orbital is now unoccupied. It also creates a "particle state" in orbital m, since this is now occupied and was unoccupied in the Fermi Vacuum. \\

This definition will bring new features to Wick's theorem. Indexes $a, b, c \dots$ will denote newly occupied orbitals. Indexes $i, j, k \dots$ will denote newly formed holes. The operator $\textbf{a}^{\dag}_i$ can then be thought of as annihilating a hole. $\textbf{a}_a$ can be thought of annihilating a particle. Likewise $\textbf{a}^{\dag}_a$ and $\textbf{a}_i$ can be thought of as creating a particle or creating a hole. \\

This differs from the concept of $\textbf{a}^{\dag}$ always being a creation operator, since $\textbf{a}^{\dag}_i$ can be thought of as annihilating a hole state. This changes our Wick's Theorem calculations, since we still have the only terms not zero being those with one annihilation operator followed by one creation operator. There are only two possibilities of this happening.

\begin{equation}
\contraction{}{\textbf{a}^{\dag}_i}{}{\textbf{a}_j}
\textbf{a}^{\dag}_i \textbf{a}_j
 = \textbf{a}^{\dag}_i \textbf{a}_j
 - \{\textbf{a}^{\dag}_i \textbf{a}_j \}_{\nu} = \textbf{a}^{\dag}_i \textbf{a}_j
 +  \textbf{a}_j \textbf{a}^{\dag}_i = \delta_{ij} . \label{fermi1}
\end{equation}

\begin{equation}
\contraction{}{\textbf{a}_a}{}{\textbf{a}^{\dag}_b}
\textbf{a}_a \textbf{a}^{\dag}_b = \textbf{a}_a \textbf{a}^{\dag}_b - \{\textbf{a}_a \textbf{a}^{\dag}_b\}_{\nu} = \textbf{a}_a \textbf{a}^{\dag}_b + \textbf{a}^{\dag}_b \textbf{a}_a = \delta_{ab} . \label{fermi2}
\end{equation}
Any other contraction will be 0 using rules analogous to Eq. \eqref{wickexample1}

\subsection{Normal Ordered $\textbf{H}$}
As noted in Eq. \eqref{annih} $\textbf{H}$ can be expressed in terms of creation and annihilation operators. This expression is known as the secound-quantized form of the electronic Hamiltionan and will be repeated here, but the indices will be changed because $a,b,i,j$ have now been denoted a new meaning.

\begin{equation}
\textbf{H} = \sum_{pq} <p|\textbf{h}|q> \textbf{a}^{\dag}_q \textbf{a}_p + 
\frac{1}{4} \sum_{pqrs} \langle pq||rs \rangle \textbf{a}^{\dag}_p \textbf{a}^{\dag}_q \textbf{a}_s \textbf{a}_r .
\end{equation}
We now wish to use Wick's theorem on this operator to simplify. From the one-electron term we use:

\begin{equation}
\textbf{a}^{\dag}_p \textbf{a}_q = \{\textbf{a}^{\dag}_p \textbf{a}_q \} + \{\contraction{}{\textbf{a}^{\dag}_p}{}{\textbf{a}_q}
 \textbf{a}^{\dag}_p \textbf{a}_q 
\} . 
\end{equation}
Eq. \eqref{fermi1} states that $\{\contraction{}{\textbf{a}^{\dag}_p}{}{\textbf{a}_q}
\textbf{a}^{\dag}_p \textbf{a}_q \}$ is not equal to zero only if both operators act on a hole space, then $\{\contraction{}{\textbf{a}^{\dag}_p}{}{\textbf{a}_q}
\textbf{a}^{\dag}_p \textbf{a}_q \}$ = $\delta_{pq}$. This means 

\begin{equation}
\sum_{pq}
\{\contraction{}{\textbf{a}^{\dag}_p}{}{\textbf{a}_q}
\textbf{a}^{\dag}_p \textbf{a}_q \} = \sum_i \langle i|h|i \rangle .
\end{equation}
Inserting this in $\textbf{H}$ we get:

\begin{equation}
\textbf{H} = \sum_{pq} <p|\textbf{h}|q> 
\{\textbf{a}^{\dag}_p \textbf{a}_q \}
+ \sum_i \langle i|h|i \rangle
 + 
\frac{1}{4} \sum_{pqrs} \langle pq||rs \rangle \textbf{a}^{\dag}_p \textbf{a}^{\dag}_q \textbf{a}_s \textbf{a}_r . \label{temp_h}
\end{equation}
The Wick's theorem will also be applied to the two electron part, $\textbf{a}^{\dag}_p \textbf{a}^{\dag}_q \textbf{a}_s \textbf{a}_r$. Included here are only the non-zero terms.

\begin{align}
\textbf{a}^{\dag}_p \textbf{a}^{\dag}_q \textbf{a}_s \textbf{a}_r = & \{\textbf{a}^{\dag}_p \textbf{a}^{\dag}_q \textbf{a}_s \textbf{a}_r\} 
+ \{
\contraction{}{\textbf{a}^{\dag}_p}{\textbf{a}^{\dag}_q}{\textbf{a}_s}
\textbf{a}^{\dag}_p \textbf{a}^{\dag}_q \textbf{a}_s 
\textbf{a}_r
\}
+ \{
\textbf{a}^{\dag}_p
\contraction{}{\textbf{a}^{\dag}_q}{}{\textbf{a}_s}
\textbf{a}^{\dag}_q \textbf{a}_s 
\textbf{a}_r
\} \nonumber \\ &
+ \{
\contraction{}{\textbf{a}^{\dag}_p}{\textbf{a}^{\dag}_q \textbf{a}_s}{\textbf{a}_r}
\textbf{a}^{\dag}_p \textbf{a}^{\dag}_q \textbf{a}_s 
\textbf{a}_r
\}
+ \{\textbf{a}^{\dag}_p
\contraction{}{\textbf{a}^{\dag}_q}{\textbf{a}_s}{\textbf{a}_r}
\textbf{a}^{\dag}_q \textbf{a}_s \textbf{a}_r
\}
+ \{
\contraction{}{\textbf{a}^{\dag}_p}{\textbf{a}^{\dag}_q}{\textbf{a}_s}
\contraction[2ex]{\textbf{a}^{\dag}_p}{\textbf{a}^{\dag}_q}{\textbf{a}_s}{\textbf{a}_r}
\textbf{a}^{\dag}_p \textbf{a}^{\dag}_q \textbf{a}_s 
\textbf{a}_r
\} \nonumber \\ &
+ \{
\contraction[2ex]{\textbf{a}^{\dag}_p}{\textbf{a}^{\dag}_q}{\textbf{a}_s}{\textbf{a}_r}
\contraction{}{\textbf{a}^{\dag}_p}{\textbf{a}^{\dag}_q \textbf{a}_s}{\textbf{a}_r}
\textbf{a}^{\dag}_p \textbf{a}^{\dag}_q \textbf{a}_s 
\textbf{a}_r
\} . \nonumber
\end{align}
These can be simplified using Eq. \eqref{fermi1}, and the rules for index swapping within a contraction noted in Eq. \eqref{wikssign}.

\begin{align}
\textbf{a}^{\dag}_p \textbf{a}^{\dag}_q \textbf{a}_s \textbf{a}_r = &\{\textbf{a}^{\dag}_p \textbf{a}^{\dag}_q \textbf{a}_s \textbf{a}_r\} 
+ \delta_{pi} \delta_{ps} \{ \textbf{a}^{\dag}_q \textbf{a}_r \}
\nonumber \\ & 
+ \delta_{qi} \delta_{qs}
\{ \textbf{a}^{\dag}_p \textbf{a}_r \}
+ \delta_{pi} \delta_{pr} 
\{ \textbf{a}^{\dag}_q \textbf{a}_s \}
+ \delta_{qi} \delta_{qr}
\{ \textbf{a}^{\dag}_p \textbf{a}_s \} \nonumber \\ &
- \delta_{pi} \delta_{ps} \delta_{qj} \delta_{qr}
+ \delta_{pi} \delta_{pr} \delta_{qj} \delta_{qs} .
\end{align}
The two electron part can now be replaced.

\begin{align}
\frac{1}{4} \sum_{pqrs} \langle pq||rs \rangle \textbf{a}^{\dag}_p \textbf{a}^{\dag}_q \textbf{a}_s \textbf{a}_r = &
\frac{1}{4}
\sum_{pqrs} \langle pq||rs \rangle \{\textbf{a}^{\dag}_p \textbf{a}^{\dag}_q \textbf{a}_s \textbf{a}_r\}
\nonumber \\ &
- \frac{1}{4} \sum_{iqr} \langle iq||ri \rangle \{\textbf{a}^{\dag}_q \textbf{a}_r \} \nonumber \\ &
+ \frac{1}{4} \sum_{ipr} \langle pi||ri \rangle \{\textbf{a}^{\dag}_p \textbf{a}_r \} \nonumber \\ &
+ \frac{1}{4} \sum_{iqs} \langle iq||is \rangle \{\textbf{a}^{\dag}_q \textbf{a}_s \} \nonumber \\ &
- \frac{1}{4} \sum_{ips} \langle pi||is \rangle \{\textbf{a}^{\dag}_p \textbf{a}_s \} \nonumber \\ &
- \frac{1}{4} \sum_{ij} \langle ij||ji \rangle \nonumber \\ &
+ \frac{1}{4} \sum_{ij} \langle ij||ij \rangle . \nonumber
\end{align}
From the symmetry in the single bar four index integrals it can be shown that these symmetries hold for the double bar integrals:

\begin{equation}
\langle pq||rs \rangle =
\langle qp||sr \rangle =
- \langle pq||sr \rangle =
- \langle qp || rs \rangle , \label{dirac_symetry}
\end{equation}
and 

\begin{equation}
\langle pq || rs \rangle = \langle rs || pq \rangle ,
\end{equation}

making eightfold symmetry. Using Eq. \eqref{dirac_symetry}, re indexing terms and combining leaves the two electron part as the following:

\begin{align}
\frac{1}{4} \sum_{pqrs} \langle pq||rs \rangle \textbf{a}^{\dag}_p \textbf{a}^{\dag}_q \textbf{a}_s \textbf{a}_r = & \frac{1}{4}
\sum_{pqrs} \langle pq||rs \rangle \{\textbf{a}^{\dag}_p \textbf{a}^{\dag}_q \textbf{a}_s \textbf{a}_r\}
 \\ &
+ \sum_{ipr} \langle pi||ri \rangle \{\textbf{a}^{\dag}_p \textbf{a}_r \} \nonumber \\ &
+ \frac{1}{2} \sum_{ij} \langle ij||ij \rangle . \nonumber
\end{align}
This can be inserted in Eq. \eqref{temp_h}.

\begin{align}
\textbf{H} = & \sum_{pq} <p|\textbf{h}|q> 
\{\textbf{a}^{\dag}_p \textbf{a}_q \}
+ \sum_i \langle i|h|i \rangle
 + \frac{1}{4}
\sum_{pqrs} \langle pq||rs \rangle \{\textbf{a}^{\dag}_p \textbf{a}^{\dag}_q \textbf{a}_s \textbf{a}_r\}
 \\ &
+ \sum_{ipr} \langle pi||ri \rangle \{\textbf{a}^{\dag}_p \textbf{a}_r \}
+ \frac{1}{2} \sum_{ij} \langle ij||ij \rangle . \nonumber
\end{align}
The first and fourth term on the right hand side are the normal ordered form of the Fock operator. If we also include the second term we have the HF energy.

\begin{equation}
\textbf{H} = \sum_{pq} f_{pq} 
\{\textbf{a}^{\dag}_p \textbf{a}_q \}
 + \frac{1}{4}
\sum_{pqrs} \langle pq||rs \rangle \{\textbf{a}^{\dag}_p \textbf{a}^{\dag}_q \textbf{a}_s \textbf{a}_r\}
+ \langle \Psi_{HF} | \textbf{H} |\Psi_{HF} \rangle .
\end{equation}
We rename these terms.

\begin{equation}
\textbf{H} = \textbf{F}_N + \textbf{V}_N + \langle \Psi_{HF} | \textbf{H} |\Psi_{HF} \rangle .
\end{equation}
The normal ordered Hamiltonian is defined from this:

\begin{equation}
\textbf{H}_N \equiv 
\textbf{H} - \langle \Psi_{HF} | \textbf{H} |\Psi_{HF} \rangle = 
\textbf{F}_N + \textbf{V}_N . \label{normal_order_hamiltonian}
\end{equation}

\subsection{CCSD Hamiltonian}

The CCSD Hamiltonian is now defined as such

\begin{equation}
\bar{H} \equiv e^{-\textbf{T}} \textbf{H}_N e^{\textbf{T}}  .
\end{equation}
Using the CCSD cluster operator, $\textbf{T} = \textbf{T}_1 + \textbf{T}_2$. This can be inserted in equation Eq. \eqref{variationalccsd}. 

\begin{align}
\bar{H} = & 
\textbf{H}_N 
+ \left[ \textbf{H}_N, \textbf{T}_1 \right] 
+ \left[ \textbf{H}_N, \textbf{T}_2 \right] 
+ \frac{1}{2} \left[ [\textbf{H}_N, \textbf{T}_1], \textbf{T}_1 \right]  \\ &
+ \frac{1}{2} \left[ [\textbf{H}_N, \textbf{T}_1], \textbf{T}_2  \right]
+ \frac{1}{2} \left[ [\textbf{H}_N, \textbf{T}_2], \textbf{T}_1 \right]
+ \frac{1}{2} \left[ [\textbf{H}_N, \textbf{T}_2], \textbf{T}_2 \right] \dots \nonumber
\end{align}
$\textbf{T}_1$ and $\textbf{T}_2$ does commute, so we can combine terms. The full $\bar{H}$ then becomes:

\begin{align}
\bar{H} = & 
\textbf{H}_N 
+ \left[ \textbf{H}_N, \textbf{T}_1 \right] 
+ \left[ \textbf{H}_N, \textbf{T}_2 \right] 
+ \frac{1}{2} \left[ [\textbf{H}_N, \textbf{T}_1], \textbf{T}_1 \right] \label{temp_hamil_ccsd} \\ &
+ \left[ [\textbf{H}_N, \textbf{T}_1], \textbf{T}_2  \right]
+ \frac{1}{2} \left[ [\textbf{H}_N, \textbf{T}_2], \textbf{T}_2 \right] \nonumber \\ &
+ \frac{1}{6} \left[ [ [ \textbf{H}_N,[\textbf{T}_1], \textbf{T}_1], \textbf{T}_1 \right]
+ \frac{1}{6} \left[ [ [ \textbf{H}_N,[\textbf{T}_2], \textbf{T}_2], \textbf{T}_2 \right] \nonumber \\ &
+ \frac{1}{2} \left[ [ [ \textbf{H}_N,[\textbf{T}_1], \textbf{T}_1], \textbf{T}_2 \right]
+ \frac{1}{2} \left[ [ [ \textbf{H}_N,[\textbf{T}_1], \textbf{T}_2], \textbf{T}_2 \right] \nonumber \\ &
+ \frac{1}{24} \left[ [ [ [\textbf{H}_N, \textbf{T}_1], \textbf{T}_1], \textbf{T}_1], \textbf{T}_1  \right]
+ \frac{1}{24} \left[ [ [ [\textbf{H}_N, \textbf{T}_2], \textbf{T}_2], \textbf{T}_2], \textbf{T}_2  \right] \nonumber \\ &
+ \frac{1}{6} \left[ [ [ [\textbf{H}_N, \textbf{T}_1], \textbf{T}_1], \textbf{T}_1], \textbf{T}_2  \right]
+ \frac{1}{6} \left[ [ [ [\textbf{H}_N, \textbf{T}_1], \textbf{T}_1], \textbf{T}_2], \textbf{T}_2  \right] \nonumber \\ &
+ \frac{1}{4} \left[ [ [ [\textbf{H}_N, \textbf{T}_1], \textbf{T}_2], \textbf{T}_2], \textbf{T}_2  \right] . \nonumber
\end{align}
$\bar{H}$ will still analytically truncates after up to and including four nested commutators. When using $\textbf{H}_N$ it is better to rewrite Eqs. \eqref{t1defi} and \eqref{t2defi} using contractions.

\begin{equation}
\textbf{T}_1 \equiv \sum_{ai} t_i^a \textbf{a}^{\dag}_a \textbf{a}_i = \sum_{ai} \left( t_i^a \{\textbf{a}^{\dag}_a \textbf{a}_i\} + \{
\contraction{}{\textbf{a}^{\dag}_a}{}{\textbf{a}_i}
\textbf{a}^{\dag}_a \textbf{a}_i
\} \right) = \sum_{ai} t_i^a \{\textbf{a}^{\dag}_a \textbf{a}_i\} .
\end{equation}
The contraction term will be 0 based the discussion before Eq. \eqref{fermi1}. Also similarly for $\textbf{T}_2$

\begin{equation}
\textbf{T}_2 = \frac{1}{4} \sum_{abij} \{
\textbf{a}^{\dag}_a \textbf{a}^{\dag}_b
\textbf{a}_i \textbf{a}_j \} .
\end{equation}
The commutators can then be calculated, starting with $[\textbf{H}_N, \textbf{T}_1]$.

\begin{equation}
[\textbf{H}_N, \textbf{T}_1] = \textbf{H}_N \textbf{T}_1 - \textbf{T}_1 \textbf{H}_N .
\end{equation}
Using the definition of contractions on both these terms we can simplify this expression.

\begin{align}
[\textbf{H}_N, \textbf{T}_1] & =
\left( \{\textbf{H}_N\textbf{T}_1\} + \{
\contraction{}{\textbf{H}_N}{}{\textbf{T}_1}
\textbf{H}_N\textbf{T}_1\} \right) - \left(\{\textbf{T}_1\textbf{H}_N\} + \{
\contraction{}{\textbf{T}_1}{}{\textbf{H}_N}
\textbf{T}_1\textbf{H}_N\} \right) \nonumber \\ &
= \{
\contraction{}{\textbf{H}_N}{}{\textbf{T}_1}
\textbf{H}_N\textbf{T}_1\}
- \{
\contraction{}{\textbf{T}_1}{}{\textbf{H}_N}
\textbf{T}_1\textbf{H}_N\} .
\end{align}
Eqs. \eqref{fermi1} and \eqref{fermi2} explains the only terms that will not be 0 when calculating the contractions. $\{ \contraction{}{\textbf{T}_1}{}{\textbf{H}_N}
\textbf{T}_1\textbf{H}_N\}$ will be 0 since $\textbf{T}_1$ does not contain any creation or annihilation operator that when placed on the left creates a non-zero contraction when using Wick's Theorem. The same argument applies to $\textbf{T}_2$. 

\begin{equation}
[\textbf{H}_N, \textbf{T}_1] = 
\{
\contraction{}{\textbf{H}_N}{}{\textbf{T}_1}
\textbf{H}_N\textbf{T}_1\} 
= \left( \textbf{H}_N \textbf{T}_1 \right)_C
.
\end{equation}

\begin{equation}
[\textbf{H}_N, \textbf{T}_2] = 
\{
\contraction{}{\textbf{H}_N}{}{\textbf{T}_2}
\textbf{H}_N\textbf{T}_2\} 
= \left( \textbf{H}_N \textbf{T}_2 \right)_C
.
\end{equation}
It then becomes clear that the only surviving terms when calculating all the commutators will be terms with $\textbf{H}_N$ in the leftmost position. \\

A new notation is also introduced, $()_C$. This notations means that each cluster operator inside the parentheses should have at least one contraction each to $\textbf{H}_N$ when applying Wick's Theorem. This holds for up to four cluster operators, which makes the truncation even more sensible. \\

The final form of $\bar{H}$ becomes

\begin{equation}
\begin{split}
\bar{H} = 
\big( \textbf{H}_N + \textbf{H}_N \textbf{T}_1 + \textbf{H}_N \textbf{T}_2
+ \frac{1}{2} \textbf{H}_N \textbf{T}_1^2
+ \frac{1}{2} \textbf{H}_N \textbf{T}_2^2
+ \textbf{H}_N \textbf{T}_1 \textbf{T}_2 \\
+ \frac{1}{6} \textbf{H}_N \textbf{T}_1^3
+ \frac{1}{6} \textbf{H}_N \textbf{T}_2^3
+ \frac{1}{2} \textbf{H}_N \textbf{T}_1^2 \textbf{T}_2
+ \frac{1}{2} \textbf{H}_N \textbf{T}_1 \textbf{T}_2^2 \\ 
+ \frac{1}{24} \textbf{H}_N \textbf{T}_1^4
+ \frac{1}{24} \textbf{H}_N \textbf{T}_2^4
+ \frac{1}{4} \textbf{H}_N \textbf{T}_1^2 \textbf{T}_2^2
+ \frac{1}{6} \textbf{H}_N \textbf{T}_1^3 \textbf{T}_2
+ \frac{1}{6} \textbf{H}_N \textbf{T}_1 \textbf{T}_2^3 \big)_C  .
\end{split} \label{CCSDHamiltonian}
\end{equation}

\subsection{CCSD Energy}
Using the definition of $\textbf{H}_N$, Eq. \eqref{normal_order_hamiltonian}, and $\bar{H}$, Eq. \eqref{CCSDHamiltonian}, we can now construct a programmable expression for the energy. 

\begin{equation}
E_{CCSD} - E_0 = \langle \Psi_0 | \bar{H} | \Psi_0 \rangle .
\end{equation}
The terms in Eq. \eqref{CCSDHamiltonian} are here calculated separately. 

\begin{equation}
\langle \Psi_0 | \textbf{H}_N | \Psi_0 \rangle = 0 .
\end{equation}
From the construction of the normal ordered Hamiltonian this term will be 0.

\begin{align}
\langle \Psi_0 | (\textbf{H}_N \textbf{T}_1)_C | \Psi_0 \rangle & = \langle \Psi_0 | \left((\textbf{F}_N + \textbf{V}_N ) \textbf{T}_1 \right)_C | \Psi_0 \rangle \nonumber \\ &
= \langle \Psi_0 | (\textbf{F}_N \textbf{T}_1)_C | \Psi_0 \rangle + \langle \Psi_0 | (\textbf{V}_N \textbf{T}_1)_C | \Psi_0 \rangle . \label{ene_1}
\end{align}

\begin{equation}
(\textbf{F}_N \textbf{T}_1)_C  = \sum_{pq} \sum_{ai} f_{pq} t_i^a  \{ \textbf{a}^{\dag}_p \textbf{a}_q\}  \{\textbf{a}^{\dag}_a \textbf{a}_i \} . 
\end{equation}
Wick's Theorem is applied to simplify the expression. Only non zero terms are included.

\begin{align}
\{ \textbf{a}^{\dag}_p \textbf{a}_q\}  \{\textbf{a}^{\dag}_a \textbf{a}_i \} & = &
\{ \textbf{a}^{\dag}_p \textbf{a}_q \textbf{a}^{\dag}_a \textbf{a}_i \} + \{ \textbf{a}^{\dag}_p
\contraction{}{\textbf{a}_q}{}{\textbf{a}^{\dag}_a}
\textbf{a}_q \textbf{a}^{\dag}_a
\textbf{a}_i \}  \\ & &
+  \{
\contraction{}{\textbf{a}^{\dag}_p}{ \textbf{a}_q \textbf{a}^{\dag}_a}{\textbf{a}_i}
\textbf{a}^{\dag}_p \textbf{a}_q \textbf{a}^{\dag}_a \textbf{a}_i \} 
+ \{
\contraction{}{\textbf{a}^{\dag}_p}{\textbf{a}_q \textbf{a}^{\dag}_a}{\textbf{a}_i}
\contraction{\textbf{a}^{\dag}_p}{\textbf{a}_q}{} {\textbf{a}^{\dag}_a}
\textbf{a}^{\dag}_p \textbf{a}_q \textbf{a}^{\dag}_a \textbf{a}_i \} \nonumber \\ & = &
\{ \textbf{a}^{\dag}_p \textbf{a}_q \textbf{a}^{\dag}_a \textbf{a}_i \} + \delta_{pi} \{\textbf{a}_q \textbf{a}_a^{\dag} \} \nonumber \\ & &
+ \delta_{qa} \{\textbf{a}_p^{\dag} \textbf{a}_i \} 
+ \delta_{pi} \delta_{qa} . \nonumber
\end{align}
Inserting this gives 

\begin{equation}
(\textbf{F}_N \textbf{T}_1)_C = \sum_{pq} \sum_{ai} f_{pq} t_i^a \left(\{ \textbf{a}^{\dag}_p \textbf{a}_q \textbf{a}^{\dag}_a \textbf{a}_i \} + \delta_{pi} \{\textbf{a}_q \textbf{a}_a^{\dag} \} 
+ \delta_{qa} \{\textbf{a}_p^{\dag} \textbf{a}_i \} 
+ \delta_{pi} \delta_{qa} \right) .
\end{equation}
When calculating $\langle \Psi_0 | (\textbf{F}_N \textbf{T}_1)_C | \Psi_0 \rangle$ only terms that solely consists of $\delta$'s will be non 0, since our basis is orthogonal. 

\begin{equation}
\langle \Psi_0 | (\textbf{F}_N \textbf{T}_1)_C | \Psi_0 \rangle =  \sum_{pq} \sum_{ai} f_{pq} t_i^a\delta_{pi} \delta_{qa} = \sum_{ai} f_{ai} t_i^a . \label{f_t1}
\end{equation}
$\langle \Psi_0 | (\textbf{V}_N \textbf{T}_1)_C | \Psi_0 \rangle$ must also be calculated.

\begin{equation}
(\textbf{V}_N \textbf{T}_1 )_C = \frac{1}{4} \sum_{pqrs} \sum_{ai} \langle pq||rs \rangle t_i^a \{\textbf{a}^{\dag}_p \textbf{a}^{\dag}_q \textbf{a}_s \textbf{a}_r \} \{\textbf{a}_a^{\dag} \textbf{a}_i \} .
\end{equation}

\begin{align}
\{\textbf{a}^{\dag}_p \textbf{a}^{\dag}_q \textbf{a}_s \textbf{a}_r \} \{\textbf{a}_a^{\dag} \textbf{a}_i \} & = \{
\textbf{a}^{\dag}_p \textbf{a}^{\dag}_q \textbf{a}_s \textbf{a}_r \textbf{a}_a^{\dag} \textbf{a}_i 
\} 
+ 
\{ \textbf{a}^{\dag}_p \textbf{a}^{\dag}_q
\contraction{}{\textbf{a}_s}{\textbf{a}_r}{\textbf{a}_a^{\dag}}
 \textbf{a}_s \textbf{a}_r \textbf{a}_a^{\dag}
  \textbf{a}_i 
\}  \\ &
+ 
\{ \textbf{a}^{\dag}_p \textbf{a}^{\dag}_q \textbf{a}_s
\contraction{}{\textbf{a}_r}{}{\textbf{a}_a^{\dag}}
\textbf{a}_r \textbf{a}_a^{\dag}
\textbf{a}_i
\}
+ \{ \textbf{a}^{\dag}_p
\contraction{}{\textbf{a}^{\dag}_q}{\textbf{a}_s \textbf{a}_r \textbf{a}_a^{\dag}}{\textbf{a}_i}
\textbf{a}^{\dag}_q \textbf{a}_s \textbf{a}_r \textbf{a}_a^{\dag} \textbf{a}_i \} \nonumber \\ &
+ \{
\contraction{}{\textbf{a}^{\dag}_p}{\textbf{a}^{\dag}_q \textbf{a}_s \textbf{a}_r \textbf{a}_a^{\dag}}{\textbf{a}_i}
\textbf{a}^{\dag}_p \textbf{a}^{\dag}_q \textbf{a}_s \textbf{a}_r \textbf{a}_a^{\dag} \textbf{a}_i
\}
+ \{
\contraction{}{\textbf{a}^{\dag}_p}{\textbf{a}^{\dag}_q \textbf{a}_s \textbf{a}_r \textbf{a}_a^{\dag}}{\textbf{a}_i}
\contraction{\textbf{a}^{\dag}_p \textbf{a}^{\dag}_q \textbf{a}_s}{\textbf{a}_r}{}{\textbf{a}_a^{\dag}}
\textbf{a}^{\dag}_p \textbf{a}^{\dag}_q \textbf{a}_s \textbf{a}_r \textbf{a}_a^{\dag} \textbf{a}_i
\} \nonumber \\ &
+ \{
\contraction{}{\textbf{a}^{\dag}_p}{\textbf{a}^{\dag}_q \textbf{a}_s \textbf{a}_r \textbf{a}_a^{\dag}}{\textbf{a}_i}
\contraction{\textbf{a}^{\dag}_p \textbf{a}^{\dag}_q}{\textbf{a}_s}{\textbf{a}_r}{\textbf{a}_a^{\dag}}
\textbf{a}^{\dag}_p \textbf{a}^{\dag}_q \textbf{a}_s \textbf{a}_r \textbf{a}_a^{\dag} \textbf{a}_i
\} . \nonumber
\end{align}
From the derivation of $\langle \Psi_0 | (\textbf{F}_N \textbf{T}_1)_C | \Psi_0 \rangle$ we noticed that the only term that survived was the term where every construction/annihilation operator was linked by a contraction. In this case we have no such terms. Hence the contribution from $\langle \Psi_0 | (\textbf{V}_N \textbf{T}_1)_C | \Psi_0 \rangle$ will be 0. \\

Inserting Eq. \eqref{f_t1} and $\langle \Psi_0 | (\textbf{V}_N \textbf{T}_1)_C | \Psi_0 \rangle = 0$ into Eq. \eqref{ene_1} gives 

\begin{equation}
\langle \Psi_0 | (\textbf{H}_N \textbf{T}_1)_C | \Psi_0 \rangle = \sum_{ai} f_{ai} t_i^a . \label{Energy_Contribution_1}
\end{equation}
Here the contribution from $\langle \Psi_0 | (\textbf{H}_N \textbf{T}_2)_C | \Psi_0 \rangle$ is calculated.

\begin{equation}
\langle \Psi_0 | (\textbf{F}_N \textbf{T}_2)_C | \Psi_0 \rangle = \frac{1}{4} \sum_{pq} \sum_{abij} f_{pq} t_{ij}^{ab} \{ \textbf{a}^{\dag}_p \textbf{a}_q \}
\{ \textbf{a}^{\dag}_a \textbf{a}^{\dag}_b \textbf{a}_i \textbf{a}_j \} .
\end{equation}
This is again a similar situation that will result in 0 contribution. The reason is that any two operators $\textbf{A}$ and $\textbf{B}$ that contain a different number of annihilation/creation operators will not create any fully contracted terms (terms that solely consists of $\delta$'s) when applying Wick's Theorem. This means because of orthogonality the contribution to $E_{CCSD}$ from terms like this will always be 0. \\

$\langle \Psi_0 | (\textbf{V}_N \textbf{T}_2)_C | \Psi_0 \rangle$ however has an equal number of operators. From this we will have a contribution.

\begin{equation}
\langle \Psi_0 | (\textbf{V}_N \textbf{T}_2)_C | \Psi_0 \rangle = \frac{1}{16} \sum_{pqrs} \sum_{abij} \langle pq||rs \rangle t_{ij}^{ab} \langle \Psi_0|
\{
\textbf{a}^{\dag}_p \textbf{a}^{\dag}_q
\textbf{a}_s \textbf{a}_r \}
\{
\textbf{a}^{\dag}_a \textbf{a}^{\dag}_b
\textbf{a}_j \textbf{a}_i \}
| \Psi_0 \rangle \nonumber
\end{equation}
Wick's Theorem is applied. Only the four terms that are fully contracted and non-zero are listed.

\begin{align}
\{
\textbf{a}^{\dag}_p \textbf{a}^{\dag}_q
\textbf{a}_s \textbf{a}_r \}
\{
\textbf{a}^{\dag}_a \textbf{a}^{\dag}_b
\textbf{a}_i \textbf{a}_j \} & = &
\{
\contraction[5ex]{\textbf{a}^{\dag}_p \textbf{a}^{\dag}_q 
\textbf{a}_s}{\textbf{a}_r}{}{\textbf{a}^{\dag}_a}
\contraction[4ex]{\textbf{a}^{\dag}_p \textbf{a}^{\dag}_q}{\textbf{a}_s}{\textbf{a}_r
\textbf{a}^{\dag}_a}{\textbf{a}^{\dag}_b}
\contraction[2ex]{\textbf{a}^{\dag}_p}{\textbf{a}^{\dag}_q}{\textbf{a}_s \textbf{a}_r
\textbf{a}^{\dag}_a \textbf{a}^{\dag}_b}{\textbf{a}_j}
\contraction{}{\textbf{a}^{\dag}_p}{\textbf{a}^{\dag}_q 
\textbf{a}_s \textbf{a}_r \textbf{a}^{\dag}_a \textbf{a}^{\dag}_b \textbf{a}_j}{\textbf{a}_i}
\textbf{a}^{\dag}_p \textbf{a}^{\dag}_q 
\textbf{a}_s \textbf{a}_r
\textbf{a}^{\dag}_a \textbf{a}^{\dag}_b
\textbf{a}_j \textbf{a}_i
\}
+
\{
\contraction[4ex]{\textbf{a}^{\dag}_p \textbf{a}^{\dag}_q}{\textbf{a}_s}{\textbf{a}_r}{\textbf{a}^{\dag}_a}
\contraction[5ex]{\textbf{a}^{\dag}_p \textbf{a}^{\dag}_q 
\textbf{a}_s}{\textbf{a}_r}{\textbf{a}^{\dag}_a}{\textbf{a}^{\dag}_b}
\contraction[2ex]{\textbf{a}^{\dag}_p}{\textbf{a}^{\dag}_q}{\textbf{a}_s \textbf{a}_r
\textbf{a}^{\dag}_a \textbf{a}^{\dag}_b}{\textbf{a}_j}
\contraction{}{\textbf{a}^{\dag}_p}{\textbf{a}^{\dag}_q 
\textbf{a}_s \textbf{a}_r \textbf{a}^{\dag}_a \textbf{a}^{\dag}_b \textbf{a}_j}{\textbf{a}_i}
\textbf{a}^{\dag}_p \textbf{a}^{\dag}_q 
\textbf{a}_s \textbf{a}_r
\textbf{a}^{\dag}_a \textbf{a}^{\dag}_b
\textbf{a}_j \textbf{a}_i
\} \nonumber \\ & &
\{
\contraction[5ex]{\textbf{a}^{\dag}_p \textbf{a}^{\dag}_q 
\textbf{a}_s}{\textbf{a}_r}{}{\textbf{a}^{\dag}_a}
\contraction[4ex]{\textbf{a}^{\dag}_p \textbf{a}^{\dag}_q}{\textbf{a}_s}{\textbf{a}_r
\textbf{a}^{\dag}_a}{\textbf{a}^{\dag}_b}
\contraction[2ex]{\textbf{a}^{\dag}_p}{\textbf{a}^{\dag}_q}{\textbf{a}_s \textbf{a}_r
\textbf{a}^{\dag}_a \textbf{a}^{\dag}_b\textbf{a}_j}{\textbf{a}_i}
\contraction{}{\textbf{a}^{\dag}_p}{\textbf{a}^{\dag}_q 
\textbf{a}_s \textbf{a}_r \textbf{a}^{\dag}_a \textbf{a}^{\dag}_b}{\textbf{a}_j}
\textbf{a}^{\dag}_p \textbf{a}^{\dag}_q 
\textbf{a}_s \textbf{a}_r
\textbf{a}^{\dag}_a \textbf{a}^{\dag}_b
\textbf{a}_j \textbf{a}_i
\}
+
\{
\contraction[4ex]{\textbf{a}^{\dag}_p \textbf{a}^{\dag}_q}{\textbf{a}_s}{\textbf{a}_r}{\textbf{a}^{\dag}_a}
\contraction[5ex]{\textbf{a}^{\dag}_p \textbf{a}^{\dag}_q 
\textbf{a}_s}{\textbf{a}_r}{\textbf{a}^{\dag}_a}{\textbf{a}^{\dag}_b}
\contraction[2ex]{\textbf{a}^{\dag}_p}{\textbf{a}^{\dag}_q}{\textbf{a}_s \textbf{a}_r
\textbf{a}^{\dag}_a \textbf{a}^{\dag}_b\textbf{a}_j}{\textbf{a}_i}
\contraction{}{\textbf{a}^{\dag}_p}{\textbf{a}^{\dag}_q 
\textbf{a}_s \textbf{a}_r \textbf{a}^{\dag}_a \textbf{a}^{\dag}_b}{\textbf{a}_j}
\textbf{a}^{\dag}_p \textbf{a}^{\dag}_q 
\textbf{a}_s \textbf{a}_r
\textbf{a}^{\dag}_a \textbf{a}^{\dag}_b
\textbf{a}_j \textbf{a}_i
\} \nonumber \\
& = & \delta_{pi} \delta_{qj} \delta_{sb} \delta_{ra} 
- \delta_{pi} \delta_{gj} \delta_{rb} \delta_{sa}
+ \delta_{pj} \delta_{qi} \delta_{rb} \delta_{sa}
- \delta_{pj} \delta_{qi} \delta_{ra} \delta_{sb} .
\end{align}
Inserting this provides

\begin{align}
\langle \Psi_0 | (\textbf{V}_N \textbf{T}_2)_C | \Psi_0 \rangle & =  \frac{1}{16} \sum_{pqrs} \sum_{abij}  \langle pq||rs \rangle t_{ij}^{ab} \langle \Psi_0|
\delta_{pi} \delta_{qj} \delta_{sb} \delta_{ra} 
- \delta_{pi} \delta_{gj} \delta_{rb} \delta_{sa}\nonumber \\ & 
+ \delta_{pj} \delta_{qi} \delta_{rb} \delta_{sa}
- \delta_{pj} \delta_{qi} \delta_{ra} \delta_{sb}
| \Psi_0 \rangle \nonumber \\ &
= \frac{1}{16} \sum_{abij} ( 
\langle ij || ab \rangle
- \langle ij || ba \rangle
+ \langle ji || ba \rangle
- \langle ji || ab \rangle ) t_{ij}^{ab} \nonumber \\ &
= \frac{1}{4} \sum_{abij} t_{ij}^{ab} \langle ij||ab \rangle .
\end{align}
Here symmetry considerations was used. This means 

\begin{equation}
\langle \Psi_0 | (\textbf{H}_N \textbf{T}_2)_C | \Psi_0 \rangle = \frac{1}{4} \sum_{abij} t_{ij}^{ab} \langle ij||ab \rangle . \label{Energy_Contribution_2}
\end{equation}
Next contribution from $\langle \Psi_0 | (\textbf{H}_N \textbf{T}_1^2)_C | \Psi_0 \rangle$. From the expression of $\bar{H}$ there is a $\frac{1}{2}$ in front of this term.

\begin{align}
\frac{1}{2} \langle \Psi_0 | (\textbf{H}_N \textbf{T}_1^2)_C | \Psi_0 \rangle = & \frac{1}{8} \sum_{pqrs} \sum_{ai} \sum_{bj} \langle pq || rs \rangle t_i^a t_j^b \nonumber \\ & 
\langle \Psi_0| 
 \{\textbf{a}^{\dag}_a \textbf{a}^{\dag}_b \textbf{a}_i \textbf{a}_j \}
\{\textbf{a}^{\dag}_a \textbf{a}_i \}
\{\textbf{a}^{\dag}_b \textbf{a}_j \}
| \Psi_0 \rangle .
\end{align}
Again Wick's Theorem is used. We note that from $\bar{H}$ there are four creation/annihilation operators. From each $\textbf{T}_1$ there are two, and combined from all there are four. This means we will have non-zero terms, these are listed here.

\begin{align}
\{\textbf{a}^{\dag}_a \textbf{a}^{\dag}_b \textbf{a}_i \textbf{a}_j \}
\{\textbf{a}^{\dag}_a \textbf{a}_i \}
\{\textbf{a}^{\dag}_b \textbf{a}_j \}
 = &
\{
\contraction{}{\textbf{a}^{\dag}_p}{\textbf{a}^{\dag}_q 
\textbf{a}_s \textbf{a}_r
\textbf{a}^{\dag}_a \textbf{a}_i
\textbf{a}^{\dag}_b}{\textbf{a}_j}
\contraction[2ex]{\textbf{a}^{\dag}_p}{\textbf{a}^{\dag}_q}{\textbf{a}_s \textbf{a}_r
\textbf{a}^{\dag}_a}{\textbf{a}_i}
\contraction[4ex]{\textbf{a}^{\dag}_p \textbf{a}^{\dag}_q}{\textbf{a}_s}{\textbf{a}_r
\textbf{a}^{\dag}_a \textbf{a}_i}{\textbf{a}^{\dag}_b}
\contraction[5ex]{\textbf{a}^{\dag}_p \textbf{a}^{\dag}_q 
\textbf{a}_s}{\textbf{a}_r}{}{\textbf{a}^{\dag}_a}
\textbf{a}^{\dag}_p \textbf{a}^{\dag}_q 
\textbf{a}_s \textbf{a}_r
\textbf{a}^{\dag}_a \textbf{a}_i
\textbf{a}^{\dag}_b \textbf{a}_j 
\}
+ 
\{
\contraction[4ex]{\textbf{a}^{\dag}_p \textbf{a}^{\dag}_q}{\textbf{a}_s}{\textbf{a}_r
\textbf{a}^{\dag}_a \textbf{a}_i}{\textbf{a}^{\dag}_b}
\contraction[5ex]{\textbf{a}^{\dag}_p \textbf{a}^{\dag}_q 
\textbf{a}_s}{\textbf{a}_r}{}{\textbf{a}^{\dag}_a}
\contraction{}{\textbf{a}^{\dag}_p}{\textbf{a}^{\dag}_q 
\textbf{a}_s \textbf{a}_r
\textbf{a}^{\dag}_a}{\textbf{a}_i}
\contraction[2ex]{\textbf{a}^{\dag}_p}{\textbf{a}^{\dag}_q}{\textbf{a}_s \textbf{a}_r
\textbf{a}^{\dag}_a \textbf{a}_i
\textbf{a}^{\dag}_b}{\textbf{a}_j}
\textbf{a}^{\dag}_p \textbf{a}^{\dag}_q 
\textbf{a}_s \textbf{a}_r
\textbf{a}^{\dag}_a \textbf{a}_i
\textbf{a}^{\dag}_b \textbf{a}_j 
\} \nonumber \\ & 
\{
\contraction[4ex]{\textbf{a}^{\dag}_p \textbf{a}^{\dag}_q}{\textbf{a}_s}{\textbf{a}_r
\textbf{a}^{\dag}_a \textbf{a}_i}{\textbf{a}^{\dag}_b}
\contraction[5ex]{\textbf{a}^{\dag}_p \textbf{a}^{\dag}_q 
\textbf{a}_s}{\textbf{a}_r}{}{\textbf{a}^{\dag}_a}
\contraction{}{\textbf{a}^{\dag}_p}{\textbf{a}^{\dag}_q 
\textbf{a}_s \textbf{a}_r
\textbf{a}^{\dag}_a}{\textbf{a}_i}
\contraction[2ex]{\textbf{a}^{\dag}_p}{\textbf{a}^{\dag}_q}{\textbf{a}_s \textbf{a}_r
\textbf{a}^{\dag}_a \textbf{a}_i
\textbf{a}^{\dag}_b}{\textbf{a}_j}
\textbf{a}^{\dag}_p \textbf{a}^{\dag}_q 
\textbf{a}_s \textbf{a}_r
\textbf{a}^{\dag}_a \textbf{a}_i
\textbf{a}^{\dag}_b \textbf{a}_j 
\}
+
\{
\contraction[4ex]{\textbf{a}^{\dag}_p \textbf{a}^{\dag}_q}{\textbf{a}_s}{\textbf{a}_r}{\textbf{a}^{\dag}_a}
\contraction[5ex]{\textbf{a}^{\dag}_p \textbf{a}^{\dag}_q 
\textbf{a}_s}{\textbf{a}_r}{\textbf{a}^{\dag}_a}{\textbf{a}^{\dag}_b}
\contraction[2ex]{\textbf{a}^{\dag}_p}{\textbf{a}^{\dag}_q}{\textbf{a}_s \textbf{a}_r
\textbf{a}^{\dag}_a \textbf{a}^{\dag}_b\textbf{a}_j}{\textbf{a}_i}
\contraction{}{\textbf{a}^{\dag}_p}{\textbf{a}^{\dag}_q 
\textbf{a}_s \textbf{a}_r \textbf{a}^{\dag}_a \textbf{a}^{\dag}_b}{\textbf{a}_j}
\textbf{a}^{\dag}_p \textbf{a}^{\dag}_q 
\textbf{a}_s \textbf{a}_r
\textbf{a}^{\dag}_a \textbf{a}^{\dag}_b
\textbf{a}_j \textbf{a}_i
\} \nonumber \\
 = & \delta_{pi} \delta_{qj} \delta_{sb} \delta_{ra} 
- \delta_{pi} \delta_{gj} \delta_{rb} \delta_{sa}
+ \delta_{pj} \delta_{qi} \delta_{rb} \delta_{sa}
- \delta_{pj} \delta_{qi} \delta_{ra} \delta_{sb} .
\end{align}
Which is a result we have seen before in Eq. \eqref{Energy_Contribution_2}, the only difference is the amplitudes and the factor $\frac{1}{2}$.

\begin{equation}
\langle \Psi_0 | (\textbf{H}_N \textbf{T}_1^2)_C | \Psi_0 \rangle = \frac{1}{2} \sum_{abij} t_{i}^{a} t_j^b \langle ij||ab \rangle . \label{Energy_Contribution_3}
\end{equation}
The next term is $\langle \Psi_0 | (\textbf{H}_N \textbf{T}_2^2)_C | \Psi_0 \rangle$. $\textbf{H}_N$ still only have four creation/annihilation operators. However now we have 8 in total from the cluster operators. This means we cannot have fully contracted terms, which means the entire contribution to $E_{CCSD}$ will be 0. \\

This argument will hold true for every single remaining term. Meaning we now have an expression for the energy from Eqs. \eqref{Energy_Contribution_1}, \eqref{Energy_Contribution_2} and \eqref{Energy_Contribution_3}.

\begin{equation}
E_{CCSD} = E_0 + \sum_{ai} f_{ai} t_i^a + \frac{1}{4} \sum_{abij} \langle ij||ab \rangle t_{ij}^{ab} + \frac{1}{2} \sum_{abij} \langle ij || ab \rangle t_i^a t_j^b . \label{CCSD_TOTAL_ENERGY}
\end{equation}
Here all factors are known except for $t_i^a$ and $t_{ij}^{ab}$. These must be determined.

\subsection{$t_i^a$ amplitudes}
We can find expressions for $t_i^a$ by calculating $\langle \Psi_i^a | \bar{H} | \Psi_0 \rangle = 0$. The notation $\Psi_i^a$ means a state with one hole state and one orbital state. This will be an excited state and we did assume orthogonality. The mathematics of this excited state can be described as such

\begin{equation}
\langle \Psi_i^a | = \langle \Psi_0 | \textbf{a}^{\dag}_i \textbf{a}_a . \label{first_excited_stats}
\end{equation}
A creation operator working to the left, on a bra, becomes an annihilation operator. $\langle \Psi_i^a | \bar{H} | \Psi_0 \rangle = 0$ can be solved in the same manner as we did for the energy. Starting with the first term $\langle \Psi_i^a | \textbf{H}_N | \Psi_0 \rangle$.

\subsubsection{$\langle \Psi_i^a | \textbf{H}_N | \Psi_0 \rangle$}

\begin{align}
\langle \Psi_i^a | \textbf{H}_N | \Psi_0 \rangle & = 
\sum_{pq} f_{pq} \langle \Psi_0| \{ \textbf{a}^{\dag}_i \textbf{a}_a \} \{ \textbf{a}^{\dag}_p \textbf{a}_q \} | \Psi_0 \rangle \nonumber \\ &
+ \frac{1}{4} \sum_{pqrs} \langle pq||rs \rangle  \langle \Psi_0| \{ \textbf{a}^{\dag}_i \textbf{a}_a \} \{ \textbf{a}^{\dag}_p \textbf{a}^{\dag}_q \textbf{a}_r \textbf{a}_s \} | \Psi_0 \rangle .
\end{align}
Here the first term is from $\textbf{F}_N$ and the second term from $\textbf{V}_N$. The second term will be zero. Eq. \eqref{first_excited_stats} is inserted and Wick's Theorem applied.

\begin{align}
\Rightarrow \langle \Psi_i^a | \textbf{H}_N | \Psi_0 \rangle & = \sum_{pq} f_{pq} \langle \Psi_0| \{ \textbf{a}^{\dag}_i \textbf{a}_a \} \{ \textbf{a}^{\dag}_p \textbf{a}_q \} | \Psi_0 \rangle \nonumber \\ &
= \sum_{pq} f_{pq} \langle \Psi_0| \{
\contraction{}{\textbf{a}^{\dag}_i}{\textbf{a}_a \textbf{a}^{\dag}_p}{\textbf{a}_q}
\contraction[3ex]{\textbf{a}^{\dag}_i}{\textbf{a}_a}{}{\textbf{a}^{\dag}_p}
\textbf{a}^{\dag}_i \textbf{a}_a \textbf{a}^{\dag}_p \textbf{a}_q 
 \} | \Psi_0 \rangle \nonumber \\ &
= \sum_{pq} f_{pq} \delta_{iq} \delta_{ap} \nonumber \\ &
= f_{ai} . \label{t1amp_1}
\end{align}

\subsubsection{$\langle \Psi_i^a | \textbf{H}_N \textbf{T}_1 | \Psi_0 \rangle$}

The next term includes $(\textbf{H}_N \textbf{T}_1)_c = (\textbf{F}_N \textbf{T}_1)_c + (\textbf{V}_N \textbf{T}_1)_c$. The $()_c$ notation is here applied to specify that there must be at least one contraction reaching from $\textbf{H}_N$ to $\textbf{T}_1$.

\begin{align}
\langle \Phi_i^a | (\textbf{F}_N \textbf{T}_1)_c | \Phi_0 \rangle  & = 
\sum_{pq} \sum_{jb} f_{pq} t_j^b \langle \Phi_0 | 
\{ \textbf{a}^{\dag}_i \textbf{a}_a \} (\{ \textbf{a}^{\dag}_p \textbf{a}_q \} \{
\textbf{a}^{\dag}_b \textbf{a}_j \})_c | \Phi_0 \rangle \nonumber \\ &
= \sum_{pq} \sum_{jb} f_{pq} t_j^b \left(
\{
\contraction{\textbf{a}^{\dag}_i
\textbf{a}_a}{\textbf{a}^{\dag}_p}{\textbf{a}_q
\textbf{a}^{\dag}_b}{\textbf{a}_j}
\contraction[3ex]{}{\textbf{a}^{\dag}_i}{\textbf{a}_a
\textbf{a}^{\dag}_p}{\textbf{a}_q}
\contraction[3ex]{\textbf{a}^{\dag}_i}{\textbf{a}_a}{\textbf{a}_q}{\textbf{a}^{\dag}_p \textbf{a}^{\dag}_b}
\textbf{a}^{\dag}_i
\textbf{a}_a
\textbf{a}^{\dag}_p
\textbf{a}_q
\textbf{a}^{\dag}_b
\textbf{a}_j
\}
+
\{
\contraction[2ex]{}{\textbf{a}^{\dag}_i}{\textbf{a}_a
\textbf{a}^{\dag}_p
\textbf{a}_q
\textbf{a}^{\dag}_b}{\textbf{a}_j}
\contraction{\textbf{a}^{\dag}_i}{\textbf{a}_a}{}{\textbf{a}^{\dag}_p}
\contraction{\textbf{a}^{\dag}_i
\textbf{a}_a
\textbf{a}^{\dag}_p}{\textbf{a}_q}{}{\textbf{a}^{\dag}_b}
\textbf{a}^{\dag}_i
\textbf{a}_a
\textbf{a}^{\dag}_p
\textbf{a}_q
\textbf{a}^{\dag}_b
\textbf{a}_j
\} \right) \nonumber \\ &
= \sum_{pq} \sum_{jb} f_{pq} t_j^b \left(
\delta_{iq} \delta_{ab} \delta_{pj} + 
\delta_{ij} \delta_{ap} \delta_{qb} \right) \nonumber \\ &
= - \sum_j f_{ji} t_j^a + \sum_b f_{ab} t_i^b
.
\end{align}

\begin{align}
\langle \Phi_i^a | (\textbf{V}_N \textbf{T}_1)_c | \Phi_0 \rangle  & = \frac{1}{4} \sum_{pqrs} \sum_{jb} \langle pq||rs\rangle  t_j^b \langle \Phi_0 | 
\{ \textbf{a}^{\dag}_i \textbf{a}_a \} (\{ \textbf{a}^{\dag}_p \textbf{a}^{\dag}_q
\textbf{a}_s \textbf{a}_r \} \{
\textbf{a}^{\dag}_b \textbf{a}_j \})_c | \Phi_0 \rangle \nonumber \\ &
= \frac{1}{4} \sum_{pqrs} \sum_{jb} \langle pq||rs \rangle t_j^b 
(
\{
\contraction[2ex]{}{\textbf{a}^{\dag}_i}{i \textbf{a}_a 
\textbf{a}^{\dag}_p \textbf{a}^{\dag}_q}{\textbf{a}_s}
\contraction{\textbf{a}^{\dag}_i}{\textbf{a}_a}{}{\textbf{a}^{\dag}_p}
\contraction[3ex]{\textbf{a}^{\dag}_i \textbf{a}_a 
\textbf{a}^{\dag}_p}{\textbf{a}^{\dag}_q}{\textbf{a}_s \textbf{a}_r
\textbf{a}^{\dag}_b}{\textbf{a}_j}
\contraction[5ex]{\textbf{a}^{\dag}_i \textbf{a}_a 
\textbf{a}^{\dag}_p \textbf{a}^{\dag}_q
\textbf{a}_s}{\textbf{a}_r}{}{\textbf{a}^{\dag}_b}
\textbf{a}^{\dag}_i \textbf{a}_a 
\textbf{a}^{\dag}_p \textbf{a}^{\dag}_q
\textbf{a}_s \textbf{a}_r
\textbf{a}^{\dag}_b \textbf{a}_j
\} \nonumber \\ &
+ 
\{
\contraction[2ex]{}{\textbf{a}^{\dag}_i}{\textbf{a}_a 
\textbf{a}^{\dag}_p \textbf{a}^{\dag}_q \textbf{a}_s}{\textbf{a}_r}
\contraction{\textbf{a}^{\dag}_i}{\textbf{a}_a}{}{\textbf{a}^{\dag}_p}
\contraction[3ex]{\textbf{a}^{\dag}_i \textbf{a}_a 
\textbf{a}^{\dag}_p}{\textbf{a}^{\dag}_q}{\textbf{a}_s \textbf{a}_r
\textbf{a}^{\dag}_b}{\textbf{a}_j}
\contraction[5ex]{\textbf{a}^{\dag}_i \textbf{a}_a 
\textbf{a}^{\dag}_p \textbf{a}^{\dag}_q
}{\textbf{a}_s}{}{\textbf{a}_r \textbf{a}^{\dag}_b}
\textbf{a}^{\dag}_i \textbf{a}_a 
\textbf{a}^{\dag}_p \textbf{a}^{\dag}_q
\textbf{a}_s \textbf{a}_r
\textbf{a}^{\dag}_b \textbf{a}_j
\}
+ 
\{
\contraction[2ex]
{}
{\textbf{a}^{\dag}_i}
{i \textbf{a}_a \textbf{a}^{\dag}_p \textbf{a}^{\dag}_q}
{\textbf{a}_s}
\contraction
{\textbf{a}^{\dag}_i}
{\textbf{a}_a}
{\textbf{a}^{\dag}_p}
{\textbf{a}^{\dag}_q}
\contraction[3ex]
{\textbf{a}^{\dag}_i \textbf{a}_a}
{\textbf{a}^{\dag}_p}
{\textbf{a}^{\dag}_q \textbf{a}_s \textbf{a}_r \textbf{a}^{\dag}_b}{\textbf{a}_j}
\contraction[5ex]{\textbf{a}^{\dag}_i \textbf{a}_a 
\textbf{a}^{\dag}_p \textbf{a}^{\dag}_q
\textbf{a}_s}{\textbf{a}_r}{}{\textbf{a}^{\dag}_b}
\textbf{a}^{\dag}_i \textbf{a}_a 
\textbf{a}^{\dag}_p \textbf{a}^{\dag}_q
\textbf{a}_s \textbf{a}_r
\textbf{a}^{\dag}_b \textbf{a}_j
\} \nonumber \\ &
+ 
\{
\contraction[2ex]
{}
{\textbf{a}^{\dag}_i}
{i \textbf{a}_a \textbf{a}^{\dag}_p \textbf{a}^{\dag}_q}
{\textbf{a}_s}
\contraction
{\textbf{a}^{\dag}_i}
{\textbf{a}_a}
{\textbf{a}^{\dag}_p}
{\textbf{a}^{\dag}_q}
\contraction[3ex]
{\textbf{a}^{\dag}_i \textbf{a}_a}
{\textbf{a}^{\dag}_p}
{\textbf{a}^{\dag}_q \textbf{a}_s \textbf{a}_r \textbf{a}^{\dag}_b}{\textbf{a}_j}
\contraction[5ex]{\textbf{a}^{\dag}_i \textbf{a}_a 
\textbf{a}^{\dag}_p \textbf{a}^{\dag}_q
}{\textbf{a}_s}{}{\textbf{a}_r \textbf{a}^{\dag}_b}
\textbf{a}^{\dag}_i \textbf{a}_a 
\textbf{a}^{\dag}_p \textbf{a}^{\dag}_q
\textbf{a}_s \textbf{a}_r
\textbf{a}^{\dag}_b \textbf{a}_j
\} ) \nonumber \\ &
=  \frac{1}{4} \sum_{pqrs} \sum_{jb} \langle pq||rs \rangle t_j^b (
\delta_{pj} \delta_{qa} \delta_{rb} \delta_{si} \nonumber \\ &
+ \delta_{pa} \delta_{qj} \delta_{ri} \delta_{sb}
- \delta_{pa} \delta_{qj} \delta_{rb} \delta_{si}
- \delta_{pj} \delta_{qa} \delta_{ri} \delta_{sb} ) \nonumber \\ &
= \sum_{jb} \langle ja||bi \rangle t_j^b .
\end{align}

In total the contribution to the amplitudes from $\langle \Psi_i^a | \textbf{H}_N \textbf{T}_1 | \Psi_0 \rangle$ is

\begin{equation}
\langle \Psi_i^a | (\textbf{H}_N \textbf{T}_1)_c | \Psi_0 \rangle = - \sum_j f_{ji} t_j^a + \sum_b f_{ab} t_i^b
\nonumber +  \sum_{jb} \langle ja||bi \rangle t_j^b . \label{t1amp_2}
\end{equation}

\subsubsection{$\frac{1}{2} \langle \Psi_i^a | (\textbf{H}_N \textbf{T}_1^2)_c | \Psi_0 \rangle$}
Still contributions from $(\textbf{F}_N \textbf{T}_1^2)_c$ and $(\textbf{V}_N \textbf{T}_1^2)_c$ calculated individually. The number of steps in the calculation is now reduced since $\delta_{ij}$ is understood to be a result of a non zero contraction between indices i and j.

\begin{align}
\frac{1}{2} \langle \Psi_i^a | (\textbf{F}_N \textbf{T}_1^2)_c | \Psi_0 \rangle & = 
\frac{1}{2} \sum_{pq} \sum_{jb} \sum_{kc} f_{pq} t_j^b t_k^c \langle \Psi_0 | \{\textbf{a}_i^{\dag} \textbf{a}_a \} \{\textbf{a}_p^{\dag} \textbf{a}_q \} \{\textbf{a}_b^{\dag} \textbf{a}_j \} \{\textbf{a}_c^{\dag} \textbf{a}_k \} | \Psi_0 \rangle \nonumber \\ &
= \frac{1}{2} \sum_{pq} \sum_{jb} \sum_{kc} f_{pq} t_j^b t_k^c
\left( -\delta_{pk} \delta_{qb} \delta_{ij} \delta_{ac}
- \delta_{pj} \delta_{qc} \delta_{ik} \delta_{ab} \right) \nonumber \\ &
= - \sum_{kc} f_{kc} t_i^c t_k^a .
\end{align}

\begin{align}
\frac{1}{2} \langle \Psi_i^a | (\textbf{V}_N \textbf{T}_1^2)_c | \Psi_0 \rangle &
= \frac{1}{8} \sum_{pqrs} \sum_{jb} \sum_{kc} 
\langle pq || rs \rangle t_j^b t_k^c \langle \Psi_0 | 
\{ \textbf{a}^{\dag}_i \textbf{a}_a \} \nonumber \\ &
\{\textbf{a}^{\dag}_p \textbf{a}^{\dag}_q
\textbf{a}_s \textbf{a}_r \} \{\textbf{a}^{\dag}_b \textbf{a}_j \} \{\textbf{a}^{\dag}_c \textbf{a}_k \}
| \Psi_0 \rangle \nonumber \\ &
= \frac{1}{8} \sum_{pqrs} \sum_{jb} \sum_{kc} 
\langle pq || rs \rangle t_j^b t_k^c (
\delta_{pa} \delta_{br} \delta_{sc} \delta_{qk} \delta_{ij} + \dots \nonumber \\ & 
= \sum_{jbs} \langle ja || bc \rangle t_j^b t_i^c
- \sum_{jbk} \langle jk || bi \rangle t_j^b t_k^a .
\end{align}

\subsubsection{Total}
The remaining terms are calculated similarly. For our purposes we list the result. A more complete derivation is available in Ref.\cite{non_refer_numba1}.

\begin{align}
0 = & f_{ai} + \sum_c f_{ac} t_i^c - \sum_k f_{ki} t_k^a + \sum_{kc} \langle ka||ci \rangle t_k^c + \sum_{kc} f_{kc} t_{ik}^{ac} + \frac{1}{2} \sum_{kcd} \langle ka || cd \rangle t_{ki}^{cd} \label{T1equation} \\ &
- \frac{1}{2} \sum_{klc} \langle kl||ci\rangle t_{kl}^{ca} - \sum_{kc} f_{kc} t_i^c t_k^a - \sum_{klc} \langle kl || ci \rangle t_k^c t_l^a + \sum_{kcd} \langle ka||cd \rangle t_k^c t_i^d \nonumber \\ & 
- \sum_{klcd} \langle kl || cd \rangle t_k^c t_i^d t_l^a + \sum_{klcd} \langle kl||cd\rangle t_k^c t_{li}^{da} \nonumber \\ &
 - \frac{1}{2} \sum_{klcd} \langle kl || cd \rangle t_{ki}^{cd} t_l^a 
- \frac{1}{2} \sum_{klcd} \langle kl||cd \rangle t_{kl}^{ca} t_i^d . \nonumber
\end{align}

\subsection{$t_{ij}^{ab}$ amplitudes}
$t_{ij}^{ab}$ amplitudes are generated in a similar fashion. These amplitudes are calculated by solving the equation

\begin{equation}
\langle \Psi_{ij}^{ab} | \bar{H} | \Psi_0 \rangle = 0 . \label{T2equationtosolve}
\end{equation}
This holds true because of the orthogonality. We can describe the state $\langle \Psi_{ij}^{ab}|$ in terms of creation and annihilation operators.

\begin{equation}
\langle \Psi_{ij}^{ab}| = \langle \Psi_0 | \{ \textbf{a}^{\dag}_i \textbf{a}^{\dag}_j \textbf{a}_a \textbf{a}_b \} .
\end{equation}
We can solve Eq. \eqref{T2equationtosolve}. Starting with the contribution from $\textbf{H}_N$. 

\begin{align}
\langle \Psi_{ij}^{ab} | \textbf{F}_N + \textbf{V}_N | \Psi_0 \rangle & = 
\langle \Psi_{0} | \{\textbf{a}^{\dag}_i \textbf{a}^{\dag}_j \textbf{a}_a \textbf{a}_b\} \left( \textbf{F}_N + \textbf{V}_N \right) | \Psi_0 \rangle \nonumber \\ &
= \sum_{pq} f_{pq} \langle \Psi_{0} | \{\textbf{a}^{\dag}_i \textbf{a}^{\dag}_j \textbf{a}_a \textbf{a}_b\} \left( \{ \textbf{a}^{\dag}_p \textbf{a}_q \} \right) | \Psi_0 \rangle \nonumber \\ & 
+ \frac{1}{4} \sum_{pqrs} \langle pq || rs \rangle \langle \Psi_{0} | \{\textbf{a}^{\dag}_i \textbf{a}^{\dag}_j \textbf{a}_a \textbf{a}_b\} \left( \{ \textbf{a}^{\dag}_p \textbf{a}^{\dag}_q 
\textbf{a}_s \textbf{a}_r\} \right) | \Psi_0 \rangle \nonumber \\ &
= \frac{1}{4} \sum_{pqrs} \langle pq || rs \rangle
( \delta_{pa} \delta_{qb} \delta_{ri} \delta_{sj} - \nonumber \\ &
\delta_{pb} \delta_{qa} \delta_{ri} \delta_{sj} - 
\delta_{pa} \delta_{qb} \delta_{si} \delta_{rj} +
\delta_{qa} \delta_{pb} \delta_{si} \delta_{rj} ) \nonumber \\ &
= \sum_{pqrs} \delta_{pb} \delta_{qa} \delta_{ri} \delta_{sj} \nonumber \\ &
= \langle ab || ij \rangle .
\end{align}
The contribution from $\textbf{F}_N$ will be zero. The contribution from $\textbf{H}_N \textbf{T}_1$ is more complicated.

\begin{equation}
\langle \Psi_{ij}^{ab} | (\textbf{F}_N + \textbf{V}_N)\textbf{T}_1 | \Psi_0 \rangle = \langle \Psi_{0} | \{\textbf{a}^{\dag}_i \textbf{a}^{\dag}_j \textbf{a}_a \textbf{a}_b\} \left( \textbf{F}_N + \textbf{V}_N \right) \textbf{T}_1 | \Psi_0 \rangle .
\end{equation}
These will be calculated individually. Here we will only calculate the contribution from $\textbf{V}_N \textbf{T}_1$.

\begin{align}
\langle \Psi_{0} | \{\textbf{a}^{\dag}_i \textbf{a}^{\dag}_j \textbf{a}_a \textbf{a}_b\}  \textbf{V}_N \textbf{T}_1 | \Psi_0 \rangle & = 
\frac{1}{4}
\sum_{pqrs}
\sum_{kc}
\langle pq|| rs \rangle t_k^c \langle \Phi_0 | \{
\textbf{a}^{\dag}_i
\textbf{a}^{\dag}_j
\textbf{a}_a
\textbf{a}_b \} \nonumber \\ & 
\left(
\{
\textbf{a}^{\dag}_p
\textbf{a}^{\dag}_q
\textbf{a}_s
\textbf{a}_r
\}
\{
\textbf{a}^{\dag}_c
\textbf{a}_k
\}
\right)_c | \Phi_0 \rangle
\nonumber \\ & = 
\frac{1}{4} \sum_{pqrs} \sum_{kc} \langle pq||rs\rangle t_k^c ( \delta_{pa} \delta_{qb} \delta_{rc} \delta_{sj} \delta_{ik}  \nonumber \\ & 
- \delta_{pa} \delta_{qb} \delta_{rc} \delta_{si} \delta_{jk}
- \delta_{pa} \delta_{qb} \delta_{rj} \delta_{sc} \delta_{ik}
+ \delta_{pa} \delta_{qb} \delta_{ri} \delta_{sc} \delta_{jk}
\nonumber \\ & 
+ \delta_{pa} \delta_{qk} \delta_{rj} \delta_{si} \delta_{bc}
- \delta_{pa} \delta_{qk} \delta_{ri} \delta_{sj} \delta_{bc}
+ \delta_{pb} \delta_{qa} \delta_{rj} \delta_{sc} \delta_{ik}
 \nonumber \\ & 
- \delta_{pb} \delta_{qa} \delta_{ri} \delta_{sc} \delta_{jk}
+ \delta_{pb} \delta_{qa} \delta_{rc} \delta_{si} \delta_{jk} 
+ \delta_{pb} \delta_{qk} \delta_{ri} \delta_{sj} \delta_{ac} \nonumber \\ & 
- \delta_{pb} \delta_{qk} \delta_{rj} \delta_{si} \delta_{ac}
- \delta_{pb} \delta_{qa} \delta_{rc} \delta_{sj} \delta_{ik}
+ \delta_{pk} \delta_{qa} \delta_{ri} \delta_{sj} \delta_{bc} \nonumber \\ & 
- \delta_{pk} \delta_{qb} \delta_{ri} \delta_{sj} \delta_{ac}
- \delta_{pk} \delta_{qa} \delta_{rj} \delta_{si} \delta_{bc}
+ \delta_{pk} \delta_{qb} \delta_{rj} \delta_{si} \delta_{ac}) \nonumber \\ &
= \sum_c \left( \langle ab || cj \rangle t_i^c - \langle ab || ci \rangle t_j^c \right) \nonumber \\ & 
+ \sum_k \left( \langle ij || bk \rangle t_k^a - \langle ij||ak\rangle t_k^b \right) .
\end{align}
The rest of Eq. \eqref{T2equationtosolve} can be solved in a similar manner. This is a task testing stamina and determination. Here we will simply state the result, however there is a more complete derivation using "Feynman Diagrams" available in Appendix A. \\

Before we state the final result we must define a permutation operator, $\textbf{P}$.

\begin{equation}
\textbf{P}(ab) f(a,b) = f(a,b) - f(b,a) .
\end{equation}
An example of this would be:

\begin{equation}
\textbf{P}(ab) \sum_{abij} t_i^a t_j^b f_{ai} = \sum_{abij} \left( t_i^a t_j^b f_{ai} - t_i^b t_j^a f_{bi} \right) .
\end{equation} 
Using this definition and solving the rest of Eq. \eqref{T2equationtosolve} the expression becomes the following:

\begin{align}
0 = & \langle ab || ij \rangle
+ \textbf{P}(ab) \sum_c f_{bc} t_{ij}^{ac}
- \textbf{P}(ij) \sum_k f_{kj} t_{ik}^{ab}
+ \frac{1}{2} \sum_{kl} \langle kl||ij \rangle t_{kl}^{ab} \label{T2equation} \\ &
+ \frac{1}{2} \sum_{cd} \langle ab || cd \rangle t_{ij}^{cd}
+ \textbf{P}(ij) \textbf{P}(ab) \sum_{kc}
\langle kb||cj \rangle t_{ik}^{ac} \nonumber \\ &
+ \textbf{P}(ij) \sum_c \langle ab || cj \rangle t_i^c
- \textbf{P}(ab) \sum_k \langle kb || ij \rangle t_k^a
\nonumber \\ &
+ \frac{1}{2} \textbf{P}(ij) \textbf{P}(ab) \sum_{klcd}
\langle kl || cd \rangle t_{ik}^{ac} t_{lj}^{db} 
+ \frac{1}{4} \sum_{klcd} \langle kl || cd \rangle
t_{ij}^{cd} t_{kl}^{ab} \nonumber \\ &
-  \frac{1}{2} \textbf{P}(ab)\sum_{klcd} \langle kl || cd \rangle t_{ij}^{ac} t_{kl}^{bd}
- \frac{1}{2} \textbf{P}(ij) \sum_{klcd} \langle kl || cd \rangle t_{ik}^{ab} t_{jl}^{cd} \nonumber \\ &
+ \frac{1}{2} \textbf{P}(ab) \sum_{kl}
\langle kl || ij \rangle t_k^a t_l^b 
+ \frac{1}{2} \textbf{P}(ij) \sum_{cd} \langle ab || cd \rangle t_i^c t_j^d \nonumber \\ &
- \textbf{P}(ij) \textbf{P}(ab) \sum_{kc} \langle kb || ic \rangle t_k^a t_j^c
+  \textbf{P}(ab) \sum_{kc} f_{kc} t_k^a t_{ij}^{bc} 
\nonumber \\ &
+ \textbf{P}(ij) \sum_{kc} f_{kc} t_i^c t_{jk}^{ab}
- \textbf{P}(ij) \sum_{klc} \langle kl || ci \rangle t_k^c t_{lj}^{ab}  \nonumber \\ &
+ \textbf{P} (ab) \sum_{kcd} \langle ka || cd \rangle t_k^c t_{ij}^{db} 
+ \textbf{P}(ij) \textbf{P}(ab) \sum_{kcd} \langle ak || dc \rangle t_i^d t_{jk}^{bc} \nonumber \\ &
+ \textbf{P} (ij) \textbf{P}(ab) \sum_{klc} \langle kl || ic \rangle t_l^a t_{jk}^{bc} 
+ \frac{1}{2} \textbf{P}(ij) \sum_{klc} \langle kl || cj \rangle t_i^c t_{kl}^{ab} \nonumber \\ &
- \frac{1}{2} \textbf{P}(ab) \sum_{kcd} \langle kb || cd \rangle t_k^a t_{ij}^{cd} 
- \frac{1}{2} \textbf{P}(ij) \textbf{P}(ab) \sum_{kcd} \langle kb||cd \rangle t_i^c t_k^a t_j^d \nonumber \\ &
+ \frac{1}{2} \textbf{P}(ij) \textbf{P}(ab) \sum_{klc} \langle kl || cj \rangle t_i^c t_k^a t_l^b
- \textbf{P}(ij) \sum_{klcd} \langle kl || cd \rangle t_k^c t_i^d t_{lj}^{ab} \nonumber \\ &
- \textbf{P}(ab) \sum_{klcd} \langle kl||cd \rangle t_k^c t_l^a t_{ij}^{db}
+ \frac{1}{4} \textbf{P}(ij) \sum_{klcd} \langle kl || cd \rangle t_i^c t_j^d t_{kl}^{ab} \nonumber \\ &
+ \frac{1}{4} \textbf{P}(ab) \sum_{klcd} \langle kl || cd \rangle t_k^a t_l^b t_{ij}^{cd}
+ \textbf{P}(ij) \textbf{P}(ab) \sum_{klcd} \langle kl || cd \rangle t_i^c t_l^b t_{kj}^{ad} \nonumber \\ &
+ \frac{1}{4} \textbf{P}(ij) \textbf{P} (ab) \sum_{klcd} \langle kl || cd \rangle t_i^c t_k^a t_j^d t_l^b . \nonumber
\end{align}
See Ref.\cite{non_refer_numba1} for the full derivation.

\section{Introducing denominators}
The expressions for $t_i^a$ and $t_{ij}^{ab}$ are complex and it is not easy to understand how to implement these equations effectively. The rest of this chapter and the next will be dedicated to simplifying Eqs. \eqref{T1equation} and \eqref{T2equation}.

\subsection{$t_i^a$}
Eq. \eqref{T1equation} should be rewritten considerably before it is programmable. Starting with a definition of $D_i^a$.

\begin{equation}
D_i^a \equiv f_{ii} - f_{aa} . \label{D_i_a_def} 
\end{equation}
Remembering Eq. \eqref{T1equation} we know it starts like this:

\begin{align}
0 = f_{ai} + \sum_c f_{ac} t_i^c - \sum_k f_{ki} t_k^a + \sum_{kc} \langle ka||ci \rangle t_k^c + \sum_{kc} f_{kc} t_{ik}^{ac} + \dots . \label{t1equationstart11}
\end{align}
The term $\sum_c f_{ac} t_i^c$ can be rewritten.

\begin{equation}
\sum_c f_{ac} t_i^c = f_{aa} t_i^a + \sum_c 
(1 - \delta_{ca} ) f_{ac} t_i^c .
\end{equation}
Doing the same with the term $\sum_k f_{ki} t_k^a$ and inserting in Eq. \eqref{t1equationstart11} we get:

\begin{align}
0 = f_{ai} + f_{aa} t_i^a + \sum_c 
(1 - \delta_{ca} ) f_{ac} t_i^c - f_{ii} t_i^a - \sum_k (1 - \delta_{ki}) f_{ki} t_k^a + \dots \nonumber
\end{align}
The two terms $f_{aa} t_i^a$ and $f_{ii} t_i^a$ are combined using the definition Eq. \eqref{D_i_a_def}. 

\begin{equation}
f_{aa} t_i^a - f_{ii} t_i^a = -D_i^a t_i^a .
\end{equation}
This is inserted into Eq. \eqref{T1equation}, and moved to the other side of the equation.

\begin{equation}
D_i^a t_i^a = f_{ai} + \sum_c 
(1 - \delta_{ca} ) f_{ac} t_i^c - \sum_k (1 - \delta_{ki}) f_{ki} t_k^a + \dots
\end{equation}
If we perform the same procedure for $t_{ij}^{ab}$ we can solve this iteratively until self consistency is reached.

\subsection{$t_{ij}^{ab}$}
We also implement a denominator $D_{ij}^{ab}$ in Eq. \eqref{T2equation}. 

\begin{equation}
D_{ij}^{ab} \equiv f_{ii} + f_{jj} - f_{aa} - f_{bb}
\end{equation}
Here we want to create the term $D_{ij}^{ab} t_{ij}^{ab}$. This is done with the same procedure with the two terms $\textbf{P}(ab) \sum_c f_{bc} t_{ij}^{ac}
- \textbf{P}(ij) \sum_k f_{kj} t_{ik}^{ab}$ from Eq. \eqref{T2equation}. These two terms can be expressed in a different manner.

\begin{align}
\textbf{P}(ab) \sum_c f_{bc} t_{ij}^{ac}
- \textbf{P}(ij) \sum_k f_{kj} t_{ik}^{ab} = & 
f_{aa} t_{ij}^{ab} + f_{bb} t_{ij}^{ab} + 
\textbf{P}(ab) \sum_c (1-\delta_{bc}) f_{bc} t_{ij}^{ac} \nonumber \\ &
- \textbf{P}(ij) \sum_k (1-\delta_{kj}) f_{kj} t_{ik}^{ab}
- f_{ii} t_{ij}^{ab}
- f_{jj} t_{ij}^{ab} . \nonumber
\end{align}
Then the four terms where no sums are present are combined into $D_{ij}^{ab} t_{ij}^{ab}$ and moved to the other side of the equation. This leaves another problem which we can solve iteratively.

\begin{equation}
D_{ij}^{ab} t_{ij}^{ab} = \langle ab||ij \rangle + + 
\textbf{P}(ab) \sum_c (1-\delta_{bc}) f_{bc} t_{ij}^{ac} - \dots \nonumber
\end{equation}
Here those three dots represent the rest of Eq. \eqref{T2equation}.

\subsection{Initial guess}
The initial guess from where to start the iterative process can be anything. However it is common to start an initial guess where all the amplitudes on the right side are 0. This leaves

\begin{equation}
t_i^a = \frac{f_{ai}}{D_i^a} .
\end{equation}

\begin{equation}
t_{ij}^{ab} = \frac{\langle ab || ij \rangle}{D_{ij}^{ab}} .
\end{equation}
However it is also common to simply guess $t_i^a = 0$. We will make this initial guess to make benchmarking the number of iterations easier. \\

The iterative procedure is one where $t_i^a$ and $t_{ij}^{ab}$ are updated simultaneously, and in theory we have converged once these amplitudes stop changing. However in practice we define a convergence criteria. This is then compared to the change in energy each iteration with the newly updated amplitudes. We then define convergence to when the energy stop changing, which should for all intents and purposes be an equivalent criteria. 

\section{Variational Principle}
The energy expression in CC contains the operator $e^{-\textbf{T}} \textbf{H} e^{\textbf{T}}$, which is not Hermitian. This means the variational principle no longer applies. It is possible to use the variational principle with CC, but this is a huge complication. This also means it is possible with coupled cluster to get energies lower than the true ground state energy.


 \clearemptydoublepage

% rewrite the jacobi example

\chapter{CCSD Factorization}
CCSD with a closed shell spin restricted Hartree Fock (RHF) basis can simplify our CCSD calculations considerably. In this chapter we will find an algorithm for a serial program to solve these equations using the RHF basis. A serial program is one which does not run in parallel. A lot of work have been done to reduce the number of calculations needed for each iteration. Most of this work involves the definition of intermediate variables that can be calculated separately. \\

The factorization is based on the work of J. F. Stanton and J. Gauss and is listed in Ref.\cite{ccsd_fac1}. However at the time of this thesis there where some typos in the factorized equations, so we will repeat the calculations. Also Ref.\cite{ccsd_fac3} holds information on factorization. Additional information on symmetry is found in an article by P. Carsky, Ref.\cite{ccsd_fac2}. H. Koch et al. also published a paper on CCSD using AOs in the equation, and partly avoiding the AO to MO transformation, Ref.\cite{ccsd_fac4}. This is an alternative to our solution. \\

One simplification we have already discussed is the introduction of $D_i^a$ and $D_{ij}^{ab}$. These denominators are independent of the amplitudes, hence they can be calculated and stored outside any iterative procedure.

\newpage

\section{Constructing an algorithm}
When constructing this CCSD serial algorithm we want a few things in mind. First we wish to minimize the number of FLOPS. Second we want to utilize external linear algebra libraries to perform the calculations. Third we want the option to easily modify the algorithm to work in parallel. This third brings up a few new concerns. One of which is that effective parallel programs are ones that minimize the communication between nodes, in part by  symmetry considerations. This issue will be dealt with in a later chapter in greater detail.

\subsection{Inserting denominators}

\subsubsection{$t_i^a$}
First we insert $D_i^a$ into Eq. \eqref{t1amp_1}. We also insert our notation for $\langle ab || ij \rangle$ which is

\begin{equation}
I_{ij}^{ab} = \langle ab || ij \rangle .
\end{equation}

\begin{align}
D_i^a t_i^a = & 
f_{ai} 
+ \sum_{a\not= c} f_{ac} t_i^c 
- \sum_{i \not= k} f_{ki} t_k^a 
+ \sum_{kc} I_{ka}^{ci} t_k^c 
+ \sum_{kc} f_{kc} t_{ik}^{ac} \nonumber \\ &
+ \frac{1}{2} \sum_{kcd} I_{ka}^{cd} t_{ki}^{cd} 
- \frac{1}{2} \sum_{klc} I_{kl}^{ci} t_{kl}^{ca} 
- \sum_{kc} f_{kc} t_i^c t_k^a 
- \sum_{klc} I_{kl}^{ci} t_k^c t_l^a \nonumber \\ & 
 + \sum_{kcd} I_{ka}^{cd} t_k^c t_i^d 
- \sum_{klcd} I_{kl}^{cd} t_k^c t_i^d t_l^a 
+ \sum_{klcd} I_{kl}^{cd} t_k^c t_{li}^{da} \nonumber \\ &
 - \frac{1}{2} \sum_{klcd} I_{kl}^{cd} t_{ki}^{cd} t_l^a 
- \frac{1}{2} \sum_{klcd} I_{kl}^{cd} t_{kl}^{ca} t_i^d
. 
\end{align}
The calculation of $t_i^a$ scales as $n^6$.

\subsubsection{$t_{ij}^{ab}$}
We do the exact same procedure for $t_{ij}^{ab}$ amplitudes. Insert $D_{ij}^{ab}$ and combine terms into the same sums.

\begin{align}
D_{ij}^{ab} t_{ij}^{ab} = & 
I_{ab}^{ij}
+ \frac{1}{2} \sum_{kl} I_{kl}^{ij} t_{kl}^{ab} 
+ \frac{1}{2} \sum_{cd} I_{ab}^{cd} t_{ij}^{cd}
+ \frac{1}{4} \sum_{klcd} I_{kl}^{cd}
t_{ij}^{cd} t_{kl}^{ab} 
 \nonumber \\ &
- \sum_{k \not= j} f_{kj} t_{ik}^{ab} 
+ \sum_{k \not= i} f_{ki} t_{jk}^{ab}
+ \sum_{c \not= b} f_{bc} t_{ij}^{ac}
- \sum_{c \not= a} f_{ac} t_{ij}^{bc}
 \nonumber \\ &
+ \textbf{P}(ab) 
\{
\sum_{kc} f_{kc} t_k^a t_{ij}^{bc}
- \sum_k I_{kb}^{ij} t_k^a
+ \frac{1}{2} \sum_{kl} I_{kl}^{ij} t_k^a t_l^b 
\nonumber \\ &
+ \sum_{kcd} 
(
I_{ka}^{cd} t_k^c t_{ij}^{db} 
- \frac{1}{2} I_{kb}^{cd} t_k^a t_{ij}^{cd} 
)
+ \sum_{klcd} I_{kl}^{cd} (\frac{1}{4} t_k^a t_l^b t_{ij}^{cd} - t_k^c t_l^a t_{ij}^{db} - \frac{1}{2} t_{ij}^{ac} t_{kl}^{bd})
\}
\nonumber \\ &
+ \textbf{P}(ij)
\{
\sum_c I_{ab}^{cj} t_i^c
+ \frac{1}{2} \sum_{cd} I_{ab}^{cd} t_i^c t_j^d 
+ \sum_{kc} f_{kc} t_i^c t_{jk}^{ab}
\nonumber \\ &
+ \sum_{klc}
( 
\frac{1}{2} 
I_{kl}^{cj} t_i^c t_{kl}^{ab}
- I_{kl}^{ci} t_k^c t_{lj}^{ab}
)
+ \sum_{klcd} I_{kl}^{cd}
(
\frac{1}{4} t_i^c t_j^d t_{kl}^{ab} 
- t_k^c t_i^d t_{lj}^{ab} 
- \frac{1}{2} t_{ik}^{ab} t_{jl}^{cd}
)
\}
\nonumber \\ &
+ \textbf{P}(ab) \textbf{P}(ij)
\{
\sum_{kc}
(
I_{kb}^{cj} t_{ik}^{ac}
- I_{kb}^{ic} t_k^a t_j^c
)
+ \sum_{kcd}
(
I_{ak}^{dc} t_i^d t_{jk}^{bc}
- I_{kb}^{cd} t_i^c t_k^a t_j^d
)
\nonumber \\ &
+ \sum_{klc}
(
\frac{1}{2} I_{kl}^{cj} t_i^c t_k^a t_l^b
+ I_{kl}^{ic} t_l^a t_{jk}^{bc}
)
+ \sum_{klcd} I_{kl}^{cd}
(
t_i^c t_l^b t_{kj}^{ad}
+ \frac{1}{4} t_i^c t_k^a t_j^d t_l^b
+ \frac{1}{2} t_{ik}^{ac} t_{lj}^{db} 
)
\} . 
\end{align}
Calculating this scales as $n^8$ and can go faster with the definition of intermediates. Much of the material presented is based on the work of John F. Stanton and Jurgen Gauss.

\subsection{$[W_1]$}

We first factor out and rewrite the following terms that as they stand now are calculated for all $a,b,i,j$ but only change when $i$ or $j$ changes.

\begin{align}
& 
\frac{1}{2} \sum_{kl} I_{kl}^{ij} t_{kl}^{ab} + \frac{1}{2} \textbf{P}(ij) \sum_{klc} I_{kl}^{cj} t_i^c t_{kl}^{ab} + \textbf{P}(ij) \frac{1}{4} \sum_{klcd} I_{kl}^{cd}  t_i^c t_j^d t_{kl}^{ab}
+ \frac{1}{4} \sum_{klcd} I_{kl}^{cd} t_{ij}^{cd} t_{kl}^{ab}
\nonumber \\ 
= &
\frac{1}{2} \sum_{kl} t_{kl}^{ab} \left[ I_{kl}^{ij} +  \sum_c \left(I_{kl}^{cj} t_i^c - I_{kl}^{ci} t_j^c + \frac{1}{2} \sum_{d}  I_{kl}^{cd} (t_i^c t_j^d - t_j^c t_i^d + t_{ij}^{cd})
 \right) \right]
\nonumber \\ 
= & \frac{1}{2} \sum_{kl} t_{kl}^{ab} [W_1]_{ij}^{kl} 
.
\end{align}
Meaning a new intermediate is now defined.

\begin{equation}
[W_1]^{kl}_{ij} = I_{kl}^{ij} +  \sum_c \left(I_{kl}^{cj} t_i^c - I_{kl}^{ci} t_j^c + \frac{1}{2} \sum_{d}  I_{kl}^{cd} (t_i^c t_j^d - t_j^c t_i^d + t_{ij}^{cd})
 \right) . \label{intermedw1}
\end{equation}
The calculation of this intermediate scales as $n^6$. $[W_1]$ appear one more time in our equations. These terms can also be combined and factorized.

\begin{align}
& \textbf{P}(ab) \frac{1}{2} \sum_{kl} I_{kl}^{ij} t_k^a t_l^b
+ \frac{1}{2} \sum_{klc} \textbf{P}(ij) \textbf{P}(ab) I_{kl}^{cj} t_i^c t_k^a t_l^b 
\nonumber \\ &
+ \frac{1}{4} \textbf{P}(ab) \textbf{P}(ij) \sum_{klcd} I_{kl}^{cd} t_i^c t_k^a t_j^d t_l^b
+ \textbf{P}(ab) \sum_{klcd} I_{kl}^{cd} \frac{1}{4} t_k^a t_l^b t_{ij}^{cd}
\nonumber \\ 
= &
\frac{1}{2} \sum_{kl} (t_k^a t_l^b - t_k^b t_l^a) \left[ I_{kl}^{ij} + \sum_c \left( \textbf{P}(ij) I_{kl}^{cj} t_i^c +
\frac{1}{2} \sum_d I_{kl}^{cd} ( t_i^c t_j^d - t_j^c t_i^d + t_{ij}^{cd}
\right) \right] \nonumber \\ 
= &
\frac{1}{2} \sum_{kl} (t_k^a t_l^b - t_k^b t_l^a) [W_1]_{ij}^{kl} .
\end{align}

\subsection{$[W_2]$}
Now we combine the following terms:

\begin{align}
& - \textbf{P}(ab) \sum_k t_k^a I_{kb}^{ij}
- \textbf{P}(ab) \frac{1}{2} \sum_{kcd} I_{kb}^{cd} t_{ij}^{cd} t_k^a
- \textbf{P}(ab) \textbf{P}(ij) \sum_{kc} t_k^a I_{kb}^{ic} t_j^c 
\nonumber \\
& - \textbf{P}(ab) \textbf{P}(ij) \sum_{kcd} I_{kb}^{cd} t_i^c t_j^d t_k^a
\nonumber \\
= &
- \textbf{P}(ab) \sum_k t_k^a \left[
I_{kb}^{ij} 
+ \sum_c \left( I_{kb}^{ic} t_j^c
- I_{kb}^{jc} t_i^c
+ \frac{1}{2} \sum_{d} I_{kb}^{cd} (t_{ij}^{cd}
+ t_i^c t_j^d - t_j^c t_i^d
) \right) \right]
\nonumber \\
= &
- \textbf{P}(ab) \sum_k t_k^a [W_2]_{ij}^{kb} \nonumber \\
= &
- \textbf{P}(ab) \sum_k t_k^b [W_2]_{ij}^{ak} .
\end{align}
We have now defined another intermediate.

\begin{equation}
[W_2]_{ij}^{ak} = I_{ak}^{ij} 
+ \sum_c \left( I_{ak}^{ic} t_j^c
- I_{ak}^{jc} t_i^c
+ \frac{1}{2} \sum_{d} I_{ak}^{cd} (t_{ij}^{cd}
+ t_i^c t_j^d - t_j^c t_i^d
) \right) . \label{intermedW2}
\end{equation}
This term scales as $n^6$.

\subsection{$[W_3]$}
Our next intermediate is defined by the following two terms:

\begin{equation}
\textbf{P}(ab) \textbf{P}(ij) \sum_{klc} I_{kl}^{ic} t_l^a t_{jk}^{bc}
+ \textbf{P}(ab) \textbf{P}(ij) \sum_{klcd} I_{kl}^{cd} t_i^c t_l^b t_{kj}^{ad} .
\end{equation}

Since c and d are arbitrary indices we can relabel the second term. We can also use a trick with the permutation operators where

\begin{equation}
\textbf{P}(ab) \textbf{P}(ij) f(a,b,i,j) = 
- \textbf{P}(ab) \textbf{P}(ij) f(b,a,i,j) ,
\end{equation}
and also the symmetry of I where

\begin{equation}
I_{kl}^{dc} = - I_{kl}^{cd} ,
\end{equation}

\begin{equation}
\Rightarrow =
\textbf{P}(ab) \textbf{P}(ij) \left(
\sum_{klc} I_{kl}^{ic} t_l^a t_{jk}^{bc}
+ \sum_{klcd} I_{kl}^{cd} t_i^c t_l^a t_{kj}^{bc}
\right) .
\end{equation}
Here we insert some more symmetry considerations.

\begin{equation}
\Rightarrow =
\textbf{P}(ab) \textbf{P}(ij) \left[ \sum_{klc} t_l^a \left(
- I_{kl}^{ci} t_{jk}^{bc} - \sum_d I_{kl}^{cd} t_i^c t_{jk}^{bc} \right) \right] .
\end{equation}
Factorizing out and defining new intermediate.

\begin{align}
\Rightarrow = &
- \textbf{P}(ab) \textbf{P}(ij) \left[ \sum_{klc} t_{jk}^{bc} t_l^a \left(
 I_{kl}^{ci} + \sum_d I_{kl}^{cd} t_i^c \right) \right] \nonumber \\
= &
- \textbf{P}(ab) \textbf{P}(ij) \sum_{klc}
t_{jk}^{bc} t_l^a
 [W_3]_{ci}^{kl} .
\end{align}

\begin{equation}
[W_3]_{ci}^{kl} = I_{kl}^{ci} + \sum_d I_{kl}^{cd} t_i^c  . \label{intermedW3}
\end{equation}
This term scales as $n^5$.

\subsection{$[F_1]$}
Until now all intermediates have been four dimensional. Now we define our first two dimensional intermediate. This is defined from the terms

\begin{align}
& \textbf{P}(ab) \sum_{kc} f_{kc} t_k^a t_{ij}^{bc}
- \textbf{P}(ab) \sum_{klcd} I_{kl}^{cd} t_k^c t_l^a t_{ij}^{db} \nonumber \\ 
= & 
\textbf{P}(ab) \sum_{kc} f_{kc} t_k^a t_{ij}^{bc}
+ \textbf{P}(ab) \sum_{klcd} I_{kl}^{cd} t_l^d t_k^a t_{ij}^{bc} \nonumber \\
= &
\textbf{P}(ab) \sum_{kc} t_k^a t_{ij}^{bc} \left[ f_{kc} + \sum_{ld} I_{kl}^{cd} t_l^d \right] \nonumber \\
= &
\textbf{P}(ab) \sum_{kc} t_k^a t_{ij}^{bc} [F_1]_k^c .
\end{align}

\begin{equation}
[F_1]_k^c = f_{kc} + \sum_{ld} I_{kl}^{cd} t_l^d . \label{intermedF1}
\end{equation}
This intermediate is also repeated when combining the terms

\begin{align}
& \textbf{P}(ij) \sum_{kc} f_{kc} t_i^c t_{jk}^{ab}
- \textbf{P}(ij) \sum_{klcd} I_{kl}^{cd} t_k^c t_i^d t_{lj}^{ab} \nonumber \\ 
= &
- \textbf{P}(ij) \sum_{kc} f_{kc} t_i^c t_{kj}^{ab}
- \textbf{P}(ij) \sum_{klcd} I_{kl}^{cd} t_l^d t_i^c t_{kj}^{ab} \nonumber \\
= &
- \textbf{P}(ij) \sum_{kc} t_i^c t_{kj}^{ab} \left[
f_{kc} + \sum_{cd} I_{kl}^{cd} t_l^d \right] \nonumber \\ 
= &
- \textbf{P}(ij) \sum_{kc} t_i^c t_{kj}^{ab} [F_1]_k^c
.
\end{align}
This term scales as $n^4$. Inserting this into the equations for $t_{ij}^{ab}$ leaves the following:

\begin{align}
D_{ij}^{ab} t_{ij}^{ab} = & 
I_{ab}^{ij}
+ \frac{1}{2} \sum_{kl} (t_k^a t_l^b - t_k^b t_l^a + t_{kl}^{ab}) [W_1]_{ij}^{kl}
+ \frac{1}{2} \sum_{cd} I_{ab}^{cd} t_{ij}^{cd}
 \nonumber \\ &
- \sum_{k \not= j} f_{kj} t_{ik}^{ab} 
+ \sum_{k \not= i} f_{ki} t_{jk}^{ab}
- \sum_{c \not= b} f_{bc} t_{ij}^{ac}
+ \sum_{c \not= a} f_{ac} t_{ij}^{bc}
 \nonumber \\ &
+ \textbf{P}(ab) 
\{
- \sum_k t_k^b [W_2]_{ij}^{ak}
+ \sum_{kc} t_k^a t_{ij}^{bc} [F_1]_k^c
+ \frac{1}{2} \sum_{kl} I_{kl}^{ij} t_k^a t_l^b 
\nonumber \\ &
+ \sum_{kcd} 
(
I_{ka}^{cd} t_k^c t_{ij}^{db} 
)
+ \sum_{klcd} I_{kl}^{cd} 
(
\frac{1}{4} t_k^a t_l^b t_{ij}^{cd} 
- \frac{1}{2} t_{ij}^{ac} t_{kl}^{bd}
)
\}
\nonumber \\ &
+ \textbf{P}(ij)
\{
- \sum_{kc} t_i^c t_{kj}^{ab} [F_1]_k^c
+ \sum_c I_{ab}^{cj} t_i^c
+ \frac{1}{2} \sum_{cd} I_{ab}^{cd} t_i^c t_j^d 
\nonumber \\ &
+ \sum_{klc}
( 
- I_{kl}^{ci} t_k^c t_{lj}^{ab}
)
+ \sum_{klcd} I_{kl}^{cd}
(
- \frac{1}{2} t_{ik}^{ab} t_{jl}^{cd}
)
\}
\nonumber \\ &
+ \textbf{P}(ab) \textbf{P}(ij)
\{
\sum_{kc}
(
I_{kb}^{cj} t_{ik}^{ac}
)
+ \sum_{kcd}
(
I_{ak}^{dc} t_i^d t_{jk}^{bc}
)
\nonumber \\ &
+ \sum_{klc}
(
\frac{1}{2} I_{kl}^{cj} t_i^c t_k^a t_l^b
- t_{jk}^{bc} t_l^a [W_3]_{ci}^{kl}
)
+ \sum_{klcd} I_{kl}^{cd}
(
\frac{1}{4} t_i^c t_k^a t_j^d t_l^b
+ \frac{1}{2} t_{ik}^{ac} t_{lj}^{db} 
)
\} .
\end{align}

\subsection{$[F_2]$}
We now combine the terms:

\begin{align}
& \sum_{k \not= i} f_{ki} t_{jk}^{ab}
- \sum_{k \not= j} f_{kj} t_{ik}^{ab} 
- \textbf{P}(ij) \sum_{kc} t_i^c t_{kj}^{ab} [F_1]_k^c
\nonumber \\ &
- \textbf{P}(ij) \sum_{klcd} I_{kl}^{cd} \frac{1}{2} t_{ik}^{ab} t_{jl}^{cd}
- \textbf{P}(ij) \sum_{klc} I_{kl}^{ci} t_k^c t_{lj}^{ab} .
\end{align}
We notice that the two terms $\sum_{k \not= i} f_{ki} t_{jk}^{ab}$ and $\sum_{k \not= j} f_{kj} t_{ik}^{ab}$ can be written in terms of $\textbf{P}(ij)$ and $\delta_{ki}$.

\begin{align}
\Rightarrow = &
\textbf{P}(ij) \left[
\sum_{k} (1 - \delta_{ki}) f_{ki} t_{jk}^{ab}
- \sum_{kc} t_i^c t_{kj}^{ab} [F_1]_k^c
- \sum_{klcd}  I_{kl}^{cd} \frac{1}{2} t_{ik}^{ab} t_{jl}^{cd}
- \sum_{klc} I_{kl}^{ci} t_k^c t_{lj}^{ab}
\right]
\nonumber \\
= &
\textbf{P}(ij) \left[
\sum_{k} (1 - \delta_{ki}) f_{ki} t_{jk}^{ab}
+ \sum_{kc} t_i^c t_{jk}^{ab} [F_1]_k^c
+ \sum_{klcd}  I_{kl}^{cd} \frac{1}{2} t_{jk}^{ab} t_{il}^{cd}
+ \sum_{klc} I_{kl}^{ic} t_l^c t_{jk}^{ab}
\right]
\nonumber \\
= &
\textbf{P}(ij) \sum_k t_{jk}^{ab} \left[
(1 - \delta_{ki}) f_{ki}
+ \sum_c t_i^c [F_1]_k^c
+ \frac{1}{2} \sum_{lcd} I_{kl}^{cd} t_{il}^{cd}
+ \sum_{lc} I_{kl}^{ic} t_l^c
\right]
\nonumber \\
= &
\textbf{P}(ij) \sum_k t_{jk}^{ab} [F_2]_i^k .
\end{align}

\begin{equation}
[F_2]_i^k = (1 - \delta_{ki}) f_{ki}
+ \sum_c \left[t_i^c [F_1]_k^c
+ \sum_{l} \left( I_{kl}^{ic} t_l^c 
+ \frac{1}{2} \sum_{d} I_{kl}^{cd} t_{il}^{cd}
\right)
\right] .
\label{intermedF2}
\end{equation}
This term scales as $n^5$.

\subsection{$[F_3]$}
We now combine terms within the $\textbf{P}(ab)$ operator.

\begin{align}
& - \sum_{c \not= b} f_{bc} t_{ij}^{ac}
+ \sum_{c \not= a} f_{ac} t_{ij}^{bc}
- \textbf{P}(ab) \sum_{kc} t_k^a t_{ij}^{bc} [F_1]_k^c
- \textbf{P}(ab) \sum_{klcd} \frac{1}{2} I_{kl}^{cd} t_{ij}^{ac} t_{kl}^{bd}
\nonumber \\ &
+ \textbf{P}(ab) \sum_{kcd} I_{ka}^{cd} t_k^c t_{ij}^{db} \nonumber \\
= &
\textbf{P}(ab) \left[
+ \sum_c (1-\delta_{ca}) f_{ac} t_{ij}^{bc}
- \sum_{kc} t_k^a t_{ij}^{bc} [F_1]_k^c
- \sum_{klcd} \frac{1}{2} I_{kl}^{cd} t_{ij}^{ac} t_{kl}^{bd}
+ \sum_{kcd} I_{ka}^{cd} t_k^c t_{ij}^{db} 
\right]
\nonumber \\
= &
\textbf{P}(ab) \left[
+ \sum_c (1-\delta_{ca}) f_{ac} t_{ij}^{bc}
- \sum_{kc} t_k^a t_{ij}^{bc} [F_1]_k^c
- \sum_{klcd} \frac{1}{2} I_{kl}^{cd} t_{ij}^{bc} t_{kl}^{ad}
+ \sum_{kcd} I_{ka}^{dc} t_k^d t_{ij}^{bc} 
\right]
\nonumber \\
= &
\textbf{P}(ab) \sum_c t_{ij}^{bc}
\left[
(1-\delta_{ca}) f_{ac}
+ \sum_{k} \left( - t_k^a [F_1]_k^c
+ \sum_{d} \left( I_{ka}^{dc} t_k^d 
- \sum_{l} \frac{1}{2} I_{kl}^{cd} t_{kl}^{ad} \right) \right)
\right] \nonumber \\
= &
\textbf{P}(ab) \sum_c t_{ij}^{bc} [F_3]_c^a .
\end{align}

\begin{equation}
[F_3]_c^a = (1-\delta_{ca}) f_{ac}
- \sum_{k} \left[ t_k^a [F_1]_k^c
+ \sum_{d} \left( I_{ka}^{cd} t_k^d 
- \frac{1}{2} \sum_{l} I_{kl}^{cd} t_{kl}^{ad} \right) \right] . \label{intermedF3}
\end{equation}
This term scales as $n^5$.

\subsection{$[W_4]$}
Now we combine the terms inside $\textbf{P}(ab) \textbf{P}(ij)$. 

\begin{align}
& \textbf{P}(ab) \textbf{P}(ij) \sum_{kc} I_{kb}^{cj} t_{ik}^{ac}
+ \textbf{P}(ab) \textbf{P}(ij) \sum_{kcd} I_{ak}^{dc} t_i^d t_{jk}^{bc}
\nonumber \\ &
- \textbf{P}(ab) \textbf{P}(ij) \sum_{klc} t_{jk}^{bc} t_l^a [W_3]_{ci}^{kl}
+ \textbf{P}(ab) \textbf{P}(ij) \sum_{klcd} I_{kl}^{cd} \frac{1}{2} t_{ik}^{ac} t_{lj}^{db}
\nonumber \\
= & \textbf{P}(ab) \textbf{P}(ij) t^{bc}_{jk} \left[
\sum_{kc} I_{ka}^{ci} 
+ \sum_{kcd} I_{ak}^{dc} t_i^d
- \sum_{klc} t_l^a [W_3]_{ci}^{kl}
+ \frac{1}{2} \sum_{klcd} I_{kl}^{cd} t_{lj}^{db}
\right] \nonumber \\ 
= &
\textbf{P}(ab) \textbf{P}(ij)  \sum_{kc} t^{bc}_{jk} \left[
I_{ak}^{ic} 
+ \sum_{d} I_{ak}^{dc} t_i^d
- \sum_{l} t_l^a [W_3]_{ci}^{kl}
+ \frac{1}{2} \sum_{ld} I_{kl}^{cd} t_{il}^{ad}
\right] .
\end{align}

\begin{equation}
[W_4]_{ic}^{ak} = 
I_{ak}^{ic} 
+ \sum_{d} I_{ak}^{dc} t_i^d
- \sum_{l} t_l^a [W_3]_{ci}^{kl}
+ \frac{1}{2} \sum_{ld} I_{kl}^{cd} t_{il}^{ad} .
\label{intermedW4}
\end{equation}
In this formulation symmetries where used extensively, the term scales as $n^6$.

\subsection{Inserting intermediates}
Inserting Eqs. \eqref{intermedF1}, \eqref{intermedF2}, \eqref{intermedF3}, \eqref{intermedW2}, \eqref{intermedW3}, \eqref{intermedW4} and \eqref{intermedw1} then leaves the following:

\begin{align}
D_{ij}^{ab} t_{ij}^{ab} = & 
I_{ab}^{ij} +
\frac{1}{2} \sum_{kl} (t_{kl}^{ab} + t_k^a t_l^b - t_l^a t_k^b) [W_1]_{ij}^{kl}
- \textbf{P}(ab) \sum_k t_k^b [W_2]_{ij}^{ak}
\nonumber \\ &
+ \textbf{P}(ij) \sum_k t_{jk}^{ab} [F_2]_i^k
+ \frac{1}{2} \sum_{cd} I_{ab}^{cd} t_{ij}^{cd}
+ \textbf{P}(ab) \sum_c t_{ij}^{bc} [F_3]_c^a
\nonumber \\ &
+ \textbf{P}(ij) \sum_c I_{ab}^{cj} t_i^c
+ \textbf{P}(ij) \frac{1}{2} \sum_{cd} I_{ab}^{cd} t_i^c t_j^d 
+ \textbf{P}(ab) \textbf{P}(ij) \sum_{kc} t_{jk}^{bc} [W_4]_{ic}^{ak} .
\end{align}
We can also define an intermediate $\tau_{ij}^{ab}$.

\begin{equation}
\tau_{ij}^{ab} = t_{ij}^{ab} + t_i^a t_j^b - t_j^a t_i^b . \label{intermedtau}
\end{equation}
This will reduce the number of calculations required, but not by any factor of $n$. Hence using this is a debate of speed versus memory. With this intermediate the equation for $\textbf{T}_2$ amplitudes look like this:

\begin{align}
D_{ij}^{ab} t_{ij}^{ab} = & 
I_{ab}^{ij} +
\frac{1}{2} \sum_{kl} \tau_{kl}^{ab} [W_1]_{ij}^{kl}
- \textbf{P}(ab) \sum_k t_k^b [W_2]_{ij}^{ak}
\nonumber \\ &
+ \textbf{P}(ij) \sum_k t_{jk}^{ab} [F_2]_i^k
+ \frac{1}{2} \sum_{cd} I_{ab}^{cd} \tau_{ij}^{cd}
+ \textbf{P}(ab) \sum_c t_{ij}^{bc} [F_3]_c^a
\nonumber \\ &
+ \textbf{P}(ij) \sum_c I_{ab}^{cj} t_i^c
+ \textbf{P}(ab) \textbf{P}(ij) \sum_{kc} t_{jk}^{bc} [W_4]_{ic}^{ak} . \label{LINK_THIS_SHIT_1_T2}
\end{align}

\subsection{Inserting into $t_i^a$}

If we combine the terms in the following manner

\begin{align}
D_i^a t_i^a = &
- \sum_{k} t_k^a
\left[
(1 - \delta_{ki}) f_{ki}
+ \sum_c t_i^c
f_{kc}
+ \sum_{lc} I_{kl}^{ic} t_l^c
+ \sum_{lcd} I_{kl}^{cd} (\frac{1}{2} t_{il}^{cd} + t_l^d)
\right]
\nonumber \\ &
+ f_{ai} 
+ \sum_{c \not= a} f_{ac} t_i^c
+ \sum_{kc} I_{ka}^{ci} t_k^c 
- \frac{1}{2} \sum_{klc} t_{kl}^{ca}
\left[
I_{kl}^{ci} + \sum_d I_{kl}^{cd} t_i^d
\right]
\nonumber \\ &
+ \sum_{kc} t_{ik}^{ac} 
\left[
f_{kc} + \sum_{ld} I_{kl}^{cd} t_l^d
\right]
+ \frac{1}{2} \sum_{kcd} I_{ka}^{cd} t_{ki}^{cd} 
+ \sum_{kcd} I_{ka}^{cd} t_k^c t_i^d 
 .
\end{align}
They reduce to the following:

\begin{align}
D_i^a t_i^a = &
- \sum_{k} t_k^a
[F_2]_i^k
+ f_{ai} 
+ \sum_{c \not= a} f_{ac} t_i^c
+ \sum_{kc} I_{ka}^{ci} t_k^c 
- \frac{1}{2} \sum_{klc} t_{kl}^{ca} [W_3]_{kl}^{ic}
\nonumber \\ &
+ \sum_{kc} t_{ik}^{ac} [F_1]_k^c
+ \frac{1}{2} \sum_{kcd} I_{ka}^{cd} t_{ki}^{cd} 
+ \sum_{kcd} I_{ka}^{cd} t_k^c t_i^d 
 .
\end{align}
This now scales as $n^6$. The largest scaling factor throughout our algorithm now is $n^6$ whereas prior to intermediates it was $n^8$. This is significantly faster, but we do have 8 (or 7 if $\tau_{ij}^{ab}$ is excluded) intermediates which must be stored. It should be noted that $t_k^c t_i^d = \frac{1}{2} (t_k^c t_i^d - t_i^c t_k^d)$. This can be used to insert $\tau_{ij}^{ab}$ in the equations for $t_i^a$ and the energy. 

\begin{align}
D_i^a t_i^a = &
- \sum_{k} t_k^a
[F_2]_i^k
+ f_{ai} 
+ \sum_{c \not= a} f_{ac} t_i^c
+ \sum_{kc} I_{ka}^{ci} t_k^c 
- \frac{1}{2} \sum_{klc} t_{kl}^{ca} [W_3]_{kl}^{ic}
\nonumber \\ &
+ \sum_{kc} t_{ik}^{ac} [F_1]_k^c
+ \frac{1}{2} \sum_{kcd} I_{ka}^{cd} \tau_{ki}^{cd} 
.  \label{LINK_THIS_SHIT_1_T1}
\end{align}

\section{SSLRS}

The science team of Scuseria, Scheiner, Lee, Rice and Schaefer are usually refered to as SSLRS. This team have developed some of the most efficient algorithms for our purposes, Ref.\cite{sslrs_citation2}. They have defined quite different intermediates, and I would like to also present their algorithm in short for comparison. If the reader want to develop their own CCSD algorithm this would be a good alternative. See also Ref.\cite{sslrs_citation1}.

\subsection{Description of algorithm}

In their algorithm the amplitude equations are defined with intermediates as the following:

\begin{align}
- D_i^a t_i^a = &
- f_{ai} 
- \sum_{c \not= a} f_{ac} t_i^c 
+ \sum_{k \not= i} f_{ki} t_k^a
+ \sum_{kc} f_{kc} (2t_{ik}^{ac} - \tau_{ik}^{ca})
+ \sum_k g_i^k t_k^a \label{SSRS1} \\ &
- \sum_c g_c^a t_i^c 
- \sum_{klc} \left( 2 [D_1]_{li}^{ck} - [D_1]_{ki}^{cl} \right) t_l^c t_k^a \nonumber \\ &
- \sum_{kc} [2(D_{2A} - D_{2B}) + D_{2C}]_{ci}^{ka} 
- 2 \sum_k [D_1]_{ik}^{ak}  \nonumber \\ &
+ \sum_{kc} v_{ic}^{ka} t_k^c
- \sum_{klc} v_{cl}^{ak} ( 2 t_{ki}^{cl} - t_{ik}^{cl} )
+ \sum_{ljc} v_{ci}^{jl} ( 2 t_{lj}^{ac} - t_{jl}^{ac} ) .
\nonumber
\end{align}

\begin{equation}
- D_{ij}^{ab} t_{ij}^{ab} = v_{ij}^{ab} + J_{ij}^{ab} + J_{ji}^{ba} + S_{ij}^{ab} + S_{ji}^{ba} . \label{SSRS2}
\end{equation}

\begin{equation}
E_{CCSD} = E_{HF} + 2\sum_{ia} f_{ia} t_a^i 
+ \sum_{abij} v_{ij}^{ab} ( 2 \tau_{ij}^{ab}
+ \tau_{ji}^{ab} ) . \label{SSRS3}
\end{equation}
The minus signs on the left side of the equation are present due to the fact that SSLRS use a different definition of the denominators, noted here in the equations are our definitions. Their definitions is equal to ours except for this minus sign. \\

$v_{ij}^{ab}$ is a short notation used by SSLRS for $\langle ab || ij \rangle$. It should be mentioned that the brackets surrounding for example $[D_1]$ is a notation where the brackets are a part of the variable. The rest of the intermediates are now defined.

\begin{align}
\tau_{ij}^{ab} = & t_{ij}^{ab} + t_i^a t_j^b
\\ 
J_{ij}^{ab} = &
\sum_{c \not= a} f_{ca} t_{ij}^{cb}
- \sum_{k \not= i} f_{ik} t_{kj}^{ab}
+ \sum_{kc} f_{kc} (t_{ij}^{cb} t_k^b + t_{ik}^{ab} t_j^c )
+ \sum_c g_c^b t_{ij}^{ac} - \sum_k g_j^k t_{ik}^{ab} .
\end{align}

\begin{align}
S_{ij}^{ab} = & 
\frac{1}{2} [B_2]_{ij}^{ab}
- [E_1^*]_{ij}^{ab}
+ [D_{2A}]_{ab}^{ij}
+ [F_{12}]_{ab}^{ij} 
 \\ &
+ \sum_{kc} (
[D_{2A}]_{kj}^{cb} - [D_{2B}]_{kj}^{cb}
+ 2[F_{12}]_{kj}^{cb}
- [E_1^*]_{cj}^{kb} )
( t_{ik}^{ac} - \frac{1}{2} t_{ki}^{ac} )
\nonumber \\ &
+ \sum_{kc} (\frac{1}{2} [D_{2C}]_{kj}^{cb}
- v_{kj}^{bc} - [F_{11}]_{kj}^{bc} + [E_{11}]_{kj}^{bc}) t_{ic}^{ak}
\nonumber \\ &
+ \sum_{kc} (\frac{1}{2} [D_{2C}]_{ki}^{cb} - v_{ki}^{bc} - [F_{11}]_{ki}^{bc} 
+ [E_{11}]_{ki}^{bc} ) t_{cj}^{ak}
\nonumber \\ &
+ \sum_{cd} ( \frac{1}{2} [D_2]_{ij}^{cd} 
+ \frac{1}{2} v_{ij}^{cd} + \frac{1}{2} (
[E_1]_{ij}^{cd} + [E_1]_{ji}^{dc}))
\tau_{cd}^{ab}
\nonumber \\ &
+ \sum_{kc} \{
([D_{2C}]_{ci}^{kb} - v_{ci}^{bk}) t_j^c
- ([D_{2A}]_{cj}^{kb} 
- [D_{2B}]_{cj}^{kb} ) t_i^c
- ([D_1]_{ji}^{bk} + [F_2]_{ji}^{bk}) \} t_k^a .
\nonumber
\end{align}
The definition of g depends on which index is where. Remembering $\textbf{a}$ is an index indicating an occupied orbital and $\textbf{i}$ is an index indicating a Fermi hole.

\begin{align}
g_i^a = & 
2 \sum_b [F_{11}]_{ib}^{ab} 
- \sum_b [F_{12}]_{ib}^{ba}
- \sum_b [D_{2A}]_{ib}^{ba}
+ \sum_{bc} (
[D_1]_{bc}^{ic} - 2[D_1]_{cb}^{ib} ) t_c^a
\\
g_a^i = &
\sum_c \{2([E_1]_{ic}^{ac} + [D_2]_{ic}^{ac} )
- ([E_1]_{ic}^{ca} + [D_2]_{ic}^{ca}) \} .
\end{align}
The rest of the intermediates in the SSLRS algorithm is now defined in order of appearance. 

\begin{equation}
[D_1]_{ij}^{ab} = \sum_k v_{ik}^{ab} t_j^k .
\end{equation}

\begin{equation}
[D_{2A}]_{ij}^{ab} = \sum_{kc}
v_{ci}^{ka} (2 t_{jc}^{ka} - t_{jc}^{ak}) .
\end{equation}

\begin{equation}
[D_{2B}]_{ij}^{ab} = \sum_{kc}
v_{ci}^{ak} t_{jc}^{ka} .
\end{equation}

\begin{equation}
[D_{2C}]_{ij}^{ab} = \sum_{kc}
v_{ci}^{ak} t_{jc}^{ak} .
\end{equation}

\begin{equation}
[B_2]_{ij}^{ab} = \sum_{kl} v_{kl}^{ab} t_{ij}^{kl} .
\end{equation}

\begin{equation}
[E_1^*]_{ij}^{ab} = \sum_k v_{ij}^{ak} t_k^b .
\end{equation}

\begin{equation}
[F_{12}]_{ij}^{ab} = \sum_c
v_{ic}^{ab} t_j^c .
\end{equation}

\begin{equation}
[F_{11}]_{ij}^{ab} = \sum_c v_{ic}^{ba} t_j^c .
\end{equation}

\begin{equation}
[E_{11}]_{ij}^{ab} \sum_c v_{ij}^{cb} t_c^a .
\end{equation}

\begin{equation}
[D_2]_{ij}^{ab} = \sum_{cd} v_{cd}^{ab} t_{ij}^{cd} .
\end{equation}

\begin{equation}
[E_1]_{ij}^{ab} = \sum_c v_{ic}^{ab} t_j^c .
\end{equation}

\begin{equation}
[F_2]_{ij}^{ab} = \sum_{cd} v_{cd}^{ab} \tau_{ij}^{cd} .
\end{equation}

\subsection{Scaling}
Specific notice should be paid to $[D_{2A}]$, $[D_{2B}]$, $[D_{2C}]$ and $[F_{2}]$. These intermediates scale as with a factor of $n^6$, where n is the number of orbitals. However it is not necessary to loop over all orbitals as index i,j,k refer to unoccupied orbitals. Indexes a,b,c refer to occupied orbitals. The scaling then reduces to $n_v^3 n_o^3$. Overall the Eq. \eqref{SSRS1}, \eqref{SSRS2} and \eqref{SSRS3} scales as 

\begin{equation}
\frac{1}{2} n_v^4 n_o^2 + 7 n_v^3 n_o^3 + \left( \frac{1}{2} + \frac{1}{2} \right) n_v^2 n_o^4 . 
\end{equation} 
It should be noted that SSLRS does propose this algorithm with the purpose of not only low scaling, but also avoid storage a large number of variables. In their papers they do mention another algorithm with slightly better scaling, but they do argue the extra needed variable storage this requires is a worrying aspect. For this reason the SSLRS algorithm is presented like this in this thesis.\\

Another positive aspect of this algorithm is the emphasis on matrix use. All the $n^6$ and $n^5$ scaling parts of the algorithm is designed so that matrix multiplication libraries can be used. These libraries are often specially designed to be efficient. \\

SSLRS algorithm also scales as $n^6$, but the number of intermediates are much larger. For this reason we chose the prior algorithm.

\section{TCE}
There are more than one way to factorize the CCSD equations. The best factorization actually depends on the system of interest. This has prompted the interest in the Tensor Contracted Engine, TCE. \\

TCE is an automated code generator for computational chemistry methods. Once the system of interest is known, the TCE aims to construct the optimal code. We will not go in detail on TCE, but additional information can be found in Ref.\cite{tce_citation_numbah_10}. 






 \clearemptydoublepage
\chapter{Comments Prior to Implementation}
 
\begin{quotation}
If God has made the world a perfect mechanism, he has at least
conceded so much to our imperfect intellect that in order to predict
little parts of it, we need not solve innumerable differential
equations, but can use dice with fair success.  
{\em Max Born, quoted  in H.~R.~Pagels, The Cosmic Code \cite{pagels1982}}
\end{quotation}

\abstract{This chapter aims at giving an overview on some of the most
  used methods to solve ordinary differential equations. Several
  examples of applications to physical systems are discussed, from the
  classical pendulum to the physics of Neutron stars.}


In this chapter we discuss a few external libraries used in the implementation and how they work. Also we will discuss a few guiding principles we will apply to our implementation later. We will mainly discuss armadillo, MPI and general parallel programming. We will also mention OpenMP and external math libraries. The external math library we will use is Intel MKL.

\section{Armadillo}
Armadillo is a C++ linear algebra library. The library is designed to be similar to matlab in syntax. It provides good speed relative to other libraries and makes it easy to utilize matrix or vector multiplications in an efficient way. The armadillo documentation is available in Ref.\cite{armadillo-ref1}. Armadillo is also available for other programming languages, but we will strictly focus on the C++ version.

\subsection{Armadillo Types}
Armadillo has its own objects. We will use four objects in armadillo. The first three are vector, matrix and cube. These are simply put one, two and three dimensional arrays defined by standard to contain numerical values of double precision. The last is field. A field in armadillo is a two dimensional array that can contain other things than numerical values. A field can contain things like strings, vectors, matrices or cubes. Anything that can be used in combination with the "=" operator and a copy, like memcpy(). A field can be defined like this:

\begin{equation}
field<mat> A \nonumber
\end{equation}
This defines a two dimensional field of matrices, meaning a four dimensional array. The field is called "A". This can be accessed by first two indexes for the field and next two indexes for the matrix. A(0,1) for example is the matrix with indexes 0 and 1 in the field. A(0,1)(2,3) is a double precision number with indexes 2,3 in the matrix located in indexes 0,1 in the field A.

\subsection{Matrix Operations}
Matrix multiplication can be utilized very easy in armadillo. If there are three matrices defined, A, B and C, we can easily call on matrix multiplication by stating:

\begin{lstlisting}
C = A * B 
\end{lstlisting}
Other operations available are additions, subtractions, element wise multiplications and element wise divisions. These are accessed in order like this:

\begin{lstlisting}
C = A + B;
C = A - B;
C = A % B;
C = A / B;
\end{lstlisting}
Also there is a function called accu(C). This is an accumulation function that accumulates the values of C, where C can be a vector, matrix or a cube. If used in combination we can define

\begin{lstlisting}
D = accu(A % B)
\end{lstlisting}
This leaves D as a double precision number. Here A and B are first element wise multiplied and the resulting matrix accumulated. A and B must be same size. This will be used in our implementation. Take for example the term

\begin{equation}
D_{ij}^{ab} t_{ij}^{ab} \leftarrow \sum_{cd} I_{ab}^{cd} \tau_{ij}^{cd} . \label{armadilloterm}
\end{equation}

If we store I as a field with indexing I(a,b)(c,d) and $\tau_{ij}^{cd}$ is stored as a field with indexing $\tau$(i,j)(c,d) then we can use element wise multiplication and accumulation to calculate Eq. \eqref{armadilloterm}.

\begin{equation}
\sum_{cd} I_{ab}^{cd} \tau_{ij}^{cd}
= accu(I(a,b) \% \tau(i,j)) .
\end{equation}
Here I(a,b) is a matrix and $\tau$(i,j) is a matrix of same size. A major positive of armadillo is that it is possible to link other effective external math libraries to perform the actual matrix operations. We can link BLAS, LAPACK, OpenBLAS and many others. Armadillo initiates calls to these libraries automatic and effectively if installed properly. We then get the effectiveness of the best external math libraries, and the simplicity of the armadillo syntax.

\subsection{Element access}
An interesting feature when using armadillo is the way we access elements. In C syntax one usually allocates an array using malloc(). This gives great control over memory accessing, as we can have even multidimensional arrays sequential in memory. \\

In armadillo we usually allocate a matrix with just mat A. We can have a field of matrices with field<mat>. However each element in the field must be allocated on its own. Also armadillo has a few checks in place to ensure a bug free working code. Based on performance and experience, this is not efficient. \\

Other types in armadillo such at mat or vector is quite efficient. However when using this library it is important to be aware that not all types in armadillo are as efficient when it comes to memory access. Usually it is best to stick with one or two dimensional arrays as much as possible, even make temporary vectors or matrices to avoid accessing a field to much. Trial and error is thus a good tool if we want to use armadillo for high performance computing. This comment applies to armadillo at the time of this thesis. The armadillo developers are continuously working to improve the performance. \\

Another problem we encountered in our implementation is that OpenMPI does not take armadillo types in its communication functions. Armadillo does have functions that can help modestly in this regard, like the function .memptr(). However, we found it was not an optimal combination.

\section{Parallel Computing and OpenMPI}
The Open Source Message Passing Interface, OpenMPI, will be used in our implementation, Ref.\cite{openmpi_cite}. OpenMPI is a library that makes parallel computing much easier. It removed the need for low level parallel programming. In this section we will discuss briefly why we need parallel computing and what it is.

\subsection{The CPU \label{the_cpu_section}}
The CPU, or the Central Processing Unit, is the brain of the computer. This unit processes instructions, many of which requires transfer from or to the memory on a computer. \\

The CPU integrates many components, such as registres, FPUs and caches. The CPU has a "clock" that synchronizes the logic units within the CPU each clock cycle to precess instructions. Among other things this clock allows us to accurately measure the time used from one section of the program to another. \\

Instructions are put in a pipeline for the CPU to execute. While the CPU is processing instructions, it also looks down the pipeline, to see what instructions it will need to perform soon and what values it will need. These values can then pre-emptively placed in the cache. In the cache they are faster to access when they are needed. \\

If the CPU makes a wrong guess on for example an if test, the pipeline is filled with instructions that should not be processed and must be flushed. This slows down performance. \\

A supercomputer consists of nodes. Each node has a number of CPUs. On the abel super computing cluster, each node has 16 CPUs. 

\subsection{The Compiler}
The compiler allows the CPU to understand easy syntax such as C++. The compiler takes the code as input and produce the .o file. In the .o file there are instructions for the CPU to process. The CPU only understands the .o file, as such we must always compile our code. The compiler also creates the pipeline. We want the pipeline to be as optimal as possible, for this reason we need a good compiler. \\

A normal compiler performs a three step procedure. Step one is to check the code for syntax errors, include problems and other basics. \\

Step two is to translate the code into an intermediate language. Here optimizations are performed. If we wrote a code segment like this

\begin{lstlisting}
double A, B, C;
A = 50;
B = 20;
C = B * B * B;

// More calculations

B *= A;
\end{lstlisting}

The compiler will take note that the variable A is defined early on, and used much later. A is defined, stored into memory, then read from memory and finally it takes part in calculations. The compiler can rearrange our code to optimize this segment. 

\begin{lstlisting}
double B, C;
B = 20;
C = B * B * B;

// More calculations

double A = 50;
B *= A;
\end{lstlisting}

Here the variable A is defined and used directly. It is defined were we need it and ready in the pipeline for calculations. If the pipeline for some reason is filled with wrong instructions, we will not take advantage of optimizations such as this. \\

Step three is to output the .o file. Compilers are extremely complex, we should mention this was a brief and simplified description. 

\subsection{Data}
Data is stored in memory as a sequence of 0s and 1s. One 0 or 1 occupies one bit. 8 bits is one byte. The memory is read as bytes. Even a bool which can either be true or false is one byte. A bool of value true is stored in memory as 00000001. \\

Other types of data have different sizes in memory. An int is 4 bytes, a float 4 bytes and a double 8 bytes. Data is usually stored in memory, or rapid access memory, RAM. Here we can access it faster than from disk. \\

On a supercomputer, each node has a fixed number of memory available. The CPUs on the node can share this memory, or we can distribute it into smaller chunks were each CPU has its own unshared memory. 

\subsection{Bandwidth}
The bandwidth is a measure of number of bytes transferred per second. The bandwidth is a feature of the hardware, we will look at it as a constant value. We will be dealing with software. \\

If we want to send an array of 100 doubles from one computer to another, this will be 800 bytes. If we sent it as floats, it would be 400 bytes. If we assume the bandwidth is same, it would be twice as fast to transfer floats than doubles. \\

However, we will always use double precision values. But also in situations where we can reduce the size of the array, if it for example is symmetric. If we reduce the size by half, to 50 doubles or 400 bytes, we have saved much time in communication. \\

The communication inside a node is quite fast on a supercomputer. However when we need to use multiple nodes at once there are challenges. The nodes are not at the same physical distance to each other, this means we can not achieve the same bandwidth between different nodes. Abel supercomputer has nodes stacked in a rack. The nodes inside the same rack are closer. The bandwidth is usually higher in communication inside a rack, relative to communication between nodes in different racks. 

\subsection{Designing Parallel Algorithms}
When we design a parallel algorithm we look for hotspots. These are computation intense areas, and a parallel implementation should be designed to work good around this area. \\

However we must be careful, as communication can sometimes overtake computation. This can happen even in computation intense areas. A measure known as Granularity is known as the ratio of computation versus communication.

\subsection{Performance}
A serial algorithm is evaluated by its runtime. The runtime of a parallel program depends on input size , number of processors and the communication. This is a multidimensional problem, and not so easy to measure. \\

Sometimes we can use two different algorithms to solve the same problem. One algorithm may be more efficient in serial, while the other is more efficient in parallel. \\

To measure how good performance our parallel algorithm gives, it must be measured against the best serial algorithm for the given problem. This is true even if the best serial algorithm is in no way close to the same as the parallel algorithm. \\

One could imagine a serial program that solves a problem in 10 seconds, but is impossible to run in parallel. And we can imagine another algorithm that solves it in 100 seconds, but runs easily in parallel. If we run the second algorithm with 5 CPUs, and say this takes 20 seconds. It would still be better to use the first serial algorithm. The parallel performance can only be described as not good. This is true until you can run the second algorithm in less time than 10 seconds. \\

A good model of performance we will use is the Speedup, S.

\begin{equation}
S(p) = \frac{T_0}{T_p} .
\end{equation}
With ideal performance $S(p)$ is linear, preferably $S(p) = p$. As we noted in section \ref{the_cpu_section} values needed for calculation are pre-emptively placed in the catche. If a CPU cannot fit all values needed for calculations in the catche, the CPU must get these values from main memory. This slows down performance considerably. \\

We consider a large array we want to use in calculations. It is twice as large as the catche. If we introduce two CPUs, we can split the array in two and fit it in the catche. We will then avoid the performance loss from memory accessing. This creates the possibility of super linear scaling. This is a situation where we double the number of CPUs, and get more than a doubling in performance. Figure \ref{super_linear_scaling} is an illustration of different types of scaling.

\begin{figure}[ht!]
\centering
\includegraphics[width=90mm]{scaling_plots_examples_1000.eps}
\caption{Illustration of possible scaling plots. Linear, Super linear and non linear is plotted.}
\label{super_linear_scaling}
\end{figure}

\subsection{Overhead}
If we try to solve a problem using two processor, it will normally not be twice as fast as it would be on one processor. This is because of overhead. Overhead are things like wasted computations, communication and latency. \\

Wasted computation would be additional computations required for running the algorithm in parallel. Latency is the time interval required to initiate a communication, and also to tell the processors that the communication is completed. \\

Runtime in a serial program is often denoted as $T_S$. The time from the first processor to start, until the last processor exits, is often noted a parallel runtime, $T_P$. The overhead, $T_O$, can then be described as

\begin{equation}
T_O = p T_P - T_S ,
\end{equation}
where p is the number of processors. Overhead is commonly increased as we increase the number of processors.

\subsection{General Parallel Guidelines}
For this implementation we will use a few simple guidelines with MPI. First we want to minimize the number of initiated communications. This is to reduce latency. When a communication is initiated, processors are syncronized. This means all processors enter into an MPI function at the same time. If one CPU is faster than another, this CPU will have to be idle and wait for the others to reach the communication function. This is undesireable. Also when exiting a communication, CPUs does not exit at the same time. This is another reason for minimizing the number of synchronizations. \\

Second we want to minimize the number of bytes to be communicated, mainly through symmetries. We want to design our algorithm specifically for this purpose. Third we want to use OpenMPI, which has optimized functions for communication implemented. In Appendix A we list many of these functions, with a short description of what they do.

\subsection{Optimizing Communication}

We will not go in detail on how the MPI functions are optimized. A good book on the subject is Ref. \cite{mpi_boka_cite_referanse}. We will only entertain a small example. \\

\begin{figure}[ht!]
\centering
\includegraphics[width=90mm]{mpi_communication.jpg}
\caption{Illustration of two scenarios. Scenario one is a naive implementation of a broadcast. Scenario two is one example on how performance can be improved.}
\label{mpi_communication_illustration_thingy}
\end{figure}

This example is illustrated in figure \ref{mpi_communication_illustration_thingy}. Say we have four processors. We want to broadcast a message from processor 1 to all the others. We distinguish the first processor by its rank, it is rank 1. If rank 1 sends its message to rank 2, then rank 3 and then rank 4, there must be three communications performed by rank 1. One communication must wait for the other to finish in this example. \\

However, if rank 1 sends its message to rank 2. And then rank 1 sends to rank 3 at the same time as rank 2 sends to rank 4, there has only been two individual communication procedures by rank 1. This gives a better performance. \\

Each vertical line in figure \ref{mpi_communication_illustration_thingy} represents one send and recieve with MPI. A MPI\_Send and MPI\_Recieve scales as

\begin{equation}
t = t_s + m t_b .
\end{equation}
Here $t_s$ is the startup time, $t_b$ is the bandwidth and m is the number of bytes. In scenario one we would be performing (P-1) such send/recieves, where P is the number of MPI procs.

\begin{equation}
T_1 = (P-1) \times (t_s + m t_b) .
\end{equation}
In scenario two we still perform send/recieves, but since they can now be done simontaniously the number of send/recieves scales as $\lceil log_2(P) \rceil$.

\begin{equation}
T_2 = \lceil log_2(P) \rceil \times (t_s + m t_b) .
\end{equation}

Ideally we do not want the number of sends/recieves performed by one CPU to increase at all when we increase the number of processors. \\

\newpage

\begin{figure}[ht!]
\centering
\includegraphics[width=90mm]{timeconsumption_scenario_one_vs_two.eps}
\caption{Illustration of communication in the two scenarios. Plotted are the number of send/recieves that must be performed.}
\label{mpi_communication_illustration_thingy2}
\end{figure}

We also perform performance tests of the actual performance on abel of the OpenMPI broadcast function. Results are presented in figure \ref{mpi_communication_real}. We notice there are indeed optimizations present from the non linear scaling. 

\begin{figure}[ht!]
\centering
\includegraphics[width=90mm]{mpi_bcast_scaling.eps}
\caption{Illustration of actual communication with different number of CPUs. Time is measured for 100 Broadcasts with $8 \times 70^4$ bytes.}
\label{mpi_communication_real}
\end{figure}

\subsection{Optimizing Work Distribution \label{work_dist_section_1341}}
Also in parallel programming it is important that all processors get assigned the same workload. We think of the workload as a series of jobs that can be executed in parallel. \\

First we must define what is one job and second we must distribute these jobs among processors. Imagine running the following calculation in parallel.

\begin{equation}
K = \sum_{ijkl}^N X_{ij} Y_{kl} Z_{lk} .
\end{equation}
We first factorize it.

\begin{equation}
Z = \sum_{ij}^N X_{ij} \times \sum_{kl}^N Y_{kl} Z_{lk} .
\end{equation}
We will look at two possible definitions of one job in this scenario. First, we define a job by its job ID. This job ID can for example be expressed as a function of i and j. For example

\begin{equation}
job\_ID = i + j . \label{example_job_distribution}
\end{equation}
If we choose this definition we have $N \times N$ jobs to distribute. Alternatively we can define a job ID as a function of i, j, k and l. For example

\begin{equation}
job\_ID = i + j + k + l .
\end{equation}
We would then have $N^4$ jobs to distribute. If we have $N^4$ jobs, we can  use $N^4$ CPUs at max. If we however chose to prior job definition, we could only use $N^2$ CPUs. In general we want to have as many jobs as possible to distribute, but sometimes this can lead to additional communication. \\

Another important feature is to optimize the job distribution. If we chose one job\_ID to be expressed by i and j, we can visualize the job\_ID in a matrix. Each column is a different index i, and each row is an index j. The matrix elements are the job\_IDs. If we use N = 4 and Eq. \eqref{example_job_distribution} we would get

\begin{center}
\begin{tikzpicture}

        \matrix [matrix of math nodes,left delimiter=(,right delimiter=)] (m)
        {
            0 & 1 & 2 & 3 \\
            1 & 2 & 3 & 4 \\
            2 & 3 & 4 & 5 \\
            3 & 4 & 5 & 6 \\
        };  
    \end{tikzpicture}
\end{center}
We want all jobs to have a different ID. We redefine the job\_ID to be

\begin{equation}
job\_ID = i \times N + j .
\end{equation}
Using this definition our matrix of job\_IDs becomes

\begin{center}
\begin{tikzpicture}

        \matrix [matrix of math nodes,left delimiter=(,right delimiter=)] (m)
        {
            0 & 1 & 2 & 3 \\
            4 & 5 & 6 & 7 \\
            8 & 9 & 10 & 11 \\
            12 & 13 & 14 & 15 \\
        };  
    \end{tikzpicture}
\end{center}
Each matrix element represent a job\_ID. Here each job has got its unique ID, and it is easier to distribute. When we distribute work we must use the MPI rank and total number of MPI procs, p. For example we can define a condition for each processor that must be true if the processor are to perform the job.

\begin{lstlisting}
if (job_ID % p = rank){
   // Perform job
}
\end{lstlisting}
From the perspective of our CPUs we can use this relation to identify our job distribution. Noted now in the matrix is what processor performs which job. We assume we have p = 8.

\begin{center}
\begin{tikzpicture}

        \matrix [matrix of math nodes,left delimiter=(,right delimiter=)] (m)
        {
            0 & 1 & 2 & 3 \\
            4 & 5 & 6 & 7 \\
            0 & 1 & 2 & 3 \\
            4 & 5 & 6 & 7 \\
        };  
    \end{tikzpicture}
\end{center}
We see the job distribution is optimal, because the amount of work for all CPUs are identical. If we used the job\_ID defined in Eq. \eqref{example_job_distribution} the amount of work for each processor would not be the same. This is a sub-optimal work distribution. 

\subsection{Why Parallel}
Constructing a parallel program seems like quite the challenging feature. Every year there are new processors released with improved performance. Why do we not just wait for a great CPU that can solve all our problems? The reason we do not wait for this, is that it will never happen. The problem with great performance CPUs is that their power consumption is generally very high. A CPU with twice the performance generally needs 3 or 4 times the power, according to a lecture from Intel on parallel programming, Ref.\cite{intelduden_citeation}. It is therefore much more feasable to have several CPUs with less performance, than one high performance CPU. \\

So not only does parallel programming enable us to perform calculations faster and on larger systems, it also requires less power. Power consumption is the limiting factor in CPU performance today. Parallel programming is thus very important, and likely to become even more important in the future. On a sidenote, this is a reason why GPUs have become so popular in scientific programming, GPUs are optimized for performance per watt. Christoffer Hirth wrote extensively about this in his thesis, Ref.\cite{non_refer_numba1}. His principles has not been incorporated in our implementation, but is a likely source of further performance gains.

\section{OpenMP}
OpenMP is another library for parallel programming. It is developed by Intel and can be activated in most compilers. OpenMP use shared memory model. Here the main memory is available on all processors. The key word here is main memory, as each processor has its own cache. OpenMP is very easy to get started with. We will not be using it, but more information is available in Ref.\cite{openmp_citation_po_g}.

\section{External Math Libraries}
External Math Libraries are optimized for performance. They have built-in functions to handle matrix-matrix multiplications, vector-matrix multiplications, and similar problems. The best libraries are the likes of OpenBLAS, Ref.\cite{openblas_citation}, and Intel MKL, Ref.\cite{mkl_citation}. For our implementation we will make use of MKL on the Abel computing cluster. These libraries often give a huge performance gain in matrix operations, relative to a naive for-loop implementation. \\

We also mention that MKL comes with a parallel version, in where it makes use of OpenMP. Both these libraries are developed by Intel. For our purposes we only made use of the serial version. 









\section{Input File}
This section will deal with the user friendly part of our program. This means easy input. We must be able to define what method to use, what atoms and where they are placed and other input variables. We want these to be defined in a separate textfile, to ensure the user  never needs to recompile or edit any code. The input file must be named "INCAR". \\

\begin{figure}[h!]
\begin{center}
\fbox{\includegraphics[width=\textwidth]{inputfile100.eps}}
\caption{Example of input file for our program}
\label{fig:inputfile100}
\end{center}
\end{figure}

An example of input file is given in figure \ref{fig:inputfile100}. This is the only file we need to change the system or method in use. Our program uses the standard fstream library to read the textfile. We then go through it searching for keywords. The keywords are defined to be the leftmost word in each line. \\

Basis\_Set is the first keyword. Here we can choose from a variety of basis sets and the program will make use of this. The current options are STO-3G, 3-21G, 4-31G, 6-31G, 6-311ss, 6-311-2d2p and 6-311-3d3p. Most of basis sets are implemented for all atoms for which they are available. \\

The next keyword is Method. Here the choices are HF, CCSD, CCSDT-1a, CCSDT-1b, CCSDT-2, CCSDT-3, CCSDT-4 and CCSDT. The CCSDT part will be discussed in the next chapter. \\

convergence\_criteria is defined to be $10^{n}$, where n is given in the textfile. -8.0 gives a convergence criteria of $10^{-8}$. The same convergence criteria is used for all methods. \\

Relax\_Pos is meant to call a relaxation procedure, but this is not jet implemented in the program. \\

use\_angstrom gives the user the option to give atomic coordinates in angstrom, instead of atomic units. The options here are true or false. If it is set to true the coordinates are transformed to atomic units inside the program. \\

print\_stuffies is a variable that gives the user the option if he wants extra values printed. If this is set to true there are several interesting numbers printed during calculations. If this is set to false we only print the final energy. This option is added for a situation where we want to perform several hundred smaller calculation. A situation where we most likely are only interested in a final number.. \\

Freeze\_Core is an option available for CCSD. This is not jet implemented. \\

The next few lines give the atoms and its positions in x, y and z. The first letter is used to determine the number of electrons and nuclei charge. The program does not deal with ions. The simplicity of atom positions and charge is an advantage. Usually in computational chemistry packages the number of atom types and number of atoms of each type must be defined. Here we keep it simple and user friendly. \\

The input file stops searching for keywords once they are all found. Hence the user is free to put comments for self in the input file, as long as they are not placed next to keywords or inside the ATOMS section.

\section{General Code Overview}
In this section we describe the general overview of the code. We will present this as figures, and fill inn the blanks throughout the remaining sections. Each class will be described by its input, output and internal workings. \\

\begin{figure}[h!]
\begin{center}
\fbox{\includegraphics[width=\textwidth]{structure.eps}}
\caption{Code structure}
\label{fig:structure}
\end{center}
\end{figure}

The first class in use is the initializer. This class takes the input from main and makes sure we use it correctly. If angstroms is used as units, the coordinates are transformed to atomic units. If we want extra print options, this is ensured here. We also define a Hartree Fock object in this class, since all methods in computational chemistry generally start with a HF calculation. \\

We then make sure the correct method is called, and pass the HF object. For this reason we drew an arrow from HF to initializer only in figure \ref{fig:structure}, since it is now passed as an object to the other methods. 


\section{Hartree Fock}
In this section we discuss the HF implementation in detail. Our Hartree Fock implementation is grounded in the class hartree\_fock\_solver. The main function is called Get\_Energy. In this function we will calculate the HF energy. The main outlay can be seen in figure \ref{fig:hfimp}. \\

\newpage

\begin{figure}[h!]
\begin{center}
\fbox{\includegraphics[width=\textwidth]{hf_imp.jpg}}
\caption{Basic Outlay of HF Implementation. First column is what action is done, second column is in what class this action takes place}
\label{fig:hfimp}
\end{center}
\end{figure}

The code is described in the text. We have also included key lines from the code itself to better illustrate the implementation. \\

\subsubsection{Filling numbers from EMSL}
\begin{lstlisting}
   // Set Matrix Sizes
   matrix_size_setter matset(Z, Basis_Set, n_Nuclei);
   Matrix_Size = matset.Set_Matrix_Size();

   // Fill numbers from EMSL
   Fill_Alpha Fyll(n_Nuclei, Z, Basis_Set,
Matrix_Size, matset.Return_Max_Bas_Func());
   alpha = Fyll.Fyll_Opp_Alpha();
   c = Fyll.Fyll_Opp_c();
   n_Basis = Fyll.Fyll_Opp_Nr_Basis_Functions();
   Number_Of_Orbitals = Fyll.Fyll_Opp_Antall_Orbitaler();
   Potenser = Fyll.Fyll_Opp_Potenser();
\end{lstlisting}

The first procedure performed in this function is to call the matrix\_size\_setter class. We make an object of this class and send the basis set in use and which atoms are in play. This class then returns how large our arrays must be. We then allocate these arrays. \\

The next step is going to the fill\_alpha class. This class contains data from EMSL, and fills up this data in arrays. The array alpha is filled with values for $\alpha_i$, the array c is filled with values of $c_i$ and Potenser is filled with the angular momentum. We also make a one dimensional array, Number\_Of\_Orbitals, which holds information on how many basis functions are in use for a specific atom. The array n\_Basis holds how many primitives each of these basis functions consist of. \\

\subsubsection{Normalizing GTOs}
\begin{lstlisting}
   // Normalize coefficients from EMSL
   Normalize_small_c();
\end{lstlisting}

The next step is to multiply in the normalization constant. This is multiplied in with the array c, through the function Normalize\_small\_c. In this function we have implemented the equations from section \ref{normalization_section}. \\

\subsubsection{Overlap Integrals}
The next step is to make an object of the class hartree\_integrals. Inside this class we will eventually calculate all the integrals we need. However the first step is to fill up an array of $E_t^{ij}$. These values are present in all our integrals. For this we use Eqs. \eqref{important_hf1}, \eqref{important_hf2} and \eqref{important_hf3}. The values for $E_t^{ij}$ will be calculated for all combinations of two primitive GTOS. This enables us to reuse the values in all our integrals, even the electron-electron repulsion. \\

\begin{lstlisting}
   // Precalculations
   HartInt.Fill_E_ij();

   // Overlap
   O = HartInt.Overlap_Matrix();
\end{lstlisting}

We then calculate the integrals. The overlap is stored in a matrix S and is calculated using Eq. \eqref{overlap_integral}. \\

\subsubsection{Kinetic Energy}
The two index integrals are stored in a matrix EK. EK consists of our kinetic energy and the nuclei-electron interaction. \\ 

\begin{lstlisting}
   // One electron operator
   EK = -0.5*HartInt.Kinetic_Energy()+
   - HartInt.Nuclei_Electron_Interaction();
\end{lstlisting}

The kinetic energy is calculated using Eq. \eqref{EKintegralsss}. \\

\subsubsection{Hermite Integrals}
For the nuclei-electron interaction we need to calculate the Hermite Integrals, $R_tuv^n$. We make a new function to calculate these called Set\_R\_ijk. This function implements the equations given in Eqs. \eqref{nucelec_0_int}, \eqref{nucelec_1_int}, \eqref{nucelec_2_int} and \eqref{nucelec_3_int}. We include the implementation of Eqs. \eqref{nucelec_0_int} and  \eqref{nucelec_1_int}. We put the values in a global four dimensional array R\_ijk.

\begin{lstlisting}
void Hartree_Integrals::Set_R_ijk(double p, int t, int u, int v, rowvec R1, rowvec R2)
{
   int t_max,nn,i,j,k,tt = t,uu = u,vv = v;
   t_max = t+u+v;
   double Boys_arg;
   rowvec Rcp(3);
   Rcp = R1-R2;
   Boys_arg = p*dot(Rcp, Rcp);
   Boys_arg = Boys(Boys_arg, 0);
   
   // Initialize R^n_0,0,0
   for (nn=0; nn<(t_max+1);nn++){
      R_ijk.at(nn)(0,0,0) = pow(-2*p, nn) *    F_Boys(nn);
   }

   // Fill up R^n_i,0,0
   for (i=0; i<tt; i++){
      for (nn=0; nn<(t_max-i); nn++){
         R_ijk.at(nn)(i+1,0,0) = Rcp(0) * R_ijk.at(nn+1)(i,0,0);
         if (i > 0){
            R_ijk.at(nn)(i+1,0,0) += i * R_ijk.at(nn+1)(i-1,0,0);
         }
      }
   }
   
   // Rest of Set_R_ijk function   
\end{lstlisting}
We here make use of the Boys function.

\subsubsection{Boys Function}
The function that calculates a value for the Boys function is called Boys. We have two equations we can use, Eq. \eqref{boys_int_1} and Eq. \eqref{boys_int_2}. One works for small x, the other for large x. We define everything less than x = 50 to be small, and everything greater or equal to 50 to be large. We include the implementation of large x. \\

\begin{lstlisting}
if (x > 50){
   Set_Boys_Start(N);
   F = Boys_Start / pow(2.0, N+1) * sqrt(M_PI/pow(x, 2*N+1));
}
\end{lstlisting}
For small x we Taylor expand around zero. We choose M = 100 in Eq. \eqref{boys_int_1} for the Taylor expansion. 

\begin{lstlisting}
else{
   double F=0, sum=0;
   int M;
   for (int j=0; j<100; j++){
      sum = pow(2*x, j);
      M = 2*N+1;
      while (M < (2*N+2+2*j)){
         sum /= M;
         M += 2;
      }
      F += sum;
   }
   F *= exp(-x);
}
\end{lstlisting}
We then use the recursive relation in Eq. \eqref{boys_int_3}. 

\begin{lstlisting}
F_Boys(N) = F;
for (int i = N; i > 0; i--){
   F_Boys(i-1) = (2*x*F_Boys(i) + exp(-x))/(2*i-1);
}
\end{lstlisting}
We are left with designing a value to N. N denotes the starting $F_n$ value, from which we will iterate down to the approximate solution. Popular here is putting N as some function of angular momentum, like $N = 6 \times l$. However we just put it to 30. In this value we were able to recreate all the benchmark values in Refs. \cite{boys_referanse_1} and \cite{boys_referanse_2}. These articles discuss the numerical calculation of the Boys function. Using N = 30 our results were also in agreement with the rest of the Computational Physics Group.

\subsubsection{Nuclei-Electron Interaction}
With these functions we can implement nuclei-electron interaction as given in Eq. \eqref{final_nuclei_electron_thang}. We add these into the array EK. 

\subsubsection{Electron-Electron Interaction}
The electron-electron repulsion integrals are stored in a four dimensional field, field\_Q. They are calculated through a function called Calc\_Integrals\_On\_The\_Fly. This function takes the input of four orbitals, i, j, k, l, and returns its value for $\langle i j | k l \rangle$. The function is an implementation of Eq. \eqref{electron_electron_int_1_1}, also using Eq. \eqref{electron_electron_int_1_2}. \\

\begin{lstlisting}
double Calc_Integrals_On_The_Fly(int orb1, int orb2, int orb3, int orb4)
{    
    int i,j,k,m;

    // Figure out what atom the AO belongs to, need atomic position
    i = Calc_Which_Atom_We_Are_Dealing_With(orb1);
    j = Calc_Which_Atom_We_Are_Dealing_With(orb3);
    k = Calc_Which_Atom_We_Are_Dealing_With(orb2);
    m = Calc_Which_Atom_We_Are_Dealing_With(orb4);

    // Here we calculate the two electron integrals
    // We have already stored E_ij^t so we reuse these
    // Symmetry considerations are applied elsewhere.

    int E_counter1, E_counter2; // These ensures we get the right E_ij^t
    int n,p,o,q; // Index for primitive GTO
    double temp = 0;
    E_counter1 = E_index(orb1,orb2);
    for (n=0; n<n_Basis(orb1); n++)
    {
        for (p=0; p<n_Basis(orb2); p++)
        {
            E_counter2 = E_index(orb3, orb4);
            for (o=0; o<n_Basis(orb3); o++)
            {
                for (q=0; q<n_Basis(orb4); q++)
                {
                    temp += c(orb1,n)*c(orb2,p)*c(orb3,o)*c(orb4,q)*
                            HartInt.Electron_Electron_Interaction_Single
                (orb1, orb3, orb2, orb4,
                  i, j, k, m, n, o, p,
                q, E_counter1, E_counter2);

                // E_t^ij is stored for x,y,z direction
            // Hence +3 on the counter
                    E_counter2 += 3;
                }
            }
            E_counter1 += 3;
        }
    }
    return temp; // temp is the value of <ij|kl>
}
\end{lstlisting}



We also take advantage of the eighfold symmetries, written our in Eqs. \eqref{interchangesym} and \eqref{interchangesym2}. We constructed the code like this originally to have the option to not store these integrals at all, and instead calculate them as needed. This would be a game changer in terms of what calculations are possible, since memory would now be scaling as $n^2$ instead of $n^4$. However we later decided on a memory distribution model was sufficient for our purposes, since we want to use coupled cluster. Coupled Cluster use more memory than HF, so the system size is restricted as is. 

\subsubsection{Parallel Implementation and Memory Distribution}
The hotspot in HF is the two electron integrals. We are not looking to make an optimized HF solver, but we must run this part of the calculation in parallel. \\

We want the workload of the integrals $\langle i j | k l \rangle$ distributed for a given index i and j. We only calculate one version of each symmetric term. \\

\begin{lstlisting}
    field_Q.set_size(Matrix_Size, Matrix_Size);
    for (int i = 0; i < Matrix_Size; i++)
    {
        for (int j = 0; j < Matrix_Size; j++)
        {
        // Leave parts of the field un-initialized
        // size = number of MPI procs
        // rank = my MPI rank
            if ((i+j)%size == rank)
            {
                field_Q(i,j) = zeros(Matrix_Size, Matrix_Size);
            }
        }
    }
\end{lstlisting}

The two electron integrals are a part of the Fock matrix calculation. We want to run this also in parallel, so ensure we can keep the integrals distributed in memory. We remember the Fock matrix was dependant upon 

\begin{equation}
\sum_{kl} \langle i j | k l \rangle ,
\end{equation}
and

\begin{equation}
\sum_{kl} \langle i l | k j \rangle .
\end{equation}
Because of this we define field\_Q to store the integrals as such

\begin{equation}
field\_Q(i,k)(j,l) = \langle i j | k l \rangle .
\end{equation}
We place the two indexes to be swapped in the matrix part of our armadillo field. We then store a $N^3$ sized array of temporary values, F\_temp(i,j,k). We then add the terms together in the correct order to produce $F_{ij}$. Here we can use functions like MPI\_Reduce, or make our own implementation of this function to produce the same result. \\

The important feature is that each processor only calculates terms based on the index i and k. This enables us to leave the indexes not in use in the field undefined, thus distributing the $N^4$ memory over all our P MPI processors in use (MPI procs). Each processor then only stores $\frac{N^4}{P}$ doubles. The amount of bytes for communication scales as $N^3$ doubles. \\

However we earlier calculated the integrals with a work distribution of indexes i and j. This work distribution makes it easier to use symmetries to avoid recalculation of symmetric terms. We therefore also introduce a communication procedure where we reshuffle the terms in field\_Q among the MPI procs. The amount of bytes for communication here is $\frac{1}{8} N^4$, and must be done using MPI\_Alltoallw or a similar implementation producing an identical result. Our HF implementation is not particularly optimized. Comments on we could optimize this implementation is available in the Future Prospects chapter. 

\subsubsection{Pre Iterative Steps}
The equation to solve in HF is the eigenvalue equation from Eq. \eqref{FOCK_EQUATION_STUFF}. To do this on a computer we must rewrite it slightly. The equation stands as

\begin{equation}
F C = S C \epsilon . \label{fdsaghbxcxd}
\end{equation}
We define a matrix V that satisfies

\begin{equation}
V^{\dag} S V = I , \label{fdsafafdsafdsafa}
\end{equation}
where I is the identity matrix. We insert $V^{\dag}$ to the left on both sides. Also $V V^{-1} $ is inserted into the equations. This leaves

\begin{equation}
V^{\dag} F V V^{-1} C = V^{\dag} S V V^{-1} C \epsilon .
\end{equation}
We also define 

\begin{equation}
F' = V^{\dag} F V ,
\end{equation}
and

\begin{equation}
C' = V^{-1} C .
\end{equation} 
We insert Eq. \eqref{fdsafafdsafdsafa}, F' and C' into Eq. \eqref{fdsaghbxcxd}.

\begin{equation}
F' C' = C' \epsilon .
\end{equation}
This is a true eigenvalue problem, where $\epsilon$ will be the eigenvalues of F' and C' will be the eigenfunctions. \\

We also define an intermediate P, which will be the electron density. 

\begin{equation}
P_{ij} = \sum_k^N C_i^k C_j^k ,
\end{equation}
where N is the number of electrons. We are now ready to begin an iterative procedure. This procedure will be different for RHF and UHF. 

\subsubsection{RHF Iterative Procedure}
For RHF we initially put the density P to be filled with zeroes. In RHF we will have an equal number of electrons with spin up and spin down. This simplifies our density matrix to

\begin{equation}
P_{ij} = \sum_k^{N/2} C_i^k C_j^k .
\end{equation}
We use Eq. \eqref{Fock_Restricted_1} to find the Fock matrix. We first insert P into the equation.

\begin{equation}
F_{ij} = (EK)_{ij} + \sum_{kl} P_{kl} (2 \langle i j | k l \rangle - \langle i l | k j \rangle) .
\end{equation}
We then perform the iterations until we reach self consistency. \\

\begin{algorithm}[H]
 \While{RHF\_continue = true}{
  Calculate $F$ \\
  $F' = V^{\dag} F V$ \\
  Solve $F' C' = C' \epsilon$ \\
  Compute $C = V C'$ \\
  Compute P \\
  \If{RHF = converged}{
    RHF\_continue = false
  }
 }
 \caption{Psudocode for RHF iterations}
 \label{RHF_ITERATIVE_PROCEDURE}
\end{algorithm}
After we have reached self consistency we calculate the energy.

\begin{lstlisting}
double Hartree_Fock_Solver::Calc_Energy()
{
    // Optimized RHF energy calculations
    Single_E_Energy = accu(EK % P);
    Two_E_Energy = 0.5*accu(Energy_Fock_Matrix % P) - 0.5*Single_E_Energy;
    return Single_E_Energy+Two_E_Energy;
}
\end{lstlisting}
Using armadillo the energy calculation simplifies to only two lines of code. 

\subsubsection{UHF Iterative Procedure}
For UHF we define two densities, $P^{\alpha}$ and $P^{\beta}$, which are the densities for spin up and down. 

\begin{equation}
P^{\alpha}_{ij} = \sum_k^{N_{\alpha}} C^{\alpha}_{ik} C^{\alpha}_{jk} .
\end{equation}

\begin{equation}
P^{\beta}_{ij} = \sum_k^{N_{\beta}} C^{\beta}_{ik} C^{\beta}_{jk} .
\end{equation}
Here $N_{\alpha}$ is the number of spin up particles, while $N_{\beta}$ is the number of spin down particles. These must be defined as input as must be equal to the total number of electrons in the system. We define the starting density to be random uniform numbers. We ensure the two matrices are not equal to each other for the first iteration. We use Eqs. \eqref{Fock_Restricted_2} and \eqref{Fock_Restricted_3} to find the Fock matrices. \\

\begin{algorithm}[H]
 \While{UHF\_continue = true}{
  Calculate $F_{\alpha}$ \\
  Calculate $F_{\beta}$ \\
  $F_{\alpha}' = V^{\dag} F_{\alpha} V$ \\
  $F_{\beta}' = V^{\dag} F_{\beta} V$ \\
  Solve $F_{\alpha}' C_{\alpha}' = C_{\alpha}' \epsilon_{\alpha}$ \\
  Solve $F_{\beta}' C_{\beta}' = C_{\beta}' \epsilon_{\beta}$ \\
  Compute $C_{\alpha} = V C_{\alpha}'$ \\
  Compute $C_{\beta} = V C_{\beta}'$ \\
  Compute $P_{\alpha}$ \\
  Compute $P_{\beta}$ \\
  \If{UHF = converged}{
    UHF\_continue = false
  }
 }
 \caption{Psudocode for UHF iterations}
 \label{UHF_ITERATIVE_PROCEDURE}
\end{algorithm}
After iterations we again calculate the energy. With armadillo the energy calculation simplifies to just two lines of code.

\begin{lstlisting}
double Hartree_Fock_Solver::Unrestricted_Energy()
{
    // Oprimized energy for UHF
    Single_E_Energy = accu((P_up + P_down) % EK);
    Two_E_Energy = 0.5 * accu(EnF_up % P_up) + 0.5 * accu(EnF_down % P_down) - 0.5 * Single_E_Energy;
    return Single_E_Energy + Two_E_Energy;
}
\end{lstlisting}


\subsubsection{Helping Convergence}
Sometimes our solution has problems converging. This is a numerical problem and we can introduce a few features to help the convergence along. \\

Damping is one option. This means updating the density only slightly, by inserting

\begin{equation}
P_{new}' = \gamma P_{old} + \left( 1 - \gamma \right) P_{new} .
\end{equation}
This reduce the change in density between iterations. We only used this in UHF. \\

A better alternative is the DIIS method, discussed in section \ref{diis_section_po_g}. We implemented this method for RHF. The first part of our DIIS implementation is calculating the error, $\Delta p$. 

\begin{lstlisting}
delta_p = F*P*O - O*P*F;
\end{lstlisting}

We then store the error and Fock matrices for the last M iterations. M is defined to M = 3 in our implementation. After this we construct the matrix B. 

\begin{lstlisting}
for (int i = 0; i < number_elements_DIIS; i++){
   for (int j = 0; j < number_elements_DIIS; j++){
      mat1 = Stored_Error.at(i);
      mat2 = Stored_Error.at(j);
      DIIS_B(i,j) = trace(mat1.t() * mat2);
   }
}
\end{lstlisting}

We then find the coefficients c.

\begin{lstlisting}
DIIS_c = solve(DIIS_B, DIIS_Z);
\end{lstlisting}

And finally we construct the new Fock matrix, as a linear combination of the previous Fock matrices. 

\begin{lstlisting}
F = DIIS_c.at(0) * Stored_F.at(0);
for (int i = 1; i < number_elements_DIIS; i++){
   F += DIIS_c.at(i) * Stored_F.at(i);
}
\end{lstlisting}

\newpage

\section{Atomic Orbital to Molecular Orbital}
Atomic Orbital (AO) to Molecular Orbital (MO) is required before we can do any CCSD calculations. In this section we describe how to implement this transformation. We are here looking for a highly optimized implementation. Some background is available in Ref.\cite{aotomo_1_cite}. However the author found the algorithms in the literature unsatisfactory. For this reason we will present a new algorithm. First, the simplest transformation is:

\begin{equation}
\langle ab | cd \rangle = \sum_{ijkl} C_i^a C_j^b C_k^c C_l^d \langle ij|kl \rangle .
\end{equation}
This scales as $n^8$ and can be factorized. 

\begin{equation}
\langle ab | cd \rangle = \sum_i C_i^a \sum_j C_j^b \sum_k C_k^c \sum_l C_l^d  \langle ij|kl \rangle .
\end{equation}
This is usually split into four quarter transformations. 

\begin{equation}
\langle aj|kl \rangle = \sum_{i} C_i^a \langle ij|kl \rangle .
\end{equation}

\begin{equation}
\langle ab|kl \rangle = \sum_{j} C_j^b \langle aj|kl \rangle .
\end{equation}

\begin{equation}
\langle ab|cl \rangle = \sum_{k} C_k^c \langle ab|kl \rangle .
\end{equation}

\begin{equation}
\langle ab|cd \rangle = \sum_{l} C_l^d \langle ab|cl \rangle .
\end{equation}
Each of these quarter transformations scale as $n^5$. The implementation of this must be done in an effective way in terms of speed and memory. The latter is the most important as the memory here scales as $N^4$, where N is the number of contracted GTOs, for both $\langle ij | kl\rangle$, $\langle ab | cd \rangle$ and also the intermediates in between each quarter transformation. \\

\begin{algorithm}[H]
 \KwData{Psudo Code}
 \KwResult{Algorithm for parallel AOtoMO transformation }
 \For{a=0; a<N}{
  \For{k=0; k<N}{
   \For{l=0; l<N}{
    \If{Grid k and l over threads}{
     \For{j=0; j<N}{
      \For{i=0; i<N}{
       $QT1(k,l,j) += 
       C_i^a \times \langle ij|kl \rangle$
      }
     }
     \For{j=0; j<N}{
      \For{b=0; b<N}{
       $QT2(k,l,b) += C_j^b \times QT1(k,l,j)$
       }
      }
    }
   }
  }
  Communicate $QT2(k,l,b)$
  
  \For{b=0; b<N}{
   \For{c=0; c<N}{
    \If{Grid b and c over threads}{
     \For{k=0; k<N}{
      \For{l=0; l<N}{  
       $QT3(b,c,l) = C_k^c \times QT2(k,l,b)$
       }
      }
      \For{l=0; l<N}{
        \For{d=0; d<N}{
       $QT4(b,c,d) = C_l^d \times QT3(b,c,l)$
        }
      }
     }
    }
   }
   
   Communicate $QT4(b,c,d)$\\
   
   \If{Store distributed MOs to given thread}{
    $\langle ab|cd \rangle = QT4(b,c,d)$
   }
 }
  
 \caption{Simple Psudocode for parallel AOtoMO transformation. QT1, QT2, QT3 and QT4 are intermediates}
 \label{aotomotrans}
\end{algorithm}

Algorithm \ref{aotomotrans} is a description of how we optimize this implementation. Further optimizations will come later, but first an illustration of the general idea. We first hold index $a$ constant throughout the transformation. This enables us to use $N^3$ size intermediates. \\

Second the grid over k and l is chosen because neither of these are involved as an index in $C$ for the first two quarter transformations. This makes sure that the terms of $QT2(k,l,b)$ calculated by each thread is the fully two quarter transformed term. This avoids the use of MPI\_Reduce or similar operations and means only one thread needs to communicate these two quarter transformed terms with specific $k$ and $l$, minimizing the communication. The total amount of double precision values communicated in the first communication for now is $N^3$ for each $a$, making it $N^4$ in total for all $a$. \\

After the first communication each thread has all terms in $QT2(k,l,b)$ available. We then make a new grid over $b$ and $c$ and continue calculations in parallel. The grid could be made over $a$ and $b$, but the prior makes in general a better work distribution. This is because index $a$ is held fixed. After the fourth quarter transformation each thread has the fully transformed MOs available for certain $b$ and $c$ indexes. \\

At this point we can distribute the MOs in the same grid as for $b$ and $c$, and start CCSD calculations. However because we want to have the distribution optimized for CCSD we implement another communication. This communication is $N^3$ for each $a$, making it $N^4$ in total for all $a$. \\

After the second communication we simply store the MOs in a memory distributed manner. It is also possible to write to disk. \\

$QT2$ and $QT4$ must be stored as one dimensional arrays, to minimize the number of communication procedures initiated, hence minimize latency. Also all multiplications are written using external math libraries through armadillo. We should also introduce symmetries to optimize our calculations further. The starting AOs had eight-fold symmetries. So does the resulting MOs. However these symmetries does not hold at all the quarter transformed intermediates. This complicates things slightly. \\

The second quarter transformed four dimensional array, QT2, will have symmetries in the two untouched indexes, as well as in the two transformed indexes. We were able to make use of this to reduce communication by 75\%, since symmetric terms need not be communicated twice. This also holds true at the QT4 level obviously. The algorithm using symmetries and external math libraries is presented in algorithm \ref{aotomo2}. \\

\begin{algorithm}
 \KwData{Psudo Code}
 \KwResult{Effective Algorithm for parallel AOtoMO transformation using external math libraries}
 \For{a=0; a<N}{
  \For{k=0; k<N}{
   \For{l=0; l<=k}{
    \If{Calculate on local thread}{
     A1(*) = C(a,*) $\times$ $\langle kl | ** \rangle$ \\
     A2($0 \rightarrow a$) = C($0 \rightarrow a$, *) $\times$ A1 \\
     \For{b=0; b<=a}{
       QT2(b,k,l) = A2(b) 
      }
    }
   }
  }
  
  MPI\_Allgatherv(QT2) \\
  
  \For{b=0; b<=a}{
   \For{c=0; c<N}{
    \If{Calculate on local thread}{
 	 A1(*) = C(c,*) $\times$
 	  QT2(b,*,*) \\
     A2($0 \rightarrow c$) = C($0 \rightarrow c$, *) $\times$ A1 \\
     \For{d=0; d<=c}{
       QT4(b,c,d) = A2(d) 
      }
     }
    }
   }
   MPI\_Allgatherv(QT4) \\
   
   \If{Store distributed MOs to given thread}{
    $\langle ab|cd \rangle = QT4(b,c,d)$ \\
    or write to disk
   }
 }
  
 \caption{Psudocode for parallel AO to MO transformation using armadillo. A1 and A2 are one dimensional intermediates}
 \label{aotomo2}
\end{algorithm}

The communication is somewhat tricky in this algorithm. Since we have inserted symmetries, the size of the message to be transmitted changes dependant upon the index $a$. This also applies to the displacement. We therefore store both in two dimensional arrays where $a$ is the outer index, the inner is the MPI rank. \\

We run through the algorithm one time in advance to calculate these variables. We also calculate and store where each processor will start calculations. This is done to remove any pipeline flushes, which can be caused by the CPU wrongly guessing the answer of an if test. \\

For this reason we define another two dimensional array, this one of size N times the number of MPI procs. In the first two quarter transformation, each rank here stored at what index $l$ will calculations start for a given index $k$. The next $l$ the same rank will perform calculations on will then be 

\begin{equation}
l \rightarrow l + p ,
\end{equation}
where p is the number of MPI procs. The exact same procedure is repeated for quarter transformation 3 and 4. \\

We have also in the more advanced algorithm inserted one dimensional arrays A1 and A2. Using these provide more optimize ways of accessing memory. It may at first sight seem like an additional complication to first calculate A2 as a one dimensional array and later store it in QT2, but this is a more efficient way when using armadillo. \\

The algorithm is implemented in the function 

\begin{lstlisting}
void Prepear_AOs(int nr_freeze);
\end{lstlisting}
The argument is how many core orbitals to freeze. The argument is somewhat wasted, since frozen core approximation is not implemented yet.

\section{CCSD Serial Implementation \label{optimize_serial_version_bii}}
Our CCSD implementation is quite large, actually close to 10 000 lines. However this is small compared to other optimized implementations, which are usually around 40 000 lines of code. Implementation is important in CCSD, since it scales quickly for larger systems. Additional information on on the advancement of CCSD is available in a series of books, Ref.\cite{book_om_advancements_ccsd}. The most effective implementation to the authors knowledge is the Cyclic Tensor Framework, see Refs.\cite{most_effective_ccsd_dude}, \cite{most_effective_ccsd_dude2} and \cite{most_effective_ccsd_dude3}. \\

In this thesis we will present a simple and effective implementation of CCSD in parallel. First we look at a simple serial implementation. This section discusses the serial implementation. There are two specific goals for this implementation. First getting the energy in the smallest amount of time, second being able to run larger systems. \\

Even more precise we can state that our goals are: \\
a) Never get zero in a multiplication\\
b) All multiplication should be done by external math libraries\\
c) Do not store anything more than needed\\

We first present the general structure of the code. Later we will discuss a few details about different optimizations we have implemented. These will be contrasted to what kind of optimizations is commonly implemented in CCSD. Then, there will be a pros and cons list for our implementation. The chapter will be quite technical as there are several considerations behind each optimization, and it all works in combination. 

\newpage

\subsection{Structure}

For our serial program we first define arrays to store all intermediates, MOs and amplitudes. Two arrays are defined for each amplitude, one for the old amplitudes and one for the new amplitudes. We define a convergence criteria, which stops iterations once the difference of energy from one iteration to the next is bellow this criteria. \\

\begin{algorithm}[H]
 \KwData{Psudo Code}
 \KwResult{Structure of CCSD serial program}
 \While{CCSD continue = true}{
  Set Eold = Enew \\
  Calc F1 \\
  Calc F2 \\
  Calc F3 \\
  Calc W1 \\
  Calc W2 \\
  Calc W3 \\
  Calc W4 \\
  Calc New t1 amplitudes \\
  Calc New t2 amplitudes \\
  Set t1 = t1new \\
  Set t2 = t2new \\
  Calc $\tau_{ij}^{ab}$ \\
  Calc New Energy \\
  \If{Enew - Eold < Convergence criteria}{
  	CCSD continue = false
  }
 }
 \caption{Psudocode for our serial CCSD program}
 \label{CCSD_STRUCTURE_SERIAL}
\end{algorithm}

Algorithm \ref{CCSD_STRUCTURE_SERIAL} illustrates the algorithm as psudocode. Each of the terms behind "Calc" is taken as a separate function to make the code easily readable. 

\subsection{Removing redundant zeroes \label{compact_storage}}
We now briefly reconsider the molecular integrals, which were calculated as such

\begin{equation}
\langle pq|rs \rangle = 
\sum_{\alpha \beta \xi \nu} C_{\alpha}^p C_{\beta}^q C_{\xi}^r C_{\nu}^s \langle \alpha \beta | \xi \nu \rangle .
\end{equation}
Here $\langle \alpha \beta | \xi \nu \rangle$ are our atomic orbitals (AOs). These come from our RHF calculations. $\langle pq|rs \rangle$ are the molecular orbitals (MOs). MOs here are presented as a linear combination of AOs. The MOs appear in CCSD as a double bar integral. This is defined as such

\begin{equation}
\langle pq||rs \rangle = \langle pq | rs \rangle
- \langle pq | sr \rangle  .
\end{equation}
Due to spin considerations, if we fill a matrix with $\langle pq||rs \rangle$ it will be filled with mostly zeroes. However when using an RHF based CCSD it is common that all even numbered spin orbitals have the same spin orientation. This means all odd numbered orbitals will also have the same spin orientation. This results in the zeroes forming pattern that we have identified and utilized. \\

$\langle pq || rs \rangle$ are diagonal in total spin projection. In RHF the total spin is also equal to zero. When we have all odd numbered orbitals with the same spin orientation, and same with even numbered orbitals, this has a practical implication. The implication is that the only terms that will not be equal to zero are those where the sum of the orbital indexes are equal to an even number. \\

We will now visualize this. We construct a program that performs the AO to MO transformation and print $\langle pq||rs \rangle$ for a fixed $p=1$ and $r=1$. In the span of $q$ and $s$ there is formed a matrix, we have noted the terms that will be zero and also the terms that will be non-zero with the indexes (q, s).

\[ \left( \begin{array}{ccccccc}
(0,0) & 0 & (0,2) & 0 & (0,4) & 0 & \dots \\
0 & (1,1) & 0 & (1,3) & 0 & (1,5) & \dots \\
(2,0) & 0 & (2,2) & 0 & (2,4) & 0 & \dots \\
0 & (3,1) & 0 & (3,3) & 0 & (3,5) & \dots \\
(4,0) & 0 & (4,2) & 0 & (4,4) & 0 & \dots \\
0 & (5,1) & 0 & (5,3) & 0 & (5,5) & \dots \\
\dots & \dots & \dots & \dots & \dots & \dots & \dots \end{array} \right)\]

This array is now stored in our computer as a four dimensional array that we call $I[a][b][c][d]$. We on purpose use a different index in the array than we do for the orbital, even though index a in this example is a referance to orbital p. We can note which orbitals our array-indexes refers to as such\\
a = p \\
b = r \\
c = q \\
d = s \\

This will be an array of size $(2N)^4$, where $N$ is the number of contraction Gaussian Type Orbitals (GTOs). We now perform a trick. We want our indexes of I to refer to a different orbital, in practise we want:\\
a = p\\
b = r\\
c = q/2 + (q\% 2) N\\
d = s/2 + (s\% 2) N\\

Where \% is the binary operator and we use integer division by 2. The number 2 comes from two spin orbitals per spacial orbital. Now index c is no longer a referance to orbital q, but a referance to orbital [q/2 + (q \% 2) N]. If we now visualize the same double bar integral with fixed $p=1$ and $r=1$ it looks like this

\[ \left( \begin{array}{cccccccc}
(0,0) & (0,2) & (0,4) & \dots & 0 & 0 & 0 & \dots \\
(2,0) & (2,2) & (2,4) & \dots & 0 & 0 & 0 & \dots\\
(4,0) & (4,2) & (4,4) & \dots & 0 & 0 & 0 & \dots\\
\dots & \dots & \dots & \dots & \dots & \dots & \dots & \dots\\
(1,1) & (1,3) & (1,5) & \dots & 0 & 0 & 0 & \dots\\
(3,1) & (3,3) & (3,5) & \dots & 0 & 0 & 0 & \dots\\
(5,1) & (5,3) & (5,5) & \dots & 0 & 0 & 0 & \dots\\
\dots & \dots & \dots & \dots & \dots & \dots & \dots & \dots \end{array} \right)\]

Performing this trick will always result in a matrix that looks something like this. We can split this matrix into four sub-matrices, one top left, one top right, one bottom left and one bottom right. Regardless of $a$ and $b$, we will always have either the two left sub-matrices, or the two right sub-matrices always filled with zeroes. These do not need to be stored. If we ensure we \emph{only perform calculations on orbitals with a non-zero contribution} we can change our array-indexing to:\\
a = p\\
b = r\\
c = q/2 + (q\% 2) N\\
d = s/2 \\

This means for two orbital where s = 2 and s = 3 we will have the same d value. However one of these orbitals will always be zero, so if we avoid doing calculations on this there will be no problems. And we also reduce the size of the array to half. Visualizing now the same array it looks like this.
\[ \left( \begin{array}{cccc}
(0,0) & (0,2) & (0,4) & \dots \\
(2,0) & (2,2) & (2,4) & \dots \\
(4,0) & (4,2) & (4,4) & \dots \\
\dots & \dots & \dots & \dots \\
(1,1) & (1,3) & (1,5) & \dots \\
(3,1) & (3,3) & (3,5) & \dots \\
(5,1) & (5,3) & (5,5) & \dots \\
\dots & \dots & \dots & \dots  \end{array} \right)\]

And its size will be $\frac{1}{2} (2N)^2$. This kind of indexing can and should be performed on $\textbf{all}$ stored integrals, amplitudes and intermediates. This ensures all memory is reduced by at least 50 \%. Also if a,b,c,d is referencing orbitals in the same manner in all stored arrays we can still use external math libraries as before. However now we will not be passing any zeroes into these external math libraries, so calculations can be faster. This change in indexing keeps all symmetries and also allow easy row and column access. The row is accessed as usual.

\begin{tikzpicture}
        \matrix [matrix of math nodes,left delimiter=(,right delimiter=)] (m)
        {
            (0,0) &(0,2) &(0,4) &\dots \\
            (2,0) & (2,2) & (2,4) & \dots \\
            (4,0) & (4,2) & (4,4) & \dots \\
            \dots & \dots & \dots & \dots \\
            (1,1) & (1,3) & (1,5) & \dots \\
			(3,1) & (3,3) & (3,5) & \dots \\
			(5,1) & (5,3) & (5,5) & \dots \\
			\dots & \dots & \dots & \dots \\
        };  
        \draw[color=red] (m-1-1.north west) -- (m-1-3.north east) -- (m-1-4.north east) -- (m-1-4.south east) -- (m-1-1.south west) -- (m-1-1.north west);
    \end{tikzpicture}

A column is slightly different, since we only require either the top half or the bottom half of the matrix.

\begin{tikzpicture}
        \matrix [matrix of math nodes,left delimiter=(,right delimiter=)] (m)
        {
            (0,0) &(0,2) &(0,4) &\dots \\
            (2,0) & (2,2) & (2,4) & \dots \\
            (4,0) & (4,2) & (4,4) & \dots \\
            \dots & \dots & \dots & \dots \\
            (1,1) & (1,3) & (1,5) & \dots \\
			(3,1) & (3,3) & (3,5) & \dots \\
			(5,1) & (5,3) & (5,5) & \dots \\
			\dots & \dots & \dots & \dots \\
        };  
        \draw[color=red] (m-1-2.north west) -- (m-3-2.south west) -- (m-4-2.south west) -- (m-4-2.south east) -- (m-3-2.north east) -- (m-1-2.north east) -- (m-1-2.north west);
\end{tikzpicture}

This can be specified using the submatrix command in armadillo. The attributes of the double bar integrals in RHF are such that there will be some remaining zeroes after removing these. This is because total spin must be zero. However there will be another pattern formed where all remaining zeroes are placed in one of two remaining sub-matrices. This can also be accounted for, reducing memory needs by an additional $\frac{1}{8}$. These calculations can also easily be avoided using submatrix calls in armadillo.\\

Since we are using an RHF basis some matrix elements become independent of spin. This means spin up will have the same value as spin down. This happens mostly for the two dimensional intermediates, and when we store in this manner the practical implications of this becomes identical upper and lower submatrices. In this situation we do not calculate the same term twice. In the future we will refer to this method as compact storage. This method removes all zeroes without defining additional arrays, as is usually done today. 

\subsection{Pre Iterative Calculations}
Before calculations can start we perform a few tricks. We do not store our MOs in one gigantic array, instead we split it up into several smaller ones. This is done at the end of the AOtoMO transforamtion. These are variables such as MO3 and MO4, decleared in the header file ccsd\_memory\_optimized.h.

\begin{lstlisting}
field<mat> MO3, MO4 ...
\end{lstlisting}
This is a very common procedure in CCSD implementation. It is performed to enable more effectively use of external math libraries. The reason it is more effective is because of memory accessing. If we want to send parts of an array into an external math library, we need first to extract which parts to send. Instead, we can define several smaller arrays like MO3. MO3 is then designed specifically to be sent directly into the external math library. \\

Ref.\cite{ccsd_fac3} also ponders this. In fact, this is such an optimization that even redundant storage of double bar integrals is often used. This means storing a value twice, just to have it easily available for passing to external math library. \\

Originally, we took advantage of this straight forward optimization. However, in the current implementation we will not be storing redundant values. Instead, we will be storing the single bar integrals, and have functions to map these into two dimensional arrays of double bar integrals ready for external math library use. We have surgically designed each and every for loop such that this mapping is redundant in terms of program efficiency. It does reduce memory requirements drastically. \\

The splitting of the integrals for us then becomes somewhat redundant in this regard, but as we will see later it is of clinical importance when we implement memory distribution. \\

Before iterations can start we also allocate memory for our intermediates and amplitudes. We use the principles of section \ref{compact_storage} here. In fact every array to come in contact with an external math library must be stored on this form.

\subsection{F1, F2 and F3}
We now start the iterative procedure. The first three intermediates are two dimensional and
very straight forward to calculate. We use Eqs. \eqref{intermedF1}, \eqref{intermedF2} and \eqref{intermedF3}. Our implementation has some additional complications which will be discussed shortly, but is still equivalent to our initial naive implementation.

\begin{lstlisting}
for (int a = 0; a < unocc_orb; a++){
  for (int m = 0; m < n_Electrons; m++){
     F1(a/2, m/2) = accu(integ2(a,m) % T_1);
  }
}
\end{lstlisting}

In this initial naive implementation integ2 is one part of the double bar integrals we pulled out. Since we want to store only the single bar integrals, we replace this with a function Fill\_integ\_2\_2D(int a, int m) to fill up a global mat integ2\_2D. This is then used in external math libraries.

\begin{lstlisting}
for (int a = 0; a < unocc_orb; a++){
  for (int m = 0; m < n_Electrons; m++){
     Fill_integ_2_2D(a, m);
     F1(a/2, m/2) = accu(integ2_2D % T_1);
  }
}
\end{lstlisting}

$[F_2]$ and $[F_3]$ are calculated in a similar procedure.

\subsection{W1, W2, W3 and W4}
These intermediates are calculated using Eqs. \eqref{intermedw1}, \eqref{intermedW2}, \eqref{intermedW3} and \eqref{intermedW4}. We include the initial naive implementation of $[W_1]$.

\begin{lstlisting}
for (int i = 0; i < n_Electrons; i++){
   for (int j = i+1; j < n_Electrons; j++){
      Fill_integ8_2D;
      W_1(i,j)(k/2,l/2) = integ8_2D;
   }
}

for (int k = 0; k < n_Electrons; k++) {
   for (int l = 0; l < n_Electrons; l++){      
      Fill_integ6_2D(k,l);
      Fill_integ4_2D(k,l);
      for (int i = 0; i < n_Electrons; i++){
         for (int j = i+1; j < n_Electrons; j++){
            W_1(i,j)(k/2,l/2) += accu(integ6_2D.col(j)
                % T_1.col(i));
            W_1(i,j)(k/2,l/2) -= accu(integ6_2D.col(i)
                % T_1.col(j));
            W_1(i,j)(k/2,l/2) += 0.5*accu(integ4_2D
                % tau3.at(i,j));
         }
      }
   }
}
\end{lstlisting}

Here the mapping into two dimensional double bar integrals is still done as a $N^4$ procedure, whereas the calculation is now an $N^6$ procedure. For easy external math library use later we store these variables as W1(i,j)(k,l), W2(i,j)(a,m), W3(i,m)(e,n) and W4(a,i)(c,k). We also use symmetries where they can be applied.

\subsection{New amplitudes}
The T1 amplitudes are calculated using Eq. \eqref{LINK_THIS_SHIT_1_T1}. For the T2 amplitudes we use Eq. \eqref{LINK_THIS_SHIT_1_T2}. To make this amplitude most optimal for external math libraries we store it as

\begin{equation}
T2(a,i)(b,j) . \label{howtostoret2}
\end{equation}

\subsection{$\tau_{ij}^{ab}$ and Energy}
$\tau_{ij}^{ab}$ is calculated in Eq. \eqref{intermedtau}. It is stored in a variable tau3(a,b)(i,j) for optimal use in external math libraries. The energy can be calculated using Eq. \eqref{CCSD_TOTAL_ENERGY}. However we can simplify this further by introducing $\tau_{ij}^{ab}$.

\begin{equation}
E_{CCSD} = E_0 + \sum_{ai} f_{ai} t_i^a + \frac{1}{4} \sum_{abij} \langle ij || ab \rangle \tau_{ij}^{ab} .
\end{equation}
Here we have inserted $t_i^a t_j^b = \frac{1}{2} \left( t_i^a t_j^b - t_j^a t_i^b \right)$. Also the term $f_{ai}$ will always be equal to zero when the basis for our CCSD calculations are a diagonalized Fock matrix.

\begin{equation}
E_{CCSD} = E_0 + \frac{1}{4} \sum_{abij} \langle ij || ab \rangle \tau_{ij}^{ab} .
\end{equation}

\subsection{Dodging Additional Unnecessary Calculations}
In section \ref{compact_storage} we discussed how to avoid multiplication where both terms are zero. However, for CCSD we originally had several terms to be multiplied, and we factorized them. This causes another potential optimization, that we wish to introduce with a simplified example. Consider four terms, A, B, C and D, that want to multiply together.

\begin{equation}
F = A \times B \times C \times D .
\end{equation}
Imagine factorizing this would speed up our calculations.

\begin{equation}
F = A \times (B \times (C \times D ) ) .
\end{equation}
Let us define intermediate E.

\begin{equation}
E = B \times (C \times D ) .
\end{equation}
After we calculated this we are left with

\begin{equation}
F = A \times E .
\end{equation}
At the end of the calculation, it turns out A was equal to zero. This means the entire calculation was wasted, as F would have been zero anyway. Spin consideration causes this situation to occur in CCSD. Luckily because this comes from spin considerations, it is deterministic. We can identify all these situations and avoid calculations. \\

This is implemented in our code, and we want to present an example from the contributions to $t_{ij}^{ab}$ from $[W4]$.

\begin{equation}
D_{ij}^{ab} t_{ij}^{ab} \leftarrow \sum_{kc} t_{jk}^{bc} \times [W_4]_{ic}^{ak} .
\end{equation}
Index $b$ and $j$ only appear in the T2 amplitudes, while indexes $a$ and $i$ appear in the intermediate. Imagine now that index $b$ is an odd number, while index $j$ is an even number. \\

In the sum over $k$ and $c$, $k$ and $c$ can themselves be odd or even. We remember we arranged our MOs so that all odd numbers had same spin orientation. Inside the sum, whenever now $c$ is an odd number we will be exciting two electron into spin up orbitals. If $j$ was an even number we also remove an electron from a spin down orbital. \\

Regardless of index $k$ this will not result in zero spin in total and the amplitude must be equal to zero with our spin restriction. In section \ref{compact_storage} we noted a situation where one of the two sub-matrices would be zero. This is the situation. \\

Also, the indexes $a$, $b$, $i$ and $j$ must themselves result in zero spin in total, or the amplitude will be zero. This limits the number of possible combinations of $t_{jk}^{bc}$ and $[W_4]_{ic}^{ak}$ to where we can actually avoid calculating some terms of $[W_4]_{ic}^{ak}$ that are not equal to zero. \\

The easiest and most human-time effective way of implementing this is to simply go through the factorization backwards, to identify which multiplications we did not need. This has been done.

\section{CCSD Parallel Implementation}
In parallel implementation we will make extensive use of memory distribution. In CCSD it is quite normal to read some of the MOs from disk. We will not be doing this, but we will place memory distribution as our number one priority. The code was however originally designed to read from disk, so this option is left easily available.

\subsection{Memory Distribution \label{kriseseksjon}}
In a serial implementation of CCSD the leading memory consumer is an $\frac{1}{16 \times 2} n_v^4$ sized array, $\langle ab||cd \rangle$. The $\frac{1}{16}$ comes from only storing spacial single bar MOs, $\langle ab|cd \rangle$, with $n_v$ being the number of virtual spin orbitals. The factor $\frac{1}{2}$ comes from symmetry. This array is called MO9 in our implementation. \\

However, the array only appears in the calculation of $t_{ij}^{ab}$, as this is the only place where the double bar integrals has three or more virtual indexes (which are $a$, $b$, $c$ etc). This means we can ensure one processor only requires parts of the array MO9 if we distribute work here correctly. It is also possible to distribute the double bar integrals themselves, \cite{ccsd_minne_distribuert_double_bar_artikkel} presents such an algorithm. \\

\begin{lstlisting}
for (int a = 0; a < unocc_orb; a++){
   for (int b = a+1; b < unocc_orb; b++){
      // Distribute work with Work_ID variable
      if (Work_ID % size == rank){
         // Perform calculation
     // Only the processor who passes this if test
     // will need <ab||cd> with specific a and b
      }
   }
}
\end{lstlisting}

All the largest parts of the single bar integrals will be distributed in memory. This leaves the largest un-distributed arrays as our T2 amplitudes and some intermediates. Specifically the old and new $t_{ij}^{ab}$, $[W_4]$ and $\tau_{ij}^{ab}$. \\

We will be able to distribute $[W_4]$, through some quite complex operations that actually also provides quite good parallel performance. \\

Because we want to store the old T2 amplitudes as specified in Eq. \eqref{howtostoret2} we are unable to take advantage of symmetries. We are however able take advantage of one symmetry for $\tau_{ij}^{ab}$ and the new T2 amplitudes. Also we have the storage of section \ref{compact_storage}. This means non distributed memory is scaling as

\begin{equation}
M(n_v, n_o) = \left(\frac{1}{2} + \frac{1}{4} + \frac{1}{16} + \frac{1}{16} \right) n_v^2 n_o^2 \approx n_v^2 n_o^2 .
\end{equation}
The factor $\frac{1}{2}$ is the old T2 amplitudes. The $\frac{1}{4}$ is the intermediate $\tau_{ij}^{ab}$. This intermediate improves performance only modestly in our factorization, but is very helpful when optimizing the use of external math libraries. We therefore keep it. The final factors $\frac{1}{16}$ are parts of the MOs we where unable to distribute in memory and also the new T2 amplitudes. The new T2 amplitudes require less memory because we only store one version of each symmetric term. How to distribute $[W_4]$ will be discussed shortly. Contributions from $n_v n_o^3$ is ignored in the non distributed memory scaling.

\subsection{Three Part Parallel}
Our parallel implementation will be quite straight forward. We will split the iterative procedure in three. Part two is the calculation of $[W_4]$. Part three is the amplitudes. Part one is everything else. The split is performed to keep in line with our guiding parallel principles of minimizing communication initiations. In each part we will also discuss what type of performance we can expect with increased number of CPUs in use. This is known as scaling. \\

\subsubsection{Part 3}
We first look at the third parallel part, the amplitudes. Each processor allocates memory for the new T2 amplitudes. At first each processor only stores the terms it performs calculations on itself. That is, the T2 amplitudes are distributed in memory. \\

We want $I_{ab}^{cd}$ distributed in memory. The numbers for this variable is stored in MO9. To make the memory distribution easy, we distribute work for $t_{ij}^{ab}$ based on indexes $a$ and $b$. These indexes are symmetric in $t_{ij}^{ab}$. We thus only need calculations on $b > a$. Since we stored the single bar spacial MOs, we must distribute work very delicately if we are to not get to much overhead. \\

Work is distributed with in a block cyclic manner, with the block always being of size 2. This block size is identical to the number of spin MOs per spacial MO. The optimal work distribution with these limitations has the mathematical formula, with all divisions being integer divisions.

\begin{equation}
Work\_ID = \frac{a}{2} \times \frac{n_v}{2} + \frac{b}{2} - \sum_{n=0}^a \frac{n}{2} .
\end{equation}
The number two is the block size, $\frac{n_v}{2}$ is the size of a column in MO9 and the sum ensures we get the optimal distribution for $b > a$. This forumla distributes work over indexes a/2 and b/2 optimally. The work ID is used by the processors to figure out if the calculation is to be performed.

\begin{lstlisting}
// Find new T2 amplitudes, function
for (int a = 0; a < unocc_orb; a++)
{
   sum_a_n += a/2;
   A = a/2;
   AA = A* Speed_Occ - sum_a_n;

   // Potential to read from file here.
   // Read in a 3 dimensional array of single bar integrals for a
   // specific index a. Same array is used for a and a+1
   // Can use for example MPI_File_read(...)

   // a is an even number
   for (int b = a+2; b < unocc_orb; b++){ // b is even number
      B = b/2;
      Work_ID = AA+B;
      if (Work_ID % size == rank){
     // Load up 2D arrays for external math libraries
         Fill_integ3_2D(a, b);
         Fill_integ9_2D(a, b);

     // Reindexing of tau for external math library use
         Fill_2D_tau(a, b);

         for (int i = 0; i < n_Electrons; i++){ // i is even number
            for (int j = i+2; j < n_Electrons; j++){ // j is even number
               MY_OWN_MPI[index_counter] =
        (-MOLeftovers(a/2, b/2)(j/2, i/2)
        + MOLeftovers(a/2, b/2)(i/2, j/2)
        + W_5(a,b)(i/2,j/2)
                - W_5(a,b)(j/2,i/2)
        - accu(W_2(i,j)(a/2, span()) % T_1.row(b/2))
                + accu(W_2(i,j)(b/2, span()) % T_1.row(a/2))
        + 0.5*accu(W_1(i,j)(span(0, Speed_Elec-1), span()) % tau1(span(0, Speed_Elec-1), span())) // Half matrix = 0, skip this
        - accu(t2.at(b,i)(span(0, Speed_Occ-1), j/2) % D3.row(a).t())
                + accu(t2.at(a,i)(span(0, Speed_Occ-1), j/2) % D3.row(b).t())
        + accu(t2.at(a,j)(b/2, span()) % D2.row(i/2))
                - accu(t2.at(a,i)(b/2, span()) % D2.row(j/2))
        - accu(integ9_2D(span(0, Speed_Occ-1), i/2) % T_1(span(0, Speed_Occ-1), j/2))
                + accu(integ9_2D(span(0, Speed_Occ-1), j/2) % T_1(span(0, Speed_Occ-1), i/2))
        + 0.5*accu(integ3_2D(span(0, Speed_Occ-1), span()) % tau3(i,j)(span(0, Speed_Occ-1), span()))) // Half matrix = 0, skip this
        
        / (DEN_AI(a/2,i/2)+DEN_AI(b/2, j/2));

        // This is one new T2 amplitude.
        // Plus one on index counter, and calculate the next
                index_counter++;
                j++;
             }
             i++;
          }
       }
       b++;
    }
}
\end{lstlisting}

The code segment above is a part of the function for the new T2 amplitudes. This is how our actual code looks. It is designed for performance. The outer loop is index a. If we wanted to read from file, we would be reading in the single bar integrals for a specific index a, into an $N^3$ sized array. \\

The next loop is index b. Here we figure out if a local processor is to perform these calculations. If the processor shall perform calculations, we fill up the largest arrays needed for external math library use. The smaller arrays used in external math libraries are already filled. \\

The next loops are i and j. Since spin must be zero, an even number a and b only allows for even number i and j. The other combinations of odd b, even a etc are also implemented in our code but not included here. \\

Inside the four loops we calculate the amplitude $t_{ij}^{ab}$. We notice every term is written using external math libraries, with the accumulation function. We also skip calculations using the span() function where appropriate, as noted in the previous section. Finally we divide by the denominator and store the new amplitude in a one dimensional array for easier MPI function use. \\

Once calculations are completed we must gather the results and update the old T2 amplitudes. The new T2 amplitudes are all stored in a one dimensional array on each local processor, so we only need to initiate one communication procedure. The most effective would be a collective all-to-all communication, where we send in the new amplitudes and gather them/write over the old ones. For this we could use a function like MPI\_Allgatherv. However this does not work with armadillo, since we cannot map values into an armadillo field directly with MPI. \\

This complicates things slightly. We see two solutions to the problem, but neither is as efficient as the before mentioned one. \\

We can either allocate a new one-dimensional array and perform the prior solution with a mapping of the new amplitudes into the armadillo field afterwards. Or we can perform P one-to-all broadcasts, where P is the number of MPI procs. Then each processors sends its information to others, and this is mapped into the armadillo type array. We chose the prior, but it is slightly less effective. There will be a mapping procedure required. The scaling of the communication will be identical to the scaling of an MPI\_Allgatherv.

\begin{lstlisting}
MPI_Allgatherv(MY_OWN_MPI, WORK_EACH_NODE(rank), MPI_DOUBLE,
                   SHARED_INFO_MPI, Work_Each_Node_T2_Parallel, Displacement_Each_Node_T2_Parallel,
                   MPI_DOUBLE, MPI_COMM_WORLD);
\end{lstlisting}

Here my own MY\_OWN\_MPI holds the information calculated on the processor. SHARED\_INFO\_MPI will contain the full new non distributed T2 amplitudes. The new amplitudes are symmetric, and we only store one version of each symmetric term. SHARED\_INFO\_MPI is the array we counted as non distributed new T2 amplitudes in section \ref{kriseseksjon}. But our algorithm is really designed to distribute the the new T2 amplitudes. However because we use armadillo with MPI we need this extra variable.\\

Without the complication arising from armadillo and MPI we could also skip this mapping of symmetries into the old T2 amplitudes.

\begin{lstlisting}
for (int K = 0; K < size; K++){
   sum_a_n = 0;
   for (int a = 0; a < unocc_orb; a++){
      sum_a_n += a/2;
      A = a/2;
      AA = A * Speed_Occ - sum_a_n;

      for (int b = a+2; b < unocc_orb; b++){
         B = b/2;
         INDEX_CHECK = AA+B;
         if (INDEX_CHECK % size == K){
            for (int i = 0; i < n_Electrons; i++){
               for (int j = i+2; j < n_Electrons; j++){
                  temp = SHARED_INFO_MPI[index_counter];

          // Map out symmetries after communication
                  t2(a,i)(b/2, j/2) = temp;
                  t2(b,i)(a/2, j/2) = -temp;
                  t2(a,j)(b/2, i/2) = -temp;
                  t2(b,j)(a/2, i/2) = temp;

                  index_counter++;
                  j++;
               }
               i++;
            }
         }
         b++;
      }
   }
}
\end{lstlisting}

\subsubsection{Part 2}
Next is part two of the parallel implementation, which is the $[W_4]$ calculation. Here we also want to distribute this variable in memory. We want to store $[W_4]$ as described previous to make use of external math libraries most effectively.

\begin{equation}
W4(a,i)(c,k) .
\end{equation}
Most of the contributions to $[W_4]$ are themselves distributed in memory on the indexes a and k. We therefore perform calculations on a local processor in a cyclic grid over these two indexes. A local processor thus holds

\begin{equation}
W4(a,*)(*,k) .
\end{equation}
Here star means all terms in this index. If we temporarily swaps the indexes i and k we can store W4(a,k)(*,*), and calculate these terms.

\begin{lstlisting}
for (int a = 0; a < unocc_orb; a++){
   for (int m = Where_To_Start_Part2(rank,a); m < n_Electrons; m+=jump){
      // Fill 2D arrays ready for external math libraries
      // These are distributed in memory
      Fill_integ7_2D(a,m);
      Fill_integ5_2D(a,m);

      for (int e = 0; e < unocc_orb; e++){
         Fill_integ2_2D_even_even(e, m);
         for(int i = 0; i < n_Electrons; i++){
            W4(a,m)(e,i) = -integ7_2D(e/2,i/2)
        - accu(W_3.at(i,m)(e/2,span()) % T_1.row(a/2))
                + accu(integ5_2D(span(0, Speed_Occ-1),e/2) % T_1(span(0, Speed_Occ-1),i/2))
                + 0.5*accu(integ2_2D % t2.at(a,i));
            i++;
         }
         e++;
      }
   }
   a++;
}
\end{lstlisting}

The Where\_To\_Start\_Part2(rank,a) variable will be explained later. However, when this intermediate contributes to the T2 amplitudes we need the full matrix that is stored in $W4(a,i)$. \\

Therefore we perform a communication. To pick the correct MPI function we also need to know what to do with the array after the communication. Afterwards we want to multiply

\begin{equation}
\sum_{ck} W4(a,i)(c,k) \times t2(b,j)(c,k)   .\label{gasghashkashfbdbhcxxcnxcruu}
\end{equation}
This multiplication will run in parallel, with work distributed cyclically over $a$ and $i$. The optimal work distribution forumla is

\begin{equation}
Work\_ID = a \times n_o + i .
\end{equation}
The communication needed to get the correct array needed prior and after is thus an all-to-all personalized communication. We will use MPI\_Alltoallw. Each processor here sends its own personalized message to the other MPI procs. The message is reduced to a one dimensional array in MY\_OWN\_MPI before communication. \\

\begin{lstlisting}
MPI_Alltoallw(MY_OWN_MPI, Global_Worksize_2[rank], Global_Displacement_2[rank], mpi_types_array, SHARED_INFO_MPI, Global_Worksize_2_1[rank], Global_Displacement_2_1[rank], mpi_types_array, MPI_COMM_WORLD);
\end{lstlisting}

We then perform the multiplication in Eq. \eqref{gasghashkashfbdbhcxxcnxcruu}, with work distributed over a and i. We want this contribution to be added to the new T2 amplitudes, which themselves are work distributed over $a$ and $b$. This means we add another MPI\_Alltoallw communication to get the correct data to the correct processor. \\

We have introduced a temporary memory distributed variable $W5(a,b)(i,j)$ to store this contribution to $t_{ij}^{ab}$. The positive features of this algorithm is that we indeed get all the variables distributed in memory, and communication procedures initiated are two All-to-All communications. All-to-All is generally the most effective kind of communication in MPI. We note that the number of initiated communications is independent of number of processors. This is exactly in line with our parallel implementation guidelines states earlier. The scaling of this communication will be identical to the scaling of two MPI\_Alltoallw functions. This function is highly optimized. Also the work distribution is optimal.

\subsubsection{Part 1 \label{problem_part_ccsd_parallel}}
The final part of our parallel implementation consist of everything else. Here we construct $[W_1]$, $[W_2]$, $W_3]$, $[F_1]$, $[F_2]$ and $[F_3]$. The contribution from $[F_1]$ to $[F_2]$ and $[F_3]$ is a $n^3$ contribution. Thus we do not need to run this part in parallel. The energy and $\tau_{ij}^{ab}$ is also calculated in serial in the current implementation. These are $n^4$ terms, and will be the leading non parallel calculations. \\

To perform this part of the parallel implementation we make cyclical grids of different kinds. We can reuse the array of new T2 amplitudes, since it is not needed at this step in the calculations. This enables less memory usage. We fill the array with all numbers calculated on the processor. Then perform communication just as in Part 3, and map the correct numbers into the correct armadillo fields. \\

\begin{lstlisting}
MPI_Allgatherv(MY_OWN_MPI, Work_Each_Node_part1_Parallel[rank], MPI_DOUBLE, SHARED_INFO_MPI, Work_Each_Node_part1_Parallel, Displacement_Each_Node_part1_Parallel, MPI_DOUBLE, MPI_COMM_WORLD);
\end{lstlisting}

We combine all these variables into one communication to maximise the number of jobs to distribute in accordance to the principles stated in section \ref{work_dist_section_1341}. Also to minimize the latency by initializing less communication procedures. \\

However because we combine several different variables we combine jobs that are not of the same size. This causes problems in our job distribution. The job distribution of part one in our parallel implementation is sub-optimal. In larger calculations some CPUs can get twice the workload of other CPUs. We have used a few tricks to lessen this performance problem. These are things like shifting the job distribution.

\begin{lstlisting}
if ((Work_ID + Shift) % size == rank){
   // Perform job
}
\end{lstlisting}

The results however are not optimal and as such we cannot expect an optimal performance of this part of the CCSD implementation. Even with this concern combining all remaining variables into one MPI communication was still the better solution, compared to performing communications after each intermediate calculation.   

\newpage

\subsection{Extra Pre Iterative Procedures}
Before we start iterating we must map out a few new variables. These are extra calculations not needed in serial and includes variables such as displacement and size of messages in the communications. They are calculated in the class \\ ccsd\_non\_iterative\_part. \\

\begin{lstlisting}
if (Work_ID % size == rank)
{
    // Do calculation
}
\end{lstlisting}

We also map out which Work\_ID each processor are to perform calculations on. This is for example the Where\_To\_Start\_Part2(rank,a) variable. This enables us to remove all if tests like the one above from our iterative procedure. If tests inside a for loop can be very time consuming if the value true or false changes often from one index to the next. This is especially true if we have two processors. In this case, the value would change every time an index is changed. This means the number of pipeline flushes could potentially be large, dependant upon the compiler. \\

For P processors, the value of the if test changes after (P-1) index changes. Not having any if tests helps performance somewhat, in particular for a small number of MPI procs.


 \clearemptydoublepage
\chapter{CCSDT implementation guide \label{ccsdt_chapter}}
%
\abstract{When differential equations are required to satisfy boundary conditions
at more than one value of the independent variable, the resulting
problem is called  a {\sl boundary value problem}. The most common case by far is when boundary
conditions are supposed to be satisfied at two points - usually the
starting and ending values of the integration. The Schr\"{o}dinger
equation is an important example of such a case. Here the
eigenfunctions are typically restricted to be finite everywhere (in particular
at $r = 0$) and for bound states the functions must go to zero at infinity.}



In this chapter will discuss the implementation of CCSDT. CCSDT includes the triple electron excitation contribution. We will not derive the equations, these are available in Ref.\cite{CCSDT-ref4}. The chapter is based on this book and a series of papers, such as Ref.\cite{CCSDT-ref3} and referances therein, by Jozef Noga and Rodney Bartlett. Also Refs. \cite{CCSDT-ref5} and \cite{CCSDT-ref1} with erratium \cite{CCSDT-ref2}, and references therein. \\

This chapter is written as a guide for implementing CCSDT with simplicity and benchmarks. CCSDT requires much computation, and contains complex equations. There are however approximations made to the CCSDT equations and these have been given their own name, CCSDT-n methods. Here n = 1a,1b,2,3,4. The chapter starts with the ready CCSD equations and expands through the CCSDT-n methods one by one. \\

The CCSDT equations can be quite difficult to extract from the literature in an implementation ready form. Each section in this chapter will start with a description of which new terms are included and supply them in a factorized form ready for implementation. We will continuously use equations from Ref.\cite{CCSDT-ref4}. \\

Throughout the chapter there are provided benchmarks for each contribution added. We hope this chapter will be useful for anyone who wish to create a working CCSDT code. The additional code needed for systems not based on Hartree Fock will be provided in the final section.

\section{System for Benchmarks}
We first define our system and must ensure the input is correct if we wish to recreate the benchmarked energy. The geometries are taken from Ref.\cite{CCSDT-ref10}. We will study $H_2O$ with coordinates \\

\begin{center}
  \begin{tabular}{ c c c c }
  \hline
     Atom & x & y & z \\ \hline
     O & 0 & 0 & -0.009 \\
     H & 1.515263 & 0 & -1.058898 \\
     H & -1.515263 & 0 & -1.058898 \\
     \hline \\
  \end{tabular} 
\end{center} 

Coordinates are given in atomic units. We use a convergence criteria $10^{-7}$. The basis set is available on EMSL as DZ (Dunning). In DALTON it is available as DZ(Dunning) without the space between. A RHF calculation using this input gives energy in atomic units:

\begin{equation}
E_{RHF} = -76.0098 .
\end{equation}

The CCSD correction to the system should be:

\begin{equation}
E_{CCSD} = -0.146238 .
\end{equation}

The benchmarks can be verified in Ref.\cite{CCSDT-ref3}. We will also supply an additional benchmark of the same system outside of equilibrium. The same input is used except the geometry is now: \\

\begin{center}
  \begin{tabular}{ c c c c }
  \hline
     Atom & x & y & z \\ \hline
     O & 0 & 0 & 0 \\
     H & -2.27289   & 0 & 1.574847 \\
     H & 2.27289 & 0 & 1.574847 \\
     \hline \\
  \end{tabular} 
\end{center} 

All calculations starts with an initial guess of $t_{ijk}^{abc} = 0$.

\section{Theory}
CCSDT, as opposed to CCSD, includes $\textbf{T}_3$. 

\begin{equation}
\textbf{T} = \textbf{T}_1 + \textbf{T}_2 + \textbf{T}_3 .
\end{equation}
All the terms from CCSD will also be included in CCSDT. Indexes a,b,c,d,e,f are understood to go over virtual orbitals. Indexes i,j,k,l,m,n go over orbitals occupied in the RHF basis. The amplitude equations are defined as:

\begin{align}
\langle \Psi_i^a | \textbf{H}_N (1 + \textbf{T}_2 + \textbf{T}_1 \textbf{T}_2 + \frac{1}{2} \textbf{T}_1^2 + \frac{1}{6} \textbf{T}_1^3 + \textbf{T}_3) | \Psi_0 \rangle_C = 0.
\end{align}
Here $\textbf{T}_3$ is the new contribution, compared to the CCSD equations.

\begin{align}
\langle \Psi_{ij}^{ab} | \textbf{H}_N (
1 + \textbf{T}_2 + \frac{1}{2} \textbf{T}_2^2
+ \textbf{T}_1 + \textbf{T}_1 \textbf{T}_2 + \frac{1}{2} \textbf{T}_1^2 
+ \frac{1}{2} \textbf{T}_1^2 \textbf{T}_2 \nonumber \\
+ \frac{1}{6} \textbf{T}_1^3 + \frac{1}{24} \textbf{T}_1^4 + \textbf{T}_3 + \textbf{T}_1 \textbf{T}_3) | \Psi_0 \rangle_C = 0.
\end{align}
The new contributions are $\textbf{T}_3$ and $\textbf{T}_1 \textbf{T}_3$. 

\begin{align}
\langle \Psi_{ijk}^{abc} | \textbf{H}_N (
\textbf{T}_2 + \textbf{T}_3 + \textbf{T}_2 \textbf{T}_3 + \frac{1}{2} \textbf{T}_2^2
+ \textbf{T}_1 \textbf{T}_2 + \textbf{T}_1 \textbf{T}_3 + \frac{1}{2} \textbf{T}_1^2 \textbf{T}_2 
\nonumber \\
+ \frac{1}{2} \textbf{T}_1 \textbf{T}_2^2 + \frac{1}{2} \textbf{T}_1^2 \textbf{T}_3 + \frac{1}{6} \textbf{T}_1 \textbf{T}_2) | \Psi_0 \rangle_c = 0.
\end{align}
All of these are new contributions. The energy expression remains unchanged from CCSD. CCSDT-n methods include more and more of these new contributions. \\

For the interested reader who wish to verify the equations we will soon present with those given in Ref.\cite{CCSDT-ref4}, we add a few permutation operator tricks. One form of the permutation operator in the CCSDT equations are $\textbf{P}(a/bc)$. $\textbf{P}(a/bc)$ is defined as

\begin{equation}
\textbf{P}(a/bc) f(a,b,c) = f(a,b,c) - f(b,a,c) - f(c,b,a) .
\end{equation}
or as the permutation where a is exchanged by b and a is exchanged by c. Two of these,  $\textbf{P}(a/bc) \textbf{P}(k/ij)$, gives in total nine permutations. \\

Another form of the permutation operator is $\textbf{P}(abc)$. This is defined as the 6 permutations

\begin{align}
\textbf{P}(abc) f(a,b,c) = & f(a,b,c) - f(b,a,c) - f(a,c,b)  \nonumber \\ & - f(c,b,a) + f(b,c,a) + f(c,a,b) . \label{temporary_equation_label_for_rfeffsf}
\end{align}

These two permutation operators can be interchanged if one rewrite Eq. \eqref{temporary_equation_label_for_rfeffsf}. 

\begin{align}
\textbf{P}(abc) f(a,b,c) = & f(a,b,c) - f(b,a,c) - f(a,c,b)  \nonumber \\ & - f(c,b,a) + f(b,c,a) + f(c,a,b) \nonumber \\ &
= \left[ f(cab) - f(acb) - f(bac) \right] \nonumber \\ &
- \left[ f(cba) - f(bca) - f(abc) \right]
\nonumber \\ &
= \textbf{P}(c/ab) \left[ f(cab) - f(cba) \right] .
\end{align}

\section{CCSDT-1a}
The simplest inclusion of triples is CCSDT-1a. This method includes the contribution from $\textbf{T}_3$ in $t_i^a$, $\textbf{T}_3$ in $t_{ij}^{ab}$ and $\textbf{T}_2$ in $t_{ijk}^{abc}$. This can be expressed as

\begin{equation}
t_{ijk}^{abc} D_{ijk}^{abc} = 
\textbf{P}(a/bc) \textbf{P}(k/ij) \sum_e I_{bc}^{ek} t_{ij}^{ae}
- \textbf{P}(c/ab) \textbf{P}(i/jk) \sum_m I_{mc}^{jk} t_{im}^{ab} .
\end{equation}
Here the denominator is defined as the Moller-Plesset denominator. 

\begin{equation}
D_{ijk}^{abc} = f_{ii} + f_{jj} + f_{kk} - f_{aa} - f_{bb} - f_{cc} .
\end{equation}

To make life simpler in more advanced CCSDT-n algorithms we already insert intermediates. We define two intermediates 

\begin{equation}
[X1]_{ab}^{ei} = I_{e,i}^{ab} ,
\end{equation}
and
\begin{equation}
[X2]_{ij}^{am} = I_{am}^{ij} .
\end{equation}

This means our equation for $t_{ijk}^{abc}$ is now

\begin{align}
D_{ijk}^{abc} t_{ijk}^{abc} = & \textbf{P}(a/bc) \textbf{P}(k/ij) \sum_e [X1]^{ek}_{bc} t_{ij}^{ae} \\ \nonumber &
- \textbf{P}(c/ab) \textbf{P}(i/jk) \sum_m [X2]_{jk}^{mc} t_{im}^{ab} .
\end{align}

CCSDT-1a makes no changes in the calculation of the energy, however the $\textbf{T}_3$ contributions are added to $t_i^a$ and parts of the $\textbf{T}_3$ contribution are added to $t_{ij}^{ab}$. To indicate the terms from the original CCSD method should also be included we use $\leftarrow$. 

\begin{equation}
D_i^a t_i^a \leftarrow \frac{1}{4} \sum_{bcjk}  I_{j,k}^{b,c} t_{ijk}^{abc} m
\end{equation}
and
\begin{align}
D_{ij}^{ab} t_{ij}^{ab} \leftarrow & 
\frac{1}{2} \sum_{kcd} I_{bk}^{cd} t_{ijk}^{acd}
- \frac{1}{2} \sum_{kcd} I_{ak}^{cd} t_{ijk}^{bcd} - \frac{1}{2} \sum_{mkc} I_{mk}^{jc} t_{imk}^{abc} + \frac{1}{2} \sum_{mkc} I_{mk}^{ic} t_{jmk}^{abc} .
\end{align}

CCSDT-1a actually gives a surprisingly good approximation to the full CCSDT energy, since the terms between the two usually adds and subtracts a similar amount to the energy. \\

The equations for $t_i^a$ is the same for CCSDT as for CCSDT-1a. Implementing the new equations should give the following result in atomic units for the system in equilibrium:

\begin{equation}
E_{CCSDT-1a} = -0.147577 .
\end{equation}
For the system outside of equilibrium we should have energy

\begin{equation}
E_{CCSDT-1a} = -0.209537 .
\end{equation}

\section{CCSDT-1b}
CCSDT-1b adds the remaining contribution to $t_{ij}^{ab}$. 

\begin{align}
D_{ij}^{ab} t_{ij}^{ab} \leftarrow
\sum_{klcd} I_{kl}^{cd} t_{ijk}^{abc} t_l^d  \frac{1}{2} \sum_{klcd} I_{kl}^{cd} \left(
 t_{ikj}^{bcd} t_l^a - t_{ikj}^{acd} t_l^b 
 + t_{kli}^{adb} t_j^c - t_{klj}^{adb} t_i^c \right) .
\end{align}

The energy correction at equilibrium is now:

\begin{equation} 
E_{CCSDT-1b} = -0.147580 .
\end{equation}

The same system outside of equilibrium should have a correction energy of

\begin{equation}
E_{CCSDT-1b} = -0.209517  .
\end{equation}

\section{CCSDT-2}
For CCSDT-2 we add all contributions from T2 that does not include T1 to $t_{ijk}^{abc}$. This explicitly includes the terms

\begin{align}
D_{ijk}^{abc} t_{ijk}^{abc} \leftarrow & 
\textbf{P}(i/jk) \textbf{P}(abc) \sum_{lde} I_{lb}^{de} t_{il}^{ad} t_{jk}^{ec}
+ \textbf{P}(ijk) \textbf{P}(a/bc) \sum_{lmd} I_{lm}^{dj} t_{il}^{ad} t_{mk}^{bc} \nonumber \\ &
- \frac{1}{2} \textbf{P}(i/jk) \textbf{P}(c/ab) \sum_{lde} I_{lc}^{de} t_{il}^{ab} t_{jk}^{de} \nonumber \\ & 
 + \frac{1}{2} \textbf{P}(k/ij) \textbf{P}(a/bc) \sum_{lmd} I_{lm}^{dk} t_{ij}^{ad} t_{lm}^{bc}  .
\end{align}

In our implementation this means changing X1 and X2, since we introduced these intermediates earlier. It should be noted that since there are currently no $\textbf{T}_3$ contributions to $t_{ijk}^{abc}$ these amplitudes does not need to be stored for each iteration. This feature applies to CCSDT-n methods up but not including CCSDT-4. The new intermediates for CCSDT-2 will be:

\begin{align}
[X1]_{ab}^{ie} = I_{ab}^{ie} +
\frac{1}{2} \sum_{mn} t_{mn}^{ab} I_{mn}^{ei} ,
\end{align}
and
\begin{align}
[X2]_{ij}^{am} = I_{ma}^{ij} + \frac{1}{2} \sum_{ef} t_{ij}^{ef} I_{ma}^{ef}  .
\end{align}
We also introduce two new intermediates.

\begin{equation}
[X12]_{ab}^{id} = \sum_{ld} I_{lb}^{ed} t_{il}^{ae} ,
\end{equation}
and
\begin{equation}
[X13]_{ij}^{al} = \sum_{md} I_{ml}^{dj} t^{ad}_{im} .
\end{equation}
The expression for $t_{ijk}^{abc}$ should be changed accordingly, leaving the contribution from CCSDT-1b untouched. This is indicated by the $\leftarrow$. 

\begin{align}
t_{ijk}^{abc} \leftarrow &
\sum_e \textbf{P}(i/jk) \textbf{P}(abc) [X12]_{ab}^{ie} t_{jk}^{ec}
+ \sum_m \textbf{P} (ijk) \textbf{P}(a/bc) [X13]_{ij}^{am} t_{mk}^{bc} .
\end{align}
The energy correction should now be in equilibrium:

\begin{equation}
E_{CCSDT-2} = -0.147459 .
\end{equation}
While outside of equilibrium:

\begin{equation}
E_{CCSDT-2} = -0.208938 .
\end{equation}
The latter was achieved in our program in 30 iterations.

\section{CCSDT-3}
CCSDT-3 adds all remaining contributions to $t_{ijk}^{abc}$ that does not themselves contain $T_3$ amplitudes. These  all contain T1. The equations are available in \cite{CCSDT-ref4}. For our purposes we continue with the implementation ready equations. For CCSDT-3 we must make a tweak in our intermediates, X12 and X13. The terms from CCSDT-2 remain, but we add some new ones. We also introduce two new intermediates.

\begin{align}
[X12]_{ab}^{ie} \leftarrow  & - \sum_l I_{al}^{id} t_l^b - \sum_{le} I_{lb}^{ed} t_i^e t_l^a - \sum_{lme} I_{lm}^{ed} t_m^b t_{il}^{ae} ,
\end{align}

\begin{align}
[X13]_{ij}^{am} \leftarrow & 
\sum_{md} I_{ml}^{dj} t_i^d t_m^a
- \sum_d I_{al}^{id} t_j^d
- \sum_{mde} I_{ml}^{de} t_{im}^{ad} t_j^e ,
\end{align}

\begin{align}
[X14]_{ab}^{id} = & 
\sum_{elm} I_{lm}^{ed} \left[
+ t_i^e t_l^a t_m^b
- t_l^e t_{im}^{ab} 
+ \frac{1}{2} I_{lm}^{ed} t_i^e t_{lm}^{ab}
\right]
\nonumber \\ &
+ \sum_{lm} I_{lm}^{id} t_l^a t_m^b
+ \sum_e I_{ab}^{ed} t_i^e ,
\end{align}
and
\begin{align}
[X15]_{ij}^{am} = &
\sum_m I_{ml}^{ij} t_m^a
- \sum_{ed} I_{al}^{de} t_i^d t_j^e
+ \frac{1}{2} \sum_{med} I_{ml}^{de} \tau_{ij}^{de} t_m^a  .
\end{align}
These two intermediates must be included in our $t_{ijk}^{abc}$ equation.

\begin{equation}
t_{ijk}^{abc} \leftarrow
\sum_e \textbf{P}(a/bc) \textbf{P}(k/ij)
[X14]_{ab}^{ie} t_{jk}^{ec}
+ \sum_m \textbf{P}(c/ab) \textbf{P}(i/jk)
[X15]_{ij}^{am} t_{mk}^{bc} .
\end{equation}
Inserting these equations we should now have the following energy correction in equilibrium:

\begin{equation}
E_{CCSDT-3} = -0.147450 .
\end{equation}
and outside equilibrium:

\begin{equation}
E_{CCSDT-3} = -0.208876 .
\end{equation}
For the reader who wish to optimize a CCSDT program, the four intermediates [X12], [X13], [X14] and [X15] can actually be placed inside [X1] and [X2], using the permutation operator tricks defined in Eq. \eqref{temporary_equation_label_for_rfeffsf}. \\

\section{CCSDT-4}
For CCSDT-4 we will add the terms that are linear in T3. This corresponds to the terms

\begin{align}
D_{ijk}^{abc} t_{ijk}^{abc} \leftarrow &
\sum _{ld}
\textbf{P}(i/jk) \textbf{P}(a/bc) 
I_{al}^{id} t^{dbc}_{ljk}
+
\frac{1}{2} \sum_{mk}
\textbf{P}(k/ij) I_{lm}^{ij} t_{lmk}^{abc}
\nonumber \\ &
+
\frac{1}{2} \sum_{de} 
\textbf{P}(c/ab) I_{ab}^{de} t^{dec}_{ijk} .
\end{align}
We will introduce these terms as three new intermediates, because there are more terms in full CCSDT that can use this factorization. These are defined as

\begin{equation}
[X3]_{ij}^{lm} = \frac{1}{2} I_{lm}^{ij} ,
\end{equation}

\begin{equation}
[X4]_{ab}^{de} = \frac{1}{2} I_{ab}^{de} ,
\end{equation}
and
\begin{equation}
[X6]_{al}^{id} = I_{al}^{id} .
\end{equation}
Inserting this in the amplitude equation leaves

\begin{align}
D_{ijk}^{abc} t_{ijk}^{abc} \leftarrow &
\sum _{ld}
\textbf{P}(i/jk) \textbf{P}(a/bc) 
[X6]_{al}^{id} t^{dbc}_{ljk}
+
 \sum_{mk}
\textbf{P}(k/ij) [X3]_{ij}^{lm} t_{lmk}^{abc}
\nonumber \\ &
+
\sum_{de}
\textbf{P}(c/ab)[X4]_{ab}^{de}  t^{dec}_{ijk} .
\end{align}
We again perform energy calculations on the same system as before. At equilibrium the energy correction should now be

\begin{equation}
E_{CCSDT-4} = -0.147613 .
\end{equation}
And off-equilibrium we should have

\begin{equation}
E_{CCSDT-4} = -0.209668 .
\end{equation}

\section{Full CCSDT}
For full CCSDT we introduce all the remaining terms. We will add these into our existing intermediates, and define a few new ones.

\begin{equation}
[X1]_{ab}^{ic} \leftarrow 
\sum_{lme}
\frac{1}{2}
I_{lm}^{ce} t_{lmi}^{aec} .
\end{equation}

\begin{equation}
[X2]_{ij}^{am} \leftarrow 
\sum_{lde}
\frac{1}{2}
I_{ml}^{de} t_{ilk}^{dea} .
\end{equation}

\begin{equation}
[X3]_{ij}^{lm} \leftarrow
\sum_d \left( I_{lm}^{dj} t_i^d
- I_{lm}^{di} t_j^d \right)
+ \sum_{de} \frac{1}{2} I_{lm}^{de} \tau_{ij}^{de} .
\end{equation}

\begin{equation}
[X4]_{ab}^{de} \leftarrow
\sum_l \left( 
I_{lb}^{de} t_l^a
- I_{la}^{de} t_l^b \right)
+ \sum_{ml} \frac{1}{2} I_{lm}^{de} \tau_{lm}^{ab} .
\end{equation}

\begin{equation}
[X6]_{al}^{id} \leftarrow 
\sum_{e} I_{al}^{ed} t_i^e
- \sum_m I_{ml}^{id} t_m^a
+ \sum_{em} I_{ml}^{ed} t_{im}^{ae}
- \sum_{em} I_{ml}^{ed} t_i^e t_m^a .
\end{equation}
We also introduce two new intermediates, [X7] and [X8].

\begin{equation}
[X7]_i^m = - \sum_{ld} I_{lm}^{di} t_l^d
- \frac{1}{2} \sum_{lde} I_{lm}^{de} \tau_{li}^{de} .
\end{equation}

\begin{equation}
[X8]_a^e = \sum_{ld} I_{la}^{de} t_l^d
- \frac{1}{2} \sum_{dlm} I_{lm}^{de} \tau_{lm}^{da} .
\end{equation}
These are added to $t_{ijk}^{abc}$ with the following permutation operators in front

\begin{equation}
D_{ijk}^{abc} t_{ijk}^{abc} \leftarrow \sum_e \textbf{P}(a/bc) [X8]_a^e t_{ijk}^{ebc}
+ \sum_m \textbf{P}(i/jk) [X7]_i^m  
t_{mjk}^{abc} .
\end{equation}
Implementing these terms, we get the full CCSDT energy correction in equilibrium

\begin{equation}
E_{CCSDT} = -0.147594 .
\end{equation}
This is identical to benchmark. Outside of equilibrium we get

\begin{equation}
E_{CCSDT} = -0.2095(20) .
\end{equation}
Here we marked a parenthesis around (20) due to the fact that Bartlett in his letters gives this energy as

\begin{equation}
E_{Bartlett} = -0.209519 .
\end{equation}
Meaning we have a difference of -0.000001 to Bartletts results. We will assume this is caused by round off errors.  

\section{Excluded Terms}
Some terms are zero when using a HF basis, because the Fock eigenvalues are diagonalized. If we want to perform CCSDT calculations for anything other than a HF basis, we must add these terms

\begin{equation}
[X1]_{ab}^{ic} \leftarrow - \sum_{ld} \langle l | F | d \rangle t_{li}^{ab} .
\end{equation}

\begin{equation}
[X15]_{ij}^{al} \leftarrow - \sum_{md} \langle m | F | d \rangle t_{ij}^{ad} .
\end{equation}

\begin{equation}
[X8]_a^d \leftarrow - \sum_l \langle l | F | d \rangle t_l^a .
\end{equation}

\begin{align}
t_{ijk}^{abc} \leftarrow &
\sum_d \textbf{P}(c/ab)
(1 - \delta_{cd}) \langle c | F | d \rangle t_{ijk}^{abd}
- \sum_l \textbf{P}(k/ij) 
(1 - \delta_{kl}) \langle k | F | l \rangle  t_{ijl}^{abc}
\nonumber \\ &
-
\sum_{ld}
\textbf{P}(i/jk)
\langle l | F | d \rangle t_i^d  t3^{abc}_{ljk} .
\end{align}


\chapter{Benchmarks}
In this chapter we will benchmark our code. We have already performed calculations using CCSDT and verified them with benchmark values. The full CCSDT method took advantage of all our implementation except for the UHF part. In this chapter we will look at the codes performance and find out for sure does it works for more systems. Also what are the strengths and what are the weaknesses of our implementations. We will look at all our methods, RHF, UHF, CCSD, and CCSDT. Also we will test our memory distributed AOtoMO transformation algorithm to its limits. In general we will benchmark our code against LSDALTON, Ref.\cite{LSDALTON_CITATION}, but we will also provide additional benchmarks in each section. CCSDT is neither available in DALTON nor LSDALTON. \\

We also mention no special flags are used to compile. We will supply plenty of performance results. The flags used were 

\begin{lstlisting}
CFLAGS = -pipe -O2 -Wall -W 
\end{lstlisting}

\section{Small systems}
We first perform some initial testing on small systems of water and $H_2$. These will be compared with the LSDALTON package, aswell as Ref.\cite{ccsd_benchmark_url_stuff} and Ref.\cite{CCSDT-ref1}. \\

We use these coordinates for all tests except with the DZ basis set, where we use the coordinates we used throughout chapter \ref{ccsdt_chapter}.  

\begin{center}
  \begin{tabular}{ c c c c }
  \hline
     Atom & x & y & z \\ \hline
     O & 0 & 0 & 0 \\
     H & 0 & 1.079252144093028 & 1.474611055780858 \\
     H & 0 & 1.079252144093028 & -1.474611055780858 \\
     \hline \\
  \end{tabular} 
\end{center} 

These are in atomic units. The coordinates are taken from Ref.\cite{ccsd_benchmark_url_stuff}.

\begin{center}
\begin{tabular}{ l c c r }
	\hline
  	Basis Set & RHF & CCSD Correction & Benchmark \\ \hline
  	STO-3G & -74.9627 & -0.0501273 & \cite{ccsd_benchmark_url_stuff} \\ 
  	4-31G & -75.9081 & -0.13668 & LSDALTON \\ 
  	6-31G & -75.9845 & -0.13603 & LSDALTON \\ 
  	DZ & -76.0098 & -0.146238 & \cite{CCSDT-ref1} \\ \hline
  	\\
	\end{tabular}
\end{center}

Our results are in agreement with the references down to the final decimal. We examine the RHF calculations with the STO-3G basis set more closely, in its components.  \\

\begin{center}
\begin{tabular}{ l r }
  	One-electron energy & = -122.219 \\ 
  	Two-electron energy & = 38.1615 \\
  	Repulsion energy & = 9.09485 \\
  	\\
	\end{tabular}
\end{center}

This is in perfect agreement with reference. With DIIS turned on this was achieved in 13 iterations with RHF. With DIIS turned off we needed 21 iterations.

\section{Hydrogen molecule}
Our next calculations will be on the diatomic hydrogen molecule. These results will be benchmarked against an FCI study from 1968, Ref.\cite{fci_h2_molecule_stuff}. Pople and others pioneered the effective use of Gaussian Type Orbitals throughout the 1970s. The FCI calculation used Slater Type Orbitals. We will plot the energy as a function of R, where R is the distance between the two nuclei in a.u. The results are available in figure \ref{fig:h2poten}. We will use the 6-311++G(2d,2p) basis set for both HF and CCSD calculations, and a convergence criteria of $10^{-5}$. Calculations are performed from R = 0.6 to R = 4.5 with 0.1 a.u. intervals. \\

\begin{figure}[h!]
\begin{center}
\fbox{\includegraphics[width=\textwidth]{H2_CCSD_HF_plot.eps}}
\caption{Energypotential for $H_2$ Molecule. 
$^a$FCI results from Ref.\cite{fci_h2_molecule_stuff}}
\label{fig:h2poten}
\end{center}
\end{figure}

The energy minimum is located at R = 1.4 au with an energy of -1.17086. This was benchmarked against LSDALTON. We could also perform CCSDT calculations on this molecule, but we only have two electrons. This means the answer will be the same as a CCSD calculation, because we do not have three electrons to excite making all the $t_{ijk}^{abc}$ amplitudes 0. This was also confirmed by our program. \\

Also, this is true for all higher versions of coupled cluster, meaning the only difference between these results and the  reference value should be a truncated basis set. We therefore repeat our calculation with the largest Pople basis set the author is aware of, 6-311++G(3df,3pd). This calculation is marked with (a). We use R = 1.4011, which is the equilibrium distance according to our FCI benchmark. We also try the aug-cc-pVQZ basis set. This calculations is marked with (b). Our calculations results in an energy in atomic units:

\begin{equation}
E_{CCSD,(a)} = -1.17264 .
\end{equation}

\begin{equation}
E_{CCSD,(b)} = -1.17394.
\end{equation}
According to the benchmark FCI energy with these nuclei positions is -1.17447. \\

We did benchmark our results against LSDALTON and the two programs were in agreement for the same basis set. On our energy plot FCI results are plotted for $R \in [1.0, 2.0]$, however, results for R = 1.1 was lacking.

\section{First row Diatomic molecules}
The natural next step is to introduce heavier atoms in our diatomic molecule. In these systems CCSD no longer include the full correlation, and we will have more sources of error than just the basis set truncation. We use a decent sized basis set, 6-311++G(2d,2p) and a convergence criteria of $10^{-6}$. Our results are benchmarked against a paper with DMC calculation and marked with $^a$, see Ref.\cite{first_row_diatomic_referance_stuff}. \\

\begin{center}
\begin{tabular}{ l c  c c c r }
	\hline
  	Molecule & R [au] & $E_{HF}$ & $E_{CCSD}$ & 	$E_0^a$ & $E_R^b$ \\ \hline
  	$Li_2$ & 5.051 & -14.8701 & -14.9322 & -14.995 &  0.50 \\\hline
  	$Be_2$ & 4.63 & -29.1321 & -29.2646 & -29.338 &  0.64 \\ \hline
  	$B_2$ & 3.005 & -48.8656 & -49.1738 & -49.415 &  0.56 \\ \hline
  	$C_2$ & 2.3481 & -75.3973 & -75.7703 & -75.923 &  0.71 \\ \hline
  	$N_2$ & 2.068 & -108.979 & -109.367 & -109.542 & 0.70 \\ \hline
  	$O_2$ & 2.282 & -149.58 & -150.058 & -150.326 & 0.64  \\ \hline
  	$F_2$ & 2.68 & -198.741 & -199.272 & -199.529 & 0.67 \\ \hline
  	\\
	\end{tabular}
\end{center}

In this table the molecule is listed to the left. R is the distance between the nuclei. $E_{HF}$ and $E_{CCSD}$ is the HF and CCSD energies for the system. $E_0$ is our benchmark value, from Ref.\cite{first_row_diatomic_referance_stuff}. This is the exact, non-relativistic, infinite nuclei mass energy. We define $E_R$ to be the percentage of correlation recovered using CCSD.

\begin{equation}
E_R = \frac{E_{CCSD} - E_{HF}}{E_0 - E_{HF}} .
\end{equation}
We also benchmark our results against Henrik Mathias Eidings RHF - MP2 and MP3 results for $O_2$, as seen in Ref.\cite{hmeiding}. His results are marked as $^a$. We use the larger 6-311++G(3df,3pd) basis set, same as is used by Henrik. 

\begin{center}
\begin{tabular}{ l c c c r }
	\hline
  	Molecule & HF & MP2$^a$ & MP3$^a$ & CCSD \\ \hline
  	$O_2$ & -149.588 & -150.142 & -150.130 & -150.138 \\\hline
  	\\
	\end{tabular}
\end{center}

$O_2$ is commonly known as an open shell molecule, meaning our spin restriction is likely to cause problems. We therefore repeat this calculation using our unrestricted Hartree Fock implementation. \\

The first calculation for $O_2$ is performed with singlet spin orientation, meaning total spin is 0. All other input remains the same. 

\begin{equation}
E_{UHF,0} = -149.647 .
\end{equation}
The triplet state, where total spin is 1, gives energy

\begin{equation}
E_{UHF,1} = -149.674  .
\end{equation}
These results are in good agreement with Henriks calculations. The singlet calculation differ from his calculations with 0.001 in energy. In our iterations we had Henriks energy exact, but we used a stronger convergence criteria and continued iterations. The number of iterations needed was about 80. 

\section{$C_{20}$ Ground State} 
$C_{20}$ is a particularly interesting molecule. Calculations using different computational methods seem to provide very different answers to the geometry of the ground state, as is noted in Ref. \cite{c20article_cite_this}. There are three structures in considerations. They are ring, bowl and cage. All these structures are present in experiment, but ring seems to be the most likely orientation. This is followed by bowl as the second most likely, and cage as third most likely.  \\

\begin{figure}[h!]
\begin{center}
\fbox{\includegraphics[width=\textwidth]{c20_ring.eps}}
\caption{Orientation for $C_{20}$ in ring formation.}
\label{fig:c20ring}
\end{center}
\end{figure}

However methods considered highly accurate, such as DMC, does not agree with experiment. Different methods do not even agree with each other, as is seen in Ref.\cite{c20coordinatesarticlezz}. One theory for the disagreement is that experiments are not done at 0 Kelvin, while quantum chemistry ground state calculations are. We will perform RHF and CCSD calculations on the system. \\

We will use the 6-31G basis set, with a convergence criteria of $10^{-4}$. For a reference we calculate the ground state energy of a single carbon atom.

\begin{figure}[h!]
\begin{center}
\fbox{\includegraphics[width=\textwidth]{c20_bowl.eps}}
\caption{Orientation for $C_{20}$ in bowl formation.}
\label{fig:c20bowl}
\end{center}
\end{figure}

\begin{equation}
E_{HF} = -37.58 .
\end{equation}

\begin{equation}
20 \times E_{HF} = -751.6 .
\end{equation}

\begin{equation}
E_{CCSD} = -37.65 .
\end{equation}

\begin{equation}
20 \times E_{CCSD} = -753.0 .
\end{equation}

We would also like to visualize the three different orientations. All coordinates are taken from Refs.\cite{c20coordinatesarticlezz} and \cite{c20coordinatesarticlezz10}. Both these references use the same coordinates. The geometries for bowl, cage and ring are illustrated in figures \ref{fig:c20bowl}, \ref{fig:c20cage} and \ref{fig:c20ring}.\\




\begin{figure}[h!]
\begin{center}
\fbox{\includegraphics[width=\textwidth]{c20_cage.eps}}
\caption{Orientation for $C_{20}$ in cage formation.}
\label{fig:c20cage}
\end{center}
\end{figure}

\begin{center}
\begin{tabular}{ l c r }
	\hline
  	Orientation & RHF & CCSD \\ \hline
  	ring & -756.454 & -758.261  \\ \hline
  	cage & -756.122 & -758.004  \\ \hline
  	bowl & -756.312 & -758.195  \\ \hline
  	\\
	\end{tabular}
\end{center}

Calculations on ring was performed without DIIS in HF, whereas cage and bowl had DIIS enabled. This was required to achieve convergence. We see the energy is lower than 20 times the energy of a single carbon for all three orientations. Our results are in good agreement with Ref.\cite{c20coordinatesarticlezz10} and experimental results. We have indeed found the ring to be the lowest energy orientation. However, we did not use a particularly large basis set. Thus our energies are not as low as we would want. According to Ref.\cite{c20article_cite_this} the energies for this system should be around -759. \\

We may not have settled the debate about the ground state orientation for $C_{20}$ at 0 Kelvin temperature, but we did manage to perform calculations on a somewhat larger molecule, showcasing some of our programs memory distributed potential. 

\section{Energy as function of number of AOs}
The size of the basis set has a great impact on our calculations. However at some point, the basis set is so large that making it even larger will not provide any noticeable improvement in accuracy. Any increase in basis set size will however always increase the runtime of our program. For this reason there is great interest in knowing what size basis set gives what level of accuracy. \\

To study this we have performed calculations on a single water molecule, $H_2O$. We have performed calculations using the STO-3G, 6-31G, 6-311G**, 6-311++G(2d,2p) and 6-311++G(3df,3pd). These basis sets are listed from smallest to largest. Results are provided in fig. \ref{fig:convplot}. \\

\begin{figure}[h!]
\begin{center}
\fbox{\includegraphics[width=\textwidth]{convergenceplot.eps}}
\caption{Energy of $H_2O$ as a function of number of AOs}
\label{fig:convplot}
\end{center}
\end{figure} 

We check the convergence with respect to the number of AOs for both HF and CCSD. We see the HF energy has better convergence than CCSD. CCSD is not particularly well converged at all, this was to be expected however based on our $H_2$ results from earlier. \\

The largest calculation used approximate 80 AOs. With 80 AOs this would be $n_o = 10$ and $n_v = 150$, meaning $\frac{n_v}{n_o} = 15$. When performing CCSD calculations this fraction is normally expected to be between 5 - 10.

\section{Hartree Fock Performance Testing}
In this section we will test the performance of our Hartree Fock program. We will first look at memory usage, and afterwards discuss the performance of both RHF and UHF for different systems. We have not spent much time on optimizing this part of the program, except for the parallel implementation and memory distribution. \\

\begin{figure}[h!]
\begin{center}
\fbox{\includegraphics[width=\textwidth]{available_memory.eps}}
\caption{Memory Requirements for HF as a function of N}
\label{fig:memory_needs}
\end{center}
\end{figure}

In figure \ref{fig:memory_needs} we have plotted the memory needs for our HF program as a function N for the serial version. N is the number of AOs. We will perform our calculation on the Abel computing cluster, Ref.\cite{abel_po_g_citation1234567}. This cluster consist of nodes. Each nodes has 16 CPUs and 64 GB of memory. This means each CPU has 4 GB of memory, if we discount that a small portion of the memory is already occupied by processes already running on the node. \\

On our figure we have marked the total available memory for a different number of CPUs. We are able to distribute the absolute dominant memory consuming array. This means the crossing point between available memory and needed memory is a good indication of the number of AOs we can run with this number of CPUs. For example 512 CPUs crosses the blue line at N = 700. This means we can do approximately 700 AOs with 512 CPUs. \\

It is possible on abel to ask for more memory for each CPU. There are also high-memory nodes available. In theory we could ask for all 64 GB of memory and only use one CPU on the node. This would however leave the other 15 CPUs unusable for other users, since there is no memory left. 

\subsection{HF performance}
To test the performance of our HF implementation we perform calculations on $O_2$ with the 6-311++G(3df,3pd) basis set. We use convergence criteria of $10^{-8}$. Only the four index integrals are run in parallel, so there are some serial calculation involved. The UHF implementation has more communication than RHF. We plot both performances in the same plot for comparison. The raw data is included in the table, and the timings involve all calculations in HF. Results are plotted in figure \ref{fig:hf_performance_stuff_jesus1234}.

\begin{center}
\begin{tabular}{ l c c }
	\hline
  	P & RHF time [s] & UHF time [s] \\ \hline
  	1 & 675.78 & 854.31 \\ 
  	2 & 322.16 & 411.68 \\ 
  	4 & 182.36 & 324.26 \\
  	8 & 129.94 & 123.69 \\ 
  	16 & 60.79 & 73.52 \\ 
  	32 & 36.58 & 49.87 \\ 
  	64 & 27.55 & 55.32 \\ 
  	128 & 25.02 & 44.59 \\
  	256 & 34.29 & 49.1  \\ \hline
  	\\
	\end{tabular}
\end{center}

\newpage

\begin{figure}[h!]
\begin{center}
\fbox{\includegraphics[width=\textwidth]{hartree_fock_parallel_performance.eps}}
\caption{Time in seconds for a HF calculation on $O_2$ with the 6-311++G(3df,3pd) basis set using $2^p$ processors. Both RHF and UHF calculations included. Results are not averaged over multiple runs.}
\label{fig:hf_performance_stuff_jesus1234}
\end{center}
\end{figure}


\section{AOtoMO Performance Testing \label{aotomoperformancetesting}}
In this section we will test the performance of our AOtoMO transformation algorithm for the four index integrals, as a function of number of CPUs. The raw data is included in the table, and we plot the data in two plots. The quantities of interest are time used in calculation versus time used in communication. Calculations spread over more CPUs can be performed faster. However more CPUs will require more communication. Calculations on 251 and 569 AOs are done on the $C_{12} H_{22} O_{11}$ molecule, sucrose. We use the 6-31G and the 6-311++G** basis sets. The 130 AO calculation is done on an imaginary molecule, simply to measure performance. \\

 \begin{center}
  \begin{tabular}{ c  c  c  c }
  \hline
     p & AOs = 130 & AOs = 251 & AOs = 569  \\ \hline
     1 & 123.55 & 3458  & - \\
     2 & 69.64 & 1800  & - \\
	 4 & 37.73 & 923  & -  \\
     8 & 25.51 & 618  & -  \\
    16 & 17.11 & 389  & 21076 \\
    32 & 14.52 & 297  & 15556 \\
    64 & 10.35 & 266  & 14961 \\
    128 & 10.62 & 232  & 11294
\\
    256 & - & -  & 9413
\\
    512 & - & -  & 9346
    
     \\ \hline \\
  \end{tabular} 
\end{center} 

We define wall time as the time from the first CPU start until the last CPU finish. We will use this as a measurement of performance of our algorithm. We will also define the point where wall time increase with increased number of CPUs as the time communication overtakes computation. \\

We should first remind ourselves the scaling here is $N^5$, where N is the number of AOs. The communication however scales as $N^4$. We see from our benchmarks that larger number of AOs scales better with increased number of CPUs. \\

\begin{figure}[h!]
\begin{center}
\fbox{\includegraphics[width=\textwidth]{scaling_1.eps}}
\caption{AOtoMO transformation scaling for small number of AOs and up to 128 processors, p. Plotted on y - axis is T/$T_0$, where $T_0$ is the wall time for the serial version. On the x - axis we have $2^p$, where p is the number of CPUs}
\label{fig:aotomo1}
\end{center}
\end{figure}

We notice especially that the wall time increase from 64 to 128 CPUs for 130 AOs. This does not happen for 251 AOs. Indicating a better scaling for higher number of AOs, as expected from the algorithm. We also note that in a less optimized AOtoMO transformation, communication would overtake computation much quicker. \\

Our memory distributed model makes us able to run larger systems using increased number of CPUs. The memory usage of the algorithm is closely related to that of our Hartree Fock implementation. This is due to the fact that in our HF program we stored all terms of the four index integrals, whereas after HF is completed we will delete the terms corresponding to one symmetry, reducing the memory requirements by $\frac{1}{2}$. The transformed integrals are also stored using this one symmetry, and need $\frac{N^4}{2}$ memory. These two combined equals the memory needed for HF. \\

For implementation reasons there will be an additional $N^3$ array needed. This is because we will need one $N^3$ intermediate for armadillo, and one for the communication in MPI. This changes memory requirements  modestly, but becomes a problem for approximate 1000 AOs. \\

For 569 AOs we are unable to perform the calculation in serial. We start calculations using 16 CPUs. From figure \ref{fig:aotomo1} we notice the best scaling is behind us at this number of CPUs. However we can make an educated guess that running this sized system in serial would require several days of computations. \\

\begin{figure}[h!]
\begin{center}
\fbox{\includegraphics[width=\textwidth]{scaling_2.eps}}
\caption{AOtoMO transformation wall time scaling for 569 AOs and $2^p$ CPUs. 16 to 512 CPUs used. }
\label{fig:aotomo2}
\end{center}
\end{figure}

Our results for 569 AOs confirm that the scaling is better with increased number of AOs. Even from 256 to 512 CPUs we improve overall performance.  \\ 

We also list a few single calculations for different number of AOs and CPUs, to see the performance of the transformation for a variety of calculations. The basis set used for the calculations was 6-311++G(2d,2p). \\

  \begin{center}
  \begin{tabular}{ c c c c }
  \hline
     Molecule & CPUs & AOs & AOtoMO \\ \hline
     $CH_4$ & 16 & 69 & 0.98 s \\
     $C_2H_6$ & 64 & 118 & 9.71 s \\
     $C_2 O H_6$ & 64 & 147 & 19.9 s \\
     $\left(H_2O\right)_8$ & 256 & 392 & 23 min \\
     $C_{20}$ & 256 & 580 & 238 min \\
     $\left(H_2O\right)_{15}$ & 512 & 735 & 35.8 hour \\
     \hline \\
  \end{tabular} 
\end{center} 

We see up to 600 AOs is very doable calculations. At 735 AOs the transformation itself takes more than one day. However, these timings are very good relative to others, Ref.\cite{aotomo_2_cite}. Full AOtoMO transformations are rare in the literature, especially for high number of AOs. It is therefore difficult to make comparisons. \\ 

Alternatively, on the benchmark page for ACESIII, Ref.\cite{aces_non_ref}, they list the wall time of a few CCSD calculations. CCSD regularly use the full AOtoMO transformation, and ACESIII is a very optimized CCSD program. Thus it is interesting to see if CCSD or AOtoMO would dominate wall time in a calculation with an optimized CCSD version. \\

An $Ar_6$ calculation with 300 AOs is listed on ACESIII benchmark site with 128 CPUs as 5.9 min per CCSD iteration. Using our AOtoMO transformation, 251 AOs was performed in 3.8 min. \\

They also list calculations on sucrose, $C_{12} H_{22} O_{11}$, using the 546 AOs, and 23 orbitals frozen. This means 523 unfrozen spin orbitals. They report 29.4 minutes for one iteration of 256 CPUs. Our four index transformation spent 156 minutes on 569 AOs. \\

These two results indicate the wall time for our transformation for this number of AOs lies approximated between 2-4 CCSD iterations, which is very reasonable. CCSD usually requires 10-20 iterations for convergence in equilibrium. Most of the benchmarks on the ACESIII site for CCSD with up to 512 CPUs is between 300 - 600 AOs, all of which is doable with this algorithm. \\

It should also be noted that our full AOtoMO transformation only depends upon number of AOs, whereas CCSD depends on the combination of occupied versus virtual orbitals. In a CCSD calculation with $n_v >> n_o$ AOtoMO would be more time consuming relative to a CCSD iteration. 

\section{CCSD Serial Performance \label{ccsdserialperformance_ppp}}
In this section we will test the CCSD performance in serial. We will try a cluster of water molecules, of size $(H_2O)_N$. We will use the 6-311++G(2d,2p) basis set. We also note the serial time of the AOtoMO transformation. 

\begin{center}
\begin{tabular}{ l c c c c c }
	\hline
  	System & AOs & $n_o$ & $n_v$ & CCSD iteration [s] & AOtoMO [s] \\ \hline
  	$(H_2O)$ & 49 & 10 & 88      & 1.17 & 0.75  \\ 
  	$(H_2O)_2$ & 98 & 20 & 176   & 37 & 14.41 \\ 
  	$(H_2O)_3$ & 147 & 30 & 264  & 632 & 297  \\
  	$(H_2O)_4$ & 196 & 40 & 352  & 2255 & 1454 \\ 
  	\hline
  	\\
	\end{tabular}
\end{center}

We test the serial implementation against an unoptimized but factorized CCSD program. This implementation is available through "CCSD1" as method in the input file. The equations are fully factorized, so the theoretical scaling of the equations is still $N^6$, where N is the number of AOs. However we have not implemented the use of external math libraries, compact storage and the other optimizations discussed in section \ref{optimize_serial_version_bii}. These calculations are not performed on abel. We also use the external math library BLAS. The performance here on one CPU is generally better than on abel supercomputer, but we do not have the high number of CPUs available. 

\begin{center}
\begin{tabular}{ l c c c c c }
	\hline
  	System & $n_o$ & AOs & Unoptimized iter [s] & Optimized iter [s] & Fraction \\ \hline
  	Ne & 10 & 29 & 3.5 & 0.1 & 35 \\
  	$H_2O$ & 10 & 49 & 40 & 0.78 & 51 \\
  	$C_2H_4$ & 16 & 62 & 670 & 5.4 & 124 \\
  	$C_2H_4$ & 16 & 98 &  & 48 &  \\
  	\hline
  	\\
	\end{tabular}
\end{center}

\section{CCSD Parallel Performance}
Our CCSD serial implementation is among the fastest. We want to test its parallel performance in detail. First we run a small system of $H_2O$ with the 6-311++G(3df,3pd) basis set for a different number of processors. We use up to 64 processors on this system. The raw data is included.
   
\begin{center}
\begin{tabular}{ l c}
	\hline
  	P & CCSD iteration time [s] \\ \hline
  	1 & 4.44  \\ 
  	2 & 2.35   \\ 
  	4 & 1.30   \\
  	8 & 0.88    \\ 
  	16 & 0.48   \\ 
  	32 & 0.27   \\ 
  	64 & 0.18   \\ \hline
  	\\
	\end{tabular}
\end{center}

\begin{figure}[h!]
\begin{center}
\fbox{\includegraphics[width=0.7\textwidth]{h2o_ccsd_time_per_iter.eps}}
\caption{Time per iteration of a CCSD run on $H_2O$ with the 6-311++G(3df,3pd) basis set using $2^p$ processors.}
\label{fig:ccsd_h2o_time_per_iter}
\end{center}
\end{figure}

\newpage

The iteration time for 64 processors oscillated between 0.17 and 0.18. 0.18 was more frequent. \\

\begin{figure}[h!]
\begin{center}
\fbox{\includegraphics[width=\textwidth]{h2o_speedup_ccsd.eps}}
\caption{The speedup, S, as a function of number of processors for a CCSD iteration on $H_2O$ with the 6-311++G(3df,3pd) basis set.}
\label{fig:ccsd_h2o_time_speedup}
\end{center}
\end{figure}

We also test a system of two water molecules. We use the same basis set, to achieve a system of exactly twice the size. We are only interested in the performance of our program, so we can place the two molecules in any location. The raw performance data is included in the table.

\begin{center}
\begin{tabular}{ l c}
	\hline
  	P & CCSD iteration time [s] \\ \hline
  	1 & 300   \\ 
  	2 & 225   \\ 
  	4 & 120   \\
  	8 &  59.0  \\ 
  	16 & 30.2   \\ 
  	32 & 15.6   \\ 
  	64 & 8.3  \\
  	128 & 4.8 \\ \hline
  	\\
	\end{tabular}
\end{center}


\begin{figure}[h!]
\begin{center}
\fbox{\includegraphics[width=\textwidth]{ccsd_iter_2_time.eps}}
\caption{Time per iteration of a CCSD run on $(H_2O)_2$ with the 6-311++G(3df,3pd) basis set using $2^p$ processors.}
\label{fig:ccsd_2h2o_time_per_iter}
\end{center}
\end{figure}

In figure \ref{fig:ccsd_2h2o_time_per_iter} we see some weird behaviour when going from one to two CPUs. This is due to the sub-optimal work distribution noted in section \ref{problem_part_ccsd_parallel}. However from four to eight CPUs, and higher, we do not have this problem. This is because the sub-optimal distribution does not get worse with an increased number of CPUs, it stays at the same sub-optimal level. We still have a good performance with time per iteration approaching a value close to zero, but with an optimal work distribution in part 1 of the implementation we would approach zero even faster. How sub-optimal the distribution is depends on the system. However distribution of part 2 and 3 is always optimal. 

\newpage


\begin{figure}[h!]
\begin{center}
\fbox{\includegraphics[width=\textwidth]{ccsd_scaling_2.eps}}
\caption{The speedup, S, as a function of number of processors for a CCSD iteration on $(H_2O)_2$ with the 6-311++G(3df,3pd) basis set.}
\label{fig:ccsd_2h2o_time_speedup}
\end{center}
\end{figure}

\newpage

\section{Potential Energy Plots}
Here we present two potential energy plots for the HF and BH molecules. We are in agreement with the benchmarked values in  Ref.\cite{potential_energy_citation_plots}. Plots are presented in figures \ref{fig:bhpoten} and \ref{fig:hfpoten}. \\

\begin{figure}[h!]
\begin{center}
\fbox{\includegraphics[width=\textwidth]{BH_CCSD_HF_plot.eps}}
\caption{Energypotential for BH Molecule, RHF and RHF-CCSD}
\label{fig:bhpoten}
\end{center}
\end{figure}

CCSD generally improve our results. However there are some features with our CCSD calculations we would like to investigate further. We here presents plots of the number of iterations as a function of R for the HF molecule. We also plot the correction to the HF energy, $E_{CCSD}$, as a function of R. \\ 


\begin{figure}[h!]
\begin{center}
\fbox{\includegraphics[width=\textwidth]{hf_pot_en_plot_kek.eps}}
\caption{Potential energy of HF Molecule. RHF and RHF-CCSD}
\label{fig:hfpoten}
\end{center}
\end{figure}

\newpage

\begin{figure}[h!]
\begin{center}
\fbox{\includegraphics[width=0.9\textwidth]{hf_iterationplot.eps}}
\caption{HF Molecule}
\label{fig:hfiter}
\end{center}
\end{figure}

\begin{figure}[h!]
\begin{center}
\fbox{\includegraphics[width=0.9\textwidth]{hf_correctionplot.eps}}
\caption{HF Molecule}
\label{fig:hfcorr}
\end{center}
\end{figure}

We notice the number of iterations required for self consistency increase when moving away from the equilibrium geometry. If we move to far away the CCSD correction begin either diverging or oscillating between solutions. We have plotted the number of iterations required for self consistency in figure \ref{fig:hfiter}. The CCSD correction energy is plotted in figure \ref{fig:hfcorr}. From figure \ref{fig:hfiter} we see that CCSD is an equilibrium geometry method. It does not always work outside of equilibrium. If the molecule is to far away from equilibrium we can expect a diverging or oscillating CCSD energy. 

\chapter{Results}
In this chapter we present our new results. We will experiment with our code, test its performance against existing software and look into features we found interesting during development. We will provide a contribution to the advancement of computational chemistry and specifically the Computational Physics Group at UiO. 

\section{Single Atoms For Future Reference}
In this section we do calculations on single atoms. Our hope is that these results may be used as a benchmark for future calculations, using different methods. Since Pople basis sets are not available for all atoms, we can only perform reliable calculations where they are available. \\

For even number electrons we will use RHF as a referance for CCSD, or CCSDT. For odd number electrons we will use UHF. In UHF we will assume we have one extra electron with spin up, relative to the number of spin down. \\

We will where available provide a reference energy from Jorgen Hoberget, comparing our results to his DMC calculations, Ref.\cite{dmc_jorgens_resultater_master}. 

\begin{center}
\begin{tabular}{ c c c c c c }
	\hline
  	Z & Atom & Basis Set & Method & Energy & DMC \\ \hline
  	1 & H & 6-311++G(3df,3pd) & UHF & -0.499818
  	&  \\
  	2 & He & 6-311G** & RHF-CCSD & -2.89057 
  	& -2.9036(2) \\ \hline
  	3 & Li & 6-311++G(2d,2p) & UHF & -7.4321 
  	& \\
  	4 & Be & 6-311++G(2d,2p) & RHF-CCSDT & -14.6341 & -14.657(2) \\
  	5 & B & 6-311++G(2d,2p) & UHF & -24.5313& \\
  	6 & C & 6-311++G(2d,2p) & RHF-CCSDT & -37.7383 &\\
  	7 & N & 6-311++G(2d,2p) & UHF & -54.3402 &\\
  	8 & O & 6-311++G(2d,2p) & RHF-CCSD & -74.884& \\
  	9 & F & 6-311++G(2d,2p) & UHF & -99.4014& \\
  	10 & Ne & 6-311++G(2d,2p) & RHF-CCSDT & -128.798 & 128.765(4) \\  \hline
  	11 & Na & 6-311++G(2d,2p) & UHF & -161.847& \\
  	12 & Mg & 6-311++G(2d,2p) & RHF-CCSDT & -199.774 & -199.904(8)\\
  	13 & Al & 6-311++G(2d,2p) & UHF & -241.874& \\
  	14 & Si & 6-311++G(2d,2p) & RHF-CCSD & -289.014 &\\
  	15 & P & 6-311++G(2d,2p) & UHF & -340.68& \\ 
  	16 & S & 6-311++G(2d,2p) & RHF-CCSD & -397.692 &\\
  	17 & Cl & 6-311++G(2d,2p) & UHF & -459.476& \\
  	18 & Ar & 6-311++G(2d,2p) & RHF-CCSD & -527.056& -527.30(4) \\
  	\hline
  	19 & K & 6-311++G(2d,2p) & UHF & -559.15& \\
  	20 & Ca & 6-311++G(2d,2p) & RHF-CCSD & -677.096& \\ \hline
  	32 & Ge & 6-311G** & RHF-CCSD & -2075.66& \\
  	33 & As & 6-311G** & UHF & -2234.12& \\
  	34 & Se & 6-311G** & RHF-CCSD & -2400.17& \\
  	35 & Br & 6-311G** & UHF & -2572.36 &\\
  	36 & Kr & 6-311G** & RHF-CCSD & -2752.44& -2749.9(2) \\
  	\hline
  	\\
	\end{tabular}
\end{center}


\section{Methods}
We will test all our methods. In coupled cluster there are two approximations, a truncated basis set and the limitation of how many excitations are included. In the benchmark chapter we looked into convergence with respect to the basis set. Now we also want to check convergence when we include higher order excitations. \\

We start with the LiH molecule. We will use the largest Pople type basis set, 6-311++G(3dp,3df). We will use Ref.\cite{very_accurate_lih_poten} as a reference. The equilibrium distance given in this paper is R = 3.015. \\

The LiH molecule contains four electrons. Here two can be considered core electrons and two valence electrons. 
 
\begin{center}
\begin{tabular}{ l c c}
	\hline
  	Method & Correction & Energy \\ \hline
  	HF & - &  -7.986 376 7\\
  	CCSD     &  -0.053 444 1  & -8.039 820 8 \\ 
    CCSDT-1a &  -0.053 536 7  & -8.039 913 4 \\ 
  	CCSDT-1b &  -0.053 536 9  & -8.039 913 6 \\
  	CCSDT-2  &  -0.053 537 0  & -8.039 913 7        \\ 
  	CCSDT-3  &  -0.053 537 4  & -8.039 914 1        \\ 
  	CCSDT-4  &  -0.053 557 3  & -8.039 934 0    \\ 
  	CCSDT    &  -0.053 555 5  &  -8.039 932 2    \\ 
  	\cite{very_accurate_lih_poten}, est $E_0$ & - & -8.070 548 0 \\ \hline
  	\\
	\end{tabular}
\end{center}

We also perform calculations on the BH molecule using all our methods. We are interested in a small area around the equilibrium distance found previously in figure \ref{fig:bhpoten}. We will compare RHF, CCSD, CCSDT-1a and CCSDT. We also include MP2 calculations performed with LSDALTON for comparison. Results are available in figures \ref{fig:zom1} and \ref{fig:zom2}. 

\begin{figure}[h!]
\begin{center}
\fbox{\includegraphics[width=0.9\textwidth]{zoomed_BH_figure.eps}}
\caption{BH Potential Energy Minimum plot. Methods in use RHF, MP2, CCSD and CCSDT. Distances in Angstrom.}
\label{fig:zom1}
\end{center}
\end{figure}

\begin{figure}[h!]
\begin{center}
\fbox{\includegraphics[width=0.9\textwidth]{zoomed_BH_figure2.eps}}
\caption{BH Potential Energy Minimum plot. Methods in use CCSD, CCSDT-1a and CCSDT. Distances in Angstrom.}
\label{fig:zom2}
\end{center}
\end{figure}






We have performed calculations with a resolution of 0.05 Angstrom. Around the minimum the resolution is increased to 0.01 Angstrom.  For this system we see different methods provide different results. \\

The improvement from HF to MP2 is larger than from MP2 to CCSD. Also we see the improvement from MP2 to CCSD is larger than the improvement from CCSD to CCSDT. We see a convergence of the energy with the respect to the contributions included. This is illustrated in figure \ref{fig:zom3}. \\

We also notice that for HF and MP2 the energy minimum is R = 1.22 Angstrom. For CCSD and CCSDT the minimum is shifted to 1.23 Angstrom. 

\newpage

\begin{figure}[h!]
\begin{center}
\fbox{\includegraphics[width=0.9\textwidth]{zoom_BH_figure3.eps}}
\caption{BH Energy minimum with respect to Method plot. Methods in use HF, MP2, CCSD, CCSDT-1a and CCSDT. MP2 results from LSDALTON. }
\label{fig:zom3}
\end{center}
\end{figure}

\section{CCSD Performance}
In this section we will test our CCSD performance against ACESIII. There are a few challenges when making this test, especially that we do not have implemented frozen core in our optimized AOtoMO transformation. We cannot perform the AOtoMO calculation with the unoptimized algorithm for the systems available on the ACESIII benchmark site. They are simply to large. We need the optimized AOtoMO algorithm, and as such we cannot perform frozen core calculations. The benchmarked values on ACESIII benchmark site are with frozen core. \\

However since CCSD calculations depend on $n_o$ and $n_v$, we will create a system of the same size and compare performance. We will benchmark against a $C_{20}$ calculation with ACESIII from 2009, \cite{aces_non_ref}. We are interested in program performance. Their calculations are performed with the following system specifics: \\

\begin{center}
\begin{tabular}{ l c}
	\hline
  	System 1 & $C_{20}$\\
  	Electrons & 120 \\
  	Basis Set & cc-pVDZ\\
  	Contracted GTOs & 280 \\
  	Frozen Orbitals & 20 core\\
  	$n_o$ for CCSD & 80\\
  	$n_v$ for CCSD & 440\\ \hline
  	\\
	\end{tabular}
\end{center}

The hardware in use here is the Kraken supercomputer, \cite{kraken_citation}. In 2009 Kraken was the most powerful supercomputer in the world managed by an academia. Each node on Kraken has the following hardware specifications

\begin{center}
\begin{tabular}{ l }
	\hline
  	Two 2.6 GHz six-core AMD Opteron processors (Istanbul)\\
    12 cores\\
    16 GB of memory\\
    Connection via Cray SeaStar2+ router\\ \hline
  	\\
	\end{tabular}
\end{center}

We will use a slightly different system, but we have ensured $n_v$ and $n_o$ is approximately the same as for the prior system.

\begin{center}
\begin{tabular}{ l c}
	\hline
  	System 2 & $C_{13} H_2$\\
  	Electrons & 80 \\
  	Basis Set & 6-311G** \\
  	Contracted GTOs & 259 \\
  	Frozen Orbitals & 0 \\
  	$n_o$ for CCSD & 80 \\
  	$n_v$ for CCSD & 438 \\ \hline
  	\\
	\end{tabular}
\end{center}

We perform our calculations on the abel computing cluster, \cite{abel_po_g_citation1234567}. Both ACESIII and our calculations are performed with 200 CPUs. We note our system has two virtual orbitals less. \\

\begin{center}
\begin{tabular}{ l c}
	\hline
  	Code & Time per Iteration [min] \\
  	ACESIII & 1.6 \\
  	Our Results & 4.17 (250 s) \\ \hline
	\end{tabular}
\end{center}

We note the AOtoMO transformation took 140 seconds. This is less than one iteration in our program, and close to one iteration in ACESIII. \\

Compiler flags can optimized our code further. Compiler flags allows the compiler to perform further optimizations, and can sometimes affect the resulting energy. We want the performance to be a realistic measure of how fast our program can produce real results. For this reason we did not use any flags in the compiler until now, but we will try this calculation once more using the flags recommended by abel. 

\begin{lstlisting}
CFLAGS = -pipe -O2 -xAVX -mavx -fomit-frame-pointer -fno-alias -Wall -W
\end{lstlisting}
These compiler settings produce a runtime per iteration of $T_P = $ 230 s (3.8 m).


\bibliographystyle{spphys}
%\bibliographystyle{spmpsci}
 \bibliography{../../Library_of_Topics/mylib}
\end{document}

%%%%%%%%%%%%%%%%%%%%%% appendix.tex %%%%%%%%%%%%%%%%%%%%%%%%%%%%%%%%%
%
% sample appendix
%
% Use this file as a template for your own input.
%
%%%%%%%%%%%%%%%%%%%%%%%% Springer-Verlag %%%%%%%%%%%%%%%%%%%%%%%%%%

\appendix
\motto{All's well that ends well}
\chapter{Chapter Heading}
\label{introA} % Always give a unique label
% use \chaptermark{}
% to alter or adjust the chapter heading in the running head

Use the template \emph{appendix.tex} together with the Springer document class SVMono (monograph-type books) or SVMult (edited books) to style appendix of your book in the Springer layout.


\section{Section Heading}
\label{sec:A1}
% Always give a unique label
% and use \ref{<label>} for cross-references
% and \cite{<label>} for bibliographic references
% use \sectionmark{}
% to alter or adjust the section heading in the running head
Instead of simply listing headings of different levels we recommend to let every heading be followed by at least a short passage of text. Furtheron please use the \LaTeX\ automatism for all your cross-references and citations.


\subsection{Subsection Heading}
\label{sec:A2}
Instead of simply listing headings of different levels we recommend to let every heading be followed by at least a short passage of text. Furtheron please use the \LaTeX\ automatism for all your cross-references and citations as has already been described in Sect.~\ref{sec:A1}.

For multiline equations we recommend to use the \verb|eqnarray| environment.
\begin{eqnarray}
\vec{a}\times\vec{b}=\vec{c} \nonumber\\
\vec{a}\times\vec{b}=\vec{c}
\label{eq:A01}
\end{eqnarray}

\subsubsection{Subsubsection Heading}
Instead of simply listing headings of different levels we recommend to let every heading be followed by at least a short passage of text. Furtheron please use the \LaTeX\ automatism for all your cross-references and citations as has already been described in Sect.~\ref{sec:A2}.

Please note that the first line of text that follows a heading is not indented, whereas the first lines of all subsequent paragraphs are.

% For figures use
%
\begin{figure}[t]
\sidecaption[t]
%\centering
% Use the relevant command for your figure-insertion program
% to insert the figure file.
% For example, with the option graphics use
\includegraphics[scale=.65]{figure}
%
% If not, use
%\picplace{5cm}{2cm} % Give the correct figure height and width in cm
%
\caption{Please write your figure caption here}
\label{fig:A1}       % Give a unique label
\end{figure}

% For tables use
%
\begin{table}
\caption{Please write your table caption here}
\label{tab:A1}       % Give a unique label
%
% For LaTeX tables use
%
\begin{tabular}{p{2cm}p{2.4cm}p{2cm}p{4.9cm}}
\hline\noalign{\smallskip}
Classes & Subclass & Length & Action Mechanism  \\
\noalign{\smallskip}\hline\noalign{\smallskip}
Translation & mRNA$^a$  & 22 (19--25) & Translation repression, mRNA cleavage\\
Translation & mRNA cleavage & 21 & mRNA cleavage\\
Translation & mRNA  & 21--22 & mRNA cleavage\\
Translation & mRNA  & 24--26 & Histone and DNA Modification\\
\noalign{\smallskip}\hline\noalign{\smallskip}
\end{tabular}
$^a$ Table foot note (with superscript)
\end{table}
%


\backmatter%%%%%%%%%%%%%%%%%%%%%%%%%%%%%%%%%%%%%%%%%%%%%%%%%%%%%%%
%%%%%%%%%%%%%%%%%%%%%%%acronym.tex%%%%%%%%%%%%%%%%%%%%%%%%%%%%%%%%%%%%%%%%%
% sample list of acronyms
%
% Use this file as a template for your own input.
%
%%%%%%%%%%%%%%%%%%%%%%%% Springer %%%%%%%%%%%%%%%%%%%%%%%%%%

\Extrachap{Glossary}


Use the template \emph{glossary.tex} together with the Springer document class SVMono (monograph-type books) or SVMult (edited books) to style your glossary\index{glossary} in the Springer layout.


\runinhead{glossary term} Write here the description of the glossary term. Write here the description of the glossary term. Write here the description of the glossary term.

\runinhead{glossary term} Write here the description of the glossary term. Write here the description of the glossary term. Write here the description of the glossary term.

\runinhead{glossary term} Write here the description of the glossary term. Write here the description of the glossary term. Write here the description of the glossary term.

\runinhead{glossary term} Write here the description of the glossary term. Write here the description of the glossary term. Write here the description of the glossary term.

\runinhead{glossary term} Write here the description of the glossary term. Write here the description of the glossary term. Write here the description of the glossary term.
%
\Extrachap{Solutions}

\section*{Problems of Chapter~\ref{intro}}

\begin{sol}{prob1}
The solution\index{problems}\index{solutions} is revealed here.
\end{sol}


\begin{sol}{prob2}
\textbf{Problem Heading}\\
(a) The solution of first part is revealed here.\\
(b) The solution of second part is revealed here.
\end{sol}



\printindex

%%%%%%%%%%%%%%%%%%%%%%%%%%%%%%%%%%%%%%%%%%%%%%%%%%%%%%%%%%%%%%%%%%%%%%

\end{document}


