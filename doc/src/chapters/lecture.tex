%%%%%%%%%%%%%%%%%%%% book.tex %%%%%%%%%%%%%%%%%%%%%%%%%%%%%
%
% sample root file for the chapters of your "monograph"
%
% Use this file as a template for your own input.
%
%%%%%%%%%%%%%%%% Springer-Verlag %%%%%%%%%%%%%%%%%%%%%%%%%%


% RECOMMENDED %%%%%%%%%%%%%%%%%%%%%%%%%%%%%%%%%%%%%%%%%%%%%%%%%%%
\documentclass[graybox,envcountchap,sectrefs]{svmono}

% choose options for [] as required from the list
% in the Reference Guide

\usepackage{mathptmx}
\usepackage{helvet}
\usepackage{courier}
%
\usepackage{type1cm}         

\usepackage{makeidx}         % allows index generation
\usepackage{graphicx}        % standard LaTeX graphics tool
                             % when including figure files
\usepackage{multicol}        % used for the two-column index
\usepackage[bottom]{footmisc}% places footnotes at page bottom

% see the list of further useful packages
% in the Reference Guide
 \usepackage{chapterbib}
 \usepackage{verbatim}

\usepackage[usenames,dvipsnames,x11names]{xcolor}
\usepackage{tikz}
\usetikzlibrary{arrows,snakes,shapes}

 \usepackage{listings}
 \usepackage{epic}
 \usepackage{eepic}
 \usepackage{a4wide}
 \usepackage{color}
 \usepackage{amsmath}
 \usepackage{amssymb}
 \usepackage[dvips]{epsfig}
 \usepackage[T1]{fontenc}
 \usepackage{cite} % [2,3,4] --> [2--4]
 \usepackage{shadow}
 \usepackage{hyperref}
 \usepackage{bezier}
 \usepackage{pstricks}
 %\usepackage{refcheck}
 \setcounter{tocdepth}{2}
%\usepackage{gnuplot-lua-tikz}

\makeindex             % used for the subject index
                       % please use the style svind.ist with
                       % your makeindex program

%%%%%%%%%%%%%%%%%%%%%%%%%%%%%%%%%%%%%%%%%%%%%%%%%%%%%%%%%%%%%%%%%%%%%

\begin{document}

\author{Author name(s)}
\title{Book title}
\subtitle{-- Monograph --}
\maketitle

\frontmatter%%%%%%%%%%%%%%%%%%%%%%%%%%%%%%%%%%%%%%%%%%%%%%%%%%%%%%


%%%%%%%%%%%%%%%%%%%%%%% dedic.tex %%%%%%%%%%%%%%%%%%%%%%%%%%%%%%%%%
%
% sample dedication
%
% Use this file as a template for your own input.
%
%%%%%%%%%%%%%%%%%%%%%%%% Springer %%%%%%%%%%%%%%%%%%%%%%%%%%

\begin{dedication}
Use the template \emph{dedic.tex} together with the Springer document class SVMono for monograph-type books or SVMult for contributed volumes to style a quotation or a dedication\index{dedication} at the very beginning of your book in the Springer layout
\end{dedication}





%%%%%%%%%%%%%%%%%%%%%%foreword.tex%%%%%%%%%%%%%%%%%%%%%%%%%%%%%%%%%
% sample foreword
%
% Use this file as a template for your own input.
%
%%%%%%%%%%%%%%%%%%%%%%%% Springer %%%%%%%%%%%%%%%%%%%%%%%%%%

\foreword

%% Please have the foreword written here
Use the template \textit{foreword.tex} together with the Springer document class SVMono (monograph-type books) or SVMult (edited books) to style your foreword\index{foreword} in the Springer layout. 

The foreword covers introductory remarks preceding the text of a book that are written by a \textit{person other than the author or editor} of the book. If applicable, the foreword precedes the preface which is written by the author or editor of the book.


\vspace{\baselineskip}
\begin{flushright}\noindent
Place, month year\hfill {\it Firstname  Surname}\\
\end{flushright}



\preface
%  last update : 24/8/2013  mhj

\begin{quotation}
So, ultimately, in order to understand nature it may be necessary to
have a deeper understanding of mathematical relationships. But the
real reason is that the subject is enjoyable, and although we humans
cut nature up in different ways, and we have different courses in
different departments, such compartmentalization is really artificial,
and we should take our intellectual pleasures where we find them. 
{\em Richard Feynman, The Laws of Thermodynamics.}
\end{quotation}

Why a preface you may ask? Isn't that just a mere exposition of a
raison d'$\mathrm{\hat{e}}$tre of an author's choice of material,
preferences, biases, teaching philosophy etc.?  To a large extent I
can answer in the affirmative to that. A preface ought to be personal.
Indeed, what you will see in the various chapters of these notes
represents how I perceive computational physics should be taught.

 This set of lecture notes serves the scope of presenting to you and
train you in an algorithmic approach to problems in the sciences,
represented here by the unity of three disciplines, physics,
mathematics and informatics. This trinity outlines the emerging field
of computational physics.

Our insight in a physical system, combined with numerical mathematics
gives us the rules for setting up an algorithm, viz.~a set of rules
for solving a particular problem.  Our understanding of the physical
system under study is obviously gauged by the natural laws at play,
the initial conditions, boundary conditions and other external
constraints which influence the given system. Having spelled out the
physics, for example in the form of a set of coupled partial
differential equations, we need efficient numerical methods in order
to set up the final algorithm.  This algorithm is in turn coded into a
computer program and executed on available computing facilities.  To
develop such an algorithmic approach, you will be exposed to several
physics cases, spanning from the classical pendulum to quantum
mechanical systems. We will also present some of the most popular
algorithms from numerical mathematics used to solve a plethora of
problems in the sciences.  Finally we will codify these algorithms
using some of the most widely used programming languages, presently C,
C++ and Fortran and its most recent standard Fortran
2008\footnote{Throughout this text we refer to Fortran 2008 as
Fortran, implying the latest standard.}. However, a high-level and fully
object-oriented language like Python is now emerging as a good
alternative although C++ and Fortran still outperform Python when it
comes to computational speed.  In this text we offer an approach where
one can write all programs in C/C++ or Fortran.  We will also show you
how to develop large programs in Python interfacing C++ and/or Fortran
functions for those parts of the program which are CPU intensive.
Such an approach allows you to structure the flow of data in a
high-level language like Python while tasks of a mere repetitive and
CPU intensive nature are left to low-level languages like C++ or
Fortran. Python allows you also to smoothly interface your program
with other software, such as plotting programs or operating system
instructions. A typical Python program you may end up writing contains
everything from compiling and running your codes to preparing the body
of a file for writing up your report.



Computer simulations are nowadays an integral part of contemporary
basic and applied research in the sciences.  Computation is becoming
as important as theory and experiment. In physics, computational
physics, theoretical physics and experimental physics are all equally
important in our daily research and studies of physical
systems. Physics is the unity of theory, experiment and
computation\footnote{We mentioned previously the trinity of physics,
mathematics and informatics. Viewing physics as the trinity of theory,
experiment and simulations is yet another example. It is obviously
tempting to go beyond the sciences. History shows that triunes,
trinities and for example triple deities permeate the Indo-European
cultures (and probably all human cultures), from the ancient Celts and
Hindus to modern days.  The ancient Celts revered many such trinues,
their world was divided into earth, sea and air, nature was divided in
animal, vegetable and mineral and the cardinal colours were red,
yellow and blue, just to mention a few.  As a curious digression, it
was a Gaulish Celt, Hilary, philosopher and bishop of Poitiers (AD
315-367) in his work {\em De Trinitate} who formulated the Holy
Trinity concept of Christianity, perhaps in order to accomodate
millenia of human divination practice.}.  Moreover, the ability "to
compute" forms part of the essential repertoire of research
scientists. Several new fields within computational science have
emerged and strengthened their positions in the last years, such as
computational materials science, bioinformatics, computational
mathematics and mechanics, computational chemistry and physics and so
forth, just to mention a few.  These fields underscore the importance
of simulations as a means to gain novel insights into physical
systems, especially for those cases where no analytical solutions can
be found or an experiment is too complicated or expensive to carry
out.  To be able to simulate large quantal systems with many degrees
of freedom such as strongly interacting electrons in a quantum dot
will be of great importance for future directions in novel fields like
nano-techonology.  This ability often combines knowledge from many
different subjects, in our case essentially from the physical
sciences, numerical mathematics, computing languages, topics from
high-performace computing and some knowledge of computers.


In 1999, when I started this course at the department of physics in
Oslo, computational physics and computational science in general were
still perceived by the majority of physicists and scientists as topics
dealing with just mere tools and number crunching, and not as subjects
of their own.  The computational background of most students enlisting
for the course on computational physics could span from dedicated
hackers and computer freaks to people who basically had never used a
PC. The majority of undergraduate and graduate students had a very
rudimentary knowledge of computational techniques and methods.
Questions like 'do you know of better methods for numerical
integration than the trapezoidal rule' were not uncommon. I do happen
to know of colleagues who applied for time at a supercomputing centre
because they needed to invert matrices of the size of $10^4\times
10^4$ since they were using the trapezoidal rule to compute
integrals. With Gaussian quadrature this dimensionality was easily
reduced to matrix problems of the size of $10^2\times 10^2$, with much
better precision.

More than a decade later most students have now been exposed to a
fairly uniform introduction to computers, basic programming skills and
use of numerical exercises.  Practically every undergraduate student
in physics has now made a Matlab or Maple simulation of for example
the pendulum, with or without chaotic motion.  Nowadays most of you
are familiar, through various undergraduate courses in physics and
mathematics, with interpreted languages such as Maple, Matlab and/or
Mathematica. In addition, the interest in scripting languages such as
Python or Perl has increased considerably in recent years.  The modern
programmer would typically combine several tools, computing
environments and programming languages. A typical example is the
following. Suppose you are working on a project which demands
extensive visualizations of the results. To obtain these results, that
is to solve a physics problems like obtaining the density profile of a
Bose-Einstein condensate, you need however a program which is fairly
fast when computational speed matters.  In this case you would most
likely write a high-performance computing program using Monte Carlo
methods in languages which are tailored for that. These are
represented by programming languages like Fortran and C++.  However,
to visualize the results you would find interpreted languages like
Matlab or scripting languages like Python extremely suitable for your
tasks.  You will therefore end up writing for example a script in
Matlab which calls a Fortran or C++ program where the number crunching
is done and then visualize the results of say a wave equation solver
via Matlab's large library of visualization tools. Alternatively, you
could organize everything into a Python or Perl script which does
everything for you, calls the Fortran and/or C++ programs and performs
the visualization in Matlab or Python. Used correctly, these tools,
spanning from scripting languages to high-performance computing
languages and vizualization programs, speed up your capability to
solve complicated problems.  Being multilingual is thus an advantage
which not only applies to our globalized modern society but to
computing environments as well.  This text shows you how to use C++
and Fortran as programming languages.

There is however more to the picture than meets the eye.  Although
interpreted languages like Matlab, Mathematica and Maple allow you
nowadays to solve very complicated problems, and high-level languages
like Python can be used to solve computational problems, computational
speed and the capability to write an efficient code are topics which
still do matter. To this end, the majority of scientists still use
languages like C++ and Fortran to solve scientific problems.  When you
embark on a master or PhD thesis, you will most likely meet these
high-performance computing languages.  This course emphasizes thus the
use of programming languages like Fortran, Python and C++ instead of
interpreted ones like Matlab or Maple. You should however note that
there are still large differences in computer time between for example
numerical Python and a corresponding C++ program for many numerical
applications in the physical sciences, with a code in C++ or Fortran
being the fastest.

Computational speed is not the only reason for this choice of
programming languages. Another important reason is that we feel that
at a certain stage one needs to have some insights into the algorithm
used, its stability conditions, possible pitfalls like loss of
precision, ranges of applicability, the possibility to improve the
algorithm and taylor it to special purposes etc etc.  One of our major
aims here is to present to you what we would dub 'the algorithmic
approach', a set of rules for doing mathematics or a precise
description of how to solve a problem. To device an algorithm and
thereafter write a code for solving physics problems is a marvelous
way of gaining insight into complicated physical systems. The
algorithm you end up writing reflects in essentially all cases your
own understanding of the physics and the mathematics (the way you
express yourself) of the problem.  We do therefore devote quite some
space to the algorithms behind various functions presented in the
text. Especially, insight into how errors propagate and how to avoid
them is a topic we would like you to pay special attention to. Only
then can you avoid problems like underflow, overflow and loss of
precision. Such a control is not always achievable with interpreted
languages and canned functions where the underlying algorithm and/or
code is not easily accesible.  Although we will at various stages
recommend the use of library routines for say linear
algebra\footnote{Such library functions are often taylored to a given
machine's architecture and should accordingly run faster than user
provided ones.}, our belief is that one should understand what the
given function does, at least to have a mere idea.  With such a
starting point, we strongly believe that it can be easier to develope
more complicated programs on your own using Fortran, C++ or Python.

We have several other aims as well, namely:
\begin{itemize}
\item We would like to give you  an opportunity to gain a 
      deeper understanding of the physics you have learned in other
      courses. In most courses one is normally confronted with simple
      systems which provide exact solutions and mimic to a certain
      extent the realistic cases. Many are however the comments like
      'why can't we do something else than the particle in a box
      potential?'.  In several of the projects we hope to present some
      more 'realistic' cases to solve by various numerical
      methods. This also means that we wish to give examples of how
      physics can be applied in a much broader context than it is
      discussed in the traditional physics undergraduate curriculum.
\item To encourage you to "discover" physics in a way similar to how 
researchers learn in the context of research.
\item Hopefully also to introduce numerical methods and new areas of physics that 
      can be studied with the methods discussed.
\item To teach   structured programming in the context of doing science. 
\item The projects we propose are meant to mimic to a certain extent 
      the situation encountered during a thesis or project work. You
      will tipically have at your disposal 2-3 weeks to solve
      numerically a given project. In so doing you may need to do a
      literature study as well. Finally, we would like you to write a
      report for every project.
\end{itemize}
Our overall goal is to encourage you to learn about science through
experience and by asking questions. Our objective is always
understanding and the purpose of computing is further insight, not
mere numbers!  Simulations can often be considered as
experiments. Rerunning a simulation need not be as costly as rerunning
an experiment.


 
Needless to say, these lecture notes are upgraded continuously, from
typos to new input.  And we do always benefit from your comments,
suggestions and ideas for making these notes better.  It's through the
scientific discourse and critics we advance.  Moreover, I have
benefitted immensely from many discussions with fellow colleagues and
students. In particular I must mention Hans Petter Langtangen, Anders
Malthe-S\o renssen, Knut M\o rken and \O yvind Ryan, whose input
during the last fifteen years has considerably improved these lecture
notes.  Furthermore, the time we have spent and keep spending together
on the Computing in Science Education project at the University, is
just marvelous. Thanks so much. Concerning the Computing in Science
Education initiative, you can read more
at \url{http://www.mn.uio.no/english/about/collaboration/cse/}.


Finally, I would like to add a petit note on referencing. These notes
have evolved over many years and the idea is that they should end up
in the format of a web-based learning environment for doing
computational science. It will be fully free and hopefully represent a
much more efficient way of conveying teaching material than
traditional textbooks.  I have not yet settled on a specific format,
so any input is welcome. At present however, it is very easy for me to
upgrade and improve the material on say a yearly basis, from simple
typos to adding new material.  When accessing the web page of the
course, you will have noticed that you can obtain all source files for
the programs discussed in the text.  Many people have thus written to
me about how they should properly reference this material and whether
they can freely use it. My answer is rather simple.  You are
encouraged to use these codes, modify them, include them in
publications, thesis work, your lectures etc.  As long as your use is
part of the dialectics of science you can use this material freely.
However, since many weekends have elapsed in writing several of these
programs, testing them, sweating over bugs, swearing in front of a
f*@?\%g code which didn't compile properly ten minutes before monday
morning's eight o'clock lecture etc etc, I would dearly appreciate in
case you find these codes of any use, to reference them properly. That
can be done in a simple way, refer to M.~Hjorth-Jensen, {\em
Computational Physics}, University of Oslo (2013). The weblink to the
course should also be included. Hope it is not too much to ask
for. Enjoy!

%%%%%%%%%%%%%%%%%%%%%%acknow.tex%%%%%%%%%%%%%%%%%%%%%%%%%%%%%%%%%%%%%%%%%
% sample acknowledgement chapter
%
% Use this file as a template for your own input.
%
%%%%%%%%%%%%%%%%%%%%%%%% Springer %%%%%%%%%%%%%%%%%%%%%%%%%%

\extrachap{Acknowledgements}

Use the template \emph{acknow.tex} together with the Springer document class SVMono (monograph-type books) or SVMult (edited books) if you prefer to set your acknowledgement section as a separate chapter instead of including it as last part of your preface.



\tableofcontents

%%%%%%%%%%%%%%%%%%%%%%acronym.tex%%%%%%%%%%%%%%%%%%%%%%%%%%%%%%%%%%%%%%%%%
% sample list of acronyms
%
% Use this file as a template for your own input.
%
%%%%%%%%%%%%%%%%%%%%%%%% Springer %%%%%%%%%%%%%%%%%%%%%%%%%%

\extrachap{Acronyms}

Use the template \emph{acronym.tex} together with the Springer document class SVMono (monograph-type books) or SVMult (edited books) to style your list(s) of abbreviations or symbols in the Springer layout.

Lists of abbreviations\index{acronyms, list of}, symbols\index{symbols, list of} and the like are easily formatted with the help of the Springer-enhanced \verb|description| environment.

\begin{description}[CABR]
\item[ABC]{Spelled-out abbreviation and definition}
\item[BABI]{Spelled-out abbreviation and definition}
\item[CABR]{Spelled-out abbreviation and definition}
\end{description}


\mainmatter%%%%%%%%%%%%%%%%%%%%%%%%%%%%%%%%%%%%%%%%%%%%%%%%%%%%%%%
%%%%%%%%%%%%%%%%%%%%%part.tex%%%%%%%%%%%%%%%%%%%%%%%%%%%%%%%%%%
% 
% sample part title
%
% Use this file as a template for your own input.
%
%%%%%%%%%%%%%%%%%%%%%%%% Springer %%%%%%%%%%%%%%%%%%%%%%%%%%

\begin{partbacktext}
\part{Introduction to programming and numerical methods}
\noindent Use the template \emph{part.tex} together with the Springer document class SVMono (monograph-type books) or SVMult (edited books) to style your part title page and, if desired, a short introductory text (maximum one page) on its verso page in the Springer layout.

\end{partbacktext}


\chapter{Introduction}

\section{Choice of programming language}

As programming language we have ended up with preferring 
C++, but all examples discussed in the text have their 
corresponding Fortran and Python programs on the webpage of this text.
 
Fortran (FORmula TRANslation) was introduced in 1957 and remains in many 
scientific computing environments the language of choice.
The latest standard, see Refs.~\cite{f95ref,metcalf1996,marshall1995,f2003}, 
includes extensions that are
familiar to users of C++. 
Some of the most important features of Fortran  include recursive
subroutines, dynamic storage allocation and pointers, 
user defined data structures, modules,
and the ability to manipulate entire arrays. 
However, there are several good reasons for 
choosing C++ as programming language for scientific and engineering
problems. Here are some:
\begin{itemize}
\item C++ is now the dominating language in Unix and Windows environments. It is widely available and is
the language of choice for system programmers.  It is very widespread for developments of non-numerical  software 
\item The C++ syntax has inspired lots of popular languages, such as Perl, Python and Java.
\item It is an extremely portable language, all Linux and Unix operated machines have a 
C++ compiler.
\item In the last years there has been an enormous effort towards developing numerical libraries
for C++. Numerous tools (numerical libraries such as MPI\cite{gropp1999,mpiref,cmpi}) are written in C++
and interfacing them requires knowledge of C++. 
Most C++ and Fortran compilers compare fairly well when it comes to speed and
numerical efficiency. Although Fortran 77 and C are regarded as slightly faster than C++ or Fortran,
compiler improvements during the last few years have diminshed such differences. The Java numerics project
has lost some of its steam recently, and Java is therefore normally slower than C++ or Fortran.
\item Complex variables, one of Fortran's strongholds, can also be defined in the new 
ANSI C++ standard. 
\item C++ is a language which catches most of the errors as early as possible, typically at compilation
time. Fortran has some of these features if one omits implicit variable declarations.
\item C++ is also an object-oriented language, to be contrasted with C and Fortran.
This means that it supports three fundamental ideas, namely objects, class hierarchies and polymorphism.
Fortran has, through the \verb? MODULE?  declaration the capability of defining classes, but lacks 
inheritance, although polymorphism is possible. Fortran is then considered as an object-based
programming language, to be contrasted with C++ which has the capability of relating classes
to each other in a hierarchical way.
\end{itemize}

An important aspect of C++ is its richness with more than 60 keywords allowing for a good balance between object orientation
and numerical efficiency. Furthermore, careful programming can results in an efficiency close to
Fortran 77.  The language is well-suited for large projects and has presently good standard libraries suitable
for computational science projects, although many of these still lag behind the large body of libraries for numerics
available to Fortran programmers. However, it is not difficult to interface libraries written in Fortran with C++
codes, if care is exercised.
Other weak sides are the fact that it can be easy to write inefficient code  and that there are many ways of writing the
same things, adding to the confusion for beginners  and professionals as well.  The language is also under continuous
development, which often causes portability problems.

C++ is also a difficult language to learn. Grasping the basics is rather straightforward, but takes time
to master. A specific problem which often causes 
unwanted or odd errors is dynamic memory management.

The efficiency of C++ codes are close to those provided by Fortran. This means often that a code
written in Fortran 77 can be faster, however  for large numerical projects C++ and Fortran 
are to be preferred. If speed is an issue, one could port critical parts of the code to Fortran 77.



\chapter{Practical Hartree-Fock approaches}\label{chap:hartreefock}

\abstract{This chapters aims at catching two birds with a stone;  to introduce to you essential features of the programming languages
C++ and Fortran with a brief reminder on Python specific topics, and to stress problems like
overflow, underflow, round off errors and eventually loss of precision due to the finite amount 
of numbers a computer can represent.  
The programs we discuss are tailored to these aims.}

\section{Getting Started}

The Hartree-Fock method, initially developed by Hartree \cite{hartree} and improved by Fock \cite{Fock},
is probably the most popular \emph{ab initio} method of quantum chemistry. There are mainly two reasons for this.
Firstly, it  provides an excellent first approximation to the wave function and energy of the system, often accounting
for about 90\%-99\% of the total energy. Secondly, in cases where an even higher degree of precision is needed, the result from a Hartree-Fock calculation
is a very good starting point for other so-called \emph{post-Hartree-Fock methods}. We will look into one such method, namely perturbation
theory, in chapter \ref{chapter:perturbation_theory}.

The general form of the equations is
\begin{equation}
 \mathcal{F}\psi_k = \varepsilon_k \psi_k,
\end{equation}
where $\mathcal{F}$ is the Fock operator, defined as
\begin{equation}
\begin{split}
 \mathcal{F}(\vec x)\psi_k(\vec x)  = & \Big[-\frac{1}{2}\nabla^2 - \sum_{n=1}^K\frac{Z_n}{|\vec R_n - \vec r|}\Big]\psi_k(\vec x) \\
                                      & +  \sum_{l=1}^N\int d\vec x'|\psi_l(\vec x')|^2\frac{1}{|\vec r - \vec r'|}\psi_k(\vec x) \\
                                      & - \sum_{l=1}^N \int d\vec x'\psi^*_l(\vec x')\frac{1}{|\vec r - \vec r'|}\psi_k(\vec x')\psi_l(\vec x).
\end{split}
\end{equation}
These are a set of coupled one-electron eigenvalue equations for the spin orbitals $\psi_k$.
The equations are non-linear because the orbitals we are seeking are
actually needed in order to obtain the operator $\mathcal{F}$ which determine them. They are therefore often referred to as self consistent field (SCF)
equations, and they must be solved iteratively.

The term in the square brackets is the one-body operator which we have called $h(\vec r)$ in equation (\ref{eq:H1}). The two extra sums are due to the interactions between
the electrons. The first of these represent the Coulomb potential from the mean field set up by the electrons of the system. The last is similar to the first
except that the indices of two orbitals have been switched. This is a direct consequence of the fact that in the derivation of the equations, the state
is assumed to be a Slater determinant. Note that due to the last sum, the Fock operator is non-local, that is to say, the value of $\mathcal{F}(\vec x)\psi_k(\vec x)$
depends on the value of $\psi_k(\vec x')$ at all coordinates $\vec x' \in \mathbb{R}^3 \oplus \{\uparrow, \downarrow\}$.

In the following section the general Hartree-Fock equations presented above are derived. Thereafter, we will see how to reformulate the equations to a more
implementation friendly form. We do this by removing the spin part so that the resulting equations are in terms of spatial orbitals only.
However, before doing this, it is necessary to decide how to relate the spin orbitals to the spatial orbitals. We discuss the two most common ways to
do this. These result in the so-called restricted and unrestricted Slater determinants, which are discussed in section \ref{sec:res_and_unres_dets}.
In section \ref{sec:RHF} we show how the restricted determinant leads to the restricted Hartree-Fock (RHF) equations, which is a set of integro-differential equations for
the spatial orbitals. In order to solve the equations, the spatial orbitals are expanded in a known basis, which leads to a set of self consistent algebraic equations
called the \emph{Roothaan equations}. Thereafter, in section \ref{sec:UHF}, we derive the unrestricted Hartree-Fock (UHF)
equations from the unrestricted determinant. These are also solved by introducing a set of known basis functions, leading to the so-called
\emph{Pople-Nesbet} equations, which is the unrestricted analogue of the Roothaan equations.

The theory of this chapter is covered by Szabo and Ostlund \cite{Szabo} and Thijssen \cite{Thijssen}.


% The Hartree-Fock method is a popular method for calculating the ground state and ground state energy of many-particle systems.
% It is a variational method which aims to find the minimum of the expectation value of the Hamiltonian. However, instead of minimizing in
% the space of all possible wave functions, the search is restricted to the set of all Slater determinants.
% 
% The Hartree-Fock equations are given by
% \begin{equation}
%  \mathcal{F}\psi_k = \varepsilon_k \psi_k,
% \end{equation}
% where $\mathcal{F}$ is the Fock operator, defined as
% \begin{equation}
% \begin{split}
%  \mathcal{F}(\vec x)\psi_k(\vec x)  = & \Big[-\frac{1}{2}\nabla^2 - \sum_{i=1}^N\frac{Z_n}{|\vec R_n - \vec r|}\Big]\psi_k(\vec x) \\
%                                       & +  \sum_{l=1}^N\int d\vec x'|\psi_l(\vec x')|^2\frac{1}{|\vec r - \vec r'|}\psi_k(\vec x) \\
%                                       & - \sum_{l=1}^N \int d\vec x'\psi^*_l(\vec x')\frac{1}{|\vec r - \vec r'|}\psi_k(\vec x')\psi_l(\vec x).
% \end{split}
% \end{equation}
% These are a set of coupled one-electron eigenvalue equations for the spin orbitals $\psi_k$.
% The equations are non-linear because the orbitals we are seeking are
% actually needed in order to obtain the operator $\mathcal{F}$ which determine them. They are therefore often referred to as self consistent field (SCF)
% equations and must be solved iteratively. The procedure goes along the following lines. First make an initial guess for the spin orbitals $\psi_k$ and calculate the Fock
% operator $\mathcal{F}$. Then, solve the Hartree-Fock equations to obtain a new set of spin orbitals and use these as input to calculate a new Fock operator.
% The process continues like this until a convergence criterium is reached.
% 
% The term in the square brackets is the one-body operator which we have called $h(\vec r)$ in equation (\ref{eq:H1}). The two extra sums are due to the interactions between
% the electrons. The first of these represent the Coulomb potential from the mean field set up by the electrons of the system. The last is similar to the first
% except that the indices of two orbitals have been switched. This is a direct consequence of the fact that in the derivation of the equations, the state
% is assumed to be a Slater determinant. Note that due to the last sum, the Fock operator is non-local, that is to say, the value of $\mathcal{F}(\vec x)\psi_k(\vec x)$
% depends on the value of $\psi_k(\vec x')$ at all coordinates $\vec x' \in \mathbb{R}^3 \oplus \{\uparrow, \downarrow\}$.
% 
% As discussed in section \ref{sec:fermion_basis}, any state $\ket{\Psi}$ can be written as a linear combination of Slater determinants
% \begin{equation}
%  \ket{\Psi} = C_0\ket{c} + \sum_{ia}C^a_i\ket{\Psi^a_i} + \sum_{ijab}C^{ab}_{ij}\ket{\Psi^{ab}_{ij}} + \dots
% \end{equation}
% The most crude approximation is to neglect all terms except the reference Slater $\ket{c}$. The Hartree-Fock method answers the following problem:
% Given the approximation $\ket{\Psi} \approx \ket{c}$, what is the optimal set of spin orbitals to choose as constituents of $\ket{c}$, and what is the resulting approximation
% for the energy, $E_{HF}$? The energy $E_{HF}$ is defined as the energy found from an exact solution of the Hartree-Fock equations (which presupposes a complete basis).
% 
% In most cases the Hartree-Fock method provides an excellent first approximation to the wave function and energy of the system, and it often accounts
% for about 99\% of the total energy. However, in quantum chemistry an even higher degree of precision is sometimes needed. In such cases, the solution of
% the Hartree-Fock equations is a very good starting point to use as input to other so-called \emph{post-Hartree-Fock methods}. The other
% methods discussed in this thesis can be considered to belong to this class of methods.
% 
% The difference between the exact energy $E$ and the Hartree-Fock energy $E_{HF}$ is referred to as the correlation energy $\Delta E_{corr}$:
% \begin{equation}
%  \Delta E_{corr} = E - E_{HF}.
% \end{equation}


\section{Derivation of the Hartree-Fock equations}
\label{sec:hartree_fock_derivation}
As discussed in section \ref{sec:fermion_basis}, the exact ground state can be written as a
linear combination of Slater determinants
\begin{equation}
 \ket{\Phi_0} = C_0\ket{\Psi_0} + \sum_{ia}C^a_i\ket{\Psi^a_i} + \sum_{i<j,a<b}C^{ab}_{ij}\ket{\Psi^{ab}_{ij}} + \dots
\end{equation}
where $\ket{\Psi_0}$ is some chosen reference determinant. In the Hartree-Fock method all determinants except $\ket{\Psi_0}$ are neglected, and the
spin orbitals from which $\ket{\Psi_0}$ is constructed are chosen in such a way that the expectation value $E_0 = \bra{\Psi_0}H\ket{\Psi_0}$ comes as
close to the excact energy $\mathscr E_0$ as possible. According to the variational principle, the expectation value $E_0$ is an upper bound
to the exact energy $\mathscr E_0$, which means that the optimal choice of spin orbitals are those which minimise $E_0$. When $E_0$ is at its minimum, any
infinitesimal variation of the spin orbitals will leave $E_0$ unchanged, which in mathematical terms means that
\begin{equation}
 \delta E_0 = \sum_{k=1}^N[\bra{\delta\psi_k}\mathcal F\ket{\psi_k} + \bra{\psi_k}\mathcal F\ket{\delta\psi_k}] = 0.
\end{equation}
If each variation could be chosen independently, this would immediately give us $N$ equations to be solved for the $N$ spin orbitals.
Unfortunately, the spin orbitals cannot be varied independently, but must remain orthonormal throughout the variation. The orthonormality condition reads
\begin{equation}
 \langle\psi_k|\psi_l\rangle - \delta_{kl} = 0.
\end{equation}
This type of \emph{constrained} optimisation problem can be solved elegantly by the method of Lagrangian multipliers. A good review of the
method is given in Boas \cite{boas2005mathematical}. The method simply says that if we construct the new functional
\begin{equation}
\label{eq:L_operator}
 \mathscr L = E_0 - \sum_{k,l=1}^N\Lambda_{lk}[\langle\psi_k|\psi_l\rangle - \delta_{kl}],
\end{equation}
we can find the stationary value of $E_0$ by solving the \emph{unconstrained} variational problem for $\mathscr L$. By unconstrained we mean that the variation
of each spin orbital can be chosen freely. The multipliers $\Lambda_{lk}$ are called Lagrange multipliers and will also be determined as part of the solution.

% In the Hartree-Fock method,
% all but the first term is neglected and the orbitals from which this determinant is
% constructed are chosen such that the reference energy $E_0 = \bra{\Psi_0}H\ket{\Psi_0}$ comes
% as close as possible to the exact energy. The variational principle says that the reference
% energy $E_0$ gives an upper bound to the exact energy. We are therefore seeking the spin orbitals
% which minimize $E_0$.

Recall from equation (\ref{eq:E_ref}) that the reference energy is given by
\begin{equation}
 E_0 = \bra{\Psi_0}H\ket{\Psi_0} = \sum_{k=1}^N\bra{\psi_k} h\ket{\psi_k} + \frac{1}{2}\sum_{k,l=1}^N[\bra{\psi_k\psi_l}g\ket{\psi_k\psi_l}
                                    - \bra{\psi_k\psi_l}g\ket{\psi_l\psi_k}].
\end{equation}
Taking the variation of this yields
\begin{equation}
\begin{split}
 \delta E_0  = & \sum_{k=1}^N\bra{\delta \psi_k}h\ket{\psi_k} \\
          &   + \frac{1}{2}\sum_{k,l = 1}^N[\bra{\delta \psi_k \psi_l}g\ket{\psi_k \psi_l} + \bra{\psi_k \delta \psi_l}g\ket{\psi_k \psi_l} \\
          &  \hspace{12mm} - \bra{\delta \psi_k \psi_l}g\ket{\psi_l \psi_k} - \bra{\psi_k \delta \psi_l}g\ket{\psi_l \psi_k}] + \text{c.c.}\\
        = & \sum_{k=1}^N\bra{\delta \psi_k}h\ket{\psi_k} \\
          & + \frac{1}{2}\sum_{k,l = 1}^N[\bra{\delta \psi_k \psi_l}g\ket{\psi_k \psi_l} + \bra{\delta\psi_l \psi_k}g\ket{\psi_l \psi_k} \\
          & \hspace{12mm} - \bra{\delta \psi_k \psi_l}g\ket{\psi_l \psi_k} - \bra{\delta\psi_l \psi_k}g\ket{\psi_k \psi_l}] + \text{c.c.}\\
        = & \sum_{k=1}^N\bra{\delta \psi_k}h\ket{\psi_k} + \sum_{k,l = 1}^N[\bra{\delta \psi_k \psi_l}g\ket{\psi_k\psi_l} - \bra{\delta \psi_k \psi_l}g\ket{\psi_l \psi_k}] + \text{c.c.}
\end{split}
\end{equation}
where c.c. represents complex conjugate terms.
In the expression after the second equality sign, we have used that $\bra{\psi_k\delta\psi_l}g\ket{\psi_k\psi_l} = \bra{\delta\psi_l\psi_k}g\ket{\psi_l\psi_k}$, and the last
line follows from the fact that the indices $k$ and $l$ can be switched since they are dummy indices.
We next define the two operators
\begin{eqnarray}
\label{eq:J_operator}
 \mathcal{J}(\vec x)\psi_k(\vec x) & = & \sum_{l=1}^N\Big[\int d\vec x' \psi^*_l(\vec x') g(\vec r, \vec r')\psi_l(\vec x')\Big]\psi_k(\vec x), \\ \label{eq:K_operator}
 \mathcal{K}(\vec x)\psi_k(\vec x) & = & \sum_{l=1}^N\Big[\int d\vec x' \psi^*_l(\vec x') g(\vec r, \vec r')\psi_k(\vec x')\Big]\psi_l(\vec x),
\end{eqnarray}
so that the variation of the energy can be written more compactly as
\begin{equation}
 \delta E_0 = \sum_{k=1}^N\bra{\delta \psi_k} h + \mathcal J - \mathcal K \ket{\psi_k} + \text{c.c.}
\end{equation}
Next we consider the variation of the constraint:
\begin{equation}
 \sum_{k,l=1}^N[\Lambda_{lk}\langle\delta\psi_k|\psi_l\rangle + \Lambda_{lk}\langle\psi_k|\delta\psi_l\rangle].
\end{equation}
We will show that the second term of this equation is in fact the complex conjugate of the first. To realise this, consider the functional $\mathscr L$
of equation (\ref{eq:L_operator}). First note that it must be real because $E_0$ is real and the added constraints are equal to zero.
Taking the complex conjugate of $\mathscr L$ therefore gives
\begin{equation}
\begin{split}
 \mathscr L = & E_0 - \sum_{k,l=1}^N\Lambda^*_{lk}[\langle\psi_k|\psi_l\rangle^* - \delta_{kl}] \\
            = & E_0 - \sum_{k,l=1}^N\Lambda^*_{lk}[\langle\psi_l|\psi_k\rangle - \delta_{lk}] \\
            = & E_0 - \sum_{k,l=1}^N\Lambda^*_{kl}[\langle\psi_k|\psi_l\rangle - \delta_{kl}].
\end{split}
\end{equation}
This form of $\mathscr L$ is identical to the original of (\ref{eq:L_operator}) except that $\Lambda_{lk}$ has been replaced by $\Lambda^*_{kl}$.
Since both will yield exactly the same Lagrange multipliers (assuming that the solution is unique), this means that
\begin{equation}
 \Lambda_{lk} = \Lambda^*_{kl},
\end{equation}
that is to say, $\Lambda_{lk}$ are the elements of a Hermitian matrix. Thus we can write the variation of the constraint as
\begin{equation}
\begin{split}
 \sum_{k,l=1}^N[\Lambda_{lk}\langle\delta\psi_k|\psi_l\rangle + \Lambda_{lk}\langle\psi_k|\delta\psi_l\rangle] = &
 \sum_{k,l=1}^N\Lambda_{lk}\langle\delta\psi_k|\psi_l\rangle + \sum_{k,l=1}^N\Lambda^*_{kl}\langle\delta\psi_l|\psi_k\rangle^* \\
= & \sum_{k,l=1}^N\Lambda_{lk}\langle\delta\psi_k|\psi_l\rangle + \sum_{k,l=1}^N\Lambda^*_{lk}\langle\delta\psi_k|\psi_l\rangle^* \\
= & \sum_{k,l=1}^N\Lambda_{lk}\langle\delta\psi_k|\psi_l\rangle + \text{c.c.}
\end{split}
\end{equation}
Putting it all together, the variation of $\mathscr L$ is now
\begin{equation}
\begin{split}
 \delta \mathscr L &  = \sum_{k=1}^N\bra{\delta \psi_k}\Big[ (h + \mathcal J - \mathcal K)\ket{\psi_k}
                     - \sum_{l=1}^N\Lambda_{lk}\ket{\psi_l}\Big] + \text{c.c.} \\
                   & = 0.
\end{split}
\end{equation}
By defining the Fock operator
\begin{equation}
\label{eq:Fock_operator}
\mathcal{F} = h + \mathcal J - \mathcal K
\end{equation}
the above equation can be written even more compactly as
\begin{equation}
\begin{split}
  \delta \mathscr L & =\sum_{k=1}^N\bra{\delta \psi_k}\Big[\mathcal F\ket{\psi_k}
                       - \sum_{l=1}^N\Lambda_{lk}\ket{\psi_l}\Big] + \text{c.c.} \\
                    & = 0.
\end{split}
\end{equation}
Since the variations of the spin orbitals can be chosen freely, each term
in the square brackets must be equal to zero, which implies that
\begin{equation}
\label{eq:non_canonical_hartree_fock}
 \mathcal F\psi_k = \sum_{l=1}^N\Lambda_{lk}\psi_l.
\end{equation}
This equation\footnote{We are actually talking about a set of $N$ equations for all the spin orbitals
$\{\psi_k\}_{k=1}^N$, but since we observe that they are all equal, we may refer to \emph{the equation}.}
is not on the same form as the one introduced at the beginning of the chapter. This reason is as follows.
Given a solution $\{\psi_k\}$ of the equation above, it is possible to obtain a new set of spin orbitals
$\{\psi'_k\}$ via a unitary transformation
\begin{equation}
 \psi'_k = \sum_l \psi_l\ U_{lk}
\end{equation}
which keeps the expectation value $E_0 = \bra{\Psi'_0}H\ket{\Psi'_0}$ as well as the form of the Fock operator
unchanged, see Szabo and Ostlund \cite{Szabo}. Thus there is some flexibility in the choice of spin orbitals.
One particular choice of spin orbitals are the eigenfunctions of the Fock operator
\begin{equation}
\label{eq:canonical_hartree_fock}
 \mathcal F\psi_k = \varepsilon_k\psi_k,
\end{equation}
which are guaranteed to exist since $\mathcal F$ is Hermitian.
These particular spin orbitals are solutions of equation (\ref{eq:non_canonical_hartree_fock}) for the specific case where
$\Lambda_{lk} = \varepsilon_k\delta_{lk}$. Equation (\ref{eq:canonical_hartree_fock}) is called the canonical Hartree-Fock
equation.

If we are studying a molecular system, that is, if the system has more than a single nucleus,
the eigenfunctions of the Hartree-Fock equations are called \emph{molecular orbitals} (MOs). The word
molecular is used to emphasise that the orbitals are characterisitic of the molecular system.
It is important to distinguish between these and the familiar atomic orbitals because they are usually
very different. This means that for molecular systems it is no longer helpful to think of the
electrons as occupying atomic orbitals; the atomic orbitals are solutions of the Hartree-Fock equations
for \emph{isolated atoms}, but the molecule is an entirely different system with often entirely 
different solutions.

When the Slater determinant $\ket{\Psi_0}$ is composed of the $N$ lowest
eigenfunctions of the Hartree-Fock equations we will call it the Hartree-Fock determinant,
and we will refer to $E_0$ as the Hartree-Fock energy. This will be the case for the
remainder of the thesis unless stated otherwise.

% Note that the Hartree-Fock equations are the same for all orbitals, which is reassuring. Recall that the Slater determinant was motivated from the fact that we wanted
% a many-particle wave function which did not distinguish between the particles. Had we gone through the exact same variational procedure as above, but instead started with the energy
% corresponding to the Hartree product, the equations would be different for different orbitals.


The Hartree-Fock energy can be written in terms of the operators $\mathcal J$ and $\mathcal K$ as
\begin{equation}
 E_0 = \sum_{k=1}^N\bra{\psi_k}h + \frac{1}{2}(\mathcal J - \mathcal K) \ket{\psi_k},
\end{equation}
which shows that the eigenvalues of the Hartree-Fock equations (\ref{eq:canonical_hartree_fock}) do not add up to the ground state energy; the term $(\mathcal J-\mathcal K)$ in the Fock operator
is a factor of two too large. However, the energy can be calculated via the eigenvalues in the following two equivalent ways:
\begin{equation}
\label{eq:E_ref2}
\begin{split}
 E_0 & = \frac{1}{2}\sum_{k=1}^N[\varepsilon_k + \bra{\psi_k}h\ket{\psi_k}] \\
         & = \sum_{k=1}^N[\varepsilon_k - \frac{1}{2}\bra{\psi_k}\mathcal J - \mathcal K\ket{\psi_k}].
 \end{split}
\end{equation}




\section{Restricted and unrestricted determinants}
\label{sec:res_and_unres_dets}
In (\ref{eq:canonical_hartree_fock}) the Hartree-Fock equations are written on their most general form. The unknowns are the eigenvalues $\varepsilon_k$ and the spin orbitals $\psi_k$.
However, before solving the equations, it is useful to rewrite them in terms of spatial orbitals $\phi_k$ instead of spin orbtitals $\psi_k$. This is done by
integrating out the spin part, as will be shown in the next section. But first we must decide how to construct the spin orbitals from the spatial orbitals. There are two
ways to do this: One can either form so-called restricted spin orbitals or unrestricted spin orbitals. The two approaches will lead to two different Hartree-Fock methods,
namely the restricted Hartree-Fock method (RHF) and the unrestricted Hartree-Fock method (UHF), respectively.

\subsection{Restricted determinants}
Recall from equation (\ref{eq:spin_orbital}) that a spin orbital $\psi_k$ is a spatial orbital $\phi_k$ multiplied with a spin function which is either spin up, $\alpha$,
or spin down, $\beta$. This means that we can create spin orbitals in the following way
\begin{equation}
\label{eq:restricted_spinorbitals}
\psi_k(\vec x) = \left\{\begin{array}{c} \phi_l(\vec r)\alpha(s) \\
                                          \text{or} \\
                                          \phi_l(\vec r)\beta(s). \\
                       \end{array}\right.
\end{equation} 
Spin orbitals on this form are called restricted spin orbitals, and the Slater determinants they form are called restricted determinants. In such determinants, a spatial orbital is either
occupied by a single electron or two electrons, see figure \ref{fig:restricted_determinant}. A determinant which has
every spatial orbital doubly occupied, is called a closed shell determinant (left figure), whereas a
determinant that has one or more partially filled spatial orbitals, is called an open shell determinant (right figure). 
If the system has an odd number of electrons, the determinant will always be open shell.
However, an even number of electrons does not imply that the determinant is closed shell; if degeneracies apart from that due to spin are present, it can still be open shell.
% Recall from equation (\ref{eq:spin_orbital}) that a spin orbital $\psi_k$ is a spatial orbital $\phi_k$ multiplied with a spin function which is either spin up, $\alpha$,
% or spin down, $\beta$. This means that a set of $N$ spin orbitals can be generated from a set of $\lceil N/2\rceil$ spatial orbitals in the following way:
% \begin{equation}
% \label{eq:restricted_spinorbitals}
%  \{\psi_{2k-1}(\vec x),\,\psi_{2k}(\vec x)\} = \{\phi_k(\vec r)\alpha(s),\,\phi_k(\vec r)\beta(s)\}, \qquad k = 1,\dots,\lceil N/2\rceil.
% \end{equation} 
% Spin orbitals on this form are called restricted spin orbitals, and the Slater determinants they form are called restricted determinants. In such determinants, a spatial orbital is either
% occupied by a single electron or two electrons, one with spin up and one with spin down, see figure \ref{fig:restricted_determinant}. A determinant which has
% every spatial orbital doubly occupied, is called a closed shell determinant (left figure), whereas a
% determinant that has one or more partially filled spatial orbitals, is called an open shell determinant (right figure). 
% If the system has an odd number of electrons, the determinant will always be open shell.
% However, an even number of electrons does not imply that the determinant is closed shell; if degeneracies apart from that due to spin are present, it can still be open shell.
\begin{figure}
 \begin{center}
  \includegraphics[scale=0.7]{hartree_fock/figures/restricted_determinant.pdf}
  \caption{Illustration of the restricted determinant comprised of spin orbitals on the form (\ref{eq:restricted_spinorbitals}).
  The left and right figures illustrate closed and open shell determinants, respectively.}
  \label{fig:restricted_determinant}
 \end{center}
\end{figure}

Throughout this thesis we will limit the use of restricted determinants (and restricted Hartree-Fock) to closed shell systems. This means that the
spin orbitals are given by
\begin{equation}
\label{eq:restricted_spinorbitals2}
 \{\psi_{2k-1}(\vec x),\,\psi_{2k}(\vec x)\} = \{\phi_k(\vec r)\alpha(s),\,\phi_k(\vec r)\beta(s)\}, \qquad k = 1,\dots,N/2,
\end{equation}
where $N$ is the number of electrons.


\subsection{Unrestricted determinants}
In equation (\ref{eq:restricted_spinorbitals}) the spin-up electrons are described by the same set of spatial orbitals as the spin-down electrons.
For closed shell systems this is often a good assumption.
However, consider the open shell determinant illustrated to the right of figure \ref{fig:restricted_determinant}. The electron occupying the spin orbital 
$\phi_3\alpha$ will have an exchange interaction with the other spin-up electrons, but not with the spin-down electrons.
Hence, it could be energetically favourable to let the spin-up levels shift with respect to the spin-down levels, as shown in figure \ref{fig:unrestricted_determinant}.
This can be accomplished by letting the spin-up and spin-down electrons be described by different sets of spatial orbitals.
Spin orbitals formed in this way, are called unrestricted spin orbitals, and the Slater determinants they form are called unrestricted determinants.
\begin{equation}
\label{eq:unrestricted_spinorbitals}
\psi_k(\vec x) = \left\{\begin{array}{c} \phi^\alpha_l(\vec r)\alpha(s) \\
                                          \text{or} \\
                                          \phi^\beta_l(\vec r)\beta(s). \\
                       \end{array}\right.
\end{equation}
% In addition to  the case of open shell systems, unrestricted determinants will give lower energies when studying the dissociation of molecules. To see why, consider the
% $H_2$-molecule as an example. The restricted orbitals will force both electrons to be located at either of the nuclei with equal probability, and although
% this is reasonable for small internuclear distances, as the distance increases, the electrons will in fact be located at differenct nuclei. This can only be
% realised by the unrestricted spin orbitals which allow the electrons to have different spatial orbitals.
\begin{figure}
 \begin{center}
  \includegraphics[scale=0.7]{hartree_fock/figures/unrestricted_determinant.pdf}
  \caption{Illustration of the unrestricted determinant comprised of spin orbitals on the form (\ref{eq:unrestricted_spinorbitals}).
  The left and right figures illustrate closed and open shell determinants, respectively.}
  \label{fig:unrestricted_determinant}
 \end{center}
\end{figure}


% \subsection{Determinants and spin operators}
% The Hamiltonians which are considered in this thesis have no spin dependence. This means that it commutes with the total spin operator $\mathcal S^2$ and the spin operator
% in the z-direction $\mathcal S_z$. If we assume that the ground state of the Hamiltonian $\ket{\Psi_0}$ is nondegenerate, this means that $\ket{\Psi_0}$ is
% an eigenstate of $\mathcal S^2$ and $\mathcal S_z$ as well. To see this consider the following:
% \begin{equation}
% \begin{split}
%  H(\mathcal S^2\ket{\Psi_0}) = & ([\mathcal S^2,H] + \mathcal S^2H)\ket{\Psi_0} \\
%                              = & (0 + \mathcal S^2H)\ket{\Psi_0} \\
%                              = & E_0 \mathcal S^2\ket{\Psi_0}.
% \end{split}
% \end{equation}
% This shows that $\mathcal S^2\ket{\Psi_0}$ is also an eigenstate of $H$ with eigenvalue $E_0$, and since the ground state of $H$ is nondegenerate, this means that
% \begin{equation}
%  \mathcal S^2\ket{\Psi_0} = A \ket{\Psi_0}
% \end{equation}
% for some constant $A$.

\section{Slater determinants and the spin operators}
\label{sec:determinants_and_spin}
In this section we define the total spin operator for a system of $N$ particles and discuss how it acts on the restricted
and unrestricted determinants.
The reader is referred to the texts by Griffiths \cite{griffiths} and Shankar \cite{shankar} for introductory treatments of spin.

\subsection{Single-particle spin operators}
The spin operator for a single particle is given by
\begin{equation}
 \vec s = s_x\vec i + s_y\vec j + s_z\vec k,
\end{equation}
where $s_x$, $s_y$ and $s_z$ are the operators for the spin components along the coordinate axes. The latter satisfy the commutation relations
\begin{equation}
\label{eq:spin_commutation}
 [s_x,\, s_y] = i s_z, \qquad [s_y,\, s_z] = i s_x, \qquad [s_z,\, s_x] = i s_y.
\end{equation}
The squared magnitude of the spin operator is a scalar operator
\begin{equation}
 \vec s^2 = s_x^2 + s_y^2 + s_z^2.
\end{equation}
It commutes with all of the component operators, and it is therefore possible to find a common set of eigenstates for
$\vec s^2$ and \emph{one} of the operators $s_x$, $s_y$ or $s_z$. The standard choice in the literature is $s_z$.
A particle with spin $s$ has eigenvalues $s(s+1)$ and $m_s$
\begin{equation}
\begin{split}
 s^2 \ket{s,m_s} & = s(s+1) \ket{s,m_s}, \\
 s_z \ket{s,m_s} & = m_s \ket{s,m_s},
\end{split}
\end{equation}
where $m_s$ can take the values $\{-s,\, -s+1,\dots, s-1,\,s\}$. Electrons have spin 1/2, and the Hilbert space describing the
spin of electrons is therefore spanned by the two states $\ket{1/2,1/2}$ and $\ket{1/2,-1/2}$, which we until now have simply called $\ket{\alpha}$
and $\ket{\beta}$ for convenience.



\subsection{Many-particle spin operators}
Analogously, the total spin operator for a system of $N$ particles is given by
\begin{equation}
 \vec S = \sum_{i=1}^N \vec s(i),
\end{equation}
where $\vec s(i)$ is the spin operator of particle $i$. Also, the squared magnitude of the total spin operator is given by
\begin{equation}
 \vec S^2 = S_x^2 + S_y^2 + S_z^2,
\end{equation}
where
\begin{equation}
 S_I = \sum_{i=1}^N s_I(i), \qquad I\in\{x,\,y,\,z\}.
\end{equation}
Since the Hamiltonian does not depend on any of the spin coordinates, it commutes with $\vec S^2$ as well as $S_z$
\begin{equation}
 [H,\,\vec S^2] = [H, S_z] = 0.
\end{equation}
This means that the exact eigenstates of the Hamiltonian $\ket{\Phi_i}$ are also eigenstates of $\vec S^2$ and $S_z$ \cite{Szabo}
\begin{align}
 \vec S^2\ket{\Phi_i} & = S(S+1)\ket{\Phi_i},\\
 S_z\ket{\Phi_i} & = M_S\ket{\Phi_i},
\end{align}
where $S$ and $M_S$ are the quantum numbers for the total spin and its projection along the $z$-axis, respectively. A natural question
to ask at this point is whether or not the restricted and unrestricted determinants are eigenstates of $\vec S^2$ and $S_z$. The answer to this question is as follows:
\begin{enumerate}
 \item Both restricted and unrestricted determinants are eigenstates of $S_z$.
 \item The restricted closed shell determinant is an eigenstate of $\vec S^2$ with eigenvalue 0, making it a pure singlet state.
 \item The unrestricted determinant is generally not an eigenstate of $\vec S^2$.
\end{enumerate}
A consequence of point three is that unrestricted determinants can often have a total spin which is larger than the exact value. This is often referred
to as spin contamination of the unrestricted determinant. The reader is referred to appendix \ref{chapter:appendix_spin} for a further discussion of this topic.




\section{Restricted Hartree-Fock (RHF)}
\label{sec:RHF}
The most general form of the Hartree-Fock equations is given in (\ref{eq:canonical_hartree_fock}), where the Fock operator $\mathcal{F}$ is given by (\ref{eq:Fock_operator}) and
(\ref{eq:J_operator}) - (\ref{eq:K_operator}). We will now derive the equations which result from the restricted assumption in (\ref{eq:restricted_spinorbitals}).
Without loss of generality, we may assume that the spin orbital on which the Fock operator acts is spin up:
\begin{equation}
\mathcal F(\vec x)\phi_k(\vec r) \alpha(s) = \varepsilon_k\phi_k(\vec r)\alpha(s).
\end{equation}
Writing out the Fock operator explicitly:
\begin{equation}
\begin{split}
 \mathcal{F}(\vec x)\phi_k(\vec r)\alpha(s)  = &\ h(\vec r)\,\phi_k(\vec r)\alpha(s) \\
                      &  + \Big[\sum_{l=1}^N\int d\vec x'\psi^*_l(\vec x')g(\vec r, \vec r')\psi_l(\vec x')\Big]\phi_k(\vec r)\alpha(s) \\
                       &    - \Big[\sum_{l=1}^N\int d\vec x'\psi^*_l(\vec x')g(\vec r, \vec r')\phi_k(\vec r')\alpha(s)\Big]\psi_l(\vec x) \\
                       = &\  \varepsilon_k\phi_k(\vec r)\alpha(s).
%                    & = & h(\vec r)\,\phi_k(\vec r)\xi_k(s) + 2\Big[\sum_{l=1}^{N/2}\int d\vec r'\phi^*_l(\vec r')g(\vec r, \vec r')\phi_l(\vec r')\Big]\phi_k(\vec r)\xi_k(s) \\
%                    &   & - \Big[\sum_{l=1}^{N/2}\int d\vec r'\phi^*_l(\vec r')g(\vec r, \vec r')\phi_k(\vec r')\Big]\phi_l(\vec r)\xi_k(s).
\end{split}
\end{equation}
Our goal is to integrate out the spin of this equation. To do this, we must also express the spin orbitals in the sums in terms of spatial orbitals. We will assume that
$N$ is an even number of electrons and that we are dealing with a closed shell determinant. This means that both sums, which run from 0 to $N$, can be split
into two sums which run from 0 to $N/2$:
\begin{equation}
\begin{split}
 \mathcal{F}(\vec x)\phi_k(\vec r)\alpha(s)  = &\ h(\vec r)\,\phi_k(\vec r)\alpha(s) \\
                      &  + \Big[\sum_{l=1}^{N/2}\int d\vec x'\phi^*_l(\vec r')\alpha^*(s')g(\vec r, \vec r')\phi_l(\vec r')\alpha(s')\Big]\phi_k(\vec r)\alpha(s) \\
                      &  + \Big[\sum_{l=1}^{N/2}\int d\vec x'\phi^*_l(\vec r')\beta^*(s')g(\vec r, \vec r')\phi_l(\vec r')\beta(s')\Big]\phi_k(\vec r)\alpha(s) \\
                       & - \Big[\sum_{l=1}^{N/2}\int d\vec x'\phi^*_l(\vec r')\alpha^*(s')g(\vec r, \vec r')\phi_k(\vec r')\alpha(s')\Big]\phi_l(\vec r)\alpha(s) \\
                       & - \Big[\sum_{l=1}^{N/2}\int d\vec x'\phi^*_l(\vec r')\beta^*(s')g(\vec r, \vec r')\phi_k(\vec r')\alpha(s')\Big]\phi_l(\vec r)\beta(s) \\
                     = &\ \varepsilon_k\phi_k(\vec r)\alpha(s).
\end{split}
\end{equation}
We note that the first two sums are in fact equal when the spin coordinate $s'$ is integrated out. Furthermore, the last sum is equal to zero because the integration
is over unequal spins. Thus we are left with
\begin{equation}
\begin{split}
 \mathcal{F}(\vec x)\phi_k(\vec r)\alpha(s)  = &\ h(\vec r)\,\phi_k(\vec r)\alpha(s) \\
                      &  + 2\Big[\sum_{l=1}^{N/2}\int d\vec r'\phi^*_l(\vec r')g(\vec r, \vec r')\phi_l(\vec r')\Big]\phi_k(\vec r)\alpha(s) \\
                       & - \Big[\sum_{l=1}^{N/2}\int d\vec r'\phi^*_l(\vec r')g(\vec r, \vec r')\phi_k(\vec r')\Big]\phi_l(\vec r)\alpha(s) \\
                       = &\ \varepsilon_k\phi_k(\vec r)\alpha(s).
\end{split}
\end{equation}
If we now multiply both sides of the equation by $\alpha^*(s)$ and integrate over the spin coordinate $s$ we finally arrive at
\begin{equation}
\begin{split}
 \big[\sum_{s=\uparrow\downarrow}\alpha^*(s)\mathcal F(\vec x)\alpha(s)\Big]\phi_k(\vec r) = &\ h(\vec r)\,\phi_k(\vec r) \\
                       & + 2\Big[\sum_{l=1}^{N/2}\int d\vec r'\phi^*_l(\vec r')g(\vec r, \vec r')\phi_l(\vec r')\Big]\phi_k(\vec r) \\
                       & - \Big[\sum_{l=1}^{N/2}\int d\vec r'\phi^*_l(\vec r')g(\vec r, \vec r')\phi_k(\vec r')\Big]\phi_l(\vec r) \\
                     = &\ \varepsilon_k\phi_k(\vec r).
\end{split}
\end{equation}
Defining the restricted spatial Fock operator as
\begin{equation}
 F(\vec r) = \sum_{s=\uparrow\downarrow}\alpha^*(s)\mathcal F(\vec x)\alpha(s),
\end{equation}
the equation can be written as
\begin{equation}
\label{eq:RHF}
 F(\vec r)\phi_k(\vec r) = \varepsilon_k\phi_k(\vec r),
\end{equation}
where
\begin{equation}
\begin{split}
 F(\vec r)\phi_k(\vec r) = & h(\vec r)\,\phi_k(\vec r)  + 2\Big[\sum_{l=1}^{N/2}\int d\vec r'\phi^*_l(\vec r')g(\vec r, \vec r')\phi_l(\vec r')\Big]\phi_k(\vec r) \\
                       & - \Big[\sum_{l=1}^{N/2}\int d\vec r'\phi^*_l(\vec r')g(\vec r, \vec r')\phi_k(\vec r')\Big]\phi_l(\vec r).
\end{split}
\end{equation}
The spatial Fock operator can be written more compactly as
\begin{equation}
\label{eq:restricted_Fock_operator}
 F(\vec r) = h(\vec r) + 2 J(\vec r) - K(\vec r),
\end{equation}
where
\begin{align}
 J(\vec r)\phi_k(\vec r) & = \sum_{l=1}^{N/2}\int d\vec r'\phi^*_l(\vec r')g(\vec r, \vec r')\phi_l(\vec r')\Big]\phi_k(\vec r), \\
 K(\vec r)\phi_k(\vec r) & = \sum_{l=1}^{N/2}\int d\vec r'\phi^*_l(\vec r')g(\vec r, \vec r')\phi_k(\vec r')\Big]\phi_l(\vec r).
\end{align}
Note the factor 2 in front of the direct term, or more to the point, the absence of the factor 2 in front of the exchange term. This is due to the fact that
the exchange interaction is only present between electrons of equal spins.

To find the energy, consider the first line of equation (\ref{eq:E_ref2}):
\begin{equation}
\begin{split}
 E_0 = & \frac{1}{2}\sum_{k=1}^N[\varepsilon_k + \bra{\psi_k}h\ket{\psi_k}] \\
         = & \frac{1}{2}\sum_{k=1}^{N}\bra{\psi_k}(\mathcal{F} + h)\ket{\psi_k}.
\end{split}
\end{equation}
If, as we did above, insert the assumption (\ref{eq:restricted_spinorbitals}) and split the sum in two sums, one with spin up and one with spin down, we get
\begin{equation}
\label{eq:E_ref3}
 E_0 = \sum_{k=1}^{N/2}\bra{\phi_k}(F + h)\ket{\phi_k}.
\end{equation}


\subsection{Introducing a basis}
We have now eliminated the spin from the general Hartree-Fock equations (\ref{eq:canonical_hartree_fock}) and arrived at equation (\ref{eq:RHF}), which represents
a set of integro-differential equations for the spatial orbitals. There is presently no feasible way to solve these as they stand.
However, by expressing the orbitals in terms of some known basis
\begin{equation}
\label{eq:linear_expansion}
 \phi_k(\vec r) = \sum_{\mu=1}^M C_{\mu k}\chi_\mu(\vec r),
\end{equation}
where $M$ is the number of basis functions, the equations can be converted to a set of algebraic equations. Inserting this expansion into equation (\ref{eq:RHF}) gives
\begin{equation}
 \sum_{\nu=1}^M F(\vec r)\chi_\nu(\vec r) C_{\nu k} = \varepsilon_k \sum_{\nu=1}^M C_{\nu k} \chi_{\nu}(\vec r).
\end{equation}
If we now multiply by $\chi^*_\mu(\vec r)$ and integrate with respect to $\vec r$, we get the so-called \emph{Roothaan equations} \cite{roothan}
\begin{equation}
 \sum_{\nu=1}^M F_{\mu\nu}C_{\nu k} = \varepsilon_k\sum_{\nu=1}^M S_{\mu\nu}C_{\nu k},
\end{equation}
or in matrix notation
\begin{equation}
\label{eq:Roothaan}
 \mathbf{FC}_k = \varepsilon_k\mathbf{SC}_k,
\end{equation}
where
\begin{equation}
 F_{\mu\nu} = \int d\vec r \chi^*_\mu(\vec r)F(\vec r)\chi_\nu(\vec r)
\end{equation}
is the Fock matrix and
\begin{equation}
 S_{\mu\nu} = \int d\vec r \chi^*_\mu(\vec r)\chi_\nu(\vec r)
\end{equation}
is the overlap matrix. If the basis is orthonormal, the overlap matrix is the identity matrix. However, in molecular calculations Gaussian functions are
most often used, and these are not orthogonal. In section \ref{sec:gen_eig} we show how equation (\ref{eq:Roothaan}) can be transformed to a regular
eigenvalue problem
\begin{equation}
 \mathbf{F'C'_k} = \varepsilon_k\mathbf C'_k.
\end{equation}


Let us take a closer look at the Fock matrix:
\begin{equation}
 \begin{split}
  F_{\mu\nu} = & \int d\vec r \chi^*_\mu(\vec r) F(\vec r)\chi_\nu(\vec r)\\
                  = & \int d\vec r \chi^*_\mu(\vec r)[h(\vec r) + 2J(\vec r) - K(\vec r)]\chi_\nu(\vec r) \\
                  = & \bra\mu h\ket\nu + 2 \bra\mu J\ket\nu - \bra\mu K\ket\nu.
 \end{split}
\end{equation}
To get the final expression for the matrix, we insert the expansion (\ref{eq:linear_expansion}) into the operators $J(\vec r)$ and $K(\vec r)$.
For $J(\vec r)$ this gives
\begin{equation}
\begin{split}
 J(\vec r)\chi_\nu(\vec r) & = \Big[\sum_{k=1}^{N/2}\int d\vec r'\phi^*_k(\vec r')g(\vec r, \vec r')\phi_k(\vec r')\Big]\chi_\nu(\vec r) \\
                             & = \Big[\sum_{k=1}^{N/2}\sum_{\sigma,\lambda=1}^M\int d\vec r'\chi^*_\sigma(\vec r')g(\vec r, \vec r')\chi_\lambda(\vec r')\Big]C^*_{\sigma k}C_{\lambda k}\chi_\nu(\vec r),
\end{split}
\end{equation}
so that
\begin{equation}
 \bra\mu J\ket\nu = \sum_{k=1}^{N/2}\sum_{\sigma,\lambda=1}^M\bra{\mu\sigma}g\ket{\nu\lambda}C^*_{\sigma k}C_{\lambda k},
\end{equation}
where the matrix element $\bra{\mu\sigma}g\ket{\nu\lambda}$ is defined in equation (\ref{eq:pqgrs_def}).
The expression for $\bra\mu K\ket\nu$ is similar, except that the indices $\nu$ and $\lambda$ switch places in the integral:
\begin{equation}
 \bra\mu K\ket\nu = \sum_{k=1}^{N/2}\sum_{\sigma,\lambda=1}^M\bra{\mu\sigma}g\ket{\lambda\nu}C^*_{\sigma k}C_{\lambda k}.
\end{equation}
Hence, the Fock matrix is given explicitly in terms of the basis functions by
\begin{equation}
 F_{\mu\nu} = \bra\mu h\ket\nu + \sum_{k=1}^{N/2}\sum_{\sigma,\lambda=1}^M[2\bra{\mu\sigma}g\ket{\nu\lambda} - \bra{\mu\sigma}g\ket{\lambda\nu}]C^*_{\sigma k}C_{\lambda k}.
\end{equation}
It is useful to define the so-called density matrix
\begin{equation}
\label{eq:density_matrix}
 P_{\lambda\sigma} = 2\sum_{k=1}^{N/2}C_{\lambda k}C^*_{\sigma k},
\end{equation}
which allows us to write the Fock matrix more compactly as
\begin{equation}
\label{eq:Fock_matrix}
 F_{\mu\nu} = \bra\mu h\ket\nu + \frac{1}{2}\sum_{\sigma,\lambda=1}^M P_{\lambda\sigma}[2\bra{\mu\sigma}g\ket{\nu\lambda} - \bra{\mu\sigma}g\ket{\lambda\nu}].
\end{equation}

The energy is found by expanding equation (\ref{eq:E_ref3}) in the known basis:
\begin{equation}
 \label{eq:E_ref4}
 E_0 = \frac{1}{2}\sum_{\mu,\nu=1}^M P_{\nu\mu}[\bra\mu h \ket\nu + F_{\mu\nu}].
\end{equation}






% Inserting this into equation (\ref{eq:RHF}), multiplying by $\chi^*_p(\vec r)$ and integrating with respect to $\vec r$ leads to the following set of eigenvalue equations
% \begin{equation}
%  \mathbf{FC_k} = \varepsilon_k\mathbf{SC_k}
% \end{equation}
% where the elements of the Fock matrix $\mathbf{F}$ are
% \begin{equation}
%  F_{pq} = \bra{p}h\ket{q} + \sum_l\sum_{rs}(2\bra{pr}g\ket{qs} - \bra{pr}g\ket{sq})C^*_{rl}C_{sl}
% \end{equation}
% and the elements of the overlap matrix $\mathbf{S}$ are
% \begin{equation}
%  S_{pq} = \langle p|q\rangle.
% \end{equation}
% It is convenient to introduce the density matrix $P_{rs} = 2\sum_lC^*_{rl}C_{sl}$. The Fock matrix is then written more compactly as
% \begin{equation}
%  F_{pq} = \bra{p}h\ket{q} + \frac{1}{2}\sum_{rs}P_{rs}(2\bra{pr}g\ket{qs} - \bra{pr}g\ket{sq}).
% \end{equation}
% The energy is given by
% \begin{equation}
%  E_{ref} = \sum_{pq}P_{pq}\bra{p}h\ket{q} + \frac{1}{2}\sum_{pqrs}P_{pq}P_{rs}(\bra{pr}g\ket{qs} - \frac{1}{2}\bra{pr}g\ket{sq})
% \end{equation}


\section{Unrestricted Hartree-Fock (UHF)}
\label{sec:UHF}
We now derive the unrestricted Hartree-Fock equations which result from the assumption (\ref{eq:unrestricted_spinorbitals}). The Hartree-Fock equations (\ref{eq:canonical_hartree_fock})
are now split into two sets of equations
\begin{align}
\mathcal F^\alpha(\vec x) \phi^\alpha_k(\vec r)\alpha(s) & = \varepsilon^\alpha_k\phi^\alpha_k(\vec r)\alpha(s), \qquad k=1,2,\dots, N_\alpha, \\
\mathcal F^\beta(\vec x) \phi^\beta_k(\vec r)\beta(s) & = \varepsilon^\beta_k\phi^\beta_k(\vec r)\beta(s), \qquad k=1,2,\dots, N_\beta,
\end{align}
where $N^\alpha$ and $N^\beta$ are the number of electrons with spin up and spin down, respectively.
Again, we want to take out the spin part of the equations. We can insert the unrestricted spin orbitals (\ref{eq:unrestricted_spinorbitals}) and do the calculations
explicitly in the same way as we did for the restricted case. However, using the insight gained during our previous calculation, we can come up with the answer directly. Consider
for example the operator $\mathcal F^\alpha(\vec x)$. It contains a kinetic term and the potential due to the atomic nuclei. Furthermore, it contains
a direct interaction term due to the mean field set up by all electrons, both spin up and spin down. Finally it contains an exchange term due to the mean field
set up by the electrons \emph{with spin up only}. Thus we can conclude that the equations for the spatial orbitals $\phi^\alpha_k$ are given by
\begin{equation}
\label{eq:HF^alpha}
 F^\alpha(\vec r)\phi^\alpha_k(\vec r) = \varepsilon^\alpha_k\phi^\alpha_k(\vec r),
\end{equation}
where the unrestricted spin up Fock operator is defined as
\begin{equation}
\label{eq:Foperator^alpha}
 F^\alpha(\vec r) = h(\vec r) + [J^\alpha(\vec r) - K^\alpha(\vec r)] + J^\beta(\vec r),
\end{equation}
with
\begin{align}
 J^\alpha(\vec r)\phi^\alpha_k(\vec r) = & \sum_{l=1}^{N^\alpha}\Big[\int d\vec r' \phi^\alpha_l(\vec r')g(\vec r, \vec r')\phi^\alpha_l(\vec r')\Big]\phi^\alpha_k(\vec r), \\
 J^\beta(\vec r)\phi^\alpha_k(\vec r) = & \sum_{l=1}^{N^\beta}\Big[\int d\vec r' \phi^\beta_l(\vec r')g(\vec r, \vec r')\phi^\beta_l(\vec r')\Big]\phi^\alpha_k(\vec r), \\
 K^\alpha(\vec r)\phi^\alpha_k(\vec r) = & \sum_{l=1}^{N^\alpha}\Big[\int d\vec r' \phi^\alpha_l(\vec r')g(\vec r, \vec r')\phi^\alpha_k(\vec r')\Big]\phi^\alpha_l(\vec r).
\end{align}
The equations for the spatial orbitals $\phi^\beta_k$ are the same, except that the indices $\alpha$ and $\beta$ switch places.


\subsection{Introducing a basis}
To solve the unrestricted Hartree-Fock equations we expand the spatial orbitals in terms of a known basis
\begin{align}
 \phi^\alpha_k(\vec r) & = \sum_{\mu=1}^M C^\alpha_{\mu k} \chi_\mu(\vec r), \label{eq:phi^alpha_expansion}\\
 \phi^\beta_k(\vec r) & = \sum_{\mu=1}^M C^\beta_{\mu k} \chi_\mu(\vec r), \label{eq:phi^beta_expansion}
\end{align}
just as we did in the restricted case. Inserting the expansion (\ref{eq:phi^alpha_expansion}) into equation (\ref{eq:HF^alpha}) gives
\begin{equation}
 \sum_{\nu=1}^M F^\alpha(\vec r)\chi_\nu(\vec r)C^\alpha_{\nu k} = \varepsilon^\alpha_k\sum_{\nu=1}^M C^\alpha_{\nu k}\chi_\nu(\vec r).
\end{equation}
If we multiply this equation by $\chi^*_\mu(\vec r)$ and integrate over $\vec r$, we get
\begin{equation}
 \sum_{\nu=1}^M F^\alpha_{\mu\nu}C^\alpha_{\nu k} = \varepsilon^\alpha_k\sum_{\nu=1}^M S_{\mu\nu}C^\alpha_{\nu k},
\end{equation}
where $S_{\mu\nu}$ is the overlap matrix and $F^\alpha_{\mu\nu}$ is the matrix representation of the unrestricted spatial Fock operator
\begin{equation}
\label{eq:Fmatrix^alpha}
 F^\alpha_{\mu\nu} = \int d\vec r \chi^*_\mu(\vec r) F^\alpha(\vec r)\chi_\nu(\vec r).
\end{equation}
The corresponding equations for the spin down particles are derived in exactly the same way, of course. In total we have the two sets of equations
\begin{align}
 \mathbf F^\alpha \mathbf C^\alpha_k = & \varepsilon^\alpha_k\mathbf{SC}^\alpha_k, \label{eq:pople-nesbet_up}\\
 \mathbf F^\beta \mathbf C^\beta_k = & \varepsilon^\beta_k\mathbf{SC}^\beta_k, \label{eq:pople-nesbet_down}
\end{align}
called the \emph{Pople-Nesbet equations} \cite{pople_nesbet}, which are the unrestricted generalisation of the Roothaan equations derived in the previous section. They are nonlinear
and coupled since the matrices $\mathbf F^\alpha$ and $\mathbf F^\beta$ are functions of both $\{\mathbf C^\alpha_k\}$ and $\{\mathbf C^\beta_k\}$.

The final expression for the Fock matrix $\mathbf F^\alpha$ is obtained by inserting the expansion (\ref{eq:phi^alpha_expansion}) into equation (\ref{eq:Fmatrix^alpha}).
Recalling that the Fock operator $F^\alpha(\vec r)$ is given by (\ref{eq:Foperator^alpha}) this leads to
\begin{equation}
 F^\alpha_{\mu\nu} = \bra{\mu}h\ket{\nu} + \sum_{k=1}^{N^\alpha}\sum_{\sigma,\lambda=1}^M\langle\mu\sigma||\nu\lambda\rangle(C^\alpha_{\sigma k})^* C^\alpha_{\lambda k}
                                         + \sum_{k=1}^{N^\beta}\sum_{\sigma,\lambda=1}^M\bra{\mu\sigma}g\ket{\nu\lambda}(C^\beta_{\sigma k})^* C^\beta_{\lambda k},
\end{equation}
and similarly we find
\begin{equation}
 F^\beta_{\mu\nu} = \bra{\mu}h\ket{\nu} + \sum_{k=1}^{N^\beta}\sum_{\sigma,\lambda=1}^M\langle\mu\sigma||\nu\lambda\rangle(C^\beta_{\sigma k})^* C^\beta_{\lambda k}
                                         + \sum_{k=1}^{N^\alpha}\sum_{\sigma,\lambda=1}^M\bra{\mu\sigma}g\ket{\nu\lambda}(C^\alpha_{\sigma k})^* C^\alpha_{\lambda k},
\end{equation}
where $\bra{\mu\sigma}g\ket{\nu\lambda}$ and $\langle\mu\sigma||\nu\lambda\rangle$ are defined in equations (\ref{eq:pqgrs_def}) and (\ref{eq:pqgrs_as_def}),
respectively. If we introduce the density matrices
\begin{align}
  P^\alpha_{\lambda\sigma} = & \sum_{k=1}^{N^\alpha}C^\alpha_{\lambda k} (C^\alpha_{\sigma k})^*, \label{eq:density_matrix_up}\\
  P^\beta_{\lambda\sigma} = & \sum_{k=1}^{N^\beta}C^\beta_{\lambda k} (C^\beta_{\sigma k})^*, \label{eq:density_matrix_down}\\
  P^T_{\lambda\sigma} = & P^\alpha_{\lambda\sigma} + P^\beta_{\lambda\sigma},
\end{align}
the Fock matrices can be written more compactly as
\begin{equation}
\label{eq:Fock_matrix_up}
 F^\alpha_{\mu\nu} = \bra{\mu}h\ket{\nu} + \sum_{\sigma,\lambda=1}^M\left[\langle\mu\sigma||\nu\lambda\rangle P^\alpha_{\lambda\sigma}
                                                                      + \bra{\mu\sigma}g\ket{\nu\lambda}P^\beta_{\lambda\sigma}\right],
\end{equation}
and
\begin{equation}
\label{eq:Fock_matrix_down}
 F^\beta_{\mu\nu} = \bra{\mu}h\ket{\nu} + \sum_{\sigma,\lambda=1}^M\left[\langle\mu\sigma||\nu\lambda\rangle P^\beta_{\lambda\sigma}
                                                                  + \bra{\mu\sigma}g\ket{\nu\lambda}P^\alpha_{\lambda\sigma}\right].
\end{equation}


The energy can be found from the expectation value of these matrices, keeping in mind that the double counting of the interaction terms must be taken into account.
The expression is
\begin{equation}
 E_0 = \frac{1}{2}\sum_{k=1}^{N^\alpha}\sum_{\mu,\nu=1}^M \big[\bra\mu h\ket\nu + F^\alpha_{\mu\nu} \big](C^\alpha_{\mu k})^* C^\alpha_{\nu k}
          +\frac{1}{2}\sum_{k=1}^{N^\beta}\sum_{\mu,\nu=1}^M\left[\bra\mu h\ket\nu + F^\beta_{\mu\nu}\right](C^\beta_{\mu k})^* C^\beta_{\nu k},
\end{equation}
or in terms of the density matrices
\begin{equation}
 E_0 = \frac{1}{2}\sum_{\mu,\nu=1}^M\left[P^T_{\nu\mu}\bra\mu h\ket\nu + P^\alpha_{\nu\mu}F^\alpha_{\mu\nu} + P^\beta_{\nu\mu}F^\beta_{\mu\nu}\right].
\end{equation}




\section{Solving the generalised eigenvalue problem}
\label{sec:gen_eig}

In this section we show how the generalised eigenvalue problem
\begin{equation}
 \mathbf{FC}_k = \varepsilon_k\mathbf{SC}_k,
\end{equation}
can be transformed to the regular eigenvalue problem
\begin{equation}
 \mathbf F'\mathbf C'_k = \varepsilon_k\mathbf C'_k.
\end{equation}
We can achieve this if there exists a matrix $\mathbf X$ such that
\begin{equation}
 \mathbf{X^\dagger S X = I},
\end{equation}
because, if we then let $\mathbf{C_k = X C'_k}$, we get
\begin{equation}
\begin{split}
 \mathbf{FC}_k & = \varepsilon_k\mathbf{SC}_k \\
 \mathbf{FXC}'_k & = \varepsilon_k\mathbf{SXC}'_k \\
 \mathbf{X^\dagger FXC}'_k & = \varepsilon_k\mathbf{X^\dagger SXC}'_k \\
 \mathbf F' \mathbf C'_k & = \varepsilon_k\mathbf C'_k,
\end{split}
\end{equation}
where
\begin{equation}
 \mathbf{F' = X^\dagger FX}.
\end{equation}
It only remains to show that the matrix $\mathbf X$ does indeed exist and how to construct it. First note that the overlap matrix $\mathbf S$ is Hermitian:
\begin{equation}
\begin{split}
 S_{\mu\nu} = & \int d\vec r \chi^*_\mu(\vec r) \chi_\nu(\vec r) \\
            = & \int d\vec r \chi_\nu(\vec r) \chi^*_\mu(\vec r) \\
            = & S^*_{\nu\mu}.
\end{split}
\end{equation}
This means that there exists a unitary matrix $\mathbf U$ such that
\begin{equation}
\label{eq:similarity_transf}
 \mathbf{U^\dagger S U = s},
\end{equation}
where $\mathbf s =$ diag$(s_1, s_2, \dots, s_M)$ is a diagonal matrix containing the eigenvalues of $\mathbf S$, which are all real, and the columns of $\mathbf U$ are the eigenvectors of $\mathbf S$.
Furthermore, the eigenvalues $\{s_i\}$ are positive. To see this, consider the expansion of some function $f(\vec r)$, not identically equal to zero, in terms of the basis functions $\chi_\mu(\vec r)$:
\begin{equation}
 f(\vec r) = \sum_\mu A_\mu\chi_\mu(\vec r).
\end{equation}
No matter how the coefficients are chosen, the norm of $f(\vec r)$ will be positive. In particular, if we choose $\mathbf{A}=[A_\mu]$ to be equal to the $i$'th eigenvector of $\mathbf S$, we get
\begin{equation}
\begin{split}
 0 < \langle f|f\rangle = & \sum_{\mu\nu}A^*_\mu S_{\mu\nu} A_\nu = \mathbf{A^\dagger S A} \\
                    = & s_i\mathbf{A^\dagger A} =  s_i ||\mathbf A||^2.
\end{split}                  
\end{equation}
Now, since all eigenvalues are positive, one can define the matrix
\begin{equation}
 \mathbf s^{-1/2} = \left[\begin{array}{c c c c}
                          s_1^{-1/2} &&& \\
                          & s_2^{-1/2} && \\
                          && \ddots & \\
                          &&& s_M^{-1/2} 
                         \end{array}\right].
\end{equation}
Multiplying equation (\ref{eq:similarity_transf}) from the left and right by $\mathbf s^{-1/2}$ yields
\begin{equation}
\begin{split}
 \mathbf s^{-1/2} \mathbf U^\dagger \mathbf{S U s}^{-1/2} =  \mathbf I \\
 (\mathbf{Us}^{-1/2})^\dagger \mathbf S (\mathbf{Us}^{-1/2}) =  \mathbf I,
\end{split}
\end{equation}
which means that
\begin{equation}
 \mathbf X = \mathbf{Us}^{-1/2}.
\end{equation}
%%%%%%%%%%%%%%%%%%%%% appendix.tex %%%%%%%%%%%%%%%%%%%%%%%%%%%%%%%%%
%
% sample appendix
%
% Use this file as a template for your own input.
%
%%%%%%%%%%%%%%%%%%%%%%%% Springer-Verlag %%%%%%%%%%%%%%%%%%%%%%%%%%

\appendix
\motto{All's well that ends well}
\chapter{Chapter Heading}
\label{introA} % Always give a unique label
% use \chaptermark{}
% to alter or adjust the chapter heading in the running head

Use the template \emph{appendix.tex} together with the Springer document class SVMono (monograph-type books) or SVMult (edited books) to style appendix of your book in the Springer layout.


\section{Section Heading}
\label{sec:A1}
% Always give a unique label
% and use \ref{<label>} for cross-references
% and \cite{<label>} for bibliographic references
% use \sectionmark{}
% to alter or adjust the section heading in the running head
Instead of simply listing headings of different levels we recommend to let every heading be followed by at least a short passage of text. Furtheron please use the \LaTeX\ automatism for all your cross-references and citations.


\subsection{Subsection Heading}
\label{sec:A2}
Instead of simply listing headings of different levels we recommend to let every heading be followed by at least a short passage of text. Furtheron please use the \LaTeX\ automatism for all your cross-references and citations as has already been described in Sect.~\ref{sec:A1}.

For multiline equations we recommend to use the \verb|eqnarray| environment.
\begin{eqnarray}
\vec{a}\times\vec{b}=\vec{c} \nonumber\\
\vec{a}\times\vec{b}=\vec{c}
\label{eq:A01}
\end{eqnarray}

\subsubsection{Subsubsection Heading}
Instead of simply listing headings of different levels we recommend to let every heading be followed by at least a short passage of text. Furtheron please use the \LaTeX\ automatism for all your cross-references and citations as has already been described in Sect.~\ref{sec:A2}.

Please note that the first line of text that follows a heading is not indented, whereas the first lines of all subsequent paragraphs are.

% For figures use
%
\begin{figure}[t]
\sidecaption[t]
%\centering
% Use the relevant command for your figure-insertion program
% to insert the figure file.
% For example, with the option graphics use
\includegraphics[scale=.65]{figure}
%
% If not, use
%\picplace{5cm}{2cm} % Give the correct figure height and width in cm
%
\caption{Please write your figure caption here}
\label{fig:A1}       % Give a unique label
\end{figure}

% For tables use
%
\begin{table}
\caption{Please write your table caption here}
\label{tab:A1}       % Give a unique label
%
% For LaTeX tables use
%
\begin{tabular}{p{2cm}p{2.4cm}p{2cm}p{4.9cm}}
\hline\noalign{\smallskip}
Classes & Subclass & Length & Action Mechanism  \\
\noalign{\smallskip}\hline\noalign{\smallskip}
Translation & mRNA$^a$  & 22 (19--25) & Translation repression, mRNA cleavage\\
Translation & mRNA cleavage & 21 & mRNA cleavage\\
Translation & mRNA  & 21--22 & mRNA cleavage\\
Translation & mRNA  & 24--26 & Histone and DNA Modification\\
\noalign{\smallskip}\hline\noalign{\smallskip}
\end{tabular}
$^a$ Table foot note (with superscript)
\end{table}
%


\backmatter%%%%%%%%%%%%%%%%%%%%%%%%%%%%%%%%%%%%%%%%%%%%%%%%%%%%%%%
%%%%%%%%%%%%%%%%%%%%%%acronym.tex%%%%%%%%%%%%%%%%%%%%%%%%%%%%%%%%%%%%%%%%%
% sample list of acronyms
%
% Use this file as a template for your own input.
%
%%%%%%%%%%%%%%%%%%%%%%%% Springer %%%%%%%%%%%%%%%%%%%%%%%%%%

\Extrachap{Glossary}


Use the template \emph{glossary.tex} together with the Springer document class SVMono (monograph-type books) or SVMult (edited books) to style your glossary\index{glossary} in the Springer layout.


\runinhead{glossary term} Write here the description of the glossary term. Write here the description of the glossary term. Write here the description of the glossary term.

\runinhead{glossary term} Write here the description of the glossary term. Write here the description of the glossary term. Write here the description of the glossary term.

\runinhead{glossary term} Write here the description of the glossary term. Write here the description of the glossary term. Write here the description of the glossary term.

\runinhead{glossary term} Write here the description of the glossary term. Write here the description of the glossary term. Write here the description of the glossary term.

\runinhead{glossary term} Write here the description of the glossary term. Write here the description of the glossary term. Write here the description of the glossary term.

\Extrachap{Solutions}

\section*{Problems of Chapter~\ref{intro}}

\begin{sol}{prob1}
The solution\index{problems}\index{solutions} is revealed here.
\end{sol}


\begin{sol}{prob2}
\textbf{Problem Heading}\\
(a) The solution of first part is revealed here.\\
(b) The solution of second part is revealed here.
\end{sol}


\printindex

%%%%%%%%%%%%%%%%%%%%%%%%%%%%%%%%%%%%%%%%%%%%%%%%%%%%%%%%%%%%%%%%%%%%%%

\end{document}





 \documentclass[10pt,english,a4wide,psfig,twoside]{book}


\usepackage{textcomp,type1ec,pdfpages}
\usepackage{bera}

\definecolor{dkgreen}{rgb}{0,0.6,0}
\definecolor{gray}{rgb}{0.5,0.5,0.5}
\definecolor{mauve}{rgb}{0.58,0,0.82}

 \lstset{language=c++}
 \lstset{alsolanguage=[90]Fortran}
 \lstset{alsolanguage=python}
% \lstset{basicstyle=\small}
 \lstset{backgroundcolor=\color{white}}
 \lstset{frame=single}
 \lstset{stringstyle=\ttfamily}
 \lstset{keywordstyle=\color{red}\bfseries}
 \lstset{commentstyle=\itshape\color{blue}}
 \lstset{showspaces=false}
 \lstset{showstringspaces=false}
 \lstset{showtabs=false}
 \lstset{breaklines}
 

% Default settings for code listings
% \lstnewenvironment{Python}[1]{
\lstset{%frame=tb,
  language=c++,
  alsolanguage=python,
  %aboveskip=3mm,
 % belowskip=3mm,
  showstringspaces=false,
  columns=flexible,
  basicstyle={\footnotesize\ttfamily},
  numbers=none,
  numberstyle=\tiny\color{gray},
  commentstyle=\color{dkgreen},
  stringstyle=\color{mauve},
  frame=single,  
  breaklines=true,
  %%%% FOR PYTHON 
  otherkeywords={\ , \}, \{},
  keywordstyle=\color{blue},
  emph={void, ||, &&, break, class,continue, delete, else,
  for, if, include, return,try,while},
  emphstyle=\color{black}\bfseries,
  emph={[2]True, False, None, self},
  emphstyle=[2]\color{dkgreen},
  emphstyle=[2]\color{red},
  emph={[3]from, import, as},
  emphstyle=[3]\color{blue},
  upquote=true,
  morecomment=[s]{"""}{"""},
  commentstyle=\color{green}\slshape, %%% cambie gray por green
  emph={[4]1, 2, 3, 4, 5, 6, 7, 8, 9, 0},
  emphstyle=[4]\color{blue},
  breakatwhitespace=true,
  tabsize=2
}

\renewcommand{\lstlistlistingname}{Code Listings}
\renewcommand{\lstlistingname}{Code Listing}
\definecolor{gray}{gray}{0.5}
\definecolor{green}{rgb}{0,0.5,0}

\lstnewenvironment{Python}[1]{
\lstset{
language=python,
basicstyle=\footnotesize\setstretch{1},
stringstyle=\color{red},
showstringspaces=false,
alsoletter={1234567890},
otherkeywords={\ , \}, \{},
keywordstyle=\color{blue},
emph={access,and,break,class,continue,def,del,elif ,else,%
except,exec,finally,for,from,global,if,import,in,is,%
lambda,not,or,pass,print,raise,return,try,while},
emphstyle=\color{black}\bfseries,
emph={[2]True, False, None, self},
emphstyle=[2]\color{red},
emph={[3]from, import, as},
emphstyle=[3]\color{blue},
upquote=true,
morecomment=[s]{"""}{"""},
commentstyle=\color{dkgreen}\slshape, % el color era gray pero lo cambie a verde
emph={[4]1, 2, 3, 4, 5, 6, 7, 8, 9, 0},
emphstyle=[4]\color{blue},
framexleftmargin=1mm, framextopmargin=1mm, rulesepcolor=\color{blue},
breakatwhitespace=true,
tabsize=2
}}{}


\lstnewenvironment{C++}[1]{
\lstset{
language=c++,
% basicstyle=\ttfamily\small\setstretch{1},
basicstyle=\footnotesize\setstretch{1},
stringstyle=\color{red},
showstringspaces=false,
alsoletter={1234567890},
otherkeywords={\ , \}, \{},
keywordstyle=\color{blue},
emph={access,and,break,class,continue,def,del,elif ,else,%
except,exec,finally,for,from,global,if,import,in,is,%
lambda,not,or,pass,print,raise,return,try,while},
emphstyle=\color{black}\bfseries,
emph={[2]True, False, None, self},
emphstyle=[2]\color{red},
emph={[3]from, import, as},
emphstyle=[3]\color{blue},
upquote=true,
morecomment=[s]{"""}{"""},
commentstyle=\color{dkgreen}\slshape, % el color era gray pero lo cambie a verde
emph={[4]1, 2, 3, 4, 5, 6, 7, 8, 9, 0},
emphstyle=[4]\color{blue},
% literate=*{:}{{\textcolor{blue}:}}{1}%
% {=}{{\textcolor{blue}=}}{1}%
% {-}{{\textcolor{blue}-}}{1}%
% {+}{{\textcolor{blue}+}}{1}%
% {*}{{\textcolor{blue}*}}{1}%
% {!}{{\textcolor{blue}!}}{1}%
% {(}{{\textcolor{blue}(}}{1}%
% {)}{{\textcolor{blue})}}{1}%
% {[}{{\textcolor{blue}[}}{1}%
% {]}{{\textcolor{blue}]}}{1}%
% {<}{{\textcolor{blue}<}}{1}%
% {>}{{\textcolor{blue}>}}{1},%
framexleftmargin=1mm, framextopmargin=1mm, rulesepcolor=\color{blue},
breakatwhitespace=true,
tabsize=2
}}{}




\usepackage{tikz}
\usetikzlibrary{shapes,arrows}

% Define block styles
\tikzstyle{decision} = [diamond, draw, fill=blue!20,
    text width=3.5em, text badly centered, node distance=2.5cm, inner sep=0pt]
\tikzstyle{block} = [rectangle, draw, fill=blue!20,
    text width=8em, text centered, rounded corners, minimum height=4em]
\tikzstyle{line} = [draw, very thick, color=black!50, -latex']
\tikzstyle{cloud} = [draw, ellipse,fill=red!20, node distance=2.5cm,
    minimum height=2em]

\def\radius{.7mm} 
\tikzstyle{branch}=[fill,shape=circle,minimum size=3pt,inner sep=0pt]


\newcommand{\bfv}[1]{\boldsymbol{#1}} 
\newcommand{\Div}[1]{\nabla \bullet \vbf{#1}}           % define divergence
\newcommand{\Grad}[1]{\boldsymbol{\nabla}{#1}}
 \newcommand{\OP}[1]{{\bf\widehat{#1}}}
 \newcommand{\be}{\begin{equation}}
 \newcommand{\ee}{\end{equation}}
\newcommand{\beN}{\begin{equation*}}
\newcommand{\bea}{\begin{eqnarray}}
\newcommand{\beaN}{\begin{eqnarray*}}
\newcommand{\eeN}{\end{equation*}}
\newcommand{\eea}{\end{eqnarray}}
\newcommand{\eeaN}{\end{eqnarray*}}
\newcommand{\bdm}{\begin{displaymath}}
\newcommand{\edm}{\end{displaymath}}
\newcommand{\bsubeqs}{\begin{subequations}}
\newcommand{\esubeqs}{\end{subequations}}
\newcommand{\Obs}[1]{\langle{\Op{#1}\rangle}}             % define observable
\newcommand{\PsiT}{\bfv{\Psi_T}(\bfv{R})}                       % symbol for trial wave function
%\newcommand{\braket}[2]{\langle{#1}|\Op{#2}|{#1}\rangle}
\newcommand{\Det}[1]{{|\bfv{#1}|}}
\newcommand{\uvec}[1]{\mbox{\boldmath$\hat{#1}$\unboldmath}}
\newcommand{\Op}[1]{{\bf\widehat{#1}}}    
\newcommand{\eqbrace}[4]{\left\{
\begin{array}{ll}
#1 & #2 \\[0.5cm]
#3 & #4
\end{array}\right.}
\newcommand{\eqbraced}[4]{\left\{
\begin{array}{ll}
#1 & #2 \\[0.5cm]
#3 & #4
\end{array}\right\}}
\newcommand{\eqbracetriple}[6]{\left\{
\begin{array}{ll}
#1 & #2 \\
#3 & #4 \\
#5 & #6
\end{array}\right.}
\newcommand{\eqbracedtriple}[6]{\left\{
\begin{array}{ll}
#1 & #2 \\
#3 & #4 \\
#5 & #6
\end{array}\right\}}

\newcommand{\mybox}[3]{\mbox{\makebox[#1][#2]{$#3$}}}
\newcommand{\myframedbox}[3]{\mbox{\framebox[#1][#2]{$#3$}}}

%% Infinitesimal (and double infinitesimal), useful at end of integrals
%\newcommand{\ud}[1]{\mathrm d#1}
\newcommand{\ud}[1]{d#1}
\newcommand{\udd}[1]{d^2\!#1}

%% Operators, algebraic matrices, algebraic vectors

%% Operator (hat, bold or bold symbol, whichever you like best):
\newcommand{\op}[1]{\widehat{#1}}
%\newcommand{\op}[1]{\mathbf{#1}}
%\newcommand{\op}[1]{\boldsymbol{#1}}

%% Vector:
\renewcommand{\vec}[1]{\boldsymbol{#1}}

%% Matrix symbol:
\newcommand{\matr}[1]{\boldsymbol{#1}}
%\newcommand{\bb}[1]{\mathbb{#1}}

%% Determinant symbol:
\renewcommand{\det}[1]{|#1|}

%% Means (expectation values) of varius sizes
\newcommand{\mean}[1]{\langle #1 \rangle}
\newcommand{\meanb}[1]{\big\langle #1 \big\rangle}
\newcommand{\meanbb}[1]{\Big\langle #1 \Big\rangle}
\newcommand{\meanbbb}[1]{\bigg\langle #1 \bigg\rangle}
\newcommand{\meanbbbb}[1]{\Bigg\langle #1 \Bigg\rangle}

%% Shorthands for text set in roman font
\newcommand{\prob}[0]{\mathrm{Prob}} %probability
\newcommand{\cov}[0]{\mathrm{Cov}}   %covariance
\newcommand{\var}[0]{\mathrm{Var}}   %variancd

%% Big-O (typically for specifying the speed scaling of an algorithm)
\newcommand{\bigO}{\mathcal{O}}

%% Real value of a complex number
\newcommand{\real}[1]{\mathrm{Re}\!\left\{#1\right\}}

%% Quantum mechanical state vectors and matrix elements (of different sizes)
\newcommand{\brab}[1]{\big\langle #1 \big|}
\newcommand{\brabb}[1]{\Big\langle #1 \Big|}
\newcommand{\brabbb}[1]{\bigg\langle #1 \bigg|}
\newcommand{\brabbbb}[1]{\Bigg\langle #1 \Bigg|}
\newcommand{\ketb}[1]{\big| #1 \big\rangle}
\newcommand{\ketbb}[1]{\Big| #1 \Big\rangle}
\newcommand{\ketbbb}[1]{\bigg| #1 \bigg\rangle}
\newcommand{\ketbbbb}[1]{\Bigg| #1 \Bigg\rangle}
\newcommand{\overlap}[2]{\langle #1 | #2 \rangle}
\newcommand{\overlapb}[2]{\big\langle #1 \big| #2 \big\rangle}
\newcommand{\overlapbb}[2]{\Big\langle #1 \Big| #2 \Big\rangle}
\newcommand{\overlapbbb}[2]{\bigg\langle #1 \bigg| #2 \bigg\rangle}
\newcommand{\overlapbbbb}[2]{\Bigg\langle #1 \Bigg| #2 \Bigg\rangle}
\newcommand{\bracket}[3]{\langle #1 | #2 | #3 \rangle}
\newcommand{\bracketb}[3]{\big\langle #1 \big| #2 \big| #3 \big\rangle}
\newcommand{\bracketbb}[3]{\Big\langle #1 \Big| #2 \Big| #3 \Big\rangle}
\newcommand{\bracketbbb}[3]{\bigg\langle #1 \bigg| #2 \bigg| #3 \bigg\rangle}
\newcommand{\bracketbbbb}[3]{\Bigg\langle #1 \Bigg| #2 \Bigg| #3 \Bigg\rangle}
\newcommand{\projection}[2]
{| #1 \rangle \langle  #2 |}
\newcommand{\projectionb}[2]
{\big| #1 \big\rangle \big\langle #2 \big|}
\newcommand{\projectionbb}[2]
{ \Big| #1 \Big\rangle \Big\langle #2 \Big|}
\newcommand{\projectionbbb}[2]
{ \bigg| #1 \bigg\rangle \bigg\langle #2 \bigg|}
\newcommand{\projectionbbbb}[2]
{ \Bigg| #1 \Bigg\rangle \Bigg\langle #2 \Bigg|}


%\proton{xposition,yposition}
\newcommand{\proton}[1]{%
    \shade[ball color=red] (#1) circle (.25);\draw (#1) node{$+$};
}

%\neutron{xposition,yposition}
\newcommand{\neutron}[1]{%
    \shade[ball color=green] (#1) circle (.25);
}

%\electron{xwidth,ywidth,rotation angle}
\newcommand{\electron}[3]{%
    \draw[rotate = #3](0,0) ellipse (#1 and #2)[color=blue];
    \shade[ball color=Gold2] (0,#2)[rotate=#3] circle (.1);
}

\newcommand{\nucleus}{%
    \neutron{0.1,0.3}
    \proton{0,0}
    \neutron{0.3,0.2}
    \proton{-0.2,0.1}
    \neutron{-0.1,0.3}
    \proton{0.2,-0.15}
    \neutron{-0.05,-0.12}
    \proton{0.17,0.21}
}

%\photoelectron{xwidth,ywidth,rotation angle}
\newcommand{\photoelectron}[3]{%
    \draw[rotate = #3](0,0) ellipse (#1 and #2)[color=blue];%
    \draw[snake=coil,%
        line after snake=0pt, segment aspect=0,%
        segment length=20pt,color=red!50!blue](#3:#1)-- +(-6,0)%
        node[fill=white!70!Gold2,draw=red!80!white, above=0.2cm,pos=0.5]%
            {Incoming $\gamma$-photon};%
    \draw[-stealth,Gold2](#3:#1) -- ++ (5,0.625);%
    \shade[ball color=Gold2](#3:#1)  --  ++(4,0.5)%
        node[fill=white!70!Gold2,draw=red!80!white,%
        text width=3cm, below right=0.2cm]%
            {Photoelectron from an inner shell} circle(0.1);%
    \fill  (#1,0)[rotate=#3,color=white,opaque] circle (.1);%
    \draw  (#1,0)[rotate=#3,color=Gold2] circle (.1) ;%
}

%\comptonelectron{xwidth,ywidth,rotation angle}
\newcommand{\comptonelectron}[3]{%
    \draw[rotate = #3](0,0) ellipse (#1 and #2)[color=blue];%
    \draw[snake=coil, line after snake=0pt,%
        segment aspect=0, segment length=10pt,color=red!50!blue]%
        (#3:#1)-- +(-6,0)%
        node[fill=white!70!Gold2,draw=red!80!white, above=0.2cm,pos=0.5]%
            {Incoming $\gamma$-photon};%
    \draw[-stealth,Gold2](#3:#1) -- ++ (5,2.5);%
    \shade[ball color=Gold2](#3:#1)  --  ++(4,2.0)%
        node[fill=white!70!Gold2,draw=red!80!white, text width=3cm,%
        below right=0.2cm]{Scattered electron from an outer shell} circle(0.1);%
    \fill  (#1,0)[rotate=#3,color=white,opaque] circle (.1);%
    \draw  (#1,0)[rotate=#3,color=Gold2] circle (.1) ;%
    \draw[snake=coil, line after snake=1mm, segment aspect=0,%
        segment length=15pt,color=red!50!blue,-stealth] (#3:#1)-- ++(6,-3)%
        node[fill=white!70!Gold2,draw=red!80!white, right=1cm,pos=0.5]%
            {Scattered $\gamma$-photon};%
}

%\paircreation{impact parameter}
\newcommand{\paircreation}[1]{%
    \draw[snake=coil, line after snake=0pt, segment aspect=0,%
        segment length=5pt,color=red!50!blue] (0,#1)-- +(-6,0)%
        node[fill=white!70!Gold2,draw=red!80!white, above=0.2cm,pos=0.5]%
            {Incoming $\gamma$-photon};%
    \draw[-stealth,Gold2](0,#1) -- ++ (5,2.5);%
    \shade[ball color=Gold2](0,#1)  --  ++(4,2.0)%
        node[fill=white!70!Gold2,draw=red!80!white, below right=0.2cm]%
            {Positron} circle(0.1);%
    \draw[-stealth,Gold2](0,#1) -- ++ (4,-2.0);%
    \shade[ball color=Gold2](0,#1)  --  ++(3.2,-1.6)%
        node[fill=white!70!Gold2!,draw=red!80!white, above right=0.2cm]%
            {Electron} circle(0.1);%
}





 \renewcommand{\rmdefault}{ptm} % Times
 \newcommand{\clearemptydoublepage}{\newpage{\pagestyle{empty}\cleardoublepage}}
 \newcommand{\mymarkright}[1]{\markright{\thesection\ -- #1}}
 \newcommand{\mymarkboth}[2]{\markboth{#1}{#2}}
 %
 % redefine page style
 %
 \usepackage{fancyhdr}
 % choose pagestyle fancy and define it
 \pagestyle{fancy}
 \fancyhead{} %clear everything
 \fancyfoot{} %clear everything
 \addtolength{\headheight}{1.5pt}
 \addtolength{\headwidth}{2.5\marginparsep}
 \renewcommand{\chaptermark}[1]{\mymarkboth{#1}{}}
 \renewcommand{\sectionmark}[1]{\mymarkright{#1}}
 \fancyfoot[EL,OR]{\bfseries\itshape\thepage}
 \fancyhead[EL]{\nouppercase{\bfseries\itshape\leftmark}}
 \fancyhead[OR]{\nouppercase{\bfseries\itshape\rightmark}}
 \renewcommand{\headrulewidth}{1pt}
 % redefine plain page style. used on first pages of chapters et.c.
 \fancypagestyle{plain}{\fancyhf{}\fancyfoot[EL,OR]{\bfseries\itshape\thepage}%
 \renewcommand{\headrulewidth}{0pt}
 }
 %
 % redefine bullet. looks better with an ndash.
 %
 \renewcommand{\labelitemi}{\bfseries--}
 %
 % define styles for section and subsection headings
 %
 %\font\tenhv  = phvb at 12pt
 %\newcommand{\hvfont}{\fontfamily{phv}\selectfont}
 %\newcommand{\hvfont}{\fontfamily{phv}\fontseries{bc}\selectfont}
 \newcommand{\secstyle}{\normalfont\itshape\bfseries\large}
 %\newcommand{\secstyle}{\hvfont\itshape\large}
 \newcommand{\subsecstyle}{\normalfont\itshape\large}
 \newcommand{\paragraphstyle}{\normalfont\itshape}
 %
 % redefine look of sections, subsections and paragraphs.
 %
 \makeatletter
 \renewcommand{\section}{%
 \@startsection
   {section}%   the name
   {1}%         the level
   {0pt} %{-\Myindent}%       the indent
   {-2\baselineskip}%   beforeskip
   {\baselineskip}%    afterskip
   {\secstyle}}%   style
 \renewcommand{\subsection}{%
 \@startsection
   {subsection}%   the name
   {2}%         the level
   {0pt}%{-\Myindent}%       the indent
   {-\baselineskip}%   beforeskip
   {0.5\baselineskip}%    afterskip
   {\subsecstyle}}%   style
 \renewcommand{\paragraph}{%
 \@startsection
   {paragraph}%   the name
   {4}%         the level
   {0pt}%{-\Myindent}%       the indent
   {-\baselineskip}%   beforeskip
   {-1em}%    afterskip
   {\paragraphstyle}}%   style
 \makeatother
 %%
 %% modify margins...
 %%
 \addtolength{\oddsidemargin}{-0.5cm}
 %\addtolength{\evensidemargin}{-0.5cm}
 \addtolength{\textwidth}{0.5cm}


 \begin{document}


 \thispagestyle{empty}
 %    \vspace*{\stretch{1}}
     \rule{\linewidth}{1mm}
     \begin{flushright}
           \Huge COMPUTATIONAL PHYSICS\\[5mm]
            Morten Hjorth-Jensen
     \end{flushright}
     \rule{\linewidth}{1mm}
     \vspace*{\stretch{2}}
 \begin{center}
 \begin{figure}[hb]
 \includegraphics[scale=1.0]{Nebbdyr2.ps}
 \end{figure}
 \end{center}
       \begin{center}
 %        \Large{www.computationalphysics.net}
         \Large{University of Oslo, Fall 2010}
       \end{center}

 \pagenumbering{roman}

 \clearemptydoublepage

 \pagestyle{fancy}

 %  the preface

 \section*{Preface}

 \preface
%  last update : 24/8/2013  mhj

\begin{quotation}
So, ultimately, in order to understand nature it may be necessary to
have a deeper understanding of mathematical relationships. But the
real reason is that the subject is enjoyable, and although we humans
cut nature up in different ways, and we have different courses in
different departments, such compartmentalization is really artificial,
and we should take our intellectual pleasures where we find them. 
{\em Richard Feynman, The Laws of Thermodynamics.}
\end{quotation}

Why a preface you may ask? Isn't that just a mere exposition of a
raison d'$\mathrm{\hat{e}}$tre of an author's choice of material,
preferences, biases, teaching philosophy etc.?  To a large extent I
can answer in the affirmative to that. A preface ought to be personal.
Indeed, what you will see in the various chapters of these notes
represents how I perceive computational physics should be taught.

 This set of lecture notes serves the scope of presenting to you and
train you in an algorithmic approach to problems in the sciences,
represented here by the unity of three disciplines, physics,
mathematics and informatics. This trinity outlines the emerging field
of computational physics.

Our insight in a physical system, combined with numerical mathematics
gives us the rules for setting up an algorithm, viz.~a set of rules
for solving a particular problem.  Our understanding of the physical
system under study is obviously gauged by the natural laws at play,
the initial conditions, boundary conditions and other external
constraints which influence the given system. Having spelled out the
physics, for example in the form of a set of coupled partial
differential equations, we need efficient numerical methods in order
to set up the final algorithm.  This algorithm is in turn coded into a
computer program and executed on available computing facilities.  To
develop such an algorithmic approach, you will be exposed to several
physics cases, spanning from the classical pendulum to quantum
mechanical systems. We will also present some of the most popular
algorithms from numerical mathematics used to solve a plethora of
problems in the sciences.  Finally we will codify these algorithms
using some of the most widely used programming languages, presently C,
C++ and Fortran and its most recent standard Fortran
2008\footnote{Throughout this text we refer to Fortran 2008 as
Fortran, implying the latest standard.}. However, a high-level and fully
object-oriented language like Python is now emerging as a good
alternative although C++ and Fortran still outperform Python when it
comes to computational speed.  In this text we offer an approach where
one can write all programs in C/C++ or Fortran.  We will also show you
how to develop large programs in Python interfacing C++ and/or Fortran
functions for those parts of the program which are CPU intensive.
Such an approach allows you to structure the flow of data in a
high-level language like Python while tasks of a mere repetitive and
CPU intensive nature are left to low-level languages like C++ or
Fortran. Python allows you also to smoothly interface your program
with other software, such as plotting programs or operating system
instructions. A typical Python program you may end up writing contains
everything from compiling and running your codes to preparing the body
of a file for writing up your report.



Computer simulations are nowadays an integral part of contemporary
basic and applied research in the sciences.  Computation is becoming
as important as theory and experiment. In physics, computational
physics, theoretical physics and experimental physics are all equally
important in our daily research and studies of physical
systems. Physics is the unity of theory, experiment and
computation\footnote{We mentioned previously the trinity of physics,
mathematics and informatics. Viewing physics as the trinity of theory,
experiment and simulations is yet another example. It is obviously
tempting to go beyond the sciences. History shows that triunes,
trinities and for example triple deities permeate the Indo-European
cultures (and probably all human cultures), from the ancient Celts and
Hindus to modern days.  The ancient Celts revered many such trinues,
their world was divided into earth, sea and air, nature was divided in
animal, vegetable and mineral and the cardinal colours were red,
yellow and blue, just to mention a few.  As a curious digression, it
was a Gaulish Celt, Hilary, philosopher and bishop of Poitiers (AD
315-367) in his work {\em De Trinitate} who formulated the Holy
Trinity concept of Christianity, perhaps in order to accomodate
millenia of human divination practice.}.  Moreover, the ability "to
compute" forms part of the essential repertoire of research
scientists. Several new fields within computational science have
emerged and strengthened their positions in the last years, such as
computational materials science, bioinformatics, computational
mathematics and mechanics, computational chemistry and physics and so
forth, just to mention a few.  These fields underscore the importance
of simulations as a means to gain novel insights into physical
systems, especially for those cases where no analytical solutions can
be found or an experiment is too complicated or expensive to carry
out.  To be able to simulate large quantal systems with many degrees
of freedom such as strongly interacting electrons in a quantum dot
will be of great importance for future directions in novel fields like
nano-techonology.  This ability often combines knowledge from many
different subjects, in our case essentially from the physical
sciences, numerical mathematics, computing languages, topics from
high-performace computing and some knowledge of computers.


In 1999, when I started this course at the department of physics in
Oslo, computational physics and computational science in general were
still perceived by the majority of physicists and scientists as topics
dealing with just mere tools and number crunching, and not as subjects
of their own.  The computational background of most students enlisting
for the course on computational physics could span from dedicated
hackers and computer freaks to people who basically had never used a
PC. The majority of undergraduate and graduate students had a very
rudimentary knowledge of computational techniques and methods.
Questions like 'do you know of better methods for numerical
integration than the trapezoidal rule' were not uncommon. I do happen
to know of colleagues who applied for time at a supercomputing centre
because they needed to invert matrices of the size of $10^4\times
10^4$ since they were using the trapezoidal rule to compute
integrals. With Gaussian quadrature this dimensionality was easily
reduced to matrix problems of the size of $10^2\times 10^2$, with much
better precision.

More than a decade later most students have now been exposed to a
fairly uniform introduction to computers, basic programming skills and
use of numerical exercises.  Practically every undergraduate student
in physics has now made a Matlab or Maple simulation of for example
the pendulum, with or without chaotic motion.  Nowadays most of you
are familiar, through various undergraduate courses in physics and
mathematics, with interpreted languages such as Maple, Matlab and/or
Mathematica. In addition, the interest in scripting languages such as
Python or Perl has increased considerably in recent years.  The modern
programmer would typically combine several tools, computing
environments and programming languages. A typical example is the
following. Suppose you are working on a project which demands
extensive visualizations of the results. To obtain these results, that
is to solve a physics problems like obtaining the density profile of a
Bose-Einstein condensate, you need however a program which is fairly
fast when computational speed matters.  In this case you would most
likely write a high-performance computing program using Monte Carlo
methods in languages which are tailored for that. These are
represented by programming languages like Fortran and C++.  However,
to visualize the results you would find interpreted languages like
Matlab or scripting languages like Python extremely suitable for your
tasks.  You will therefore end up writing for example a script in
Matlab which calls a Fortran or C++ program where the number crunching
is done and then visualize the results of say a wave equation solver
via Matlab's large library of visualization tools. Alternatively, you
could organize everything into a Python or Perl script which does
everything for you, calls the Fortran and/or C++ programs and performs
the visualization in Matlab or Python. Used correctly, these tools,
spanning from scripting languages to high-performance computing
languages and vizualization programs, speed up your capability to
solve complicated problems.  Being multilingual is thus an advantage
which not only applies to our globalized modern society but to
computing environments as well.  This text shows you how to use C++
and Fortran as programming languages.

There is however more to the picture than meets the eye.  Although
interpreted languages like Matlab, Mathematica and Maple allow you
nowadays to solve very complicated problems, and high-level languages
like Python can be used to solve computational problems, computational
speed and the capability to write an efficient code are topics which
still do matter. To this end, the majority of scientists still use
languages like C++ and Fortran to solve scientific problems.  When you
embark on a master or PhD thesis, you will most likely meet these
high-performance computing languages.  This course emphasizes thus the
use of programming languages like Fortran, Python and C++ instead of
interpreted ones like Matlab or Maple. You should however note that
there are still large differences in computer time between for example
numerical Python and a corresponding C++ program for many numerical
applications in the physical sciences, with a code in C++ or Fortran
being the fastest.

Computational speed is not the only reason for this choice of
programming languages. Another important reason is that we feel that
at a certain stage one needs to have some insights into the algorithm
used, its stability conditions, possible pitfalls like loss of
precision, ranges of applicability, the possibility to improve the
algorithm and taylor it to special purposes etc etc.  One of our major
aims here is to present to you what we would dub 'the algorithmic
approach', a set of rules for doing mathematics or a precise
description of how to solve a problem. To device an algorithm and
thereafter write a code for solving physics problems is a marvelous
way of gaining insight into complicated physical systems. The
algorithm you end up writing reflects in essentially all cases your
own understanding of the physics and the mathematics (the way you
express yourself) of the problem.  We do therefore devote quite some
space to the algorithms behind various functions presented in the
text. Especially, insight into how errors propagate and how to avoid
them is a topic we would like you to pay special attention to. Only
then can you avoid problems like underflow, overflow and loss of
precision. Such a control is not always achievable with interpreted
languages and canned functions where the underlying algorithm and/or
code is not easily accesible.  Although we will at various stages
recommend the use of library routines for say linear
algebra\footnote{Such library functions are often taylored to a given
machine's architecture and should accordingly run faster than user
provided ones.}, our belief is that one should understand what the
given function does, at least to have a mere idea.  With such a
starting point, we strongly believe that it can be easier to develope
more complicated programs on your own using Fortran, C++ or Python.

We have several other aims as well, namely:
\begin{itemize}
\item We would like to give you  an opportunity to gain a 
      deeper understanding of the physics you have learned in other
      courses. In most courses one is normally confronted with simple
      systems which provide exact solutions and mimic to a certain
      extent the realistic cases. Many are however the comments like
      'why can't we do something else than the particle in a box
      potential?'.  In several of the projects we hope to present some
      more 'realistic' cases to solve by various numerical
      methods. This also means that we wish to give examples of how
      physics can be applied in a much broader context than it is
      discussed in the traditional physics undergraduate curriculum.
\item To encourage you to "discover" physics in a way similar to how 
researchers learn in the context of research.
\item Hopefully also to introduce numerical methods and new areas of physics that 
      can be studied with the methods discussed.
\item To teach   structured programming in the context of doing science. 
\item The projects we propose are meant to mimic to a certain extent 
      the situation encountered during a thesis or project work. You
      will tipically have at your disposal 2-3 weeks to solve
      numerically a given project. In so doing you may need to do a
      literature study as well. Finally, we would like you to write a
      report for every project.
\end{itemize}
Our overall goal is to encourage you to learn about science through
experience and by asking questions. Our objective is always
understanding and the purpose of computing is further insight, not
mere numbers!  Simulations can often be considered as
experiments. Rerunning a simulation need not be as costly as rerunning
an experiment.


 
Needless to say, these lecture notes are upgraded continuously, from
typos to new input.  And we do always benefit from your comments,
suggestions and ideas for making these notes better.  It's through the
scientific discourse and critics we advance.  Moreover, I have
benefitted immensely from many discussions with fellow colleagues and
students. In particular I must mention Hans Petter Langtangen, Anders
Malthe-S\o renssen, Knut M\o rken and \O yvind Ryan, whose input
during the last fifteen years has considerably improved these lecture
notes.  Furthermore, the time we have spent and keep spending together
on the Computing in Science Education project at the University, is
just marvelous. Thanks so much. Concerning the Computing in Science
Education initiative, you can read more
at \url{http://www.mn.uio.no/english/about/collaboration/cse/}.


Finally, I would like to add a petit note on referencing. These notes
have evolved over many years and the idea is that they should end up
in the format of a web-based learning environment for doing
computational science. It will be fully free and hopefully represent a
much more efficient way of conveying teaching material than
traditional textbooks.  I have not yet settled on a specific format,
so any input is welcome. At present however, it is very easy for me to
upgrade and improve the material on say a yearly basis, from simple
typos to adding new material.  When accessing the web page of the
course, you will have noticed that you can obtain all source files for
the programs discussed in the text.  Many people have thus written to
me about how they should properly reference this material and whether
they can freely use it. My answer is rather simple.  You are
encouraged to use these codes, modify them, include them in
publications, thesis work, your lectures etc.  As long as your use is
part of the dialectics of science you can use this material freely.
However, since many weekends have elapsed in writing several of these
programs, testing them, sweating over bugs, swearing in front of a
f*@?\%g code which didn't compile properly ten minutes before monday
morning's eight o'clock lecture etc etc, I would dearly appreciate in
case you find these codes of any use, to reference them properly. That
can be done in a simple way, refer to M.~Hjorth-Jensen, {\em
Computational Physics}, University of Oslo (2013). The weblink to the
course should also be included. Hope it is not too much to ask
for. Enjoy!


 \clearemptydoublepage

 \tableofcontents

 \clearemptydoublepage

 \pagenumbering{arabic}
 %  Introductory chapters
         \part{Introduction to Programming and Numerical Methods}
 %%  Introduction
         \input{introduction}
 \clearemptydoublepage
 %%  introduction to C++ and F90 programming
         \input{basicCF90.tex}
 \clearemptydoublepage
 %% numerical derivation
      \input{differentiate.tex}
 \clearemptydoublepage
 %% Classes.
%\input{classes}
% \clearemptydoublepage
 %% numerical integration
      \input{integrate.tex}
 \clearemptydoublepage
 % this section should be included at an earlier stage
      \input{nonlinear.tex}
 \clearemptydoublepage
% \clearemptydoublepage
         \part{Linear Algebra and Eigenvalue problems}
 %% Linear algebra.
 \input{linalgebra}
% Put diagonalization chapter here
 \clearemptydoublepage
 %% interpolation  this chapter should come earlier
%      \input{interpolate.tex}
% \clearemptydoublepage
 %% eigenvalue systems
     \input{eigenvalue.tex}
 \clearemptydoublepage
        \part{Ordinary and Partial Differential Equations}
 %%  Differential equations 
         \input{diffeq.tex}
 \clearemptydoublepage
 %% Two point boundary value problems. 
         \input{twopboundary.tex}
 \clearemptydoublepage
 %% Partial differential equations, finite difference
 \input{partdiff.tex}
 \clearemptydoublepage
        \part{Monte Carlo Methods}
 %% Monte Carlo methods
      \input{montecarlo_intro.tex}
 \clearemptydoublepage
 %% random walks and the diffusion equation
      \input{randowalks.tex}
 \clearemptydoublepage
 %% Monte carlo applications, stat phys
      \input{stat_phys.tex}
 \clearemptydoublepage
% \chapter{Modelling Phase Transitions in Statistical Physics}\label{chap:advancedstatphys}
% \input{advancedsm.tex}
% \clearemptydoublepage
 %% Monte carlo applications, quantum mechanics
      \input{vmc.tex}
 \clearemptydoublepage




 %  Advanced topics
 \part{Advanced topics}   

 \chapter{Many-body approaches to studies of electronic systems: Hartree-Fock theory}\label{chap:advancedatoms}

 \input{advancedatoms}

 \clearemptydoublepage
 \chapter{Bose-Einstein condensation and Diffusion Monte Carlo}\label{chap:advancedqmc}

 \input{advancedqm.tex}


% \clearemptydoublepage
%\chapter{Density functional theory}\label{chap:dft}
%\input{dft}



% \clearemptydoublepage




%\chapter{Quantum Information Theory and Quantum Algorithms}\label{chap:quantinfo}


%\input{quantuminformation}





 \clearemptydoublepage



 \bibliographystyle{unsrt}

 \bibliography{../../../Library_of_Topics/mylib}

 \end{document}









